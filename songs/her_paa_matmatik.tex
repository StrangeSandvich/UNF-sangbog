\begin{song}{Her på mat'matik}
  {} % Bruges ikke, lad stå blank
  {Fætter Mikkel} % Titel, Kunstner - eks.: "Jutlandia, Kim Larsen". Hvis sangen er på sin egen melodi, brug da \SBOrgMel.
  {} % Navnet på forfatteren. Undlad kaldenavne. Brug gerne TBF. Brug "&" frem for "og". Hvis forfatter er ukendt, lad da stå tom.
  {Matematikrevyen, 2015} % Eks. "Fysikrevy, 2010" eller "2010"
  {\NotCCLIed} % Lad stå som den er

  \begin{SBVerse}
    Her på mat'matik skal man være kvik\\
    Sætte pris på elegance\\
    Ha' intuition, også ambition\\
    Gribe fat i hver en chance
  \end{SBVerse}

  \begin{SBChorus}
    For vi elsker ren logik \emph{(klap-klap)}\\
    algebra og statistik \emph{(klap-klap)}\\
    Alt der er komplekst: analyse, vækst,\\
    det er her vi har det bedst \emph{(klap-klap)}
  \end{SBChorus}

  \begin{SBVerse}
    Fra det første år, kurserne består\\
    Fællesskabet ej forglemme\\
    Der bli'r undervist, masser kage spist\\
    Det er her hvor vi har hjemme
  \end{SBVerse}

  \begin{SBChorus}
    For vi elsker ren logik\ldots
  \end{SBChorus}

  \begin{SBSection*}
    % Skriv sektioner her. Hvis du ønsker lidt mellemrum for at give luft i et langt afsnit el.lign., brug da \\\medskip
  \end{SBSection*}
\end{song}