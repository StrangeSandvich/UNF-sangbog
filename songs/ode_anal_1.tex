\begin{song}{Ode til analyse 1}
  {} % Bruges ikke, lad stå blank
  {Jeg tror det kaldes kærlighed, Shout} % Titel, Kunstner - eks.: "Jutlandia, Kim Larsen". Hvis sangen er på sin egen melodi, brug da \SBOrgMel.
  {} % Navnet på forfatteren. Undlad kaldenavne. Brug gerne TBF. Brug "&" frem for "og". Hvis forfatter er ukendt, lad da stå tom.
  {Matematikrevy, 2013} % Eks. "Fysikrevy, 2010" eller "2010"
  {\NotCCLIed} % Lad stå som den er

  \begin{SBVerse}
    Hvordan gi’r vi mening til\\
    Et uend’ligt summespil?\\
    Tallene bli’r fler og fler\\
    Tjek om delsum konverger’
  \end{SBVerse}

  \begin{SBVerse}
    Sum af n’te del i p\\
    Forholdstest og rottesten\\
    Vi lærte om en dejlig ven\\
    Den hedder Grænsesammenligningstesten
  \end{SBVerse}

  \begin{SBChorus}
    Denne sang den er til dig\\
    Åh Jan Philip Solovej\\
    Når funktionerne de ændrer sig\\
    \medskip
    Så ta’r vi det skridt for skridt\\
    Lært i Analyse 1\\
    Lad n gå mod uendeligt
  \end{SBChorus}

  \begin{SBVerse}
    Nu blev faget mer’ abstrakt\\
    Vi fandt en sum for $\pi$ eksakt\\
    Lærte om vor tavlesvamp\\
    Valget tør og våd er\ldots ikke mer’ en kamp
  \end{SBVerse}

  \begin{SBChorus}
    Denne sang den er til dig\\
    Åh Jan Philip Solovej\\
    Når funktionerne de ændrer sig\\
    \medskip
    Så ta’r vi det skridt for skridt\\
    Lært i Analyse 1\\
    Lad n gå mod uendeligt
  \end{SBChorus}

  \begin{SBSection*}
    Er du nu langt væk eller så nær?\\
    \emph{(Er du nu langt væk eller så nær?)}\\
    Er metrikken 0, så du er her?\\
    \emph{Wow-yeahhh}
  \end{SBSection*}

  \begin{SBChorus}
    Denne sang den er til dig\\
    Åh Jan Philip Solovej\\
    Når funktionerne de ændrer sig\\
    \medskip
    Så ta’r vi det skridt for skridt\\
    Lært i Analyse 1\\
    Lad n gå mod uendeligt
  \end{SBChorus}

  \begin{SBChorus}
    Denne sang den er til dig\\
    Åh Jan Philip Solovej\\
    Når funktionerne de ændrer sig\\
    \medskip
    Så ta’r vi det skridt for skridt\\
    Lært i Analyse 1\\
    Lad n gå mod uendeligt
  \end{SBChorus}

  \begin{SBSection*}
    Lad n gå mod uendeligt
  \end{SBSection*}
\end{song}
