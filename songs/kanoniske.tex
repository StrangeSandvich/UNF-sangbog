\begin{song}{Den kanoniske matematikersang}
  {} % Bruges ikke, lad stå blank
  {Bamses fødselsdag}
  {Esben Bistrup Halvorsen og Rasmus Resen Amossen}
  {}
  {\NotCCLIed}

  \begin{SBVerse}
	Som mat'matikstuderende,\\
	så er jeg svær at narre.\\
	Når andre kalder mig for nørd,\\
	så må jeg bare svare:
  \end{SBVerse}

  \begin{SBChorus}
	Hip hurra for algebra,\\
	Euler, Gauss og Galois\\
	og for rum med en kompakt\\
	deformationsretrakt.
  \end{SBChorus}

  \begin{SBVerse}
	En dag jeg sa' til kæresten:\\
	"nu vil jeg dyrke grupper".\\
	Hun kiggede forbavset op,\\
	"det ikke super duper!".
  \end{SBVerse}

  \begin{SBSection*}
	"Jamen det er ej med dig,\\
	gutterne de er på vej".\\
	"Får du ikke nok af mig?!"\\
	 -så rejste hun sin vej.\\
  \end{SBSection*}

  \begin{SBVerse}
	Nu var jeg blevet singelton,\\
	og sagde til min moder:\\
	"Jeg er disjunkt med kæresten\\
	som Peking med Nyboder.
  \end{SBVerse}

  \begin{SBSection*}
	Inklusionen er nu vendt,\\
	jeg er blevet transcendent".\\
	"Er du trans, din klamme tøs?!"\\
	-så blev jeg arveløs.
  \end{SBSection*}

  \begin{SBVerse}
	Da arven nu var faldet bort,\\
	jeg måtte til at spare.\\
	Jeg talte med min vicevært\\
	og kunne ham forklare:
  \end{SBVerse}

  \begin{SBSection*}
	"Pengemængden er kompakt,\\
	jeg vil ha' en ny kontrakt!".\\
	"Fint med mig, den kommer her:\\
	Du bor her ikke mer'!".
  \end{SBSection*}

  \begin{SBVerse}
	Nu var jeg efterhånden ved\\
	at få lidt hovedpine.\\
	Jeg tog til hospitalet og\\
	fik hjælp af en blondine.
  \end{SBVerse}

  \begin{SBSection*}
	"Hovedsmerten går for vidt,\\
	den sku' deles op i snit!".\\
	"Hvidt og snit, og så nå'et, ik'?"\\
	-der røg mit overblik.
  \end{SBSection*}

  \begin{SBVerse}
	Jeg har nu få't det hvide snit\\
	og boligen er røget,\\
	men selvom både arv og kær'-\\
	ste fløj, er jeg fornøjet!
  \end{SBVerse}

  \begin{SBChorus}
	Hip hurra for algebra,\\
	Euler, Gauss og Galois\\
	og for rum med en kompakt\\
	deformationsretrakt.
  \end{SBChorus}
\end{song}




