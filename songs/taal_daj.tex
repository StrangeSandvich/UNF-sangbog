\begin{song}{Tål daj' i læseferien}
  {} % Bruges ikke, lad stå blank
  {12 Days of Christmas} % Titel, Kunstner - eks.: "Jutlandia, Kim Larsen". Hvis sangen er på sin egen melodi, brug da \SBOrgMel.
  {Mona Holm \& Marianne Bangsø} % Navnet på forfatteren. Undlad kaldenavne. Brug gerne TBF. Brug "&" frem for "og". Hvis forfatter er ukendt, lad da stå tom.
  {\TKET{}s Julerevy, 1987} % Eks. "Fysikrevy, 2010" eller "2010"
  {\NotCCLIed} % Lad stå som den er

  \begin{SBVerse}
    På den første dag i læseferien sagde jeg til mig selv:\\
    Der er god tid, jeg pjækker i dag.
  \end{SBVerse}

  \begin{SBVerse}
    På den anden dag i læseferien sagde jeg til mig selv:\\
    Mat’matik er trist,\\
    der er god tid, jeg pjækker i dag.
  \end{SBVerse}

  \begin{SBVerse}
    På den tredje dag i læseferien sagde jeg til mig selv:\\
    Jeg er træt,\\
    mat’matik er trist,\\
    der er god tid, jeg pjækker i dag.
  \end{SBVerse}

  \begin{SBVerse}
    På den fjerde dag i læseferien sagde jeg til mig selv:\\
    Tror jeg går på druk -- \emph{Skål!}\\
    Jeg er træt,\\
    mat’matik er trist,\\
    der er god tid, jeg pjækker i dag.
  \end{SBVerse}

  \begin{SBVerse}
    På den femte dag i læseferien sagde jeg til mig selv:\\
    Åh, tømmermænd!\\
    Tror jeg går på druk -- \emph{Skål!}\\
    Jeg er træt,\\
    mat’matik er trist,\\
    der er god tid, jeg pjækker i dag.
  \end{SBVerse}

  \begin{SBVerse}
    På den sjette dag i læseferien sagde jeg til mig selv:\\
    Der er fest i aften,\\
    åh, tømmermænd!\\
    Tror jeg går på druk -- \emph{Skål!}\\
    Jeg er træt,\\
    mat’matik er trist,\\
    der er god tid, jeg pjækker i dag.
  \end{SBVerse}

  \begin{SBVerse}
    På den syvende dag i læseferien sagde jeg til mig selv:\\
    Hvor er min bog nu?\\
    Der er fest i aften,\\
    åh, tømmermænd!\\
    Tror jeg går på druk -- \emph{Skål!}\\
    Jeg er træt,\\
    mat’matik er trist,\\
    der er god tid, jeg pjækker i dag.
  \end{SBVerse}

  \begin{SBVerse}
    På den ottende dag i læseferien sagde jeg til mig selv:\\
    Der' film i TV,\\
    hvor er min bog nu?\\
    Der er fest i aften,\\
    åh, tømmermænd!\\
    Tror jeg går på druk -- \emph{Skål!}\\
    Jeg er træt,\\
    mat’matik er trist,\\
    der er god tid, jeg pjækker i dag.
  \end{SBVerse}

  \begin{SBVerse}
    På den niende dag i læseferien sagde jeg til mig selv:\\
    Savner mine venner,\\
    der' film i TV,\\
    hvor er min bog nu?\\
    Der er fest i aften,\\
    åh, tømmermænd!\\
    Tror jeg går på druk -- \emph{Skål!}\\
    Jeg er træt,\\
    mat’matik er trist,\\
    der er god tid, jeg pjækker i dag.
  \end{SBVerse}

  \begin{SBVerse}
    På den tiende dag i læseferien sagde jeg til mig selv:\\
    Mon jeg kan nå det?\\
    Savner mine venner,\\
    der' film i TV,\\
    hvor er min bog nu?\\
    Der er fest i aften,\\
    åh, tømmermænd!\\
    Tror jeg går på druk -- \emph{Skål!}\\
    Jeg er træt,\\
    mat’matik er trist,\\
    der er god tid, jeg pjækker i dag.
  \end{SBVerse}

  \begin{SBVerse}
    På den elvete dag i læseferien sagde jeg til mig selv:\\
    Nu skal der læses!\\
    Mon jeg kan nå det?\\
    Savner mine venner,\\
    der' film i TV,\\
    hvor er min bog nu?\\
    Der er fest i aften,\\
    åh, tømmermænd!\\
    Tror jeg går på druk -- \emph{Skål!}\\
    Jeg er træt,\\
    mat’matik er trist,\\
    der er god tid, jeg pjækker i dag.
  \end{SBVerse}

  \begin{SBVerse}
    På den tolvte dag i læseferien sagde jeg til mig selv:\\
    Gud, det' i morgen!\\
    Nu skal der læses!\\
    Mon jeg kan nå det?\\
    Savner mine venner,\\
    der' film i TV,\\
    hvor er min bog nu?\\
    Der er fest i aften,\\
    åh, tømmermænd!\\
    Tror jeg går på druk -- \emph{Skål!}\\
    Jeg er træt,\\
    mat’matik er trist,\\
    Så'n gik tiden -- jeg dumped' i dag!
  \end{SBVerse}
\end{song}