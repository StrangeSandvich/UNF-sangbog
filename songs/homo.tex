\begin{song}{Homo-sangen}
  {} % Bruges ikke, lad stå blank
  {\SBOrgMel} % Titel, Kunstner - eks.: "Jutlandia, Kim Larsen". Hvis sangen er på sin egen melodi, brug da \SBOrgMel.
  {Jan Midtgaard} % Navnet på forfatteren. Undlad aliasser. Brug "&" frem for "og". Hvis forfatter er ukendt, lad da stå tom.
  {TÅGEKAMMERETs Julerevy, 1999} % Eks. "Fysikrevy 2010" eller "2010"
  {\NotCCLIed} % Lad stå som den er

  \begin{SBVerse}
    Livets mange glæder man jo dele kan,\\
    men det er nu bedt, så'n mand til mand'\\
    Intet er skam større end den ægte kærlighed,\\
    spring ud af skabet, tag din læsemakker med!
  \end{SBVerse}

  \begin{SBChorus}
    Får du lyst til rigtig mand igen,\\
    er det for lidt med kun en pige-ven.\\
    Du får et helt nyt syn på lineær algebra \emph{(shi-bu-du-ah)}\\
    når du gi’r din læsemakker den bagfra!
  \end{SBChorus}

  \begin{SBVerse}
    Når du går og voldtag’r en and ved unisø’n
    ved du den er gal med den der kløen.
    Til TÅGEKAMMER-fester du scorer ikke spor,
    den sidste pige som du kyssed’ var din mor!
  \end{SBVerse}

  \begin{SBChorus}
    Får du lyst til rigtig mand igen,\ldots
  \end{SBChorus}

  \begin{SBSection*}
  Her på matematisk der er pigerne få,\\
  her går sexlivet bestemt i stå.
  \end{SBSection*}

  \begin{SBChorus}
    Du får et helt nyt syn på lineær algebra \emph{(shi-bu-du-ah)}\\
    når du gi’r din læsemakker,\\
    når du gi'r din forelæser,\\
    når du gi'r din koordinator den bagfra!
  \end{SBChorus}
\end{song}