\begin{song}{Glad for Fysik}
  {} % Bruges ikke, lad stå blank
  {Quangs sang, Anders Matthesen} % Titel, Kunstner - eks.: "Jutlandia, Kim Larsen". Hvis sangen er på sin egen melodi, brug da \SBOrgMel.
  {} % Navnet på forfatteren. Undlad kaldenavne. Brug gerne TBF. Brug "&" frem for "og". Hvis forfatter er ukendt, lad da stå tom.
  {FysikRevy, 2013} % Eks. "Fysikrevy, 2010" eller "2010"
  {\NotCCLIed} % Lad stå som den er

  \begin{SBVerse}
	Her er en historie om en dreng der hedder Finn\\
	Han er humanist på 7. år  og fatter ik' en pind\\
	KUA det er mørkt og goldt, og Finn har studiegæld\\
	Så han prøver sig med Logos, for det meste uden held
  \end{SBVerse}

  \begin{SBVerse}
	Jeg har kendt en smart jurist, hvis navn jeg nu har glemt\\
	Han er specialist i at slikke røv, og han har det ikke nemt\\
	Han pukler hele natten, til den næste morgen gryr\\
	Kopierer ting for rige folk og lever af geby'r
  \end{SBVerse}

  \begin{SBChorus}
	Så hvordan kan du dog sige, at du ikke er tilfreds?\\
	At din regnelærer er dum og sur og dine lektier gi'r dig stress?\\
	Du har alt, hvad du skal bruge, du har Schuams og hævegrej\\
	Og der er tusind humanister, der ville ønske de var dig
  \end{SBChorus}

  \begin{SBVerse}
	Tom sidder og koder med kroppen på et ton\\
	Han har slet ingen venner, og hans kær'ste er hans hånd\\
	Han scorer sjældent piger, men synes Ponyer er fedt\\
	Han er groet fast bag skærmen, hvor han koder latin-1
  \end{SBVerse}

  \begin{SBChorus}
	Så hvad driver dig til at sige at studiet er grumt?\\
	At du ikke gider Newton og synes termo, det er dumt?\\
	Du har alt, hvad du skal bruge, du har Schuams og hævegrej\\
	Og utallige dataloger ville ønske de var dig
  \end{SBChorus}
\end{song}