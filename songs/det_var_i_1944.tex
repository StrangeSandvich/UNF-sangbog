\begin{song}{Det var i 1944}
  {} % Bruges ikke, lad stå blank
  {Jutlandia, Kim Larsen} % Titel, Kunstner - eks.: "Jutlandia, Kim Larsen". Hvis sangen er på sin egen melodi, brug da \SBOrgMel.
  {DBR, MWR} % Navnet på forfatteren. Undlad kaldenavne. Brug gerne TBF. Brug "&" frem for "og". Hvis forfatter er ukendt, lad da stå tom.
  {UNF Revy, 2016} % Eks. "Fysikrevy, 2010" eller "2010"
  {\NotCCLIed} % Lad stå som den er

  \begin{SBVerse}
    Det var i 1944 eller cirka der omkring,\\
    da UNF blev stiftet\\
    Vi kommer fra foreningen, der hedder SNU,\\
    nen nu er de forældet\\
    Ung men’sker fra gym’asiet af\\
    uddeler naturvidenskab
  \end{SBVerse}

  \begin{SBChorus}
    Ja ja! Fra UNF af\\
    Vi kommer som altid med viden\\
    Newton med tyngdekraft og Bohr med fysik,\\
    og Bjerrum, han lav’d diagrammet
  \end{SBChorus}

  \begin{SBVerse}
    Vi holder foredrag og workshops, og vi tar’ på studietur\\
    Når vi formidler vor viden\\
    Rundt i hele landet, ja, med alle vor forsøg\\
    - og folk de falder på stribe\\
    Forskere, det skal vi alle vær’\\
    Come on students, vis os noget blær
  \end{SBVerse}

  \begin{SBChorus}
    Ja ja! Fra UNF af\\
    Vi kommer som altid med viden\\
    Newton med tyngdekraft og Bohr med fysik,\\
    og Bjerrum, han lav’d diagrammet
  \end{SBChorus}

  % \begin{SBChorus}
  %   Ja ja! Fra UNF af\ldots
  % \end{SBChorus}

  \begin{SBVerse}
    Vi er fir' foreninger der dækker vores land\\
    men vi vil gern’ være flere\\
    Vi ha’d’ både Midtjylland og Sønderborg en gang\\
    Så’n er det ikke mere\\
    Arrangører på 16 år\\
    laver camps, som de ikke forstår
  \end{SBVerse}

  \begin{SBChorus}
    Ja ja! Fra UNF af\\
    Vi kommer som altid med viden\\
    Newton med tyngdekraft og Bohr med fysik,\\
    og Bjerrum, han lav’d diagrammet
  \end{SBChorus}

  \begin{SBChorus}
    Ja ja! Fra UNF af\\
    Vi kommer som altid med viden\\
    Newton med tyngdekraft og Bohr med fysik,\\
    og Bjerrum, han lav’d diagrammet
  \end{SBChorus}

  % \begin{SBChorus}
  %   Ja ja! Fra UNF af\ldots
  % \end{SBChorus}

  % \begin{SBChorus}
  %   Ja ja! Fra UNF af\ldots
  % \end{SBChorus}
\end{song}