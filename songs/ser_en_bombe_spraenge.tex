\begin{song}{Når jeg ser en bombe sprænge}
  {} % Bruges ikke, lad stå blank
  {Naar jeg ser et rødt Flag smælde, Oskar Hansen} % Titel, Kunstner - eks.: "Jutlandia, Kim Larsen". Hvis sangen er på sin egen melodi, brug da \SBOrgMel.
  {} % Navnet på forfatteren. Undlad kaldenavne. Brug gerne TBF. Brug "&" frem for "og". Hvis forfatter er ukendt, lad da stå tom.
  {FysikRevy, 2006} % Eks. "Fysikrevy, 2010" eller "2010"
  {\NotCCLIed} % Lad stå som den er

  \begin{SBVerse}
    Når jeg ser på datalogen\\
    og hans kode fuld af slam,\\
    når jeg ser på biologen,\\
    der så nænsomt klapper sit lam\\
    bli'r jeg trist for alle de stakler\\
    som aldrig får lyset at se.\\
    Os som dagligt fysikproblemer takler\\
    kan kun af deres faglighed le.
  \end{SBVerse}

  \begin{SBVerse}
    Vi er lærlinge af Maxwell\\
    Vi har gå'd i Newtons spor\\
    Vi har lyttet til Jens Martin fortæl'\\
    I kan blive vor lands næste Bohr\\
    Så vor ædle historie forpligter\\
    og på fortidens skuldre vi står\\
    Evigt fremad mod stjernerne vi sigter\\
    mod nye horisonter vi går.
  \end{SBVerse}

  \begin{SBVerse}
    Når jeg ser en bombe sprænge\\
    i en ørken fuld af sand\\
    kan jeg høre drønet så længe\\
    at jeg taber den halve forstand\\
    Gennem våben vi verdenen former\\
    og erobrer hvert land på vor vej\\
    Mod den allersidste verdenskrig vi stormer,\\
    De vil se at fysikken er sej.
  \end{SBVerse}
\end{song}
