\begin{song}{Mat'matik}
  {} % Bruges ikke, lad stå blank
  {Bubbi-bjørnene, Disney} % Titel, Kunstner - eks.: "Jutlandia, Kim Larsen". Hvis sangen er på sin egen melodi, brug da \SBOrgMel.
  {} % Navnet på forfatteren. Undlad aliasser. Brug "&" frem for "og". Hvis forfatter er ukendt, lad da stå tom.
  {FysikRevy, 2008} % Eks. "Fysikrevy 2010" eller "2010"
  {\NotCCLIed} % Lad stå som den er

  \begin{SBVerse}
    Vektorfunktioner i tre dimensioner\\
    skal opereres med curl og divergens.\\
    $\nabla$\textsuperscript{2} gir koldsved på panden;\\
    planintegraler er en pestilens.
  \end{SBVerse}

  \begin{SBChorus}
    Mat'matik\\
    strider ofte mod enhver logik;\\
    men har du elektrodynamik,\\
    så skal du ku' mat'matik.
  \end{SBChorus}

  \begin{SBVerse}
    Kan en hermitisk og injektiv matrix\\
    ha' negativ trace men en nul-determinant?\\
    Solovej siger dens egenværdier\\
    vil være reelle, men er det mon sandt?
  \end{SBVerse}

  \begin{SBChorus}
    Mat'matik\\
    strider ofte mod enhver logik;\\
    men hvis du har kvantemekanik,\\
    så skal du ku' mat'matik.
  \end{SBChorus}

  \begin{SBVerse}
    $T_a$ vil gi' $SU(3)$-symmetri,\\
    og Lagrange-operator'n er Gauge-invariant.\\
    Når operator'ne virker på kvarkerne,\\
    får de en farve og smagen af kvant
  \end{SBVerse}

  \begin{SBChorus}
    Mat'matik\\
    strider ofte mod enhver logik;\\
    men i kvantekromodynamik,\\
    så skal du ku' mat'matik.
  \end{SBChorus}

  \begin{SBSection*}
    Så skal du ku' mat'matik!
  \end{SBSection*}
\end{song}