\begin{song}{Integralsangen}
  {} % Bruges ikke, lad stå blank
  {Peberkagesangen, Dyrene i Hakkebakkeskoven} % Titel, Kunstner - eks.: "Jutlandia, Kim Larsen". Hvis sangen er på sin egen melodi, brug da \SBOrgMel.
  {} % Navnet på forfatteren. Undlad kaldenavne. Brug gerne TBF. Brug "&" frem for "og". Hvis forfatter er ukendt, lad da stå tom.
  {Matematikrevyen, 2008} % Eks. "Fysikrevy, 2010" eller "2010"
  {\NotCCLIed} % Lad stå som den er

  \begin{SBVerse}
		Hvis\ldots\\
		integralet du skal løse,\\
		må du ikke bare sløse.\\
		Du skal give det dit bedste,\\
		og så ikke blive gal.\\
		Og hvis stamfunktionen findes,\\
		er det bare om at mindes:\\
		Differens af indsat grænser\\
		det er vores areal.
  \end{SBVerse}

  \begin{SBVerse}
		Hvis du ikke kender grænser,\\
		differens du ikke ænser;\\
		du skal find’ de tal der matcher\\
		de værdier som du har.\\
		Hvis konstanten ellers stemmer,\\
		vores resultat vi kender.\\
		Din instruktor bliver glad,\\
		og jeres opgave er klar.
  \end{SBVerse}
\end{song}