\begin{song}{Matrixrepræsentationsteoremet}
  {} % Bruges ikke, lad stå blank
  {I will survive, Gloria Gaynor} % Titel, Kunstner - eks.: "Jutlandia, Kim Larsen". Hvis sangen er på sin egen melodi, brug da \SBOrgMel.
  {Christian Bladt Brandt} % Navnet på forfatteren. Undlad aliasser. Brug "&" frem for "og". Hvis forfatter er ukendt, lad da stå tom.
  {\TKET{}, 2011} % Eks. "Fysikrevy 2010" eller "2010"
  {\NotCCLIed} % Lad stå som den er

  \begin{SBVerse}
    Hvis man skal transformere, altså lineært,\\
    mellem to vektorrum, så er det altså elementært.\\
    Det kan jo snildt repræsenteres ved en matrix, som I ser,\\
    og hvis I ikke helt kan se det, så beviser jeg det her:
  \end{SBVerse}

  \begin{SBVerse}
    Vi ser på $V$, et vektorrum,\\
    der har ordnet basis kaldet $E$, og $n$ som dimension.\\
    Med dimensionen $m$ og basen $F$ der har vi $W$,\\
    det vektorrum den lineær’ transformation går over i.
  \end{SBVerse}

  \begin{SBVerse}
    Så har vi $x$, hvad er nu det?\\
    En koordinatvektor i basen $E$ for vektor’n $v$ i $V$,\\
    og for vektorerne i $W$ der kan vi definer’\\
    en koordinatvektor i basen $F$, der kaldes $y$, oh yeah!
  \end{SBVerse}

  \begin{SBChorus}
    Så er vi klar\\
    til at bevise\\
    denne sætning, der er vigtig,\\
    men først skal vi lige ha’ formuleret helt præcist\\
    hvad vi gern’ vil ha’ bevist.\\
    Det kommer her,\\
    det kommer her:
  \end{SBChorus}

  \begin{SBVerse}
    Nu vil vi vise $Ax=y$ hvis og kun hvis\\
    $L$ på $v$ den ser så’n her ud i vores $F$-basis.\\
    Her la’r vi $A$ være en matrix der har søjler givet ve’\\
    transformationen anvendt på basisvektorerne fra $E$.
  \end{SBVerse}

  \begin{SBVerse}
    Så vi ta’r $L$, anvendt på $v$,\\
    og starter med at bruge linearitet på det.\\
    Og hvis vi husker på hvordan vor matrix $A$ var definer’t,\\
    så er det næste udtryk, som I ser, vist ikke helt forkert!
  \end{SBVerse}

  \begin{SBVerse}
    Det sidste skridt er trivielt!\\
    Vi bytter rundt på de to summer, ikke nog’t specielt.\\
    Nu ses det at det udtryk der’ i parentes, det er $y_i$,\\
    så $y$ lig $A$ på $x$, og jeg ka’ si':\\
    Q.E.D.
  \end{SBVerse}
\end{song}