\begin{song}{Bevis for Skæringssætningen}
  {} % Bruges ikke, lad stå blank
  {Bohemian Rhapsody, Queen} % Titel, Kunstner - eks.: "Jutlandia, Kim Larsen". Hvis sangen er på sin egen melodi, brug da \SBOrgMel.
  {} % Navnet på forfatteren. Undlad kaldenavne. Brug gerne TBF. Brug "&" frem for "og". Hvis forfatter er ukendt, lad da stå tom.
  {MatematikRevy, 2011} % Eks. "Fysikrevy, 2010" eller "2010"
  {\NotCCLIed} % Lad stå som den er

  \begin{SBSection*}
    Se på den tegning\\
    Grafen går denne vej\\
    Det er da klart nok\\
    hvor det nulpunkt må gemme sig\\
    Ja - her og her -\\
    der kan være fler' end et!\\
    Se på det største,\\
    det er da lige til\\
    Det er supremum for mængden $D$\\
    Hvad er $D$? Lad os se:\\
    Det skal vær' de $x$, hvor\\
    $f(x)$ er mindre end $0$ - \\
    end $0$
  \end{SBSection*}

  \begin{SBSection*}
    Sæt $c$ -
    lig' $sup(D)$\\
    vælg $x_n$ i mængden $D$\\
    højst $\frac{1}{n}$ fra $c$\\
    \medskip
    Følgen\\
    $x_n$ går mod $c$\\
    Så derfor konkluderer vi nu at\\
    \medskip
    Følgen\\
    $f(x_n)$\\
    Konverger' mod $f(c)$\\
    Så $f(c)$ er svagt mindre end nul og\\
    \medskip
    Det var den ulighed,\\
    kan vi også få den anden?
  \end{SBSection*}

  \begin{SBSection*}
    $c + $\\
    $\frac{1}{n}$\\
    det kalder vi $x_n$\\
    For $b > x_n$\\
    \medskip
    Følgen kaldet $x_n$,\\
    den går mod $c$\\
    så vi konluderer derfor lig'som før:\\
    \medskip
    Følgen $f(x_n)$\\
    Går mod $f(c)$\\
    Så $f(c)$ er derfor svagt stør' end $0$
  \end{SBSection*}

  \begin{SBSection*}
    Nu har jeg næsten fået klaret mit bevis\\
    \emph{Pas nu på! Pas nu på! Du har glemt en detalje!}\\
    Du skal også vise hjælpesætning 5.1.10!
  \end{SBSection*}

  \begin{SBSection*}
    \emph{(Du er givet)} Du er givet \emph{(Du er givet)} Du er givet,\\
    Du er givet $\epsilon> 0$(-0-0-0-0)
  \end{SBSection*}

  \begin{SBSection*}
    Da er vor afstand fra $f(x_n)$\\
    \emph{Hen til $f(c) < \epsilon$}\\
    Hvis blot $x_n$ kun er $\delta$ fra $c$
  \end{SBSection*}

  \begin{SBSection*}
    Så er jeg færdig her, følgen konverger' !\\
    Bevis det! Nej - det må da være klart \emph{Soleklart!}\\
    Bevis det! Det må da være klart! \emph{Soleklart!}\\
    Bevis det! Det må da være klart!\\
    Soleklart! (må da være klart!)\\
    Soleklart! (må da være klart!)
  \end{SBSection*}

  \begin{SBSection*}
    TRI-VI-EEEEEELT!\\
    NEJ NEJ NEJ NEJ NEJ NEJ NEJ!\\
    Åh, hvorfor ikke, hvorfor ikke,\\
    hvorfor ikke trivielt?\\
    Funktionen $f$ er kontinuert i punktet $c$ -\\
    i $c$ - i $c$!
  \end{SBSection*}

  \begin{SBSection*}
    Da $x_n$ konvergerer mod c ses igen\\
    at $x_n$ højst er $\delta$ fra $c$ for stort n\\
    Hvad mere? - Vi kan nu konkludere,\\
    at $f(x_n)$ - har grænseværdi $f(c)$\\
    \medskip
    Hvaaad nu? Hvaaad nu?
  \end{SBSection*}

  \begin{SBSection*}
    $f(c)$ er større\\
    eller lig med nul\\
    $f(c)$ er mindre...\\
    $f(c)$ må være... $= 0$
  \end{SBSection*}

  \begin{SBSection*}
    Hvilket skulle vises...
  \end{SBSection*}
\end{song}