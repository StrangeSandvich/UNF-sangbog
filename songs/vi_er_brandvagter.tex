\begin{song}{Vi er brandvagter}
  {} % Bruges ikke, lad stå blank
  {Melodi} % Titel, Kunstner - eks.: "Jutlandia, Kim Larsen". Hvis sangen er på sin egen melodi, brug da \SBOrgMel.
  {Forfatter} % Navnet på forfatteren. Undlad kaldenavne. Brug gerne TBF. Brug "&" frem for "og". Hvis forfatter er ukendt, lad da stå tom.
  {Anledning og år} % Eks. "Fysikrevy, 2010" eller "2010"
  {\NotCCLIed} % Lad stå som den er

  \begin{SBVerse}
    % Skriv vers her
  \end{SBVerse}

  \begin{SBChorus}
    % Skriv omkvæd her
  \end{SBChorus}

  \begin{SBSection*}
    % Skriv sektioner her. Hvis du ønsker lidt mellemrum for at give luft i et langt afsnit el.lign., brug da \\\medskip
  \end{SBSection*}
\end{song}

% Festen er slut
% For os som passer på jer nu
% Vi sidder her med vores film
% Du må sove godt
% Sidder alen’
% Tænk på festen vi har haft
% Ser I kommer meget vakst
% Og går til ro

% Vi er brandvag-(t)er
% Ja for vi er
% i U-N-F
% UNF

% Vi holder øje
% Og lytter efter brandalarm
% Hjælper jer-hvis den går igang
% Men nu er den tam

% Vi er brandvag-(t)er
% Ja for vi er
% i U-N-F
% UNF
% Vi er brandvag-(t)er
% Ja for vi er
% i U-N-F
% UNF

% Klokken slår to
% Nu er I alle gået til ro
% Og vi sidder kun os to
% Og’ vi kigger på en sko

% Vi er brandvag-(t)er
% Ja for vi er
% i U-N-F
% UNF
% Vi er brandvag-(t)er Rebecca
% For I skal ikk’ Maria
% Oh Rebecca
% Slå’s helt ihjel
% Af en brand, af en brand
% I UN-F Rebecca
% Af en brand (maria)
