\begin{song}{Der er et ølrigt land}
  {} % Bruges ikke, lad stå blank
  {Der er et yndigt land, Adam Oehlenschläger} % Titel, Kunstner - eks.: "Jutlandia, Kim Larsen". Hvis sangen er på sin egen melodi, brug da \SBOrgMel.
  {} % Navnet på forfatteren. Undlad kaldenavne. Brug gerne TBF. Brug "&" frem for "og". Hvis forfatter er ukendt, lad da stå tom.
  {} % Eks. "Fysikrevy, 2010" eller "2010"
  {\NotCCLIed} % Lad stå som den er

  \begin{SBVerse}
    Der er et ølrigt land,\\
    det står med nød og næppe\\
    blandt alt det pokkers vand\\
    blandt alt det pokkers vand.\\
    Det bugter sig i bar og kro.\\
    Det hedder gamle Danmark,\\
    og her er øllen go'\\
    ja, hver en øl er go'.
  \end{SBVerse}

  \begin{SBVerse}
    Her drak i fordums tid\\
    hver tillakkede kæmper\\
    sin mjød af fad med flid\\
    sin mjød af fad med flid.\\
    Så prøved' han, ej uden mén\\
    at finde sine bene,\\
    men faldt ved hver en sten\\
    ja, hver en bautasten.
  \end{SBVerse}

  \begin{SBVerse}
    Den øl endnu er skøn,\\
    og gid den aldrig vælter.\\
    Lad baj'ren stå så grøn\\
    lad baj'ren stå så grøn.\\
    De ædle sorters skønne øer\\
    med sutter, sulde svende\\
    og svimle danske møer\\
    ja, svimle danske møer.
  \end{SBVerse}

  \begin{SBVerse}
    Hil druk og fædreland.\\
    Hil hver en kølig bajer.\\
    Vi drikker dem vi kan\\
    vi drikker dem vi kan.\\
    Vort gamle Danmark -- SKÅL! -- bestå\\
    så længe øllet skummer,\\
    og næsen den bli'r blå\\
    med røde prikker på.
  \end{SBVerse}
\end{song}