\begin{song}{Jeg har fundet mig en bug}
  {} % Bruges ikke, lad stå blank
  {Melodi} % Titel, Kunstner - eks.: "Jutlandia, Kim Larsen". Hvis sangen er på sin egen melodi, brug da \SBOrgMel.
  {Forfatter} % Navnet på forfatteren. Undlad kaldenavne. Brug gerne TBF. Brug "&" frem for "og". Hvis forfatter er ukendt, lad da stå tom.
  {Anledning og år} % Eks. "Fysikrevy, 2010" eller "2010"
  {\NotCCLIed} % Lad stå som den er

  \begin{SBVerse}
    % Skriv vers her
  \end{SBVerse}

  \begin{SBChorus}
    % Skriv omkvæd her
  \end{SBChorus}

  \begin{SBSection*}
    % Skriv sektioner her. Hvis du ønsker lidt mellemrum for at give luft i et langt afsnit el.lign., brug da \\\medskip
  \end{SBSection*}
\end{song}

% 1. Jeg har fundet mig en bug
% den er stor og væm’lig GDB er ikke nok
% dér er buggen nemlig
% Jeg har brugt den hele nat på at stoppe huller
% Og er efterhånden træt Af etter og nuller
% 2. Jeg har voldsom kaffetrang men der er ej mere
% Koden den vil ik’ engang næsten kompilere
% Der er noget voldsomt galt på så mang’ niveauer
% Helt basalt er det fatalt
% at jeg sidder og sover
% De fandt ham den næste dag med tastetryk i panden Plud’slig vågned’ han og sa’ Fejlen var en anden!
% Husk det, gæve datalog Ofte skal du bare
% ha’ det lidt på afstand, så ser du meget klarer’!