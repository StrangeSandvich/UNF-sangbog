\begin{song}{Primtal}
  {} % Bruges ikke, lad stå blank
  {Candy, Robbie Williams} % Titel, Kunstner - eks.: "Jutlandia, Kim Larsen". Hvis sangen er på sin egen melodi, brug da \SBOrgMel.
  {} % Navnet på forfatteren. Undlad kaldenavne. Brug gerne TBF. Brug "&" frem for "og". Hvis forfatter er ukendt, lad da stå tom.
  {Matematikrevyen, 2014} % Eks. "Fysikrevy, 2010" eller "2010"
  {\NotCCLIed} % Lad stå som den er

  \begin{SBVerse}
    Lad os definere de tal der fascinerer.\\
    For primtal skal der gælde faktorer trivielle.\\
    Elegant og simpelt, det kan man let forstå.\\
    Men trods det er det fulde billed’ umuligt at opnå.\\\medskip
    \emph{Men hør nu:}\\
    Euler og Euklid de viste,\\
    blandt primtal er der slet intet sidste.\\
    Givet $n$ kan primtal skrives\\
    helt entydigt, vi har uniqueness.
  \end{SBVerse}

  \begin{SBChorus}
    Skriv nu $p$ og $q$.\\
    Der’ så meg’t der ikk’ er vist endnu.\\
    Uden dem går teori itu,\\
    for alt er smukt ved primtal.\\
    Skriv nu $p$ og $q$.\\
    De er overalt, det vides jo.\\
    Også selvom det er svært at tro,\\
    for alt er smukt ved primtal.
  \end{SBChorus}

  \begin{SBVerse}
    Deres heltalsringe dem kan man let frembringe.\\
    For vilkårligt $p$ er der meget at indse.\\
    Man kan se tendenser og smukke kongruenser.\\
    Perfekt at anerkende, forbundet med Mersenne.\\\medskip
    Goldbach havde en formodning.\\
    Og kryptologer brug’r dem i kodning.\\
    Dankortkøb forløber sikkert fordi\\
    vi slipper primtal fri. Hvad gjord’ vi uden dem?\\
    Så kom nu!
  \end{SBVerse}

  \begin{SBChorus}
    Skriv nu $p$ og $q$.\\
    Der’ så meg’t der ikk’ er vist endnu.\\
    Uden dem går teori itu,\\
    for alt er smukt ved primtal.\\
    Skriv nu $p$ og $q$.\\
    De er overalt, det vides jo.\\
    Også selvom det er svært at tro,\\
    for alt er smukt ved primtal.
  \end{SBChorus}

  % \begin{SBChorus}
  %   Skriv nu $p$ og $q$.\ldots
  % \end{SBChorus}

  \begin{SBVerse}
    $\pi(x)$ er asymptotisk.\\
    Distribueringen dog ej logisk.\\
    Sylows sætning hjælper os til at se\\
    grupper af orden $p$. Hvad gjord’ vi uden dem?\\
    \SBRepeat{\SBRepeat{\SBRepeat{Hvad gjord’ vi uden dem?}}}
  \end{SBVerse}

  \begin{SBChorus}
    Skriv nu $p$ og $q$.\\
    Der’ så meg’t der ikk’ er vist endnu.\\
    Uden dem går teori itu,\\
    for alt er smukt ved primtal.\\
    Skriv nu $p$ og $q$.\\
    De er overalt, det vides jo.\\
    Også selvom det er svært at tro,\\
    for alt er smukt ved primtal.
  \end{SBChorus}

  \begin{SBChorus}
    Skriv nu $p$ og $q$.\\
    Der’ så meg’t der ikk’ er vist endnu.\\
    Uden dem går teori itu,\\
    for alt er smukt ved primtal.\\
    Skriv nu $p$ og $q$.\\
    De er overalt, det vides jo.\\
    Også selvom det er svært at tro,\\
    for alt er smukt ved primtal.
  \end{SBChorus}

  % \begin{SBChorus}
  %   Skriv nu $p$ og $q$.\ldots
  % \end{SBChorus}

  % \begin{SBChorus}
  %   Skriv nu $p$ og $q$.\ldots
  % \end{SBChorus}
\end{song}