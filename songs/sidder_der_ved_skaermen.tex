\begin{song}{Hvem sidder der ved skærmen}
  {} % Bruges ikke, lad stå blank
  {Jens Vejmand, Carl Nielsen} % Titel, Kunstner - eks.: "Jutlandia, Kim Larsen". Hvis sangen er på sin egen melodi, brug da \SBOrgMel.
  {} % Navnet på forfatteren. Undlad kaldenavne. Brug gerne TBF. Brug "&" frem for "og". Hvis forfatter er ukendt, lad da stå tom.
  {DIKUrevy, 1973} % Eks. "Fysikrevy, 2010" eller "2010"
  {\NotCCLIed} % Lad stå som den er

  \begin{SBVerse}
    Hvem sidder der ved skærmen\\
    Med strimler om sin hånd\\
    Med hulkort op til halsen\\
    Og om sin sko et bånd?\\
    \medskip
    Det er jo datalogen\\
    Som sidder dagen lang\\
    Ved skærmen og forvandler\\
    Sit Algol 5 til Slam
  \end{SBVerse}

  \begin{SBVerse}
    Og vågner du en morgen\\
    I allerførse gry\\
    Og ser at skærmen skriver\\
    På ny, på ny, på ny\\
    \medskip
    Det er jo datalogen\\
    Som på dig sindssygt glor\\
    Og taster vildt og blodigt\\
    Ved terminalens bord
  \end{SBVerse}

  \begin{SBVerse}
    Og vandrer du til stuen\\
    Hvor alt er fint i stand\\
    Og møder du en stakkel\\
    Hvis øjne står i vand -\\
    \medskip
    Det er jo datalogen\\
    Som kommer dig på tværs\\
    Og ikke mer' kan finde\\
    En fejl i line halvfjerds
  \end{SBVerse}

  \begin{SBVerse}
    Og vender du tilbage\\
    I bygger og i blæst\\
    Mens aftenstjernen skælver\\
    Af kulde fra nordvest\\
    \medskip
    Og klinger terminalen\\
    Bag ryggen ganske nær\\
    Det er jo datalogen\\
    Som endnu sidder der
  \end{SBVerse}

  \begin{SBVerse}
    Dog dataskærmen jævned\\
    For ham den slemme vej\\
    Men da det led mod julen\\
    Da sagde skærmen nej;\\
    \medskip
    Det er jo datalogen\\
    Hans fingre slap den brat\\
    De bar ham væk fra stuen\\
    En kold december nat
  \end{SBVerse}

  \begin{SBVerse}
    Det er på opslagstavlen\\
    Et gammelt grønnet kort\\
    Hvor hulningen er ussel\\
    Og skriften ganske bort\\
    \medskip
    Det er jo datalogen\\
    Som aldrig gav et gny\\
    Men på hans plads ved skærmen\\
    Der sidder snart en ny
  \end{SBVerse}
\end{song}