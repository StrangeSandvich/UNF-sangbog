\begin{song}{Kvanter i måneskin}
  {} % Bruges ikke, lad stå blank
  {Danse i måneskin, Trine Dyrholm} % Titel, Kunstner - eks.: "Jutlandia, Kim Larsen". Hvis sangen er på sin egen melodi, brug da \SBOrgMel.
  {} % Navnet på forfatteren. Undlad kaldenavne. Brug gerne TBF. Brug "&" frem for "og". Hvis forfatter er ukendt, lad da stå tom.
  {FysikRevy, 2004} % Eks. "Fysikrevy, 2010" eller "2010"
  {\NotCCLIed} % Lad stå som den er

  \begin{SBVerse}
    Ved tavlen er jeg en charmør,\\
    og pigerne bli’r bløde som smør.\\
    Når jeg pertuberer, mere,\\
    så får jeg topkarakterer.
  \end{SBVerse}
  \begin{SBVerse}
    I dag er det fredag, jeg går\\
    ned på Cafeen? og får\\
    en Guld Tuborg fra hanen, til ganen.\\
    Og så går jeg hjem til mig selv.
  \end{SBVerse}

  \begin{SBChorus}
    Og kvanter i måneskin.\\
    Jeg regner ud og ind\\
    på energi og spin.\\
    Kvanter i måneskin.
  \end{SBChorus}

  \begin{SBVerse}
    Når piger de synes, jeg er sær,\\
    så si’r jeg: ”Det’ sådan jeg er.”\\
    Jeg regner på atomer, tro mig,\\
    der er mindst ti millioner.
  \end{SBVerse}
  \begin{SBVerse}
    Jeg sidder og læser Euklid,\\
    Hun siger, jeg spilder min tid.\\
    Hun siger: ”Jeg vil ha’ dig. Ta’ mig!”\\
    Og så går jeg hjem til mig selv!
  \end{SBVerse}

  \begin{SBChorus}
    Og kvanter i måneskin.\\
    Jeg regner ud og ind\\
    på energi og spin.\\
    Kvanter i måneskin.
  \end{SBChorus}
\end{song}