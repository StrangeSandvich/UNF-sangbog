\begin{song}{Et helt nyt liv}
  {} % Bruges ikke, lad stå blank
  {Melodi} % Titel, Kunstner - eks.: "Jutlandia, Kim Larsen". Hvis sangen er på sin egen melodi, brug da \SBOrgMel.
  {Forfatter} % Navnet på forfatteren. Undlad kaldenavne. Brug gerne TBF. Brug "&" frem for "og". Hvis forfatter er ukendt, lad da stå tom.
  {Anledning og år} % Eks. "Fysikrevy, 2010" eller "2010"
  {\NotCCLIed} % Lad stå som den er

  \begin{SBVerse}
    % Skriv vers her
  \end{SBVerse}

  \begin{SBChorus}
    % Skriv omkvæd her
  \end{SBChorus}

  \begin{SBSection*}
    % Skriv sektioner her. Hvis du ønsker lidt mellemrum for at give luft i et langt afsnit el.lign., brug da \\\medskip
  \end{SBSection*}
\end{song}

% Frivillig
% Jeg kan vise dig alt
% Denne strålende verden
% Hør, frivillig, hvornår har hjertet, 
% Sidst bestemt for dig
% Jeg kan åbne dit blik
% mod naturvidenskaben
% Ja, du fatter det næppe
% Alt det som vil vise sig

% Et helt nyt liv
% Et nyt fantastisk perspektiv
% Ingen der siger nej
% Og standser dig
% Og si'r at vi kun drømmer

% Ny
% Et helt nyt liv
% En verden som er ny for mig
% I sort og hvid og blå, kan jeg forstå
% At viden kan formidles her, med dig

% Frivillig
% At viden kan formidles her, med dig

% Ny
% Helt utrolige camps
% Ingen søvn og kun kaffe
% Venskaber kan jeg skaffe
% Mens jeg laver camps med dig

% Et helt nyt liv

% Frivillig
% Tag en ku’uglepen

% Ny
% Og hundred tusing ting at gør'

% Frivillig
% Bare vent, der er mere

% Ny
% Jeg er en arrangør
% Jeg DK kør'
% At vende om, er ingen god ide, 

% Frivillig
% Et helt nyt liv

% Ny
% Mailen lukker sig op

% Frivillig
% Hvor det utroligste ka' ske

% Ny
% Hvert projekt fascinerer

% Begge
% Jeg alle foredrag så
% Der nok at nå
% Lad mig få et helt nyt liv med dig

% Frivillig
% Et helt nyt liv

% Ny
% Med gratis mad

% Frivillig
% En helt ny vej

% Ny
% Som gør mig glad

% Frivillig
% Så tag med mig

% Ny
% Til landsmøde

% Begge
% For dig og mig
