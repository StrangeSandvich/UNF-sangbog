\begin{song}{Våbenfysik}{}
  {Jutlandia, Kim Larsen}
  {}
  {FysikRevy 2003}
  {\NotCCLIed}

  \begin{SBVerse}
	Det var i 1945 og nu ville man ha’ fred,\\
	men der var krig i Japan.\\
	Der blev samlet uran nok og det blev kastet ned,\\
	for der var krig i Japan.\\
	De fik at se hvad fysikken formår,\\
	og sjove børn de næste mange år.
  \end{SBVerse}

  \begin{SBChorus}
	Hey-Ho for våbenfysik!\\
	Vi blæser på alle traktater.\\
	Bomber her, bomber der, bomber for fred!\\
	Hvad skal man med diplomater?
  \end{SBChorus}

  \begin{SBVerse}
	Vi flyver gennem natten og gi’r din fjende smæk.\\
	Han ser os ikke komme.\\
	Vi stiller ingen spørgsmål, og så snart vi har din check,\\
	så er krigen omme.\\
	Hvis du gi’r, fyrer vi den sgu af.\\
	Det er det, de unge vil ha’.
  \end{SBVerse}

  \begin{SBChorus}
   	\textit{Hey-Ho for våbenfysik\ldots}
  \end{SBChorus}

  \begin{SBVerse}
	Vores salgskontor har åbent hele døgnet – bare ring;\\
	du skal ikke tøve.\\
	Vacuum, brint og EMP og andre sjove ting,\\
	dem vil vi gerne prøve.\\
	Her er ugens tilbudskatalog:\\
	Start tre krige og betal for to!
  \end{SBVerse}

  \begin{SBChorus}
	\textit{Hey-Ho for våbenfysik\ldots}
  \end{SBChorus}
  
  \begin{SBChorus}
	Hey-Ho for våbenfysik!\\
	Vi blæser på alle traktater.\\
	Bomber her, bomber der, bomber for fred!\\
	Vi nakker de slyngelstater!
  \end{SBChorus}

\end{song}