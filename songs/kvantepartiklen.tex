\begin{song}{Kvantepartiklen}
  {} % Bruges ikke, lad stå blank
  {Den røde tråd, Shu-Bi-Dua} % Titel, Kunstner - eks.: "Jutlandia, Kim Larsen". Hvis sangen er på sin egen melodi, brug da \SBOrgMel.
  {} % Navnet på forfatteren. Undlad kaldenavne. Brug gerne TBF. Brug "&" frem for "og". Hvis forfatter er ukendt, lad da stå tom.
  {FysikRevy, 2014} % Eks. "Fysikrevy, 2010" eller "2010"
  {\NotCCLIed} % Lad stå som den er

  \begin{SBVerse}
    Hvor mon man er før man bli'r set?\\
    I Timbuktu, måske Tibet?\\
    Fordeling af sandsynlighed?\\
    Man må for fa'n have været et sted!
  \end{SBVerse}

  \begin{SBVerse}
    I dobbeltspalten var kun mig.\\
    Hvordan sku' jeg vælge vej?\\
    Og jeg indså til mit held\\
    inteferensen med mig selv.
  \end{SBVerse}

  \begin{SBVerse}
    Potentialesymmetri\\
    kan være ulig' eller lig'.\\
    Der findes intet bedre sæt,\\
    du ved min basis er komplet.
  \end{SBVerse}

  \begin{SBChorus}
    Helt nede\\
    Livslede\\
    Hvorfor kan jeg ej kendes?\\
    Jeg længes efter vished, efter sikkerhed.
  \end{SBChorus}

  \begin{SBVerse}
    Endli' fik jeg position,\\
    blev til en deltafunktion.\\
    Operator'n gav et sted.\\
    Nu har jeg endelig fået fred.
  \end{SBVerse}

  \begin{SBChorus}
    Helt nede\\
    Livslede\\
    Min position forsvinder.\\
    Kun minder om at være endegyldigt kendt.
  \end{SBChorus}

  \begin{SBVerse}
    Hvad mon man er før man bli'r set?\\
    Partikel-bølge dualitet?\\
    Bølgefunktion, en fermion?\\
    Blot matematisk abstraktion.
  \end{SBVerse}
\end{song}