\begin{song}{Jeg elsker "science"}
  {} % Bruges ikke, lad stå blank
  {I Will Survive} % Titel, Kunstner - eks.: "Jutlandia, Kim Larsen". Hvis sangen er på sin egen melodi, brug da \SBOrgMel.
  {DBR} % Navnet på forfatteren. Undlad kaldenavne. Brug gerne TBF. Brug "&" frem for "og". Hvis forfatter er ukendt, lad da stå tom.
  {UNF Revy, 2016} % Eks. "Fysikrevy, 2010" eller "2010"
  {\NotCCLIed} % Lad stå som den er

  \begin{SBVerse}
    Først så var jeg bange, jeg var skrækslagen\\
    Jeg tænkte, det lød lig’så skrækkeligt som et stræklagen.\\
    Og jeg sad søvnløs hele natten, stirred’ tomt ind i min pejs\\
    Før hed det naturvidenskab, nu skal man sige “science”
  \end{SBVerse}

  \begin{SBVerse}
    Syn’s det var dumt - noget rigtigt lort\\
    Men nu der ingen vej tilbage, “science” er det nye sort.\\
    Jeg sku’ ha sagt noget eller gjort noget, åh gået i protest\\
    Men sket er sket, og science er jo dansk når det er bedst
  \end{SBVerse}

  \begin{SBVerse}
    Det’ ikk’ så slemt, og egentlig nemt\\
    Syv bogstaver? Det er enkelt og bekvemt\\
    Hvorfor skal man slæbe rundt på gamle danske ord?\\
    Alle snakker engelsk nu, ligegyldigt hvor de bor.
  \end{SBVerse}

  \begin{SBChorus}
    Jeg elsker science, jeg elsker science!\\
    Fysik, kemi, biologi det er jo alt for nice.\\
    Matematik, geologi og medicinsk teknologi,
    jeg elsker science. Jeg elsker science!
  \end{SBChorus}
\end{song}




