  \begin{song}{Fulbert og Beatrice}
  {} % Bruges ikke, lad stå blank
  {\SBOrgMel} % Titel, Kunstner - eks.: "Jutlandia, Kim Larsen". Hvis sangen er på sin egen melodi, brug da \SBOrgMel.
  {Jens Louis Petersen} % Navnet på forfatteren. Undlad aliasser. Brug "&" frem for "og". Hvis forfatter er ukendt, lad da stå tom.
  {1951} % Eks. "Fysikrevy 2010" eller "2010"
  {\NotCCLIed} % Lad stå som den er
  \fancyhead[CE,CO]{\CHeadFont10}

  \begin{SBVerse}
    I frankens rige, hvor floder rinde\\
    som sølverstrømme i lune dal,\\
    lå ridderborgen på bjergets tinde\\
    med slanke tårne og gylden sal.\\\medskip
    Og det var sommer med blomsterbrise\\
    og suk af elskov i urtegård.\\
    Og det var Fulbert og Beatrice,\\
    og Beatrice var sytten år.
  \end{SBVerse}

  \begin{SBVerse}
    De havde leget som børn på borgen,\\
    mens Fulbert endnu var gangerpilt.\\
    Men langvejs drog han en årle morgen,\\
    mod Saracenen han higed' vildt.\\\medskip
    Han spidded' tyrker som pattegrise,\\
    et tusind stykker blev lagt på bår,\\
    for Fulbert kæmped' for Beatrice,\\
    og Beatrice var sytten år.
  \end{SBVerse}

  \begin{SBVerse}
    Med gluttens farver på sølversaddel\\
    han havde stridt ved Jerusalem.\\
    Han kæmped' kækt uden frygt og dadel\\
    og gik til fods hele vejen hjem.\\\medskip
    Nu sad han atter på bænkens flise\\
    og viste stolt sine heltesår,\\
    som ganske henrykked' Beatrice\\
    for Beatrice var sytten år.
  \end{SBVerse}

  \begin{SBVerse}
    En kappe prydet med små opaler\\
    og smagfuldt ternet med tyrkens blod,\\
    en ring af guld og et par sandaler\\
    den ridder lagde for pigens fod.\\\medskip
    Og da hun øjnede hans caprise\\
    blev hjertet mygt i den væne mår.\\
    Af lykke dånede Beatrice,\\
    for Beatrice var sytten år.
  \end{SBVerse}

  \begin{SBVerse}
    Da banked' blodet i heltens tinding,\\
    thi ingen helte er gjort af træ.\\
    Til trods for plastre og knæforbinding\\
    sank ridder Fulbert med stil på knæ.\\\medskip
    Han kvad: ``Skønjomfru, oh skænk mig lise,\\
    thi du alene mit hjerte rår!''\\
    ``Min helt, min ridder,'' kvad Beatrice,\\
    for Beatrice var sytten år.
  \end{SBVerse}

  \begin{SBVerse}
    Og der blev bryllup i højen sale\\
    med guldpokaler og troubadour,\\
    og under sange og djærven tale\\
    blev Fulbert ført til sin jomfrus bur.\\\medskip
    Og følget hvisked' om øm kurtice\\
    og skæmtsom puslen blandt dun og vår.\\
    For det var Fulbert og Beatrice,\\
    og Beatrice var sytten år.
  \end{SBVerse}

  \begin{SBVerse}
    Men ridder Fulbert den samme aften\\
    af borgens sale blev båren død.\\
    Den megen krig havde tær't på kraften,\\
    og sejrens palmer den sidste brød.\\\medskip
    Oh bejler, lær da af denne vise:\\
    Ød ej din kraft under krigens kår.\\
    Nej, spar potensen til Beatrice,\\
    når Beatrice er sytten år.
  \end{SBVerse}
\end{song}