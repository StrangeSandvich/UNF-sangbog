%%%%%% rcsid = @(#)$Id: sample-sb.tex,v 1.23 2010-04-12 18:04:11 rathc Exp $
%%%%%%
%%
%%      ===============================
%%      Sample Songbook (sample-sb.tex)
%%      ===============================
%%
%%      Version 4.5, 30 April, 2010
%%
%%      Copyright 1992--2010 Christopher Rath <christopher@rath.ca>
%%
%%      This package is free software; you can redistribute it and/or
%%      modify it under the terms of version 2.1 of the GNU Lesser
%%	General Public License as published by the Free Software 
%%	Foundation.
%%
%%      This package is distributed in the hope that it will be
%%      useful, but WITHOUT ANY WARRANTY; without even the implied
%%      warranty of MERCHANTABILITY or FITNESS FOR A PARTICULAR
%%      PURPOSE.  See the GNU Lesser General Public License for more
%%      details.
%%
%%      This file contains a subset of the songbook we distribute
%%      at our church.  To the best of my knowledge, all of the lyrics
%%      contained herein are freely distributable.  This file has been
%%      provided as a sample of what can be produced by the chordbk,
%%      wordbk, and overhead LaTeX styles.
%%
%%      NEEDED:  The fancyhdr LaTeX style is required to properly
%%              format this file.  If you don't have that then comment
%%              out the commands in the preamble which deal with the
%%              fancyhdr style.
%%
%%%%%%
%%%%%%
%%
%%      1. Chord notation.  Within this songbook the following
%%         conventions have been adopted:
%%
%%              "Minor" is entered as "m";
%%                      e.g. Cm7 for C minor 7th.
%%              "Major" is entered as "M";
%%                      e.g. CM7 for C major 7th.
%%
%%%%%%
%%%%%%
%%      ============
%%      Bibliography
%%      ============
%%
%%      Exalt Him!: Exalt Him!  Compiled by Tom Fettke.  (c)1989
%%                      Word Music.
%%
%%      Hosanna! Music Books: Hosanna! Music Books #1--#6.
%%                      (c)1987--92 Integrity Music, Inc.
%%
%%      Worship Him II: Worship Him II.  Compiled by Jesse Peterson
%%                      and Bruce Ballinger.  (c)1989 Tempo Music
%%                      Publications.
%%
%%      Worship Songs Of The Vineyard: Worship Songs Of The Vineyard
%%                      --- Volume 2.  (c)1989 Vineyard Ministries
%%                      International.
%%
%%%%%%
%%%%%%

%%%%%%%%%%%%%%%%%%%%%%%%%%%%%%%%%%%%%%%%%%%%%%%%%%%%%%%%%%
%%%%%%%%%%%%%%%%%%%%%%%%%%%%%%%%%%%%%%%%%%%%%%%%%%%%%%%%%%
%%                                                      %%
%%           P R E A M B L E   B E G I N S              %%
%%                                                      %%
%%%%%%%%%%%%%%%%%%%%%%%%%%%%%%%%%%%%%%%%%%%%%%%%%%%%%%%%%%
%%%%%%%%%%%%%%%%%%%%%%%%%%%%%%%%%%%%%%%%%%%%%%%%%%%%%%%%%%

\documentclass[a5paper]{book}
\usepackage{latexsym,
            fancyhdr,
            titlesec,
            amsmath,
            amssymb,
            multicol,
            amsthm,
            stmaryrd,
            amsthm,
            color,
            needspace,
            stackengine,
            wasysym}
\usepackage[utf8]{inputenc}
\usepackage[T1]{fontenc}
% \usepackage[chordbk]{songbook}                  %% Words & Chords edition.
%%\usepackage[compactallsongs,chordbk]{songbook}    %% Words & Chords edition.
\usepackage[wordbk]{songbook}                 %% Words Only edition.
%%\usepackage[overhead]{songbook}               %% Overhead Transparency edition.
\usepackage{titletoc}
\usepackage{tket}  % Draws "TÅGEKAMMERET" correctly



%%%
% Revision Date and Release Date definitions.
%
%       \RelDate - The last time this songbook was released.  Set this
%                  date each time a new release/update of the songbook
%                  is generated.
%       \RevDate - The last time a particular song was revised in any
%                  way.  This command will be renewed inside every
%                  song.
%%%
\newcommand{\RelDate}{31~August,~2003}
\newcommand{\RevDate}{\today}

%%%
% C.C.L.I. license number definition; for copyright licensing info.
% One of these macros will be manually inserted into the {SBMel}
% parameter of the {song} environment.
%
%       \CCLInumber - The actual copyright license number.  Don't
%               insert this command in the {SBMel} parameter, use one
%               of the others.
%       \CCLIed - Indicates a song falls under our CCLI license.
%       \NotCCLIed - Indicates a song doesn't fall under our CCLI
%               license.  Public Domain songs fall into this category.
%       \PGranted - We have received specific permission from the
%               copyright holder to use this song.
%       \PPending - We are in the process of obtaining permission to
%               use this song.
%%%
\newcommand{\CCLInumber}{Your CCLI Number}
\newcommand{\CCLIed}{{\SBMelInfoFont (CCLI \CCLInumber)}}
\newcommand{\NotCCLIed}{\relax}
\newcommand{\PGranted}{\relax}
\newcommand{\PPending}{{\SBMelInfoFont (Permission Pending)}}

%%%
% Title page information.
%%%
\title{UNF Sangbogen}
\author{}
\date{Revideret:  \RevDate}

%%%
% Redefine fonts from SongBook style that I don't like.
%%%
\font\myTinySF=cmss8 at 8pt
\renewcommand{\SBMelInfoFont}{\tiny\myTinySF}

%%%
% Define fonts to use in the headers and footers of the songbook.
%%%
\newcommand{\LHeadFont}{\normalsize}            % = cmr12  at 12pt
\newcommand{\CHeadFont}{\normalsize\rm}         % = cmr12  at 12pt
\newcommand{\RHeadFont}{\normalsize}            % = cmr12  at 12pt
\newcommand{\LFootFont}{\scriptsize}            % = cmr8   at  8pt
\newcommand{\CFootFont}{\tiny\myTinySF}         % = cmss8  at  8pt
\newcommand{\RFootFont}{\scriptsize}            % = cmr8   at  8pt

\def\repeat{%
  \stackanchor{.}{.}%
  \rule[-\dp\strutbox]{.3pt}{\normalbaselineskip}%
  \kern0.5pt%
  \rule[-\dp\strutbox]{1pt}{\normalbaselineskip}%
  \kern1pt%
}
\def\frepeat{%
  \kern1pt%
  \rule[-\dp\strutbox]{1pt}{\normalbaselineskip}%
  \kern0.5pt%
  \rule[-\dp\strutbox]{.3pt}{\normalbaselineskip}%
  \stackanchor{.}{.}%
}
\newcommand{\SBRepeat}[1]{#1\\#1}
% \newcommand{\SBRepeat}[1]{\frepeat #1\repeat}
\setcounter{SBSongCnt}{-1}
\renewcommand{\SBWAndMTag}{Forfatter:}
\renewcommand{\SBUnknownTag}{Ukendt}
\renewcommand{\SBChorusTag}{Ref.}
\renewcommand{\SBOrgMel}{Originalmelodi}
\renewcommand{\SpaceAfterChorus}   {\vspace{0ex plus1ex minus 0.5ex}}
\renewcommand{\SpaceAfterOpGroup}  {\vspace{0ex plus1ex minus 0.5ex}}
\renewcommand{\SpaceAfterSBBracket}{\vspace{0ex plus1ex minus 0.5ex}}
\renewcommand{\SpaceAfterSection}  {\vspace{0ex plus1ex minus 0.5ex}}
\renewcommand{\SpaceAfterSong}     {\vspace{0ex plus1ex minus 0.5ex}}
\renewcommand{\SpaceAfterVerse}    {\vspace{0ex plus1ex minus 0.5ex}}

% Tell LaTeX that \medskip is a good place to make a page break
\let\oldmedskip\medskip
\renewcommand{\medskip}{\oldmedskip\pagebreak[2]}

%%%
% Turn on/off index-file generation.  Uncomment the \makeindex line to
% turn index generation on;  comment it out to turn index generation
% off.
%%%
\makeTitleIndex         %% Title and First Line Index.
\makeTitleContents      %% Table of Contents.
\makeKeyIndex           %% Index of song by key.
\makeArtistIndex	%% Index of song by artist.
\newcommand{\SBThechapter}[0]{}
\newcommand{\SBChapter}[1]{
    \startcontents
    \chapter*{#1} 
    % %%%%%% rcsid = @(#)$Id: sample-sb.tex,v 1.23 2010-04-12 18:04:11 rathc Exp $
%%%%%%
%%
%%      ===============================
%%      Sample Songbook (sample-sb.tex)
%%      ===============================
%%
%%      Version 4.5, 30 April, 2010
%%
%%      Copyright 1992--2010 Christopher Rath <christopher@rath.ca>
%%
%%      This package is free software; you can redistribute it and/or
%%      modify it under the terms of version 2.1 of the GNU Lesser
%%	General Public License as published by the Free Software 
%%	Foundation.
%%
%%      This package is distributed in the hope that it will be
%%      useful, but WITHOUT ANY WARRANTY; without even the implied
%%      warranty of MERCHANTABILITY or FITNESS FOR A PARTICULAR
%%      PURPOSE.  See the GNU Lesser General Public License for more
%%      details.
%%
%%      This file contains a subset of the songbook we distribute
%%      at our church.  To the best of my knowledge, all of the lyrics
%%      contained herein are freely distributable.  This file has been
%%      provided as a sample of what can be produced by the chordbk,
%%      wordbk, and overhead LaTeX styles.
%%
%%      NEEDED:  The fancyhdr LaTeX style is required to properly
%%              format this file.  If you don't have that then comment
%%              out the commands in the preamble which deal with the
%%              fancyhdr style.
%%
%%%%%%
%%%%%%
%%
%%      1. Chord notation.  Within this songbook the following
%%         conventions have been adopted:
%%
%%              "Minor" is entered as "m";
%%                      e.g. Cm7 for C minor 7th.
%%              "Major" is entered as "M";
%%                      e.g. CM7 for C major 7th.
%%
%%%%%%
%%%%%%
%%      ============
%%      Bibliography
%%      ============
%%
%%      Exalt Him!: Exalt Him!  Compiled by Tom Fettke.  (c)1989
%%                      Word Music.
%%
%%      Hosanna! Music Books: Hosanna! Music Books #1--#6.
%%                      (c)1987--92 Integrity Music, Inc.
%%
%%      Worship Him II: Worship Him II.  Compiled by Jesse Peterson
%%                      and Bruce Ballinger.  (c)1989 Tempo Music
%%                      Publications.
%%
%%      Worship Songs Of The Vineyard: Worship Songs Of The Vineyard
%%                      --- Volume 2.  (c)1989 Vineyard Ministries
%%                      International.
%%
%%%%%%
%%%%%%

%%%%%%%%%%%%%%%%%%%%%%%%%%%%%%%%%%%%%%%%%%%%%%%%%%%%%%%%%%
%%%%%%%%%%%%%%%%%%%%%%%%%%%%%%%%%%%%%%%%%%%%%%%%%%%%%%%%%%
%%                                                      %%
%%           P R E A M B L E   B E G I N S              %%
%%                                                      %%
%%%%%%%%%%%%%%%%%%%%%%%%%%%%%%%%%%%%%%%%%%%%%%%%%%%%%%%%%%
%%%%%%%%%%%%%%%%%%%%%%%%%%%%%%%%%%%%%%%%%%%%%%%%%%%%%%%%%%

\documentclass[12pt]{book}
\usepackage{latexsym,
            fancyhdr,
            titlesec,
            amsmath,
            amssymb,
            multicol,
            amsthm,
            stmaryrd,
            amsthm,
            color,
            needspace,
            stackengine,
            wasysym}
\usepackage[utf8]{inputenc}
\usepackage[T1]{fontenc}
% \usepackage[chordbk]{songbook}                  %% Words & Chords edition.
%%\usepackage[compactallsongs,chordbk]{songbook}    %% Words & Chords edition.
\usepackage[wordbk]{songbook}                 %% Words Only edition.
%%\usepackage[overhead]{songbook}               %% Overhead Transparency edition.



%%%
% Revision Date and Release Date definitions.
%
%       \RelDate - The last time this songbook was released.  Set this
%                  date each time a new release/update of the songbook
%                  is generated.
%       \RevDate - The last time a particular song was revised in any
%                  way.  This command will be renewed inside every
%                  song.
%%%
\newcommand{\RelDate}{31~August,~2003}
\newcommand{\RevDate}{\today}

%%%
% C.C.L.I. license number definition; for copyright licensing info.
% One of these macros will be manually inserted into the {SBMel}
% parameter of the {song} environment.
%
%       \CCLInumber - The actual copyright license number.  Don't
%               insert this command in the {SBMel} parameter, use one
%               of the others.
%       \CCLIed - Indicates a song falls under our CCLI license.
%       \NotCCLIed - Indicates a song doesn't fall under our CCLI
%               license.  Public Domain songs fall into this category.
%       \PGranted - We have received specific permission from the
%               copyright holder to use this song.
%       \PPending - We are in the process of obtaining permission to
%               use this song.
%%%
\newcommand{\CCLInumber}{Your CCLI Number}
\newcommand{\CCLIed}{{\SBMelInfoFont (CCLI \CCLInumber)}}
\newcommand{\NotCCLIed}{\relax}
\newcommand{\PGranted}{\relax}
\newcommand{\PPending}{{\SBMelInfoFont (Permission Pending)}}

%%%
% Title page information.
%%%
\title{UNF Sangbogen}
\author{}
\date{Revideret:  \RevDate}

%%%
% Redefine fonts from SongBook style that I don't like.
%%%
\font\myTinySF=cmss8 at 8pt
\renewcommand{\SBMelInfoFont}{\tiny\myTinySF}

%%%
% Define fonts to use in the headers and footers of the songbook.
%%%
\newcommand{\LHeadFont}{\normalsize}            % = cmr12  at 12pt
\newcommand{\CHeadFont}{\normalsize\rm}         % = cmr12  at 12pt
\newcommand{\RHeadFont}{\normalsize}            % = cmr12  at 12pt
\newcommand{\LFootFont}{\scriptsize}            % = cmr8   at  8pt
\newcommand{\CFootFont}{\tiny\myTinySF}         % = cmss8  at  8pt
\newcommand{\RFootFont}{\scriptsize}            % = cmr8   at  8pt

\def\repeat{%
  \stackanchor{.}{.}%
  \rule[-\dp\strutbox]{.3pt}{\normalbaselineskip}%
  \kern0.5pt%
  \rule[-\dp\strutbox]{1pt}{\normalbaselineskip}%
  \kern1pt%
}
\def\frepeat{%
  \kern1pt%
  \rule[-\dp\strutbox]{1pt}{\normalbaselineskip}%
  \kern0.5pt%
  \rule[-\dp\strutbox]{.3pt}{\normalbaselineskip}%
  \stackanchor{.}{.}%
}
\newcommand{\SBRepeat}[1]{\frepeat #1\repeat} %TODO: lav ordentlig med musik-gentagelses-tegn
\setcounter{SBSongCnt}{-1}
\renewcommand{\SBWAndMTag}{Forfatter:}
\renewcommand{\SBUnknownTag}{Ukendt}
\renewcommand{\SBChorusTag}{Ref.}
\renewcommand{\SBOrgMel}{Originalmelodi}
\renewcommand{\SpaceAfterChorus}   {\vspace{0em}}
\renewcommand{\SpaceAfterOpGroup}  {\vspace{0em}}
\renewcommand{\SpaceAfterSBBracket}{\vspace{0em}}
\renewcommand{\SpaceAfterSection}  {\vspace{0em}}
\renewcommand{\SpaceAfterSong}     {\vspace{0em}}
\renewcommand{\SpaceAfterVerse}    {\vspace{0em}}

%%%
% Turn on and define fancy page heading/footing definition.
%%%
\pagestyle{fancy}

\ifChordBk
  % It's a words & chords songbook...
  \addtolength{\headwidth}{\marginparsep}
  \addtolength{\headwidth}{\marginparwidth}
  \renewcommand{\headrulewidth}{0.4pt}
  \renewcommand{\footrulewidth}{0.4pt}
  \fancyhead[LE,RO]{\LHeadFont\emph{\leftmark\/}\SBContinueMark}
  \fancyhead[CE,CO]{\CHeadFont\thepage}
  \fancyhead[RE,LO]{\RHeadFont \chaptermark}
\else\ifOverhead
  % It's an overhead...
  \renewcommand{\footrulewidth}{0pt}
  \renewcommand{\headrulewidth}{0pt}
  \fancyhead[LE,RO]{}
  \fancyhead[CE,CO]{}
  \fancyhead[RE,LO]{}
\else\ifWordBk
  % It's a words only songbook...
  \addtolength{\headwidth}{\marginparsep}
  \addtolength{\headwidth}{\marginparwidth}
  \renewcommand{\headrulewidth}{0.4pt}
  \renewcommand{\footrulewidth}{0.4pt}
  \fancyhead[LE,RO]{\LHeadFont UNF Sangbogen}
  \fancyhead[CE,CO]{\CHeadFont\thepage}
  \fancyhead[RE,LO]{\RHeadFont TODO: kapitelnavn}
\fi\fi\fi

\fancyfoot[LE,RO]{\LFootFont ScienceCamps 2018}
\ifSongEject
  \fancyfoot[CE,CO]{\CFootFont Last Revised:  \RevDate}
\else
  \fancyfoot[CE,CO]{\CFootFont}
\fi
\fancyfoot[RE,LO]{\RFootFont TODO: Et eller andet smart kan stå her}

%%%
% Turn on/off index-file generation.  Uncomment the \makeindex line to
% turn index generation on;  comment it out to turn index generation
% off.
%%%
\makeTitleIndex         %% Title and First Line Index.
\makeTitleContents      %% Table of Contents.
\makeKeyIndex           %% Index of song by key.
\makeArtistIndex	%% Index of song by artist.

\titleformat{\chapter}
[display]
{}
{%\vspace*{\fill}
 % \titlerule[1pt]%
 % \vspace{1pt}%
 % \titlerule
 % \vspace{1pc}%
 \chaptertitlename}
{}
{\Huge}
[\clearpage]

%%%%%%%%%%%%%%%%%%%%%%%%%%%%%%%%%%%%%%%%%%%%%%%%%%%%%%%%%%
%%%%%%%%%%%%%%%%%%%%%%%%%%%%%%%%%%%%%%%%%%%%%%%%%%%%%%%%%%
%%                                                      %%
%%           D O C U M E N T   B E G I N S              %%
%%                                                      %%
%%%%%%%%%%%%%%%%%%%%%%%%%%%%%%%%%%%%%%%%%%%%%%%%%%%%%%%%%%
%%%%%%%%%%%%%%%%%%%%%%%%%%%%%%%%%%%%%%%%%%%%%%%%%%%%%%%%%%
\begin{document}

%%%
% Uncomment "\maketitle" statement to make a title page.
%%%
\maketitle
\mainmatter
\ifWordBk
  \twocolumn
\fi

%%%
% Songbook begins.
%%%

\onecolumn
\chapter*{Os mod de andre}
\twocolumn
\begin{song}{Vi kan ikke li'}
  {} % Bruges ikke, lad stå blank
  {Vores buschauffør kan ikke køre bus} % Titel, Kunstner - eks.: "Jutlandia, Kim Larsen". Hvis sangen er på sin egen melodi, brug da \SBOrgMel.
  {} % Navnet på forfatteren. Undlad aliasser. Brug "&" frem for "og". Hvis forfatter er ukendt, lad da stå tom.
  {} % Eks. "Fysikrevy 2010" eller "2010"
  {\NotCCLIed} % Lad stå som den er

 \begin{SBVerse}
    \SBRepeat{Vi kan ikke li' folk vi ik' kan li'}\\
    For vi kan sgu ikke li' dem,\\
    de skulle hellere tage at blive hjemme\\
    Vi kan ikke li' folk vi ik' kan li'
  \end{SBVerse}

  \begin{SBVerse}
    \SBRepeat{Vi kan ikke li' folk fra teologi}\\
    For de er nogle hængemuler,\\
    tror at helligånden puler\\
    Vi kan ikke li' folk fra teologi
  \end{SBVerse}

  \begin{SBVerse}
    \SBRepeat{Vi kan ikke li' folk fra biologi}\\
    For de går i strikketrøjer,\\
    og med dyrene sig fornøjer\\
    Vi kan ikke li' folk fra biologi
  \end{SBVerse}

  \begin{SBVerse}
    \SBRepeat{Vi kan ikke li' folk der har fysik}\\
    For de har sgu nogle testikler\\
    der er på størrelse med partikler\\
    Vi kan ikke li' folk der har fysik
  \end{SBVerse}

  \begin{SBVerse}
    \SBRepeat{Vi kan ikke li' folk fra musikvidenskab}\\
    For de er nogle lamme nødder,\\
    passer ik' på versefødder\\
    Vi kan ikke li' folk fra musikvidenskab
  \end{SBVerse}

  \begin{SBVerse}
    \SBRepeat{Vi kan ikke li' folk fra nano-tek}\\
    De går rundt med sure miner,\\
    laver bittesmå maskiner\\
    Vi kan ikke li' folk fra nano-tek
  \end{SBVerse}

  \begin{SBVerse}
    \SBRepeat{Vi kan ikke li' folk fra statistik}\\
    De vil hypoteser teste\\
    mens de hel're burde feste\\
    Vi kan ikke li' folk fra statistik
  \end{SBVerse}

  \begin{SBVerse}
    \SBRepeat{Vi kan ikke li' folk fra arkæologi}\\
    fordi de graver og de graver\\
    og på gamle mennesker rager\\
    Vi kan ikke li' folk fra arkæologi
  \end{SBVerse}

  \begin{SBVerse}
    \SBRepeat{Vi kan ikke li' folk fra filosofi}\\
    for de går bar' og funderer\\
    om de selv mon eksisterer\\
    Vi kan ikke li' folk fra filosofi
  \end{SBVerse}

  \begin{SBVerse}
    \SBRepeat{Vi kan ikke li' folk fra farmaci}\\
    for de går og triller piller\\
    og tabletter og pastiller\\
    Vi kan ikke li' folk fra farmaci
  \end{SBVerse}

  \begin{SBVerse}
    \SBRepeat{Vi kan ikke li' folk fra geologi}\\
    For de går og ser på stenene\\
    når de burde sprede benene\\
    Vi kan ikke li' folk fra geologi
  \end{SBVerse}

  \begin{SBVerse}
    \SBRepeat{Vi kan ikke li' folk fra astronomi}\\
    De er fjolser uden hjerner\\
    der går rundt og ser på stjerner\\
    Vi kan ikke li' folk fra astronomi
  \end{SBVerse}
%  \begin{SBVerse}
%    Vi kan ikke li' folk fra iNano\ldots\\
%    bagfra blir' de lidt for kække\\
%    synes Ångstrøm de er frække
%  \end{SBVerse}

  \begin{SBVerse}
    \SBRepeat{Vi kan ikke li' folk fra økonomi}\\
    For de regner på budgettet\\
    når de sidder på toilettet\\
    Vi kan ikke li' folk fra økonomi
  \end{SBVerse}

  \begin{SBVerse}
    \SBRepeat{Vi kan ikke li' folk fra lingvistik}\\
    Kulli waffli zarka gunku\\
    emfle birnan smöja dunku\\
    Vi kan ikke li' folk fra lingvistik
  \end{SBVerse}

  \begin{SBVerse}
    \SBRepeat{Vi kan ikke li' folk fra folk der har kemi}\\
    For de hedder Ion og Ester\\
    mens de luften stærkt forpester\\
    Vi kan ikke li' folk fra folk der har kemi
  \end{SBVerse}

 \begin{SBVerse}
    \SBRepeat{Vi kan ikke li' folk fra matematik}\\
    For de teoretiserer\\
    indtil $2$ plus $3$ bli'r $4$\\
    Vi kan ikke li' folk fra matematik
  \end{SBVerse}

% \begin{SBVerse}
%    Vi kan ikke li' folk der har IT\ldots\\
%    Uni skaffer studiejobbet\\
%    Det er næsten alt for snobbet
%  \end{SBVerse}

% \begin{SBVerse}
%    Vi kan ikke li' folk der har dat-mult\ldots\\
%    Datalogcirkusartister;\\
%    de er næsten humanister
%  \end{SBVerse}

% \begin{SBVerse}
%    Vi kan ikke li' folk der har tek-fys\ldots\\
%    for de har en bærbar PC\\
%    læser mails når de' på WC
%  \end{SBVerse}

% \begin{SBVerse}
%    Vi kan ikke li' folk fra statskundskab\ldots\\
%    for de gider kun at kom' for- di\\
%    de tror de møder Lomborg
%  \end{SBVerse}

% \begin{SBVerse}
%    Vi kan ikke li' folk fra farmaceutisk kemi\ldots\\
%    de fungerer som en buffer\\
%    der i alle emner skuffer
%  \end{SBVerse}

 \begin{SBVerse}
    \SBRepeat{Vi kan ikke li' folk fra folk fra medicin}\\
    For de går i hvide kitler,\\
    og det rimer jo på Hitler\\
    Vi kan ikke li' folk fra folk fra medicin
  \end{SBVerse}

 \begin{SBVerse}
    \SBRepeat{Vi kan ikke li' folk fra molekylær(biologi)}\\
    for de kloner små kaniner\\
    splejser dem med appelsiner\\
    Vi kan ikke li' folk fra molekylær(biologi)
  \end{SBVerse}

 \begin{SBVerse}
    \SBRepeat{Vi kan ikke li' folk fra psykologi}\\
    For de vader rundt i læder,\\
    og de hader deres fædre\\
    Vi kan ikke li' folk fra psykologi
  \end{SBVerse}

 \begin{SBVerse}
    \SBRepeat{Vi kan ikke li' folk der er jurister}\\
    for med hule paragraffer\\
    dagen lang sig selv de straffer\\
    Vi kan ikke li' folk der er jurister
  \end{SBVerse}

 \begin{SBVerse}
    \SBRepeat{Vi kan ikke li' folk der har historie}\\
    for de har kun gamle minder\\
    og et job de aldrig finder\\
    Vi kan ikke li' folk der har historie
  \end{SBVerse}

 \begin{SBVerse}
    \SBRepeat{Vi kan ikke li' folk der har idræt}\\
    For de ligner jo rødbeder,\\
    når de render rundt og sveder\\
    Vi kan ikke li' folk der har idræt
  \end{SBVerse}

 \begin{SBVerse}
    \SBRepeat{Vi kan ikke li' folk fra datalogi}$^+$
  \end{SBVerse}
\end{song}
\begin{song}{Vi er ikke humanister}
  {} % Bruges ikke, lad stå blank
  {Vi er ikke rigtig voksne, Bølle Bob} % Titel, Kunstner - eks.: "Jutlandia, Kim Larsen". Hvis sangen er på sin egen melodi, brug da \SBOrgMel.
  {Jeanette Pinderup} % Navnet på forfatteren. Undlad aliasser. Brug "&" frem for "og". Hvis forfatter er ukendt, lad da stå tom.
  {TÅGEKAMMERETs Julerevy, 2007} % Eks. "Fysikrevy 2010" eller "2010"
  {\NotCCLIed} % Lad stå som den er

  \begin{SBChorus}
    Vi er ikke humanister, vi har ikke samfundsfag,\\
    Medicin er også noget for de andre.\\
    Vi er ikke teologer, og går ik’ på Business School,\\
    og de andre synes ikke vi er cool.
  \end{SBChorus}

  \begin{SBVerse}
    Når jeg vil ud og danse hele natten, siger de:\\
    Du har ingen rytmesans! Alle tænker bare stands,\\
    men når de vil have drenge med til fester hører vi\\
    at de pludselig kan li’ datalogi.
  \end{SBVerse}

  \begin{SBVerse}
    I hverdagen der skal vi ikke sende dem et nik.\\
    Du er ikke lækker nok! Du er bare et lille pjok!\\
    Men når de skal bruge laser eller nano-mekanik,\\
    er de blevet meget glade for fysik.
  \end{SBVerse}

  \begin{SBChorus}
    Vi er ikke humanister\ldots
  \end{SBChorus}

  \begin{SBVerse}
    Hvis vi vil sige noget får vi bare deres blik:\\
    Hey - hvem tror du dog du er? Du er lige meget her!\\
    Men hvis de så har et spørgsmål til en vigtig statistik,\\
    ved de godt at svaret er på mat’matik.
  \end{SBVerse}

  \begin{SBChorus}
    Vi er ikke humanister\ldots
  \end{SBChorus}
\end{song}

\begin{song}{Er jeg humanist?}
  {} % Bruges ikke, lad stå blank
  {Har jeg bildæk?, Drengene fra Angora} % Titel, Kunstner - eks.: "Jutlandia, Kim Larsen". Hvis sangen er på sin egen melodi, brug da \SBOrgMel.
  {} % Navnet på forfatteren. Undlad kaldenavne. Brug gerne TBF. Brug "&" frem for "og". Hvis forfatter er ukendt, lad da stå tom.
  {Ingeniørrevyen, 2017} % Eks. "Fysikrevy, 2010" eller "2010"
  {\NotCCLIed} % Lad stå som den er

  \begin{SBVerse}
    Andet semester i eksamenstid'n.\\
    Vi skrev på vor's projekt, og der stod dias på menuen.\\
    Vi sku' fremlægge beregninger med $\pi$,\\
    og jeg ha'd lav'd en fucking sexet koreografi.\\\medskip
    Jeg viste dem det, men pluds'lig slog det mig,\\
    at der var fucking gay, og derfor sagde jeg:\\
    "Drenge, jeg tror jeg har et meget stort problem!\\
    Jeg tror ikke at jeg passer ind i jer's system"
  \end{SBVerse}

  \begin{SBChorus}
    Er jeg humanist? Helt ærligt, er jeg humanist?\\
    Nej, det' da helt normalt at ha' et ekstra følsimt twist.\\
    Men er jeg ik' ubrugelig? Går op i antropologi!\\
    Nej, du' da bare lidt blød i kanten, tag nu og klap i!\\
    \emph{Vi går i baren nu\ldots}
  \end{SBChorus}

  \begin{SBVerse}
    Senere i vores fredagsbar\\
    mødte jeg en pige, som jeg syn's var rar.\\
    Jeg sagde hun havde smukke øjne, så begyndte hun at grin'\\
    og sagde: "hey, hvad tror du selv, dit store arbejdsløse svin?"\\
    Senere, da vi sku' skriv' rapport,\\
    jeg lav'd en lækker forside med mange nuancer af sort.\\\medskip
    Men ku' mærke at de andre var vrede på mig\\
    selvon jeg hav'd medbragt speltkiks med økologisk butterdej.\\
    Jeg ku' ikke li' det, og spurgt' dem derfor ad\\
    hvorfor de var vred', så begyndte jeg at græd'
  \end{SBVerse}

  \begin{SBChorus}
    Er jeg humanist? Burd' jeg ik' få pisk?\\
    Nej, nogle gange er man bare en lille smule trist.\\
    Men bare se min striksweater! Jeg er en fucking trendsetter!\\
    Nej, du' da bar' en fimset fætter, tag nu og slap af.\\
  \end{SBChorus}

  \begin{SBSection*}
    Men altså, helt ærligt,\\
    synes I slet ikke jeg er humanist, eller hvad?\\
    Vi har jo sagt til dig at du ikke er humanist!\\
    Jamen bare se den forside jeg har lavet, den er helt... lyserød!\\
    Jamen du bliver jo ikke tilfreds før vi siger at du er humanist.\\
    Hva?!!\\
    \emph{Du bliver jo ikke tilfreds før vi siger at du er humanist!}
  \end{SBSection*}

  \begin{SBChorus}
    Du er humanist! En fucking ord-onanist!\\
    Du går op i følelser, farver og andet ubrugeligt pis!\\
    Du er humanist! En arbejdsløs artist!\\
    Du får jo aldrig noget job, bu-hu hvor er det trist.
  \end{SBChorus}

  \begin{SBChorus}
    I er ikke særligt søde! Skal jeg nu ha' stød?\\
    Nej det' ik' farligt, men krokodiller bli'r din død.\\
    Dit humanist-svin! Dit humanist-svin!\\
    Mangel på indtægt, det ligger faktisk til min slægt.\\
    Dit humanist-svin! Dit humanist-svin!\\
    Mangel på indtægt, det ligger faktisk til min slægt.\\
    Dit humanist-svin! Dit humanist-svin!
  \end{SBChorus}
\end{song}













\begin{song}{Humanist}
  {} % Bruges ikke, lad stå blank
  {Onani, Dario von Slutty} % Titel, Kunstner - eks.: "Jutlandia, Kim Larsen". Hvis sangen er på sin egen melodi, brug da \SBOrgMel.
  {} % Navnet på forfatteren. Undlad kaldenavne. Brug gerne TBF. Brug "&" frem for "og". Hvis forfatter er ukendt, lad da stå tom.
  {DIKUrevy, 2013} % Eks. "Fysikrevy, 2010" eller "2010"
  {\NotCCLIed} % Lad stå som den er

  \begin{SBChorus}
    Humanist - det' godt nok trist,\\
    humanist - det' godt nok trist,\\
    humanist - det' godt nok trist:\\
    Det ta'r dem 10 semestre, og kan ikke brug's til sidst.
  \end{SBChorus}

  \begin{SBVerse}
    Jeg er ham der Klaes, jeg har ECTS-behov,\\
    men kurserne jeg tager, de gi'r mig aldrig lov.\\
    De taler mig i søvne, her til forelæsningen,\\
    og når den så er ovre, har jeg det hele glemt igen.\\
    Hvorfor tage kurser, der altid dumper mig?\\
    Det er derfor, at jeg siger: "Der er en nem're vej!"
  \end{SBVerse}

  \begin{SBChorus}
    Humanist - det' godt nok trist,\\
    humanist - det' godt nok trist,\\
    humanist - det' godt nok trist:\\
    Det ta'r dem 10 semestre, og kan ikke brug's til sidst.
  \end{SBChorus}

  \begin{SBChorus}
    Humanist - det' godt nok trist,\\
    humanist - det' godt nok trist,\\
    humanist - det' godt nok trist:\\
    Det ta'r dem 10 semestre, og kan ikke brug's til sidst.
  \end{SBChorus}

  % \begin{SBChorus}
  %   Humanist - det' godt nok trist\ldots
  % \end{SBChorus}

  % \begin{SBChorus}
  %   Humanist - det' godt nok trist\ldots
  % \end{SBChorus}

  \begin{SBVerse}
    Se, de fleste går og drømmer om at arbejde med kode,\\
    men jeg vil bar' være færdig - arbejdsløshed er på mode!\\
    Jeg er ligeglad med penge, jeg har børneopsparing,\\
    og når den så løber tør, kan jeg pizza udbring'.\\
    Hvorfor skulle arbejd', og ikke ha' tid til leg?\\
    Det er derfor, at jeg siger: "Der er en nem're vej!"
  \end{SBVerse}

  \begin{SBChorus}
    Humanist - det' godt nok trist,\\
    humanist - det' godt nok trist,\\
    humanist - det' godt nok trist:\\
    Det ta'r dem 10 semestre, og kan ikke brug's til sidst.
  \end{SBChorus}

  \begin{SBChorus}
    Humanist - det' godt nok trist,\\
    humanist - det' godt nok trist,\\
    humanist - det' godt nok trist:\\
    Det ta'r dem 10 semestre, og kan ikke brug's til sidst.
  \end{SBChorus}

  % \begin{SBChorus}
  %   Humanist - det' godt nok trist\ldots
  % \end{SBChorus}

  % \begin{SBChorus}
  %   Humanist - det' godt nok trist\ldots
  % \end{SBChorus}

  \begin{SBVerse}
    Nu er det ovre, jeg er endelig blevet fri,\\
    er blevet kandidat i østrigsk eskimologi.\\
    Der findes intet job hvori denne grad kan bruges,\\
    så alle jobsamtaler sluttes af med at der buh'es.\\
    Her er det så, jeg fortryder jeg droppede ud.\\
    Hvis bare jeg kunne kode, så fremtiden lysere ud.
  \end{SBVerse}

  \begin{SBChorus}
    Datalog - han er jo god,\\
    datalog - han er jo god,\\
    datalog - får job, værsgo'!\\
    Det ta'r dem 20 semestre, men i det mindste kan de kode.
  \end{SBChorus}

  \begin{SBChorus}
    Datalog - han er jo god,\\
    datalog - han er jo god,\\
    datalog - får job, værsgo'!\\
    Det ta'r dem 20 semestre, men i det mindste kan de kode.
  \end{SBChorus}

  % \begin{SBChorus}
  %   Datalog - han er jo god\ldots
  % \end{SBChorus}
\end{song}
\begin{song}{Når en humanist adderer}
  {} % Bruges ikke, lad stå blank
  {Peberkagesangen, Dyrene i Hakkebakkeskoven} % Titel, Kunstner - eks.: "Jutlandia, Kim Larsen". Hvis sangen er på sin egen melodi, brug da \SBOrgMel.
  {\TKET{}} % Navnet på forfatteren. Undlad aliasser. Brug "&" frem for "og". Hvis forfatter er ukendt, lad da stå tom.
  {} % Eks. "Fysikrevy 2010" eller "2010"
  {\NotCCLIed} % Lad stå som den er

  \begin{SBVerse}
    Når en humanist adderer,\\
    tar’ hun først et $\pi$ i fjerde,\\
    ganger det med $n$ matricer,\\
    mens hun vælger en "sød" brøk.\\
    \medskip
    Læg det til determinanten,\\
    det er humanistisk fjanten -\\
    for hun når jo aldrig læng’re\\
    end at svaret det er $x$.
  \end{SBVerse}

  \begin{SBVerse}
    Når en biolog skal lære,\\
    hvordan man multiplicerer,\\
    skal eksempler observeres,\\
    hvor to køer bli’r til tre.\\
    \medskip
    Og gør man det en gang mere,\\
    Så bli’r tre kø’r jo til flere!\\
    Så hun kender kun’ et svar:\\
    at der er flere nu end før.
  \end{SBVerse}

  \begin{SBVerse}
    Geologer de kan regne,\\
    dem skal man ikke forklejne.\\
    De bestemmer snildt en alder,\\
    når de måler på en sten.\\
    \medskip
    Men har deres kompetencer\\
    nogen sociale nuancer?\\
    Man skal være i en grusgrav\\
    før det bliver relevant!
  \end{SBVerse}

  \begin{SBVerse}
    Men hos os fra \TKET{}\\
    giver regning ingen jamren,\\
    vores ligninger gi’r mening\\
    også med socialt aspekt.\\
    \medskip
    Vi beregner øl i kasser\\
    med potens og inderklasser,\\
    og vor \KASS har altid styr på,\\
    hvem der skylder hvad for hvad.
  \end{SBVerse}

  \begin{SBVerse}
    Så lektionen den må være:\\
    Hold mat/fys’erne i ære,\\
    de har styr på deres sager,\\
    og de drikker mange øl. \emph{(Skål!)}\\
    \medskip
    Udregningerne skal komme\\
    fra han-dyrene med vomme,\\
    der beviser de har været\\
    alt for meget på TK!
  \end{SBVerse}
\end{song}
\begin{song}{Jeg har set en rigtig .*}
  {} % Bruges ikke, lad stå blank
  {Jeg har set en rigtig negermand, Niels C. Andersen} % Titel, Kunstner - eks.: "Jutlandia, Kim Larsen". Hvis sangen er på sin egen melodi, brug da \SBOrgMel.
  {} % Navnet på forfatteren. Undlad aliasser. Brug "&" frem for "og". Hvis forfatter er ukendt, lad da stå tom.
  {DIKUrevy/Sommerfest, 2002} % Eks. "Fysikrevy 2010" eller "2010"
  {\NotCCLIed} % Lad stå som den er

  \begin{SBVerse}
    Jeg har set en rigtig fysiker.\\
    Han havd’ skæg og hår så langt som en musiker.\\
    Han var sikkert ikk’ en rigtig mand.\\
    Han ligned’ en der sku’ ha buksevand.\\\medskip
    Jeg spurgte ham: "Hvad er du da for en?\\
    Hvorfor har du så tynde blege stankelben?\\
    Kom du fra rummet? Hvem er mon din mor?"\\
    Så lo han blot og sagde disse ord:\\\medskip
    \emph{``Det er sandt, at I fandt en fysikmutant.\\
    Jeg har tragten med og hjernen sat i pant.''}
  \end{SBVerse}

  \begin{SBVerse}
    Jeg har set en rigtig smart jurist.\\
    Gik i jakkesæt og spilled sej, det var sgu’ trist.\\
    Det kan godt vær’ han får mange pi’r,\\
    men hans studiejob det er at ta’ kopier.\\\medskip
    Hans hobby er at mishandle små dyr.\\
    Før han vil hjælpe nogen kræver han gebyr.\\
    Men det er også klart han er sadist,\\
    det skal man være for at bli’ jurist.\\\medskip
    \emph{En jurist er så trist, han er AntiKrist.\\
    Gider nogen hente Max von XORcist?}
  \end{SBVerse}

  \begin{SBVerse}
    Jeg har set en rigtig humanist.\\
    Han var knokleskæv og sikkert også kommunist.\\
    Så jeg spurgte hvad hans pensum var,\\
    og han kigged’ op og gav mig dette svar:\\\medskip
    ``Til sommer skal jeg op i Anders And,\\
    et nummer af Fantomet og en Superman,\\
    og hvis en dag jeg vågner af min døs,\\
    så bli’r jeg færdig og er arbejdsløs.''\\\medskip
    \emph{Har meldt pas! Tigger hash! Lever kun på nas!\\
    Når vi styrer verden får I alle gas!}
  \end{SBVerse}

  \begin{SBVerse}
    Jeg har set en rigtig biolog,\\
    og hun rulled’ sig i mudder’t som en anden so.\\
    Hendes hår var fyldt med andemad.\\
    Hun var måske køn hvis bar’ hun fik et bad.\\\medskip
    Jeg spurgte: "Er du kvinde eller mand?\\
    Hvorfor har du det mudder nede i din spand?\\
    Har du en kær’ste, kan jeg bli’ din fyr?"\\
    "Nej tak," sa’ hun, "jeg boller kun med dyr."\\\medskip
    \emph{Biolog, er en so, sku’ man ellers tro.\\
    Hvis du ellers får hend’ ren, så er hun go’.}
  \end{SBVerse}

  \begin{SBVerse}
    Jeg er go’, for jeg er datalog.\\
    Jeg har jakkesæt \& slips \& nye gummisko.\\
    Mine bumser fik lidt kirurgi,\\
    og nu er det mig som pigerne kan li’.\\\medskip
    I Rungsted lever jeg i sus og dus\\
    med Porsche, fast forbindelse og kæmpe hus.\\
    Jeg lever højt på Internet-trafik,\\
    og tjener fedt på andre fjolsers klik.\\\medskip
    \emph{Jeg var fed, jeg var led, på succes jeg red\\
    indtil sidste år da aktierne gik ned.}
  \end{SBVerse}
\end{song}
\begin{song}{Din mor}
  {} % Bruges ikke, lad stå blank
  {Gaston, Disney} % Titel, Kunstner - eks.: "Jutlandia, Kim Larsen". Hvis sangen er på sin egen melodi, brug da \SBOrgMel.
  {Ole Søe Sørensen og Christian Bladt Brandt} % Navnet på forfatteren. Undlad aliasser. Brug "&" frem for "og". Hvis forfatter er ukendt, lad da stå tom.
  {\TKET{}, 2011} % Eks. "Fysikrevy 2010" eller "2010"
  {\NotCCLIed} % Lad stå som den er

  \begin{SBSection*}
    Hør, unge mand, hvilket sprog De dog har!\\
    Og hvad mon De bilder Dem ind?\\
    Stod det til mig, ja så ville De snart\\
    føle min hånd på din kind.
  \end{SBSection*}

  \begin{SBSection*}
    For opdrag’lse kræver en hånd der er fast,\\
    og lejlighedsvis ogs’ et bat!\\
    Jeg ved, at De nok ej kan lægges til last,\\
    for det er jo tydeligt at:
  \end{SBSection*}

  \begin{SBVerse}
    Ingen er som din mor så’n en mær som din mor.\\
    Ingen ifør’r sig kejserens klæ’r som din mor.\\
    For der er ingen skøge i Aalborg\\
    med et rygte så blakket som hun.\\
    Hun er skaldet og trind li’som Voldborg.\\
    Ud’n at sige for meget så siger jeg kun:
  \end{SBVerse}

  \begin{SBVerse}
    Ingen sug’r som din mor, ingen slug’r som din mor,\\
    ingen burde gå på slankekur som din mor.\\
    Der er ingen så fed her i kongeriget,\\
    ingen så stor som din mor.
  \end{SBVerse}

  \begin{SBVerse}
    Ingen ved som din mor hver en ed som din mor,\\
    ingen har en så udforsket sked’ som din mor.\\
    Hendes skød er som fadkoteletter;\\
    sejt og med klumper af brusk.\\
    Altså det har jeg hørt fra min fætter.\\
    Hun rider dig hårdt med en pisk som en kusk.
  \end{SBVerse}

  \begin{SBVerse}
    Ingen skrig’r som din mor på lidt lir som din mor.\\
    Ingen lær’ os om blomster og bier som din mor.\\
    Inspirerede Munch til at male Skriget!\\
    Hip hip hurra for din mor.
  \end{SBVerse}

  \begin{SBSection*}
    Som lille der fik jeg historier fortalt\\
    om lig’ne der rådner i jord.\\
    Og jeg ville langt hel’re døden ha’ valgt\\
    end at kaste et blik på din mor\ldots
  \end{SBSection*}

  \begin{SBVerse}
    Ingen har som din mor et pessar som din mor,\\
    der’ på stør’lse med et badekar som din mor.\\
    Hun er mægtig som bølgernes fråden,\\
    og har tusinde mænd i sin favn.\\
    Hun er løs, båd’ i kød’t og på tråden.\\
    Du har titusind søskende i hver en havn.
  \end{SBVerse}

  \begin{SBVerse}
    Ingen slår som din mor, og har hår som din mor,\\
    ingen sagde så meget i går som din mor!\\
    Alle helvedes pinsler er blid massage\\
    i forhold til – din mor!
  \end{SBVerse}
\end{song}
\begin{song}{Glad for Fysik}
  {} % Bruges ikke, lad stå blank
  {Quangs sang, Anders Matthesen} % Titel, Kunstner - eks.: "Jutlandia, Kim Larsen". Hvis sangen er på sin egen melodi, brug da \SBOrgMel.
  {} % Navnet på forfatteren. Undlad kaldenavne. Brug gerne TBF. Brug "&" frem for "og". Hvis forfatter er ukendt, lad da stå tom.
  {FysikRevy, 2013} % Eks. "Fysikrevy, 2010" eller "2010"
  {\NotCCLIed} % Lad stå som den er

  \begin{SBVerse}
	Her er en historie om en dreng der hedder Finn.\\
	Han er humanist på 7. år, og fatter ik' en pind.\\
	KUA det er mørkt og goldt, og Finn har studiegæld.\\
	Så han prøver sig med Logos, for det meste uden held.
  \end{SBVerse}

  \begin{SBVerse}
	Jeg har kendt en smart jurist, hvis navn jeg nu har glemt.\\
	Han er specialist i at slikke røv, og han har det ikke nemt.\\
	Han pukler hele natten, til den næste morgen gryr.\\
	Kopierer ting for rige folk og lever af geby'r.
  \end{SBVerse}

  \begin{SBChorus}
	Så hvordan kan du dog sige, at du ikke er tilfreds?\\
	At din regnelærer er dum og sur og dine lektier gi'r dig stress?\\
	Du har alt, hvad du skal bruge, du har Schuams og hævegrej.\\
	Og der er tusind humanister, der ville ønske de var dig.
  \end{SBChorus}

  \begin{SBVerse}
	Tom sidder og koder med kroppen på et ton.\\
	Han har slet ingen venner, og hans kær'ste er hans hånd.\\
	Han scorer sjældent piger, men synes Ponyer er fedt.\\
	Han er groet fast bag skærmen, hvor han koder Latin-1.
  \end{SBVerse}

  \begin{SBChorus}
	Så hvad driver dig til at sige at studiet er grumt?\\
	At du ikke gider Newton og synes termo, det er dumt?\\
	Du har alt, hvad du skal bruge, du har Schuams og hævegrej.\\
	Og utallige dataloger ville ønske de var dig.
  \end{SBChorus}
\end{song}

\onecolumn
\chapter*{Diverse}
\twocolumn
\begin{song}{Gaffa og WD-40, Oh my}
  {} % Bruges ikke, lad stå blank
  {Happy, Pharell Williams} % Titel, Kunstner - eks.: "Jutlandia, Kim Larsen". Hvis sangen er på sin egen melodi, brug da \SBOrgMel.
  {Jonatan, Jonas Kielsholm, Diderik \& Diana} % Navnet på forfatteren. Undlad kaldenavne. Brug gerne TBF. Brug "&" frem for "og". Hvis forfatter er ukendt, lad da stå tom.
  {TÅGEKAMMERET, 2014} % Eks. "Fysikrevy, 2010" eller "2010"
  {\NotCCLIed} % Lad stå som den er

  \begin{SBVerse}
    Hvad ville du gøre hvis dit hus ik havde et tag?\\
    Dine arme virker ikke for de falder af.\\
    Jeg bruger det bedste værktøj, creme d’la creme.\\
    Uh, Hvis du lytter nu, bliver det hele så nemt.
  \end{SBVerse}

  \begin{SBChorus}
    \emph{Jeg bruger Gaffa!}
    Hvis din bil den mangler dæk og dit tv det går itu.\\
    \emph{Jeg bruger Gaffa!}
    Hvis din hund den løber væk, men det var egentlig ik’ det den sku’.\\
    \emph{Jeg bruger Gaffa!}
    Hvis du ik’ ka lukke kærstens mund eller har hårvækst du gerne vil fjern’.\\
    \emph{Jeg bruger Gaffa!}
    Når du pepper dit sexliv op men du har ikke nogen gode håndjern.
  \end{SBChorus}

  \begin{SBVerse}
    Gaffa er fint, hvis det er stationært.\\
    Skal det køre rundt, bliver det en smule svært.\\
    Bedstefars gigt, det klares i en fart,\\
    buksen ryger på med mit præparat.
  \end{SBVerse}

  \begin{SBChorus}
    \emph{WD-40!} Når dit køleskab er tomt, men din rugbrødsmad mangler smør.\\
    \emph{WD-40!} Når din kærest’ har hovedpin’, men du selv er i godt humør.\\
    \emph{WD-40!} Når din hamster piver lidt, og du sidder med tømmermænd.\\
    \emph{WD-40!} Mod slidte led og mod stramme låg, det hele løses med et klem.
  \end{SBChorus}

  \begin{SBSection*}
    Ingenting, der er bedre end\\
    Ingenting, der er bedre end\\
    \emph{(WD, Gaffa, 40, Gaffa, Gaffa, WD, gaffa, 40)}
  \end{SBSection*}

  \begin{SBChorus}
    \emph{Gaffa!} Hvis du er til forelæsning og øjnene de falder i.\\
    \emph{WD-40!} Yndlings-olie-bi-produkt, for alt det andet er blasfemi!\\
    \emph{Brug kun gaffa!} Hvis du har et åbent sår, og din læge ik’ kan fikse det.\\
    \emph{WD-40!} Dit Halting-problem løses og dit NP bliver lig med P.
  \end{SBChorus}
\end{song}
\begin{song}{Doom Doom}
  {} % Bruges ikke, lad stå blank
  {Melodi} % Titel, Kunstner - eks.: "Jutlandia, Kim Larsen". Hvis sangen er på sin egen melodi, brug da \SBOrgMel.
  {Forfatter} % Navnet på forfatteren. Undlad kaldenavne. Brug gerne TBF. Brug "&" frem for "og". Hvis forfatter er ukendt, lad da stå tom.
  {Anledning og år} % Eks. "Fysikrevy, 2010" eller "2010"
  {\NotCCLIed} % Lad stå som den er

  \begin{SBVerse}
    % Skriv vers her
  \end{SBVerse}

  \begin{SBChorus}
    % Skriv omkvæd her
  \end{SBChorus}

  \begin{SBSection*}
    % Skriv sektioner her. Hvis du ønsker lidt mellemrum for at give luft i et langt afsnit el.lign., brug da \\\medskip
  \end{SBSection*}
\end{song}
\begin{song}{Jeg' en nørd}
  {} % Bruges ikke, lad stå blank
  {Pokémon} % Titel, Kunstner - eks.: "Jutlandia, Kim Larsen". Hvis sangen er på sin egen melodi, brug da \SBOrgMel.
  {} % Navnet på forfatteren. Undlad kaldenavne. Brug gerne TBF. Brug "&" frem for "og". Hvis forfatter er ukendt, lad da stå tom.
  {FysikRevy, 2010} % Eks. "Fysikrevy, 2010" eller "2010"
  {\NotCCLIed} % Lad stå som den er

  \begin{SBVerse}
    AD\&D, det er min leg,\\
    og Go det er min sport.\\
    Nyt tøj, det bytter jeg\\
    til sjældne magickort.
  \end{SBVerse}

  \begin{SBVerse}
    World of Warcraft er så fedt.\\
    Jeg spiller natten lang.\\
    Back to the Future har jeg set\\
    for 85. gang!
  \end{SBVerse}

  \begin{SBChorus}
    Jeg' en nørd! Jeg er stolt af det.\\
    Jeg læser XKCD. Jeg' en nørd!\\
    Skriver i \LaTeX,\\
    og jeg tænder på $dx$.\\\medskip
    Jeg' en nørd! Nu er jeg ved\\
    at finde en at nørde med,\\
    som ka’ ta’ min uskyldighed.\\
    Jeg' en nørd, og jeg er stolt af det!\\\medskip
    Jeg er stolt af det!
  \end{SBChorus}

  \begin{SBVerse}
    Jeg er level 10 i sværd,\\
    så spil nu ikke smart.\\
    I min bæltetaske er\\
    der terninger parat.
  \end{SBVerse}

  \begin{SBVerse}
    Sig at det skal være os,\\
    og mærk min energi.\\
    Når jeg læser Hitchhikers’\\
    Guide to the Galaxy.
  \end{SBVerse}

  \begin{SBChorus}
    Jeg' en nørd! Jeg er stolt af det.\\
    Jeg læser XKCD. Jeg' en nørd!\\
    Skriver i \LaTeX,\\
    og jeg tænder på $dx$.\\\medskip
    Jeg' en nørd! Nu er jeg ved\\
    at finde en at nørde med,\\
    som ka’ ta’ min uskyldighed.\\
    Jeg' en nørd, og jeg er stolt af det!\\\medskip
    Jeg er stolt af det!\\\medskip
    \emph{Jeg' en nørd!}
  \end{SBChorus}

  \begin{SBSection*}
    % Skriv sektioner her. Hvis du ønsker lidt mellemrum for at give luft i et langt afsnit el.lign., brug da \\\medskip
  \end{SBSection*}
\end{song}

\begin{song}{Sort snak}
  {} % Bruges ikke, lad stå blank
  {Storkespringvandet, Cæsar} % Titel, Kunstner - eks.: "Jutlandia, Kim Larsen". Hvis sangen er på sin egen melodi, brug da \SBOrgMel.
  {} % Navnet på forfatteren. Undlad kaldenavne. Brug gerne TBF. Brug "&" frem for "og". Hvis forfatter er ukendt, lad da stå tom.
  {} % Eks. "Fysikrevy, 2010" eller "2010"
  {\NotCCLIed} % Lad stå som den er

  \begin{SBVerse}
    $\aleph\ \aleph\ \aleph\ \aleph\ \chi$\\
    $\aleph\ \aleph\ \aleph\ \pi$\\
    $\aleph\ \pi\ \chi\ \aleph\ \pi\ \chi\ \pi$\\
    $\chi\ \chi\ \pi\ \aleph\ \lambda\ \aleph$
  \end{SBVerse}

  \begin{SBVerse}
    $\beta\ \beta\ \beta\ \beta\ \xi$\\
    $\beta\ \beta\ \beta\ \psi$\\
    $\beta\ \psi\ \xi\ \beta\ \psi\ \xi\ \psi$\\
    $\xi\ \xi\ \psi\ \beta\ \gamma\ \beta$
  \end{SBVerse}
\CBPageBrk
  \begin{xlatn}{Sort snak}
    {}
    {Oversat af SSM}

    \begin{SBVerse}
      Alef alef alef alef chi\\
      Alef alef alef pi\\
      Alef pi chi alef pi chi pi\\
      Chi chi pi alef lambda alef
    \end{SBVerse}
    \begin{SBVerse}
      Beta beta beta beta xi\\
      Beta beta beta psi\\
      Beta psi xi beta psi xi psi\\
      Xi xi psi beta gamma beta
    \end{SBVerse}
  \end{xlatn}
\end{song}
  \begin{song}{Fulbert og Beatrice}
  {} % Bruges ikke, lad stå blank
  {\SBOrgMel} % Titel, Kunstner - eks.: "Jutlandia, Kim Larsen". Hvis sangen er på sin egen melodi, brug da \SBOrgMel.
  {Jens Louis Petersen} % Navnet på forfatteren. Undlad aliasser. Brug "&" frem for "og". Hvis forfatter er ukendt, lad da stå tom.
  {1951} % Eks. "Fysikrevy 2010" eller "2010"
  {\NotCCLIed} % Lad stå som den er
  \fancyhead[CE,CO]{\CHeadFont10}

  \begin{SBVerse}
    I frankens rige, hvor floder rinde\\
    som sølverstrømme i lune dal,\\
    lå ridderborgen på bjergets tinde\\
    med slanke tårne og gylden sal.\\\medskip
    Og det var sommer med blomsterbrise\\
    og suk af elskov i urtegård.\\
    Og det var Fulbert og Beatrice,\\
    og Beatrice var sytten år.
  \end{SBVerse}

  \begin{SBVerse}
    De havde leget som børn på borgen,\\
    mens Fulbert endnu var gangerpilt.\\
    Men langvejs drog han en årle morgen,\\
    mod Saracenen han higed' vildt.\\\medskip
    Han spidded' tyrker som pattegrise,\\
    et tusind stykker blev lagt på bår,\\
    for Fulbert kæmped' for Beatrice,\\
    og Beatrice var sytten år.
  \end{SBVerse}

  \begin{SBVerse}
    Med gluttens farver på sølversaddel\\
    han havde stridt ved Jerusalem.\\
    Han kæmped' kækt uden frygt og dadel\\
    og gik til fods hele vejen hjem.\\\medskip
    Nu sad han atter på bænkens flise\\
    og viste stolt sine heltesår,\\
    som ganske henrykked' Beatrice\\
    for Beatrice var sytten år.
  \end{SBVerse}

  \begin{SBVerse}
    En kappe prydet med små opaler\\
    og smagfuldt ternet med tyrkens blod,\\
    en ring af guld og et par sandaler\\
    den ridder lagde for pigens fod.\\\medskip
    Og da hun øjnede hans caprise\\
    blev hjertet mygt i den væne mår.\\
    Af lykke dånede Beatrice,\\
    for Beatrice var sytten år.
  \end{SBVerse}

  \begin{SBVerse}
    Da banked' blodet i heltens tinding,\\
    thi ingen helte er gjort af træ.\\
    Til trods for plastre og knæforbinding\\
    sank ridder Fulbert med stil på knæ.\\\medskip
    Han kvad: ``Skønjomfru, oh skænk mig lise,\\
    thi du alene mit hjerte rår!''\\
    ``Min helt, min ridder,'' kvad Beatrice,\\
    for Beatrice var sytten år.
  \end{SBVerse}

  \begin{SBVerse}
    Og der blev bryllup i højen sale\\
    med guldpokaler og troubadour,\\
    og under sange og djærven tale\\
    blev Fulbert ført til sin jomfrus bur.\\\medskip
    Og følget hvisked' om øm kurtice\\
    og skæmtsom puslen blandt dun og vår.\\
    For det var Fulbert og Beatrice,\\
    og Beatrice var sytten år.
  \end{SBVerse}

  \begin{SBVerse}
    Men ridder Fulbert den samme aften\\
    af borgens sale blev båren død.\\
    Den megen krig havde tær't på kraften,\\
    og sejrens palmer den sidste brød.\\\medskip
    Oh bejler, lær da af denne vise:\\
    Ød ej din kraft under krigens kår.\\
    Nej, spar potensen til Beatrice,\\
    når Beatrice er sytten år.
  \end{SBVerse}
\end{song}
\begin{song}{$\alpha\beta$-sangen}{}
  {I en kælder sort som kul, Vilhelm Høm}
  {Tenna Schaldemose og Tine Nyeng}
  {\TKET{}s Majrevy, 2004}
  {\NotCCLIed}

  \begin{SBVerse}
    $\alpha\ \beta\ \gamma\ \delta$\hspace{-2.5pt}\protect\colorbox{white}{\phantom{$\delta$}}\\
    \hspace{-2.5pt}$\delta$\hspace{-1.25em}\colorbox{white}{\phantom{$\delta$}}$\ \varepsilon\ \zeta$\\
    $\eta\ \theta\ \iota\ \kappa$\hspace{-2.5pt}\protect\colorbox{white}{\phantom{$\kappa$}}\\
    \hspace{-3pt}$\kappa$\hspace{-1.3em}\colorbox{white}{\phantom{$\kappa$}}$\ \lambda\ \mu\ \nu\ \xi$\\
    $o\ \pi\ \rho\ \sigma$\\
    $\tau\ \upsilon\ \varphi\ \chi\ \psi$\\
    $\omega$ er sidste,\\
    $24$ på liste!
  \end{SBVerse}
\CBPageBrk
  \begin{xlatn}{alfa-beta-sangen}
    {}
    {}

    \begin{SBVerse}
      Alfa beta gamma del-\\
      ta epsilon zeta\\
      eta theta iota kap-\\
      pa lambda my ny ksi\\
      omikron pi rho sigma\\
      tau ypsilon phi chi psi\\
      omega er sidste,\\
      fireogtyve på liste!
    \end{SBVerse}
  \end{xlatn}
\end{song}
\fancyhead[CE,CO]{\CHeadFont\thepage}
\begin{song}{Ode til bacon}
  {} % Bruges ikke, lad stå blank
  {I want it that way, Backstreet Boys} % Titel, Kunstner - eks.: "Jutlandia, Kim Larsen". Hvis sangen er på sin egen melodi, brug da \SBOrgMel.
  {Rasmus Fruergaard-Pedersen} % Navnet på forfatteren. Undlad aliasser. Brug "&" frem for "og". Hvis forfatter er ukendt, lad da stå tom.
  {TÅGEKAMMERETs Julerevy, 2005} % Eks. "Fysikrevy 2010" eller "2010"
  {\NotCCLIed} % Lad stå som den er

  \begin{SBVerse}
    Gik hen i Fø\TeX.\\
    Det var en refleks.\\
    Det var med en ræson.\\
    Jeg sku’ ha’ bacon.
  \end{SBVerse}

  \begin{SBVerse}
    Jeg så i disken,\\
    helt tom. Jeg følte\\
    på min vom – var som beton.\\
    Den skal ha’ bacon.
  \end{SBVerse}

  \begin{SBChorus}
    Hvorfor er der tomt i køledisken?\\
    Hvorfor er der ingen lækkerbidsken?\\
    Og hvor er den gris der’ skåret i facon?\\
    Jeg vil ha’ bacon.
  \end{SBChorus}

  \begin{SBVerse}
    Jeg så i vantro.\\
    (Vægt)-vogterne de stod og lo.\\
    Kom nu herhen smag vor karton.\\
    Nej, jeg vil ha’ bacon.
  \end{SBVerse}

  \begin{SBChorus}
    Hvorfor er der tomt i køledisken\ldots
  \end{SBChorus}

  \begin{SBSection*}
    Smagen kan give dig lysten til livet,\\
    selvom du er vegetar, Aaah!\\
    Du blir hvad du spiser, så blir jeg en gris, ja.\\
    Det’ hel’re end hundred’ år!
  \end{SBSection*}

  \begin{SBVerse}
    Stegt let med smør til.\\
    Svøbt om pølser på grill.\\
    Stegt sprødt, stegt sprødt, stegt sprødt, stegt sprødt\\
    \ldots
  \end{SBVerse}

  \begin{SBSection*}
    \emph{Svøb flæsk i bacon!}
  \end{SBSection*}

  \begin{SBSection*}
    Stegt i smør eller helt naturel.\\
    Føler mig så høj og kulturel.\\
    Jeg ku’ synge ud fra en balkon:\\
    "Jeg vil ha’ bacon."
  \end{SBSection*}

  \begin{SBChorus}
    Smag på lidt bacon svøbt om dejlig mørksej.\\
    Smag på lidt bacon fyldt i julepostej.\\
    Smag på lidt bacon strøget let med stærk dijon, dijon.\\
    Smag på lidt bacon.
  \end{SBChorus}

  \begin{SBChorus}
    Giv mig en bacon-krydder-frikadelle.\\
    Jeg har lyst til en saltet stegt fedtcelle.\\
    Hvem laver mig en saltet flæskesværs-bonbon?\\
    Jeg hylder bacon.
  \end{SBChorus}

  \begin{SBSection*}
    For jeg vil ha’ bacon.
  \end{SBSection*}
\end{song}
\begin{song}{En veganer}
  {} % Bruges ikke, lad stå blank
  {Indianer, Tøsedrengene} % Titel, Kunstner - eks.: "Jutlandia, Kim Larsen". Hvis sangen er på sin egen melodi, brug da \SBOrgMel.
  {} % Navnet på forfatteren. Undlad kaldenavne. Brug gerne TBF. Brug "&" frem for "og". Hvis forfatter er ukendt, lad da stå tom.
  {Birevy, 2015} % Eks. "Fysikrevy, 2010" eller "2010"
  {\NotCCLIed} % Lad stå som den er

  \begin{SBVerse}
    Står ved en pølsevogn,\\
    føler mig lidt som en klovn.\\
    Står og tåger ud i luften -\\
    hvordan endte jeg her?
  \end{SBVerse}

  \begin{SBVerse}
    Bacon, pølser, flæskesteg\\
    er ikke lige nogt’ for mig.\\
    Jeg vil gerne ud i skoven\\
    hvor jeg kommer fra.
  \end{SBVerse}

  \begin{SBChorus}
    En veganer.\\
    Spiser alt der gror,\\
    planter med lidt jord.\\
    En veganer\\
    kalder alle mig
  \end{SBChorus}

  \begin{SBVerse}
    Svampene ka’ jeg li.\\
    Ba-e-o-cy-stin gør mig fri,\\
    og min hjerne den ser farver,\\
    gør mig lykkelig
  \end{SBVerse}

  \begin{SBVerse}
    Min hjerne bliver helt brændt af\\
    af svampene jeg tog idag.\\
    Før jeg følte mig så fanget,\\
    svampe gjor’ mig fri
  \end{SBVerse}

  \begin{SBChorus}
    En veganer\\
    Spiser alt der gror\\
    svampe med lidt jord\\
    En veganer\\
    hvem kalder på mig?
  \end{SBChorus}

  \begin{SBChorus}
    En veganer\\
    Spiser alt der gror\\
    svampe med lidt jord\\
    En veganer\\
    nu er jeg for sej
  \end{SBChorus}

  \begin{SBVerse}
    I skovene hvor træerne gror\\
    begik jeg mit første mord.\\
    Fælder træer og lave buske,\\
    træblod i mit spor.
  \end{SBVerse}

  \begin{SBVerse}
    Er klar til at gå berserk.\\
    De mange planter jeg vil kværk'\\
    med min økse og min flamme,\\
    skoven brænder stærkt 
  \end{SBVerse}

  \begin{SBChorus}
    En veganer\\
    jager alt der gror\\
    på vor sarte jord\\
    En veganer\\
    planter dræber jeg
  \end{SBChorus}

  \begin{SBChorus}
    En veganer\\
    jager alt der gror\\
    på vor sarte jord\\
    En veganer\\
    buske brænder jeg
  \end{SBChorus}

  \begin{SBChorus}
    En veganer\\
    jager alt der gror\\
    på vor sarte jord\\
    En veganer\\
    blomster dræber jeg
  \end{SBChorus}

  \begin{SBChorus}
    En veganer\\
    jager alt der gror\\
    på vor sarte jord\\
    En veganer\\
    træer fælder jeg
  \end{SBChorus}

  \begin{SBChorus}
    En veganer\\
    jager alt der gror\\
    på vor sarte jord\\
    En veganer\\
    ukrudt dræber jeg
  \end{SBChorus}
\end{song}

\onecolumn
\chapter*{UNF}
\twocolumn
\begin{song}{Minken Mink}
  {} % Bruges ikke, lad stå blank
  {Bjørnen Bjørn, Sigurd Barrett} % Titel, Kunstner - eks.: "Jutlandia, Kim Larsen". Hvis sangen er på sin egen melodi, brug da \SBOrgMel.
  {SSM \& SABH} % Navnet på forfatteren. Undlad aliasser. Brug "&" frem for "og". Hvis forfatter er ukendt, lad da stå tom.
  {UNF Computer Science Camp, 2015} % Eks. "Fysikrevy 2010" eller "2010"
  {\NotCCLIed} % Lad stå som den er

  \begin{SBSection*}
    \SBRepeat{Minken Mink er en mink.}\\
    Og den kan li' de fleste, den er så flink,\\
    For Minken Mink er en mink
  \end{SBSection*}

  \begin{SBSection*}
    Nogen tror at Minken er en ræv.\\
    Nogen tror at Minken er en husmår.\\
    Nogen tror at Minken er en kat.\\
    Men Minken Mink er en mink.\\
    Ja, Minken Mink er en mink!
  \end{SBSection*}
\end{song}
\begin{song}{ScienceCamps}
  {} % Bruges ikke, lad stå blank
  {Pokémon} % Titel, Kunstner - eks.: "Jutlandia, Kim Larsen". Hvis sangen er på sin egen melodi, brug da \SBOrgMel.
  {SSM, KBE, MGS, RES, MDE, DBR, HFA} % Navnet på forfatteren. Undlad aliasser. Brug "&" frem for "og". Hvis forfatter er ukendt, lad da stå tom.
  {UNF Revy, 2016} % Eks. "Fysikrevy 2010" eller "2010"
  {\NotCCLIed} % Lad stå som den er

  \begin{SBVerse}
    Jeg drager ud på livets vej,\\
    jeg har et enkelt mål\\
    ScienceCamps afholder jeg,\\
    min vilje er af stål!
  \end{SBVerse}

  \begin{SBVerse}
    Starter ud på Chemisty\\
    Og ta’r så til Fysik\\
    GDC og Medico,\\
    derefter Matematik
  \end{SBVerse}

  \begin{SBChorus}
    ScienceCamps, jeg kan klare dem!\\
    Multitasking er mit lod\\
    4 camps - åhh, jeg elsker det,\\
    bare sommer’n bli’r ved og ved\\\medskip
    4 camps, jeg ka’ klare det,\\
    det kræver bare koffein\\
    Jeg vil bare hjælpe til!\\
    ScienceCamps - kan jeg klare dem?\\
    Jeg kan klare dem!
  \end{SBChorus}

  \begin{SBVerse}
    Socialt ansvar på nanocamp,\\
    kasserer på SDC\\
    Fagligt team og så PR,\\
    koordinators vilje ske!
  \end{SBVerse}

  \begin{SBVerse}
    Første uge er ovre nu,\\
    Ser frem til kriminal\\
    Astro, Show og CSC\\
    Det bliver fænomenalt!
  \end{SBVerse}

  \begin{SBChorus}
    13 camps, jeg skal klare det!\\
    måske det er lidt hårdt\\
    20 camps, ååh, Discovery!\\
    Jeg skal smadre den rekord!\\\medskip
    40 camps, jeg har brug for hjælp!\\
    Jeg mangler stadig CSI!\\
    Robodays og Futureweek,\\
    Medicin, jeg skal bruge det!
  \end{SBChorus}

  \begin{SBVerse}
    Når presset det bliver alt for stort\\
    og jeg går psykisk ned\\
    Så tager jeg bare en ekstra camp,\\
    for jeg bliver ved og ved!
  \end{SBVerse}

  \begin{SBVerse}
    Min psykolog, han siger stop!\\
    - min læge ligeså\\
    Men jeg mangler Biotech,\\
    den må jeg lige på!
  \end{SBVerse}

  \begin{SBChorus}
    100 camps, alle skal jeg nå!\\
    Det er det jeg duer til\\
    1000 camps, åh Medicinal\\
    Robot og ISSC\\\medskip
    Alle camps, åh, jeg er lidt træt,\\
    men stemmerne bliver ved og ved\\
    Næste år, der står jeg klar!\\
    ScienceCamps, jeg kan klare dem,\\
    kan jeg klare dem? ScienceCamps!
  \end{SBChorus}
\end{song}
\begin{song}{Det var i 1944}
  {} % Bruges ikke, lad stå blank
  {Jutlandia, Kim Larsen} % Titel, Kunstner - eks.: "Jutlandia, Kim Larsen". Hvis sangen er på sin egen melodi, brug da \SBOrgMel.
  {DBR, MWR} % Navnet på forfatteren. Undlad kaldenavne. Brug gerne TBF. Brug "&" frem for "og". Hvis forfatter er ukendt, lad da stå tom.
  {UNF Revy, 2016} % Eks. "Fysikrevy, 2010" eller "2010"
  {\NotCCLIed} % Lad stå som den er

  \begin{SBVerse}
    Det var i 1944 eller cirka der omkring,\\
    da UNF blev stiftet\\
    Vi kommer fra foreningen, der hedder SNU,\\
    nen nu er de forældet\\
    Ung men’sker fra gym’asiet af\\
    uddeler naturvidenskab
  \end{SBVerse}

  \begin{SBChorus}
    Ja ja! Fra UNF af\\
    Vi kommer som altid med viden\\
    Newton med tyngdekraft og Bohr med fysik,\\
    og Bjerrum, han lav’d diagrammet
  \end{SBChorus}

  \begin{SBVerse}
    Vi holder foredrag og workshops, og vi tar’ på studietur\\
    Når vi formidler vor viden\\
    Rundt i hele landet, ja, med alle vor forsøg\\
    - og folk de falder på stribe\\
    Forskere, det skal vi alle vær’\\
    Come on students, vis os noget blær
  \end{SBVerse}

  \begin{SBChorus}
    Ja ja! Fra UNF af\ldots
  \end{SBChorus}

  \begin{SBVerse}
    Vi er fir' foreninger der dækker vores land\\
    men vi vil gern’ være flere\\
    Vi ha’d’ både Midtjylland og Sønderborg en gang\\
    Så’n er det ikke mere\\
    Arrangører på 16 år\\
    laver camps, som de ikke forstår
  \end{SBVerse}

  \begin{SBChorus}
    Ja ja! Fra UNF af\ldots
  \end{SBChorus}

  \begin{SBChorus}
    Ja ja! Fra UNF af\ldots
  \end{SBChorus}
\end{song}
\begin{song}{Din TBF}
  {} % Bruges ikke, lad stå blank
  {YMCA, Village People} % Titel, Kunstner - eks.: "Jutlandia, Kim Larsen". Hvis sangen er på sin egen melodi, brug da \SBOrgMel.
  {SSM, MWR, AAN} % Navnet på forfatteren. Undlad kaldenavne. Brug gerne TBF. Brug "&" frem for "og". Hvis forfatter er ukendt, lad da stå tom.
  {UNF Revy, 2016} % Eks. "Fysikrevy, 2010" eller "2010"
  {\NotCCLIed} % Lad stå som den er

  \begin{SBVerse}
    Du der - du skal vær’ arrangør\\
    Jeg sa’e du der - hvis du ellers da tør\\
    Jeg sa’e du der - hvis du vil være med\\
    til at gør’ en kæmpe forskel
  \end{SBVerse}

  \begin{SBVerse}
    Du der, hvis du skriver dit navn\\
    på den her er--klæring bliver dit savn\\
    for naturvi--denskab fuldstændig væk\\
    og du bliver en af os nu
  \end{SBVerse}

  \begin{SBChorus}
    For du skal bare ha' en TBF\\
    Så bliver du en del af UNF\\
    Så kan du logge ind på vor’s eget system\\
    Du kan rode rundt i vores ting\\\medskip
    Ja, du skal bare have en TBF\\
    Så bliver du en del af UNF\\
    Men pas på hvad du si’r, hvis du let bli’r genert\\
    For du bliver jo nok citeret
  \end{SBChorus}

  \begin{SBVerse}
    Du der, har du fanget det nu?\\
    Jeg si’r du der, er du klar til at brug’\\
    din tid på at sætte deadlines til folk\\
    som allig’vel bryder dem, og
  \end{SBVerse}

  \begin{SBVerse}
    har du nu helt styr på hvordan\\
    du kontakter forelæser så han\\
    siger ja tak så’n at vi kan få fyldt\\
    hele vor's sæsonprogram ud
  \end{SBVerse}

  \begin{SBChorus}
    For du har lige fået din TBF\\
    Så nu er du en del af UNF\\
    Referater er der, formularer med mer’,\\
    og din pizzafaktor er her\\\medskip
    For du har lige fået din TBF\\
    Så nu er du en del af UNF\\
    Hvis du får brug for hjælp, eller bliver i tvivl\\
    kan en QuickPoll redde dit livl
  \end{SBChorus}

  \begin{SBSection*}
    \emph{Din TBF!}
  \end{SBSection*}
\end{song}







\begin{song}{Hymne til Matematik Camp}
  {} % Bruges ikke, lad stå blank
  {I Danmark er jeg født, H. C. Andersen} % Titel, Kunstner - eks.: "Jutlandia, Kim Larsen". Hvis sangen er på sin egen melodi, brug da \SBOrgMel.
  {Søren Møller} % Navnet på forfatteren. Undlad kaldenavne. Brug gerne TBF. Brug "&" frem for "og". Hvis forfatter er ukendt, lad da stå tom.
  {Matematik Camp, 2014} % Eks. "Fysikrevy, 2010" eller "2010"
  {\NotCCLIed} % Lad stå som den er

  \begin{SBVerse}
    På campen har jeg lært, der har jeg hjemme\\
    Der er alt sandt, derfra min fremtid går.\\
    Du mat'matik, du er min egen stemme,\\
    så klart beviserne min hjerne når.\\\medskip
    Min tankes $\pi$ og $\rho$,\\
    hvor Euklids elementer\\
    står mellem algebra og rentesrenter.\\
    Dig agter jeg -- MatCamp mit origo.
  \end{SBVerse}

  \begin{SBVerse}
    Hvor lægges næste år mon sommerskolen?\\
    Mon ik' det er ved jydens Unisø.\\
    Et yndigt sted hvor alle knækker koden,\\
    så dejligt som på Danas egen ø.\\\medskip
    Min tankes $\pi$ og $\rho$,\\
    hvor aksiomer holder.\\
    Gauss gav os alt, han er sandhedens tolder.\\
    Dig agter jeg -- MatCamp mit origo.
  \end{SBVerse}

  \begin{SBVerse}
    Engang du herre var i kloges færden.\\
    Bød over dumhed - nu du kaldes led.\\
    Et lille fag som dog ved alt om verden,\\
    og hører vores sang og troskabsed.\\\medskip
    Min tankes $\pi$ og $\rho$\\
    funktioners nulpunkt finder.\\
    Gauss giv dig rødder, som vi gi'r dig minder.\\
    Dig agter jeg -- MatCamp mit origo.
  \end{SBVerse}

  \begin{SBVerse}
    Se divergens ved polyedrets kanter,\\
    der gi'r uend'ligt mange mindre skridt;\\
    og symmetri blandt vores kommutanter\\
    gør stadig majoranten lige skidt.\\\medskip
    Min tankes $\pi$ og $\rho$,\\
    hvor grænser frit bestemmes,\\
    og ingen sig ved udvælgelsen græmmes.\\
    Dig agter jeg -- MatCamp mit origo
  \end{SBVerse}

  \begin{SBVerse}
    Du camp hvor jeg har lært, hvor jeg har hjemme,\\
    hvor alt er sandt, hvorfra min fremtid går,\\
    hvor mat'matik er livets klare stemme,\\
    og klart beviserne min hjerne når\\\medskip
    Min tankes $\pi$ og $\rho$\\
    ved videnskabens kilde.\\
    I mat'matik min hjerne alt vil vide.\\
    Dig agter jeg -- MatCamp mit origo.
  \end{SBVerse}
\end{song}
\begin{song}{Hvad må man? (i UNF)}
  {} % Bruges ikke, lad stå blank
  {Fy fy skamme, Omsen og Momsen} % Titel, Kunstner - eks.: "Jutlandia, Kim Larsen". Hvis sangen er på sin egen melodi, brug da \SBOrgMel.
  {KBE, RTR, MIL, DBR, SSM, MDE, RES} % Navnet på forfatteren. Undlad kaldenavne. Brug gerne TBF. Brug "&" frem for "og". Hvis forfatter er ukendt, lad da stå tom.
  {UNF Revy, 2016} % Eks. "Fysikrevy, 2010" eller "2010"
  {\NotCCLIed} % Lad stå som den er

  \begin{SBVerse}
    Må man flyv’ på første klasse?\\
    Må man spise mad på NOMA?\\
    Må man hæld’ så meget sprut på delta’r\\
    at de går i koma?\\\medskip
    Må man gamble med vor’s penge \\
    Må man drukne sig i vodka?\\
    Må man bruge Times New Roman \\
    når man laver sig en tryksag?
  \end{SBVerse}

  \begin{SBChorus}
    Næ næ næ næ næ det må vi ikke,\\
    fy fy skamme skamme fy fy ah ah\\
    slemme slemme fy fy næ næ nix nix\\
    slut forbudt - men hva må vi så?
  \end{SBChorus}

  \begin{SBVerse}
    Må man ændre vores logo?\\
    Må man slet’ vor’s database?\\
    Må man pille på en deltager \\
    og gå til næste fase?\\\medskip
    Må man bade under sprinkleren\\
    som er i lab’ratoriet?\\
    Må man sparke til en forelæ-\\
    ser ned’ i auditoriet?
  \end{SBVerse}

  \begin{SBChorus}
    Næ næ næ næ næ det må vi ikke,\\
    fy fy skamme skamme fy fy ah ah\\
    slemme slemme fy fy næ næ nix nix\\
    slut forbudt - men hva må vi så?
  \end{SBChorus}

  % \begin{SBChorus}
  %   Næ næ næ næ næ det må vi ikke\ldots
  % \end{SBChorus}

  \begin{SBVerse}
    Må man sove til et foredrag?\\
    Må man ryge pot på brandvagt?\\
    Må man rode på vor’s lager\\
    eller arranger’ en andagt?\\\medskip
    Må man udsmide plakater\\
    eller grin’ af folk der bløder?\\
    Må man be’ om hem’lig afstemning\\
    til alle vores møder?
  \end{SBVerse}

  \begin{SBChorus}
    Næ næ næ næ næ det må vi ikke,\\
    fy fy skamme skamme fy fy ah ah\\
    slemme slemme fy fy næ næ nix nix\\
    slut forbudt - men hva må vi så?
  \end{SBChorus}

  % \begin{SBChorus}
  %   Næ næ næ næ næ det må vi ikke\ldots
  % \end{SBChorus}

  \begin{SBVerse}
    Må man melde sig til opvask?\\
    Må passe på sin kasse?\\
    Må man komm’ med gode input\\
    når vi planlægger en masse?\\\medskip
    Må man sige fra i tide?\\
    Må man stille op til DK?\\
    Må man finde sig en UNF’er\\
    som man ka’ læg’ an på?
  \end{SBVerse}

  \begin{SBChorus}
    Ja ja ja ja ja det må vi gerne\\
    næ sikke fint fint ja ja meget det blir\\
    stort stort klap klap fint fint flot flot\\
    det var godt - nå det må vi godt
  \end{SBChorus}
\end{song}
\begin{song}{Jeg elsker "science"}
  {} % Bruges ikke, lad stå blank
  {Melodi} % Titel, Kunstner - eks.: "Jutlandia, Kim Larsen". Hvis sangen er på sin egen melodi, brug da \SBOrgMel.
  {Forfatter} % Navnet på forfatteren. Undlad kaldenavne. Brug gerne TBF. Brug "&" frem for "og". Hvis forfatter er ukendt, lad da stå tom.
  {Anledning og år} % Eks. "Fysikrevy, 2010" eller "2010"
  {\NotCCLIed} % Lad stå som den er

  \begin{SBVerse}
    % Skriv vers her
  \end{SBVerse}

  \begin{SBChorus}
    % Skriv omkvæd her
  \end{SBChorus}

  \begin{SBSection*}
    % Skriv sektioner her. Hvis du ønsker lidt mellemrum for at give luft i et langt afsnit el.lign., brug da \\\medskip
  \end{SBSection*}
\end{song}

% Først så var jeg bange, jeg var skrækslagen
% Jeg tænkte, det lød lig’så skrækkeligt som et stræklagen.
% Og jeg sad søvnløs hele natten
% stirred’ tomt ind i min pejs
% Før hed det naturvidenskab,
% nu skal man sige “science”

% Syn’s det var dumt - noget rigtigt lort
% Men nu der ingen vej tilbage, “science” er det nye sort.
% Jeg sku’ ha sagt noget eller gjort noget, 
% åh gået i protest
% Men sket er sket, og science er jo dansk når det er bedst

% Det’ ikk’ så slemt, og egentlig nemt
% Syv bogstaver?
% Det enkelt og bekvemt
% Hvorfor skal man slæbe rundt på gamle danske ord?
% Alle snakker engelsk nu, 
% ligegyldigt hvor de bor.

% Jeg elsker science, jeg elsker science!
% Fysik, kemi, biologi 
% det er jo alt for nice.
% Matematik, geologi 
% og medicinsk teknologi, jeg elsker science
% Jeg elsker science!

% \begin{song}{Vi er brandvagter}
  {} % Bruges ikke, lad stå blank
  {Melodi} % Titel, Kunstner - eks.: "Jutlandia, Kim Larsen". Hvis sangen er på sin egen melodi, brug da \SBOrgMel.
  {Forfatter} % Navnet på forfatteren. Undlad kaldenavne. Brug gerne TBF. Brug "&" frem for "og". Hvis forfatter er ukendt, lad da stå tom.
  {Anledning og år} % Eks. "Fysikrevy, 2010" eller "2010"
  {\NotCCLIed} % Lad stå som den er

  \begin{SBVerse}
    % Skriv vers her
  \end{SBVerse}

  \begin{SBChorus}
    % Skriv omkvæd her
  \end{SBChorus}

  \begin{SBSection*}
    % Skriv sektioner her. Hvis du ønsker lidt mellemrum for at give luft i et langt afsnit el.lign., brug da \\\medskip
  \end{SBSection*}
\end{song}

% Festen er slut
% For os som passer på jer nu
% Vi sidder her med vores film
% Du må sove godt
% Sidder alen’
% Tænk på festen vi har haft
% Ser I kommer meget vakst
% Og går til ro

% Vi er brandvag-(t)er
% Ja for vi er
% i U-N-F
% UNF

% Vi holder øje
% Og lytter efter brandalarm
% Hjælper jer-hvis den går igang
% Men nu er den tam

% Vi er brandvag-(t)er
% Ja for vi er
% i U-N-F
% UNF
% Vi er brandvag-(t)er
% Ja for vi er
% i U-N-F
% UNF

% Klokken slår to
% Nu er I alle gået til ro
% Og vi sidder kun os to
% Og’ vi kigger på en sko

% Vi er brandvag-(t)er
% Ja for vi er
% i U-N-F
% UNF
% Vi er brandvag-(t)er Rebecca
% For I skal ikk’ Maria
% Oh Rebecca
% Slå’s helt ihjel
% Af en brand, af en brand
% I UN-F Rebecca
% Af en brand (maria)


\onecolumn
\chapter*{Studieliv og \LaTeX}
\twocolumn
\begin{song}{Tål daj' i læseferien}
  {} % Bruges ikke, lad stå blank
  {12 Days of Christmas} % Titel, Kunstner - eks.: "Jutlandia, Kim Larsen". Hvis sangen er på sin egen melodi, brug da \SBOrgMel.
  {Mona Holm \& Marianne Bangsø} % Navnet på forfatteren. Undlad kaldenavne. Brug gerne TBF. Brug "&" frem for "og". Hvis forfatter er ukendt, lad da stå tom.
  {TÅGEKAMMERETs Julerevy, 1987} % Eks. "Fysikrevy, 2010" eller "2010"
  {\NotCCLIed} % Lad stå som den er

  \begin{SBVerse}
    På den første dag i læseferien sagde jeg til mig selv:\\
    Der er god tid, jeg pjækker i dag.
  \end{SBVerse}

  \begin{SBVerse}
    På den anden dag i læseferien sagde jeg til mig selv:\\
    Mat’matik er trist,\\
    der er god tid, jeg pjækker i dag.
  \end{SBVerse}

  \begin{SBVerse}
    På den tredje dag i læseferien sagde jeg til mig selv:\\
    Jeg er træt,\\
    mat’matik er trist,\\
    der er god tid, jeg pjækker i dag.
  \end{SBVerse}

  \begin{SBVerse}
    På den fjerde dag i læseferien sagde jeg til mig selv:\\
    Tror jeg går på druk -- \emph{Skål!}\\
    Jeg er træt,\\
    mat’matik er trist,\\
    der er god tid, jeg pjækker i dag.
  \end{SBVerse}

  \begin{SBVerse}
    På den femte dag i læseferien sagde jeg til mig selv:\\
    Åh, tømmermænd!\\
    Tror jeg går på druk -- \emph{Skål!}\\
    Jeg er træt,\\
    mat’matik er trist,\\
    der er god tid, jeg pjækker i dag.
  \end{SBVerse}

  \begin{SBVerse}
    På den sjette dag i læseferien sagde jeg til mig selv:\\
    Der er fest i aften,\\
    åh, tømmermænd!\\
    Tror jeg går på druk -- \emph{Skål!}\\
    Jeg er træt,\\
    mat’matik er trist,\\
    der er god tid, jeg pjækker i dag.
  \end{SBVerse}

  \begin{SBVerse}
    På den syvende dag i læseferien sagde jeg til mig selv:\\
    Hvor er min bog nu?\\
    Der er fest i aften,\\
    åh, tømmermænd!\\
    Tror jeg går på druk -- \emph{Skål!}\\
    Jeg er træt,\\
    mat’matik er trist,\\
    der er god tid, jeg pjækker i dag.
  \end{SBVerse}

  \begin{SBVerse}
    På den ottende dag i læseferien sagde jeg til mig selv:\\
    Der' film i TV,\\
    hvor er min bog nu?\\
    Der er fest i aften,\\
    åh, tømmermænd!\\
    Tror jeg går på druk -- \emph{Skål!}\\
    Jeg er træt,\\
    mat’matik er trist,\\
    der er god tid, jeg pjækker i dag.
  \end{SBVerse}

  \begin{SBVerse}
    På den niende dag i læseferien sagde jeg til mig selv:\\
    Savner mine venner,\\
    der' film i TV,\\
    hvor er min bog nu?\\
    Der er fest i aften,\\
    åh, tømmermænd!\\
    Tror jeg går på druk -- \emph{Skål!}\\
    Jeg er træt,\\
    mat’matik er trist,\\
    der er god tid, jeg pjækker i dag.
  \end{SBVerse}

  \begin{SBVerse}
    På den tiende dag i læseferien sagde jeg til mig selv:\\
    Mon jeg kan nå det?\\
    Savner mine venner,\\
    der' film i TV,\\
    hvor er min bog nu?\\
    Der er fest i aften,\\
    åh, tømmermænd!\\
    Tror jeg går på druk -- \emph{Skål!}\\
    Jeg er træt,\\
    mat’matik er trist,\\
    der er god tid, jeg pjækker i dag.
  \end{SBVerse}

  \begin{SBVerse}
    På den elvete dag i læseferien sagde jeg til mig selv:\\
    Nu skal der læses!\\
    Mon jeg kan nå det?\\
    Savner mine venner,\\
    der' film i TV,\\
    hvor er min bog nu?\\
    Der er fest i aften,\\
    åh, tømmermænd!\\
    Tror jeg går på druk -- \emph{Skål!}\\
    Jeg er træt,\\
    mat’matik er trist,\\
    der er god tid, jeg pjækker i dag.
  \end{SBVerse}

  \begin{SBVerse}
    På den tolvte dag i læseferien sagde jeg til mig selv:\\
    Gud, det' i morgen!\\
    Nu skal der læses!\\
    Mon jeg kan nå det?\\
    Savner mine venner,\\
    der' film i TV,\\
    hvor er min bog nu?\\
    Der er fest i aften,\\
    åh, tømmermænd!\\
    Tror jeg går på druk -- \emph{Skål!}\\
    Jeg er træt,\\
    mat’matik er trist,\\\medskip
    Så'n gik tiden -- jeg dumped' i dag!
  \end{SBVerse}

  \begin{SBSection*}
    % Skriv sektioner her. Hvis du ønsker lidt mellemrum for at give luft i et langt afsnit el.lign., brug da \\\medskip
  \end{SBSection*}
\end{song}
\begin{song}{Homo-sangen}
  {} % Bruges ikke, lad stå blank
  {\SBOrgMel} % Titel, Kunstner - eks.: "Jutlandia, Kim Larsen". Hvis sangen er på sin egen melodi, brug da \SBOrgMel.
  {Jan Midtgaard} % Navnet på forfatteren. Undlad aliasser. Brug "&" frem for "og". Hvis forfatter er ukendt, lad da stå tom.
  {TÅGEKAMMERETs Julerevy, 1999} % Eks. "Fysikrevy 2010" eller "2010"
  {\NotCCLIed} % Lad stå som den er

  \begin{SBVerse}
    Livets mange glæder man jo dele kan,\\
    men det er nu bedt, så'n mand til mand'\\
    Intet er skam større end den ægte kærlighed,\\
    spring ud af skabet, tag din læsemakker med!
  \end{SBVerse}

  \begin{SBChorus}
    Får du lyst til rigtig mand igen,\\
    er det for lidt med kun en pige-ven.\\
    Du får et helt nyt syn på lineær algebra \emph{(shi-bu-du-ah)}\\
    når du gi’r din læsemakker den bagfra!
  \end{SBChorus}

  \begin{SBVerse}
    Når du går og voldtag’r en and ved unisø’n\\
    ved du den er gal med den der kløen.\\
    Til TÅGEKAMMER-fester du scorer ikke spor,\\
    den sidste pige som du kyssed’ var din mor!
  \end{SBVerse}

  \begin{SBChorus}
    Får du lyst til rigtig mand igen,\ldots
  \end{SBChorus}

  \begin{SBSection*}
  Her på matematisk der er pigerne få,\\
  her går sexlivet bestemt i stå.
  \end{SBSection*}

  \begin{SBChorus}
    Du får et helt nyt syn på lineær algebra \emph{(shi-bu-du-ah)}\\
    når du gi’r din læsemakker,\\
    når du gi'r din forelæser,\\
    når du gi'r din koordinator den bagfra!
  \end{SBChorus}
\end{song}
\begin{song}{Jeg vil læse .*}
  {} % Bruges ikke, lad stå blank
  {Regnvejrsdag i november, Pia Raug} % Titel, Kunstner - eks.: "Jutlandia, Kim Larsen". Hvis sangen er på sin egen melodi, brug da \SBOrgMel.
  {} % Navnet på forfatteren. Undlad kaldenavne. Brug gerne TBF. Brug "&" frem for "og". Hvis forfatter er ukendt, lad da stå tom.
  {DIKUrevy, 2008} % Eks. "Fysikrevy, 2010" eller "2010"
  {\NotCCLIed} % Lad stå som den er

  \begin{SBVerse}
    Jeg vil læse retorik,\\
    for jeg fatter ik' en brik.\\
    Jeg vil lær' om sprogfinesser\\
    og gå op i petitesser,\\
    jeg bli'r nok en stor professor.\\
    Jeg vil læse retorik.
  \end{SBVerse}

  \begin{SBVerse}
    Jeg vil være teolog,\\
    selvom lønnen ik' er god.\\
    Tro på gud det er det sande,\\
    også når man træder vande,\\
    missionere i fjerne lande.\\
    Jeg vil være teolog.
  \end{SBVerse}

  \begin{SBVerse}
    Jeg vil læse medicin,\\
    tjene penge som et svin.\\
    Jeg vil skær i døde men'sker,\\
    men jeg er jo også svensker,\\
    klæde mig i latexhandsker.\\
    Jeg vil læse medicin.
  \end{SBVerse}

  \begin{SBVerse}
    Jeg vil læse på kemi,\\
    hårde stoffer kan jeg li'.\\
    Jeg er skæv fra ni til sytten,\\
    finsprit hælder jeg i bøtten,\\
    så jeg ender nok på støtten.\\
    Jeg vil læse på kemi.
  \end{SBVerse}

  \begin{SBVerse}
    Jeg vil læse cand.polit.,\\
    for mit men'skesyn er skidt.\\
    Jeg vil skær' på hospitaler,\\
    og på skoler uden kvaler,\\
    mens jeg holder falske taler.\\
    Jeg vil læse cand.polit.
  \end{SBVerse}

  \begin{SBVerse}
    Jeg vil læse på fysik,\\
    lære kvantemekanik.\\
    Jeg går op i Einsteins tanker,\\
    og jeg er en rigtig dranker,\\
    kæler for mit fadølsanker.\\
    Jeg vil læse på fysik.
  \end{SBVerse}

  \begin{SBVerse}
    Jeg vil være officér\\
    i den danske dronnings hær.\\
    Jeg vil skyde talibaner',\\
    vifte med de danske faner,\\
    sprede død i lange baner.\\
    Jeg vil være officér
  \end{SBVerse}

  \begin{SBVerse}
    Historie er mit liv,\\
    støver rundt i et arkiv.\\
    Læser i de gamle skrifter,\\
    selvom jeg får mange rifter,\\
    jeg ved alt om Neros drifter.\\
    For historie er mit liv.
  \end{SBVerse}

  \begin{SBVerse}
    Jeg vil være pædagog,\\
    rende rundt i maosko.\\
    Hjælpe børn i overtøjet,\\
    så de ikke bli'r tilmøjet,\\
    du kan tro de er fornøjet.\\
    Jeg vil være pædagog.
  \end{SBVerse}

  \begin{SBVerse}
    Jeg vil være aktuar,\\
    hvor min skæbne den er klar.\\
    Forsikringsvidenskab er sagen,\\
    jeg vil ta' en bid af kagen,\\
    mens jeg sylter kundeklagen.\\
    Jeg vil være aktuar.
  \end{SBVerse}

  \begin{SBVerse}
    Jeg vil muge ud hos køer,\\
    køre rundt med trillebør.\\
    Sidde i en kæmpe traktor,\\
    moms det er en ukendt faktor,\\
    nu skal grisen ned til slagter.\\
    Jeg vil muge ud hos køer.
  \end{SBVerse}

  \begin{SBVerse}
    Jeg vil læse på MI,\\
    for musik kan alle li'.\\
    Selvom jeg var sidst i klassen\\
    kan jeg stadig score kassen,\\
    blot jeg lær' at spille bassen.\\
    Men jeg kan ikke holde takten.
  \end{SBVerse}

  \begin{SBVerse}
    Jeg vil læse æstetik\\
    i dansenes mystik.\\
    Svanens død den kan jeg lide,\\
    skønt jeg ej er en sylfide,\\
    vil jeg over scenen glide.\\
    Jeg vil læse æstetik.
  \end{SBVerse}

  \begin{SBVerse}
    Jeg vil læse nano tech,\\
    for min hjerne drak jeg væk\\
    sidte fredag på Caféen?,\\
    da jeg rigtig sku' gå te'en,\\
    og jeg vælted om i sneen.\\
    Jeg vil læse nano tech.
  \end{SBVerse}

  \begin{SBVerse}
    Jeg vil være datalog,\\
    og min fremtid den er god.\\
    gcc det er en gave,\\
    jeg blir nok en kodeslave,\\
    jeg vil gerne prøv' at snave.\\
    Jeg vil være datalog.
  \end{SBVerse}
\end{song}
\begin{song}{De tog stadig folk ind}
  {} % Bruges ikke, lad stå blank
  {Melodi} % Titel, Kunstner - eks.: "Jutlandia, Kim Larsen". Hvis sangen er på sin egen melodi, brug da \SBOrgMel.
  {Forfatter} % Navnet på forfatteren. Undlad kaldenavne. Brug gerne TBF. Brug "&" frem for "og". Hvis forfatter er ukendt, lad da stå tom.
  {Anledning og år} % Eks. "Fysikrevy, 2010" eller "2010"
  {\NotCCLIed} % Lad stå som den er

  \begin{SBVerse}
    % Skriv vers her
  \end{SBVerse}

  \begin{SBChorus}
    % Skriv omkvæd her
  \end{SBChorus}

  \begin{SBSection*}
    % Skriv sektioner her. Hvis du ønsker lidt mellemrum for at give luft i et langt afsnit el.lign., brug da \\\medskip
  \end{SBSection*}
\end{song}
\begin{song}{Eksamensangst}
  {} % Bruges ikke, lad stå blank
  {Min store kærlighed, Anders Matthesen} % Titel, Kunstner - eks.: "Jutlandia, Kim Larsen". Hvis sangen er på sin egen melodi, brug da \SBOrgMel.
  {} % Navnet på forfatteren. Undlad kaldenavne. Brug gerne TBF. Brug "&" frem for "og". Hvis forfatter er ukendt, lad da stå tom.
  {FysikRevy, 2014} % Eks. "Fysikrevy, 2010" eller "2010"
  {\NotCCLIed} % Lad stå som den er

  \begin{SBVerse}
    Uret tikker, hjertet det slår,\\
    jeg mærker pulsen inde i mit øre.\\
    Jeg får det værre som tiden går,\\
    men jeg har lovet at jeg nok forsøger.\\
    Jeg har dumpet alle mine fag.\\
    Jeg håber i dag er min lykkedag.
  \end{SBVerse}

  \begin{SBVerse}
    Solen falder blidt på min kind,\\
    og jeg mærker trykket i min blære.\\
    Jeg kan ikke nå at lette det nu;\\
    nu må det briste eller bære.\\
    Døren knirker, jeg gi'r et lille hvin.\\
    Jeg håber snart at turen er min.
  \end{SBVerse}

  \begin{SBChorus}
    Bare jeg nu kan få det sagt,\\
    det jeg har øvet og planlagt,\\
    det jeg har terpet ud og ind,\\
    eller går klappen ned igen?
  \end{SBChorus}

  \begin{SBVerse}
    Hånden ryster, halsen er tør.\\
    Jeg sveder kraftigt under mine arme.\\
    Jeg stirrer tomt lige ind i på den dør\\
    der står imellem mig og helvedets varme.\\
    De andre gør det gang på gang på gang,\\
    for mig er det stadig ulidelig tvang.
  \end{SBVerse}

  \begin{SBChorus}
    Bare jeg nu kan få det sagt,\\
    det jeg har øvet og planlagt,\\
    det jeg har terpet ud og ind,\\
    eller går klappen ned igen?
  \end{SBChorus}

  % \begin{SBChorus}
  %   Bare jeg nu kan få det sagt,\ldots
  % \end{SBChorus}

  \begin{SBSection*}
    Bare jeg ku forklare,\\
    bare censor ku' prøv' at forstå,\\
    at mundtlig eksamen ik' er sag'n\\
    når man bli'r så nervøs at man går i stå.
  \end{SBSection*}

  \begin{SBChorus}
    Bare jeg nu kan få det sagt,\\
    det jeg har øvet og planlagt,\\
    det jeg har terpet ud og ind,\\
    eller går klappen ned igen?
  \end{SBChorus}

  \begin{SBChorus}
    Bare jeg nu kan få det sagt,\\
    det jeg har øvet og planlagt,\\
    det jeg har terpet ud og ind,\\
    eller går klappen ned igen?
  \end{SBChorus}

  % \begin{SBChorus}
  %   Bare jeg nu kan få det sagt,\ldots
  % \end{SBChorus}

  % \begin{SBChorus}
  %   Bare jeg nu kan få det sagt,\ldots
  % \end{SBChorus}
\end{song}
\begin{song}{Fuck mit liv}
  {} % Bruges ikke, lad stå blank
  {Melodi} % Titel, Kunstner - eks.: "Jutlandia, Kim Larsen". Hvis sangen er på sin egen melodi, brug da \SBOrgMel.
  {Forfatter} % Navnet på forfatteren. Undlad kaldenavne. Brug gerne TBF. Brug "&" frem for "og". Hvis forfatter er ukendt, lad da stå tom.
  {Anledning og år} % Eks. "Fysikrevy, 2010" eller "2010"
  {\NotCCLIed} % Lad stå som den er

  \begin{SBVerse}
    % Skriv vers her
  \end{SBVerse}

  \begin{SBChorus}
    % Skriv omkvæd her
  \end{SBChorus}

  \begin{SBSection*}
    % Skriv sektioner her. Hvis du ønsker lidt mellemrum for at give luft i et langt afsnit el.lign., brug da \\\medskip
  \end{SBSection*}
\end{song}
\begin{song}{Hvad skal vi \TeX'e i nat?}
  {} % Bruges ikke, lad stå blank
  {Hvor skal vi sove i nat?, Laban} % Titel, Kunstner - eks.: "Jutlandia, Kim Larsen". Hvis sangen er på sin egen melodi, brug da \SBOrgMel.
  {} % Navnet på forfatteren. Undlad kaldenavne. Brug gerne TBF. Brug "&" frem for "og". Hvis forfatter er ukendt, lad da stå tom.
  {DIKUrevy, 2007} % Eks. "Fysikrevy, 2010" eller "2010"
  {\NotCCLIed} % Lad stå som den er

  \begin{SBVerse}
    Vi fik rapporten, og tænkte straks det samme;\\
    går den, så går den, og drak os bankelamme.\\
    Vi' alt for kloge, vi' nogle seje gutter.\\
    Selv i en tåge, ta'r den kun fem minutter.
  \end{SBVerse}

  \begin{SBVerse}
    Så fandt vi siden, hvor alting blev beskrevet.\\
    Vi mangled' viden, for at få alt det lavet.\\
    Vi hader livet under rapportregimer.\\
    Vi er fortvivlet, vi har kun fjorten timer.
  \end{SBVerse}

  \begin{SBChorus}
    Hvad skal vi \TeX'e i nat? Vi har jo intet lavet.\\
    Hvad skal vi \TeX'e i nat? Vi er slet ikke færdig'.\\
    Vi er jo fuldstændig sat. Vi emmed' af arrogancen,\\
    og mistede chancen for at gøre det godt.
  \end{SBChorus}

  \begin{SBChorus}
    Hvad skal vi \TeX'e i nat? Vi ku' jo lave genbrug.\\
    Hvad skal vi \TeX'e i nat? Så printer jeg den ud nu.\\
    Vi ku' jo drikke en sjat. Vi ta'r en overspringshandling,\\
    det gi'r en forvandling, så vi tror det er flot.
  \end{SBChorus}

  \begin{SBVerse}
    Vi lærte lektien, vi læser alle sider.\\
    Så nu er vi mænd der ikke går og lider.\\
    Vi starter tidligt og knokler meget længe.\\
    Vi får lidt slidgigt -- ser sjældent vore senge.
  \end{SBVerse}

  \begin{SBVerse}
    Men hov hvad er det, er det det gale pensum?\\
    Ja, så forstår jeg der var så tavst på forum.\\
    Vi starter forfra, vi føler os som tåber,\\
    og i kantinen er det kun os der råber.
  \end{SBVerse}

  \begin{SBChorus}
    Hvad skal vi \TeX'e i nat? Vi har jo intet lavet.\\
    Hvad skal vi \TeX'e i nat? Vi er slet ikke færdig'.\\
    Vi er jo fuldstændig sat. Vi emmed' af arrogancen,\\
    og mistede chancen for at gøre det godt.
  \end{SBChorus}

  \begin{SBChorus}
    Hvad skal vi \TeX'e i nat? Vi ku' jo lave genbrug.\\
    Hvad skal vi \TeX'e i nat? Så printer jeg den ud nu.\\
    Vi ku' jo drikke en sjat. Vi ta'r en overspringshandling,\\
    det gi'r en forvandling, så vi tror det er flot.
  \end{SBChorus}
\end{song}
\begin{song}{Brug \LaTeX}
  {} % Bruges ikke, lad stå blank
  {Spørg om hjælp, Anders Matthesen}
  {Morten Jensen}
  {TÅGEKAMMERET, 2015}
  {\NotCCLIed}

  \begin{SBChorus}
    Brug \LaTeX, når du skriver ind.\\
    For \TeX{} er smart, og \TeX{} er smukt, det luner i mit sind.\\
    Du må ikke bruge Word, for det er noget bøvl,\\
    og hvis du gør alligevel, så får du satme høvl!
  \end{SBChorus}

  \begin{SBVerse}
    Jeg fik engang en aflevering fra en sød student.\\
    Han havde potentiale til at blive verdenskendt;\\
    men han afleverede et dokument i Word,\\
    og det betød, hans bachelor den ej blev gennemført.
  \end{SBVerse}

  \begin{SBVerse}
    Og Lone, hun var også ganske dygtig, da hun kom.\\
    Hun var klog, og hun var sød, dog sku’ man brug’ kondom,\\
    og hendes løse læber så jeg ikk’ som et problem;\\
    men da hun tegned’ cosinus i Paint, fik jeg eksem!
  \end{SBVerse}

  \begin{SBChorus}
    Med \LaTeX, alt er som en leg;\\
    men bru’r du Maple, Word og Paint, så vil jeg straffe dig.\\
    Personalet her på stedet ved godt, hvad de skal,\\
    når nogen ikke bru’r \LaTeX, så gi’r de mig et kald.
  \end{SBChorus}

  \begin{SBVerse}
    Jørgen, han var også fin, trods han var humanist.\\
    Han vidste alt om didaktik, ja det var ganske vist;\\
    men hvis man bruger PowerPoint til slides, bli’r man til grin.\\
    De fandt ham snart med tasken fuld af ryge-heroin.
  \end{SBVerse}

  \begin{SBChorus}
    Brug \LaTeX. Lad nu vær’ at piv’,\\
    og kan du ikk’, så få det lært, for det kan red’ dit liv.\\
    Og ellers ku’ det være, at du fik besøg af mig\\
    Så ender du måske i grøften på din livets vej.
  \end{SBChorus}
\end{song}
\begin{song}{\TeX'e-sangen}
  {} % Bruges ikke, lad stå blank
  {Melodi} % Titel, Kunstner - eks.: "Jutlandia, Kim Larsen". Hvis sangen er på sin egen melodi, brug da \SBOrgMel.
  {Forfatter} % Navnet på forfatteren. Undlad kaldenavne. Brug gerne TBF. Brug "&" frem for "og". Hvis forfatter er ukendt, lad da stå tom.
  {Anledning og år} % Eks. "Fysikrevy, 2010" eller "2010"
  {\NotCCLIed} % Lad stå som den er

  \begin{SBVerse}
    % Skriv vers her
  \end{SBVerse}

  \begin{SBChorus}
    % Skriv omkvæd her
  \end{SBChorus}

  \begin{SBSection*}
    % Skriv sektioner her. Hvis du ønsker lidt mellemrum for at give luft i et langt afsnit el.lign., brug da \\\medskip
  \end{SBSection*}
\end{song}

\onecolumn
\chapter*{Fysik}
\twocolumn
\begin{song}{Våbenfysik}{}
  {Jutlandia, Kim Larsen}
  {}
  {FysikRevy 2003}
  {\NotCCLIed}

  \begin{SBVerse}
	Det var i 1945 og nu ville man ha’ fred,\\
	men der var krig i Japan.\\
	Der blev samlet uran nok og det blev kastet ned,\\
	for der var krig i Japan.\\
	De fik at se hvad fysikken formår,\\
	og sjove børn de næste mange år.
  \end{SBVerse}

  \begin{SBChorus}
	Hey-Ho for våbenfysik!\\
	Vi blæser på alle traktater.\\
	Bomber her, bomber der, bomber for fred!\\
	Hvad skal man med diplomater?
  \end{SBChorus}

  \begin{SBVerse}
	Vi flyver gennem natten og gi’r din fjende smæk.\\
	Han ser os ikke komme.\\
	Vi stiller ingen spørgsmål, og så snart vi har din check,\\
	så er krigen omme.\\
	Hvis du gi’r, fyrer vi den sgu af.\\
	Det er det, de unge vil ha’.
  \end{SBVerse}

  \begin{SBChorus}
	Hey-Ho for våbenfysik!\\
	Vi blæser på alle traktater.\\
	Bomber her, bomber der, bomber for fred!\\
	Hvad skal man med diplomater?
  \end{SBChorus}

  % \begin{SBChorus}
  % 	Hey-Ho for våbenfysik\ldots
  % \end{SBChorus}

  \begin{SBVerse}
	Vores salgskontor har åbent hele døgnet – bare ring;\\
	du skal ikke tøve.\\
	Vacuum, brint og EMP og andre sjove ting,\\
	dem vil vi gerne prøve.\\
	Her er ugens tilbudskatalog:\\
	Start tre krige og betal for to!
  \end{SBVerse}

  \begin{SBChorus}
	Hey-Ho for våbenfysik!\\
	Vi blæser på alle traktater.\\
	Bomber her, bomber der, bomber for fred!\\
	Hvad skal man med diplomater?
  \end{SBChorus}

  % \begin{SBChorus}
  % 	Hey-Ho for våbenfysik\ldots
  % \end{SBChorus}
  
  \begin{SBChorus}
	Hey-Ho for våbenfysik!\\
	Vi blæser på alle traktater.\\
	Bomber her, bomber der, bomber for fred!\\
	Vi nakker de slyngelstater!
  \end{SBChorus}

\end{song}
\begin{song}{Fasebal i Kvanteland}
  {} % Bruges ikke, lad stå blank
  {Julebal i Nisseland, Far til fire i byen} % Titel, Kunstner - eks.: "Jutlandia, Kim Larsen"
  {} % Navnet på forfatteren. Undlad aliasser. Brug "&" frem for "og".
  {FysikRevy 2004} % Eks. "Fysikrevy 2010" eller "2010"
  {\NotCCLIed} % Lad stå som den er

  \begin{SBVerse}
    Sikk' en dejlig energi,\\
    Jeg kan li' entropi\\
    Cæsium atomet står\\
    For at tiden går
  \end{SBVerse}

  \begin{SBChorus}
    Til fasebal,\\
    \hspace{1em}til fasebal i Kvanteland\\
    Elektroner\\
    \hspace{1em}flyver frit omkring\\
    Skru' feltet op\\
    \hspace{1em}og styr dem som en kvantemand\\
    Se nu bare,\\
    \hspace{1em}de flyver rundt i ring
  \end{SBChorus}

  \begin{SBSection*}
    Atomer absorberer\\
    \hspace{1em}det lys, vi sender ind\\
    Et spejl reflekterer:\\
    \hspace{1em}Pas på, du ik' bli'r blind!\\
    Til fasebal,\\
    \hspace{1em}til fase-fase-fase-fase-fasebal\\
    Bose-Einstein\\
    \hspace{1em}er lysets karneval
  \end{SBSection*}

  \begin{SBVerse}
    Her er koldt uhadada\\
    Vakuum ska' vi ha'\\
    Det kan ikke være svært\\
    Brug det, du har lært
  \end{SBVerse}

  \begin{SBChorus}
    Til fasebal,\\
    \hspace{1em}til fasebal i Kvanteland\\
    Stoffet flyder\\
    \hspace{1em}i superfludium\\
    Og hver boson,\\
    \hspace{1em}den gør jo som sin nabomand,\\
    De står sammen\\
    \hspace{1em}og rummet bliver til skum\\
  \end{SBChorus}

  \begin{SBSection*}
    Atomer absorberer\\
    \hspace{1em}det lys, vi sender ind\\
    Et spejl reflekterer:\\
    \hspace{1em}Pas på, du ik' bli'r blind!\\
    Til fasebal,\\
    \hspace{1em}til fase-fase-fase-fase-fasebal\\
    Bose-Einstein\\
    \hspace{1em}er lysets karneval
  \end{SBSection*}

  \begin{SBSection*}
    Bose-Einstein\\
    \hspace{1em}er lysets karneval\\
    Bose-Einstein\\
    \hspace{1em}er lysets karneval\\
    Bose-Einstein\\
    \hspace{1em}er lysets karneval
  \end{SBSection*}
\end{song}
\begin{song}{ASTRID}
  {} % Bruges ikke, lad stå blank
  {What Makes You Beautiful, One Direction} % Titel, Kunstner - eks.: "Jutlandia, Kim Larsen". Hvis sangen er på sin egen melodi, brug da \SBOrgMel.
  {Sabrina Tang} % Navnet på forfatteren. Undlad kaldenavne. Brug gerne TBF. Brug "&" frem for "og". Hvis forfatter er ukendt, lad da stå tom.
  {TÅGEKAMMERET, 2013} % Eks. "Fysikrevy, 2010" eller "2010"
  {\NotCCLIed} % Lad stå som den er

  \begin{SBVerse}
    Der er en pig’, som jeg kan li’.\\
    Jeg går i stå, hver gang jeg går forbi\\
    lokalet hvor jeg ved, hun står;\\
    min drømmepige på 27 år.\\\medskip
    Alle fysikfyre kender hende,\\
    Alle ved, hvem hun er.
  \end{SBVerse}

  \begin{SBChorus}
    Hun underviser i atomar kollision,\\
    ved alt om ion-beams, synkrotonradiation\\
    og spektroskopiforsøg med en elektron.\\
    Jeg bli’r hø-ø-øj, høj på energifysik.\\\medskip
    Er på en sær måde underskøn,\\
    og hun formår at ku’ vær’ både rund og køn.\\
    Og hele bygningen giver et kæmpe drøn\\
    når hun gå-å-år i gang med kvantemekanik.\\
    Åh-å-åh, ASTRID, hun er så unik.
  \end{SBChorus}

  \begin{SBVerse}
    Men det kan ske, at FFB\\
    de får serveret for meg’t LFP.\\
    Et øludbrud – hun står for skud.\\
    Hvis hun bli’r ramt flipper alle folk ud.\\\medskip
    Hun kører løs lig’som en maskine,\\
    Kan holde natten lang.
  \end{SBVerse}

  \begin{SBChorus}
    Hvis bare jeg ku’ Proton-Enhanced-Nuklear\\
    Induktion-Spektroskopere dig, var jeg klar!\\
    Har noget Large Hadron Collider ikke har.\\
    I kan nok forstå; ASTRID er min drømmetøs.\\\medskip
    Din kærlighed er så adækvat.\\
    Og selv om jeg sku’ gå hen og bliv’ kandidat,\\
    så vil du altid forblive min egen skat.\\
    Du er så-å-å pisse-sexet og famøs,\\
    Åh-å-åh, ASTRID gør mig så nervøs.
  \end{SBChorus}

  \begin{SBSection*}
    ASTRID, når vi to er sam’n er jeg så tilpas,\\
    og ASTRID2, hun vil aldrig ku’ ta’ din plads.\\
    Stod det til mig, gav vi hende til BIOGAS.\\
    Fordi åh-åh-åh ASTRID er min ring af guld!
  \end{SBSection*}

  \begin{SBChorus}
    Befinder mig i en excitations-tilstand,\\
    der’ ingen and’n synkrotonstrålingslagerring\\
    der øger min entropi lig’som ASTRID kan.\\
    Fordi åh-å-åh hun kan permutere mig.\\\medskip
    I ISAs kælder ka’ vi ta’ ned,\\
    og du må godt ta’ din søster ELISA med,\\
    hvis hun da ellers ka’ hold’ på en hem’lighed\\
    Fordi åh-å-åh, min verden graviterer\\
    nå-å-år, min verden oscillerer\\
    nå-å-år ASTRID kollidér’ med mig!
  \end{SBChorus}
\end{song}
\begin{song}{Jeg er fysiker}
  {} % Bruges ikke, lad stå blank
  {The Lumberjack Song, Monty Python} % Titel, Kunstner - eks.: "Jutlandia, Kim Larsen". Hvis sangen er på sin egen melodi, brug da \SBOrgMel.
  {} % Navnet på forfatteren. Undlad kaldenavne. Brug gerne TBF. Brug "&" frem for "og". Hvis forfatter er ukendt, lad da stå tom.
  {Matematikrevyen, 2007} % Eks. "Fysikrevy, 2010" eller "2010"
  {\NotCCLIed} % Lad stå som den er

  \begin{SBChorus}
    Jeg er fysiker og jeg er glad.\\
    Jeg kan min Schaums bog uden ad!\\\medskip
    \emph{Han er fysiker og han er glad.\\
    Han kan sin Schaums bog uden ad!}
  \end{SBChorus}

  \begin{SBVerse}
    Ampere her, promille der,\\
    jeg gør et Maple-plot.\\
    Jeg plotter i den farve,\\
    min kone si’r er flot!\\\medskip
    \emph{Ampere her, promille der,\\
    han gør et Maple-plot.\\
    Han plotter i den farve,\\
    hans kone si’r er flot!}
  \end{SBVerse}

  \begin{SBChorus}
    \emph{Han er fysiker og han er glad.\\
    Han kan sin Schaums bog uden ad!}
  \end{SBChorus}

  \begin{SBVerse}
    Eksperimentér, analyser,\\
    og spis en masse slik!\\
    Jeg læser en artikel,\\
    og leger med min rubikskube\\\medskip
    \emph{Eksperimentér, analyser,\\
    og spis en masse slik!\\
    Han læser en artikel,\\
    og leger med sin rubikskube}
  \end{SBVerse}

  \begin{SBChorus}
    Jeg er fysiker, og jeg er fin.\\
    Jeg kan formlen for min urin!\\
    \emph{Han er fysiker og han er fin.\\
    Han kan formlen for sin urin!}
  \end{SBChorus}

  \begin{SBVerse}
    Jeg rækkeudvikler på alt\\
    fordi det har jeg lært.\\
    Hvis blot jeg læste mat'matik\\
    som ham der Erik Kjær!\\\medskip
    \emph{Han rækkeudvikler på alt\\
    fordi det har han lært.\\
    Hvis blot han læste mat'matik\\
    som ham der Erik Kjær!}
  \end{SBVerse}
\end{song}
\begin{song}{$\chi$-fitter}
  {} % Bruges ikke, lad stå blank
  {Kickflipper, Razz} % Titel, Kunstner - eks.: "Jutlandia, Kim Larsen". Hvis sangen er på sin egen melodi, brug da \SBOrgMel.
  {} % Navnet på forfatteren. Undlad kaldenavne. Brug gerne TBF. Brug "&" frem for "og". Hvis forfatter er ukendt, lad da stå tom.
  {FysikRevy, 2011} % Eks. "Fysikrevy, 2010" eller "2010"
  {\NotCCLIed} % Lad stå som den er

  \begin{SBChorus}
    $\chi$fitter vildt \emph{(Yeah!)}\\
    $\chi$fitter snildt \emph{(Yeah!)}\\
    $\chi$fitter højt \emph{(Øv!)}\\
    Ikke for at blære, men vi $\chi$fitter yeah!
  \end{SBChorus}

  \begin{SBVerse}
    Mig og alle gutterne sætter attributterne\\
    På fittet som I ser, de bedste startværdier\\
    Du bliver altid mobbet, hvis dataen er flobbet\\
    Vil fittet konvergere, når løkken itererer\\\medskip
    Usikkerheden stiger\\
    på mine fitværdier\\
    Mit $\chi$-kvadrat er stort,\\
    det' noget lort \emph{(Det' noget lort!)}
  \end{SBVerse}

  \begin{SBChorus}
    $\chi$fitter vildt \emph{(Yeah!)}\\
    $\chi$fitter snildt \emph{(Yeah!)}\\
    $\chi$fitter højt \emph{(Øv!)}\\
    Ikke for at blære, men vi...\\\medskip
    $\chi$fitter vildt \emph{(Yeah!)}\\
    $\chi$fitter snildt \emph{(Yeah!)}\\
    $\chi$fitter højt \emph{(Øv!)}\\
    Ikke for at blære, men vi $\chi$fitter yeah!
  \end{SBChorus}

  \begin{SBVerse}
    P-værdi fortæller om vores fitmodeller\\
    Beskriver data godt, om plottet det bliver flot\\
    $\chi$-kvadratet minimer', rutinen sørger for det sker\\
    Syns' du jeg er dum, tjek mit parameterrum\\\medskip
    Usikkerheden stiger\\
    på mine fitværdier\\
    Mit $\chi$-kvadrat er stort,\\
    det' noget lort \emph{(Det' noget lort!)}\\\medskip
    Usikkerheden stiger\\
    på mine fitværdier\\
    Mit $\chi$-kvadrat er stort,\\
    ligesom din mor \emph{(Større end din mor!)}
  \end{SBVerse}

  \begin{SBChorus}
    $\chi$fitter vildt\ldots
  \end{SBChorus}

  \begin{SBSection*}
    Åh, jeg er bare så træt af at fitte.
  \end{SBSection*}

  \begin{SBChorus}
    $\chi$fitter vildt \emph{(Yeah!)}\\
    $\chi$fitter snildt \emph{(Yeah!)}\\
    $\chi$fitter højt \emph{(Øv!)}\\
    Ikke for at blære, men vi $\chi$fitter yeah!
  \end{SBChorus}

  \begin{SBChorus}
    $\chi$fitter vildt \emph{(Yeah!)}\\
    $\chi$fitter snildt \emph{(Yeah!)}\\
    $\chi$fitter højt \emph{(Øv!)}\\
    Ikke for at blære, men vi $\chi$fitter yeah!
  \end{SBChorus}

  % \begin{SBChorus}
  %   $\chi$fitter vildt\ldots
  % \end{SBChorus}

  % \begin{SBChorus}
  %   $\chi$fitter vildt\ldots
  % \end{SBChorus}
\end{song}
\begin{song}{Steve Hawking}
  {} % Bruges ikke, lad stå blank
  {Melodi} % Titel, Kunstner - eks.: "Jutlandia, Kim Larsen". Hvis sangen er på sin egen melodi, brug da \SBOrgMel.
  {Forfatter} % Navnet på forfatteren. Undlad kaldenavne. Brug gerne TBF. Brug "&" frem for "og". Hvis forfatter er ukendt, lad da stå tom.
  {Anledning og år} % Eks. "Fysikrevy, 2010" eller "2010"
  {\NotCCLIed} % Lad stå som den er

  \begin{SBVerse}
    % Skriv vers her
  \end{SBVerse}

  \begin{SBChorus}
    % Skriv omkvæd her
  \end{SBChorus}

  \begin{SBSection*}
    % Skriv sektioner her. Hvis du ønsker lidt mellemrum for at give luft i et langt afsnit el.lign., brug da \\\medskip
  \end{SBSection*}
\end{song}
\begin{song}{Kun fysik}
  {} % Bruges ikke, lad stå blank
  {Rigtige mænd, Disney} % Titel, Kunstner - eks.: "Jutlandia, Kim Larsen". Hvis sangen er på sin egen melodi, brug da \SBOrgMel.
  {} % Navnet på forfatteren. Undlad kaldenavne. Brug gerne TBF. Brug "&" frem for "og". Hvis forfatter er ukendt, lad da stå tom.
  {FysikRevy, 2009} % Eks. "Fysikrevy, 2010" eller "2010"
  {\NotCCLIed} % Lad stå som den er

  \begin{SBVerse}
    Danmarks statsminister\\
    vil ha' lærde mænd.\\
    Ikke humanister,\\
    der er nok af dem!\\
    Modesnak det er slet intet værd!\\
    Og fodboldfacts kan inden brug'!\\
    Det' fysik! Kun fysik!\\
    I skal ku'!
  \end{SBVerse}

  \begin{SBVerse}
    Hurtig're end Pauli\\
    Skal I regne kvant.\\
    Altid skarp og saglig,\\
    Søg hvad der er sandt.\\
    Brug af Schaums og TI-89\\
    Det er for svagt, så hør mig nu!\\
    Det' fysik!\\
    Kun fysik!\\
    I skal ku'!
  \end{SBVerse}

  \begin{SBSection*}
    Ebbe Sand laver aldrig fejl!\\
    Skal jeg virk'lig pertubere?\\
    Det med mat'matikken faldt mig aldrig nemt.\\
    Der flækkede jeg igen en negl!\\
    Mon Beckham han kan transformere?\\
    Ku' jeg dividere, var det ik' så slemt.
  \end{SBSection*}

  \begin{SBChorus}
    \emph{Kun fysik!} I skal ku' regne på kvantefelter!\\
    \emph{Kun fysik!} Og integrere enhver funktion!\\
    \emph{Kun fysik!} Og I skal læse til hjernen smelter!\\
    Før I kan få en forskningsmillion!
  \end{SBChorus}

  \begin{SBVerse}
    Eksamen om en uge.\\
    Det vil aldrig gå.\\
    Natten må I bruge,\\
    Hvis I vil bestå.\\
    I' en dum, ubrug'lig, ynk'lig flok.\\
    Så drop ud, I dumper nu!\\
    Det' fysik!\\
    Kun fysik!\\
    I skal ku'!
  \end{SBVerse}

  \begin{SBChorus}
    \emph{Kun fysik!}\ldots
  \end{SBChorus}

  \begin{SBChorus}
    \emph{Kun fysik!}\ldots
  \end{SBChorus}
\end{song}

\onecolumn
\chapter*{Matematik}
\twocolumn
\begin{song}{Den kanoniske matematikersang}
  {} % Bruges ikke, lad stå blank
  {Bamses fødselsdag}
  {Esben Bistrup Halvorsen og Rasmus Resen Amossen}
  {}
  {\NotCCLIed}

  \begin{SBVerse}
	Som mat'matikstuderende,\\
	så er jeg svær at narre.\\
	Når andre kalder mig for nørd,\\
	så må jeg bare svare:
  \end{SBVerse}

  \begin{SBChorus}
	Hip hurra for algebra,\\
	Euler, Gauss og Galois\\
	og for rum med en kompakt\\
	deformationsretrakt.
  \end{SBChorus}

  \begin{SBVerse}
	En dag jeg sa' til kæresten:\\
	"nu vil jeg dyrke grupper".\\
	Hun kiggede forbavset op,\\
	"det ikke super duper!".\\\medskip
	"Jamen det er ej med dig,\\
	gutterne de er på vej".\\
	"Får du ikke nok af mig?!"\\
	 -så rejste hun sin vej.\\
  \end{SBVerse}

  \begin{SBVerse}
	Nu var jeg blevet singelton,\\
	og sagde til min moder:\\
	"Jeg er disjunkt med kæresten\\
	som Peking med Nyboder.\\\medskip
	Inklusionen er nu vendt,\\
	jeg er blevet transcendent".\\
	"Er du trans, din klamme tøs?!"\\
	-så blev jeg arveløs.
  \end{SBVerse}

  \begin{SBVerse}
	Da arven nu var faldet bort,\\
	jeg måtte til at spare.\\
	Jeg talte med min vicevært\\
	og kunne ham forklare:\\\medskip
	"Pengemængden er kompakt,\\
	jeg vil ha' en ny kontrakt!".\\
	"Fint med mig, den kommer her:\\
	Du bor her ikke mer'!".
  \end{SBVerse}

  \begin{SBVerse}
	Nu var jeg efterhånden ved\\
	at få lidt hovedpine.\\
	Jeg tog til hospitalet og\\
	fik hjælp af en blondine.\\\medskip
	"Hovedsmerten går for vidt,\\
	den sku' deles op i snit!".\\
	"Hvidt og snit, og så nå'et, ik'?"\\
	-der røg mit overblik.
  \end{SBVerse}

  \begin{SBVerse}
	Jeg har nu få't det hvide snit\\
	og boligen er røget,\\
	men selvom både arv og kær'-\\
	ste fløj, er jeg fornøjet!
  \end{SBVerse}

  \begin{SBChorus}
	Hip hurra for algebra,\\
	Euler, Gauss og Galois\\
	og for rum med en kompakt\\
	deformationsretrakt.
  \end{SBChorus}
\end{song}





\begin{song}{Mat'matik}
  {} % Bruges ikke, lad stå blank
  {Bubbi-bjørnene, Disney} % Titel, Kunstner - eks.: "Jutlandia, Kim Larsen". Hvis sangen er på sin egen melodi, brug da \SBOrgMel.
  {} % Navnet på forfatteren. Undlad aliasser. Brug "&" frem for "og". Hvis forfatter er ukendt, lad da stå tom.
  {FysikRevy, 2008} % Eks. "Fysikrevy 2010" eller "2010"
  {\NotCCLIed} % Lad stå som den er

  \begin{SBVerse}
    Vektorfunktioner i tre dimensioner\\
    skal opereres med curl og divergens.\\
    $\nabla$\textsuperscript{2} gir koldsved på panden;\\
    planintegraler er en pestilens.
  \end{SBVerse}

  \begin{SBChorus}
    Mat'matik\\
    strider ofte mod enhver logik;\\
    men har du elektrodynamik,\\
    så skal du ku' mat'matik.
  \end{SBChorus}

  \begin{SBVerse}
    Kan en hermitisk og injektiv matrix\\
    ha' negativ trace men en nul-determinant?\\
    Solovej siger dens egenværdier\\
    vil være reelle, men er det mon sandt?
  \end{SBVerse}

  \begin{SBChorus}
    Mat'matik\\
    strider ofte mod enhver logik;\\
    men hvis du har kvantemekanik,\\
    så skal du ku' mat'matik.
  \end{SBChorus}

  \begin{SBVerse}
    $T_a$ vil gi' $SU(3)$-symmetri,\\
    og Lagrange-operator'n er Gauge-invariant.\\
    Når operator'ne virker på kvarkerne,\\
    får de en farve og smagen af kvant
  \end{SBVerse}

  \begin{SBChorus}
    Mat'matik\\
    strider ofte mod enhver logik;\\
    men i kvantekromodynamik,\\
    så skal du ku' mat'matik.
  \end{SBChorus}

  \begin{SBSection*}
    Så skal du ku' mat'matik!
  \end{SBSection*}
\end{song}
\begin{song}{Matrixrepræsentationsteoremet}
  {} % Bruges ikke, lad stå blank
  {I will survive, Gloria Gaynor} % Titel, Kunstner - eks.: "Jutlandia, Kim Larsen". Hvis sangen er på sin egen melodi, brug da \SBOrgMel.
  {Christian Bladt Brandt} % Navnet på forfatteren. Undlad aliasser. Brug "&" frem for "og". Hvis forfatter er ukendt, lad da stå tom.
  {\TKET{}, 2011} % Eks. "Fysikrevy 2010" eller "2010"
  {\NotCCLIed} % Lad stå som den er

  \begin{SBVerse}
    Hvis man skal transformere, altså lineært,\\
    mellem to vektorrum, så er det altså elementært.\\
    Det kan jo snildt repræsenteres ved en matrix, som I ser,\\
    og hvis I ikke helt kan se det, så beviser jeg det her:
  \end{SBVerse}

  \begin{SBVerse}
    Vi ser på $V$, et vektorrum,\\
    der har ordnet basis kaldet $E$, og $n$ som dimension.\\
    Med dimensionen $m$ og basen $F$ der har vi $W$,\\
    det vektorrum den lineær’ transformation går over i.
  \end{SBVerse}

  \begin{SBVerse}
    Så har vi $x$, hvad er nu det?\\
    En koordinatvektor i basen $E$ for vektor’n $v$ i $V$,\\
    og for vektorerne i $W$ der kan vi definer’\\
    en koordinatvektor i basen $F$, der kaldes $y$, oh yeah!
  \end{SBVerse}

  \begin{SBChorus}
    Så er vi klar\\
    til at bevise\\
    denne sætning, der er vigtig,\\
    men først skal vi lige ha’ formuleret helt præcist\\
    hvad vi gern’ vil ha’ bevist.\\
    Det kommer her,\\
    det kommer her:
  \end{SBChorus}

  \begin{SBVerse}
    Nu vil vi vise $Ax=y$ hvis og kun hvis\\
    $L$ på $v$ den ser så’n her ud i vores $F$-basis.\\
    Her la’r vi $A$ være en matrix der har søjler givet ve’\\
    transformationen anvendt på basisvektorerne fra $E$.
  \end{SBVerse}

  \begin{SBVerse}
    Så vi ta’r $L$, anvendt på $v$,\\
    og starter med at bruge linearitet på det.\\
    Og hvis vi husker på hvordan vor matrix $A$ var definer’t,\\
    så er det næste udtryk, som I ser, vist ikke helt forkert!
  \end{SBVerse}

  \begin{SBVerse}
    Det sidste skridt er trivielt!\\
    Vi bytter rundt på de to summer, ikke nog’t specielt.\\
    Nu ses det at det udtryk der’ i parentes, det er $y_i$,\\
    så $y$ lig $A$ på $x$, og jeg ka’ si':\\
    Q.E.D.
  \end{SBVerse}
\end{song}
\begin{song}{Er du dus med Grassmans Lemma}
  {} % Bruges ikke, lad stå blank
  {Melodi} % Titel, Kunstner - eks.: "Jutlandia, Kim Larsen". Hvis sangen er på sin egen melodi, brug da \SBOrgMel.
  {Forfatter} % Navnet på forfatteren. Undlad kaldenavne. Brug gerne TBF. Brug "&" frem for "og". Hvis forfatter er ukendt, lad da stå tom.
  {Anledning og år} % Eks. "Fysikrevy, 2010" eller "2010"
  {\NotCCLIed} % Lad stå som den er

  \begin{SBVerse}
    Er du dus med Grassmans lemma\\
    og Holgers gule bog\\
    og Borel Heines sætning\\
    så er du rigtig klog\\
    Kan du smile til et $\delta$\\
    og vinke til $dx$\\
    så har du fundet ud af det\\
    som er mere værd end – MAT A\\\medskip
    En vektor, en matrix,\\
    en n’te grads funktion,\\
    en basis, en række,\\
    en dårlig mi«knas»kro«knas»fon
  \end{SBVerse}

  \begin{SBVerse}
    Har du set når Holger summer\\
    på A og B og C\\
    så ved du hvad der kommer\\
    på tavle nr. tre.\\
    Har du forberedt dig hjemme\\
    det hjælper ikke spor\\
    for lige meget hvad han si’r\\
    så forstår du ikk’ et ord\\\medskip
    Et lemma, et $\gamma$,\\
    en summe over $X$,\\
    et primtal, en faktor,\\
    en ugeseddel – syv
  \end{SBVerse}

  \begin{SBVerse}
    Syn’s du at Leif han mumler\\
    at frikvarter er rart?\\
    Snakker du med humanister\\
    for at føle dig lidt smart?\\
    Er du træt af Leifes sweater\\
    af konkav og konveks?\\
    Så bare vent til algebra,\\
    så får du gruppe – teori\\\medskip
    Supremum og limes,\\
    et simpelt korollar,\\
    en normal fordeling,\\
    en stinkende cigar.
  \end{SBVerse}

  \begin{SBVerse}
    Er du dus med ham Igna– Ignat– Michael?\\
    Kan du programmere C\\
    Får du ondt når EMS han skriger\\
    om $O(\log n+p)$?\\
    Kan du smile til en binær\\
    og vinke til en hex?\\
    Jeg tror, at EMS blev gift\\
    på grund af børn’ne, ikke – sex
  \end{SBVerse}
\end{song}





\begin{song}{Her på mat'matik}
  {} % Bruges ikke, lad stå blank
  {Fætter Mikkel} % Titel, Kunstner - eks.: "Jutlandia, Kim Larsen". Hvis sangen er på sin egen melodi, brug da \SBOrgMel.
  {} % Navnet på forfatteren. Undlad kaldenavne. Brug gerne TBF. Brug "&" frem for "og". Hvis forfatter er ukendt, lad da stå tom.
  {Matematikrevyen, 2015} % Eks. "Fysikrevy, 2010" eller "2010"
  {\NotCCLIed} % Lad stå som den er

  \begin{SBVerse}
    Her på mat'matik skal man være kvik\\
    Sætte pris på elegance\\
    Ha' intuition, også ambition\\
    Gribe fat i hver en chance
  \end{SBVerse}

  \begin{SBChorus}
    For vi elsker ren logik \emph{(klap-klap)}\\
    algebra og statistik \emph{(klap-klap)}\\
    Alt der er komplekst: analyse, vækst,\\
    det er her vi har det bedst \emph{(klap-klap)}
  \end{SBChorus}

  \begin{SBVerse}
    Fra det første år, kurserne består\\
    Fællesskabet ej forglemme\\
    Der bli'r undervist, masser kage spist\\
    Det er her hvor vi har hjemme
  \end{SBVerse}

  \begin{SBChorus}
    For vi elsker ren logik\ldots
  \end{SBChorus}

  \begin{SBSection*}
    % Skriv sektioner her. Hvis du ønsker lidt mellemrum for at give luft i et langt afsnit el.lign., brug da \\\medskip
  \end{SBSection*}
\end{song}
\begin{song}{Jeg er en matematiker fra HCØ}
  {} % Bruges ikke, lad stå blank
  {Melodi} % Titel, Kunstner - eks.: "Jutlandia, Kim Larsen". Hvis sangen er på sin egen melodi, brug da \SBOrgMel.
  {Forfatter} % Navnet på forfatteren. Undlad kaldenavne. Brug gerne TBF. Brug "&" frem for "og". Hvis forfatter er ukendt, lad da stå tom.
  {Anledning og år} % Eks. "Fysikrevy, 2010" eller "2010"
  {\NotCCLIed} % Lad stå som den er

  \begin{SBVerse}
    % Skriv vers her
  \end{SBVerse}

  \begin{SBChorus}
    % Skriv omkvæd her
  \end{SBChorus}

  \begin{SBSection*}
    % Skriv sektioner her. Hvis du ønsker lidt mellemrum for at give luft i et langt afsnit el.lign., brug da \\\medskip
  \end{SBSection*}
\end{song}
\begin{song}{På mat'matik}
  {} % Bruges ikke, lad stå blank
  {Melodi} % Titel, Kunstner - eks.: "Jutlandia, Kim Larsen". Hvis sangen er på sin egen melodi, brug da \SBOrgMel.
  {Forfatter} % Navnet på forfatteren. Undlad kaldenavne. Brug gerne TBF. Brug "&" frem for "og". Hvis forfatter er ukendt, lad da stå tom.
  {Anledning og år} % Eks. "Fysikrevy, 2010" eller "2010"
  {\NotCCLIed} % Lad stå som den er

  \begin{SBVerse}
    % Skriv vers her
  \end{SBVerse}

  \begin{SBChorus}
    % Skriv omkvæd her
  \end{SBChorus}

  \begin{SBSection*}
    % Skriv sektioner her. Hvis du ønsker lidt mellemrum for at give luft i et langt afsnit el.lign., brug da \\\medskip
  \end{SBSection*}
\end{song}
\begin{song}{Matematikkens Historie 2}
  {} % Bruges ikke, lad stå blank
  {Major General's Song, Gilbert and Sullivan} % Titel, Kunstner - eks.: "Jutlandia, Kim Larsen". Hvis sangen er på sin egen melodi, brug da \SBOrgMel.
  {} % Navnet på forfatteren. Undlad kaldenavne. Brug gerne TBF. Brug "&" frem for "og". Hvis forfatter er ukendt, lad da stå tom.
  {Matematikrevy, 2012} % Eks. "Fysikrevy, 2010" eller "2010"
  {\NotCCLIed} % Lad stå som den er

  \begin{SBVerse}
    Jeg er godt eksempel på en klassisk matematiker\\
    Jeg har jo haft mit studie som fuldstændig analytiker\\
    I matematikhistorien der kan jeg finde skjulested\\
    Men jeg er ikke stor nok til at kunne sammenlignes med:\\
    Ramanujan, og Trachtenberg, Von Neumann og Kolmogorov\\
    Og Grothendieck, og De Moivre, Hippocrates, Lyapunov\\
    Og Weierstrass, Pythagoras, Lobaechevsky og så Cauchy\\
    Samt Cavalieri, Minkowski, Fibonacci og Jacobi 
  \end{SBVerse}

  \begin{SBVerse}
    Husk Wessel, Russell og Borel, og Aristoteles, Pascal\\
    Zermelo, Fraenkel, Gödel, Boole, og Goldbach, Hardy, og Abel\\
    Og Lindelöf, og Dirichlet, og Dedekind, og Sørensen\\
    Og Noether, Cantor, Mandelbrot, og Littlewood og Pedersen 
  \end{SBVerse}

  \begin{SBVerse}
    Og Hipparchos, og Möbius, Eudoxus, Apollonius\\
    Archimedes, og Bernoulli, Diophantos, Frobenius\\
    Og Poincaré, Galileo, Al-Khwârizmi og Fubini\\
    Brahmagupta, og Chebyshev, Caratheodory, Seki,\\
    Og Lie, Fourier og Bháscara, Iwasawa, Aryabhatta\\
    Og Liouville, og Fischer, og så Leibniz, Chern, og Atiyah\\
    Og Lebesgue, og Erdõs, Euclid, og Banach, Tarski og Sylow\\
    Og Hadamard, og Sylvester, og Weyl, Landau, og så Markov
  \end{SBVerse}

  \begin{SBVerse}
    Husk Wessel, Russell og Borel, og Aristoteles, Pascal\\
    Zermelo, Fraenkel, Gödel, Boole, og Goldbach, Hardy, og Abel\\
    Og Lindelöf, og Dirichlet, og Dedekind, og Sørensen\\
    Og Noether, Cantor, Mandelbrot, og Littlewood og Pedersen 
  \end{SBVerse}

  \begin{SBVerse}
    Wiles, Lagrange og Archytas, De Morgan, Babbage og Laplace\\
    Napier. Hilbert, Fermat, Lambert, Eilers, og l'Hôpital, Thales\\
    Og Kepler, Taylor, Escher, Lehrer, Brouwer, Nash og så Descartes\\
    Og Darboux, Jensen, Hansen, Grassmann, Jordan, Klein og Galois
  \end{SBVerse}

  \begin{SBVerse}
    Vi mangler tre af matematikkens helt centrale søjler\\
    Det' nogen af de største det er Riemann, Gauss og Euler
  \end{SBVerse}
\end{song}
\begin{song}{Primtal}
  {} % Bruges ikke, lad stå blank
  {Candy, Robbie Williams} % Titel, Kunstner - eks.: "Jutlandia, Kim Larsen". Hvis sangen er på sin egen melodi, brug da \SBOrgMel.
  {} % Navnet på forfatteren. Undlad kaldenavne. Brug gerne TBF. Brug "&" frem for "og". Hvis forfatter er ukendt, lad da stå tom.
  {Matematikrevyen, 2014} % Eks. "Fysikrevy, 2010" eller "2010"
  {\NotCCLIed} % Lad stå som den er

  \begin{SBVerse}
    Lad os definere de tal der fascinerer.\\
    For primtal skal der gælde faktorer trivielle.\\
    Elegant og simpelt, det kan man let forstå.\\
    Men trods det er det fulde billed’ umuligt at opnå.\\\medskip
    \emph{Men hør nu:}\\
    Euler og Euklid de viste,\\
    blandt primtal er der slet intet sidste.\\
    Givet $n$ kan primtal skrives\\
    helt entydigt, vi har uniqueness.
  \end{SBVerse}

  \begin{SBChorus}
    Skriv nu $p$ og $q$.\\
    Der’ så meg’t der ikk’ er vist endnu.\\
    Uden dem går teori itu,\\
    for alt er smukt ved primtal.\\
    Skriv nu $p$ og $q$.\\
    De er overalt, det vides jo.\\
    Også selvom det er svært at tro,\\
    for alt er smukt ved primtal.
  \end{SBChorus}

  \begin{SBVerse}
    Deres heltalsringe dem kan man let frembringe.\\
    For vilkårligt $p$ er der meget at indse.\\
    Man kan se tendenser og smukke kongruenser.\\
    Perfekt at anerkende, forbundet med Mersenne.\\\medskip
    Goldbach havde en formodning.\\
    Og kryptologer brug’r dem i kodning.\\
    Dankortkøb forløber sikkert fordi\\
    vi slipper primtal fri. Hvad gjord’ vi uden dem?\\
    Så kom nu!
  \end{SBVerse}

  \begin{SBChorus}
    Skriv nu $p$ og $q$.\ldots
  \end{SBChorus}

  \begin{SBVerse}
    $\pi(x)$ er asymptotisk.\\
    Distribueringen dog ej logisk.\\
    Sylows sætning hjælper os til at se\\
    grupper af orden $p$. Hvad gjord’ vi uden dem?\\
    \SBRepeat{\SBRepeat{\SBRepeat{Hvad gjord’ vi uden dem?}}}
  \end{SBVerse}

  \begin{SBChorus}
    Skriv nu $p$ og $q$.\ldots
  \end{SBChorus}

  \begin{SBChorus}
    Skriv nu $p$ og $q$.\ldots
  \end{SBChorus}
\end{song}
\begin{song}{Integralsangen}
  {} % Bruges ikke, lad stå blank
  {Melodi} % Titel, Kunstner - eks.: "Jutlandia, Kim Larsen". Hvis sangen er på sin egen melodi, brug da \SBOrgMel.
  {Forfatter} % Navnet på forfatteren. Undlad kaldenavne. Brug gerne TBF. Brug "&" frem for "og". Hvis forfatter er ukendt, lad da stå tom.
  {Anledning og år} % Eks. "Fysikrevy, 2010" eller "2010"
  {\NotCCLIed} % Lad stå som den er

  \begin{SBVerse}
    % Skriv vers her
  \end{SBVerse}

  \begin{SBChorus}
    % Skriv omkvæd her
  \end{SBChorus}

  \begin{SBSection*}
    % Skriv sektioner her. Hvis du ønsker lidt mellemrum for at give luft i et langt afsnit el.lign., brug da \\\medskip
  \end{SBSection*}
\end{song}

\onecolumn
\chapter*{Datalogi}
\twocolumn
\begin{song}{Se min kode}
  {} % Bruges ikke, lad stå blank
  {Se min kjole, Gunnar Nyborg-Jensen} % Titel, Kunstner - eks.: "Jutlandia, Kim Larsen". Hvis sangen er på sin egen melodi, brug da \SBOrgMel.
  {Jacob Johannsen og Erik Søe Sørensen} % Navnet på forfatteren. Undlad kaldenavne. Brug gerne TBF. Brug "&" frem for "og". Hvis forfatter er ukendt, lad da stå tom.
  {\TKET{}s Julerevy, 2003} % Eks. "Fysikrevy, 2010" eller "2010"
  {\NotCCLIed} % Lad stå som den er

  \begin{SBVerse}
    Se min kode - den er struktureret,\\
    alt hvad jeg skriver, det er smukt som den.\\
    Det er fordi jeg altid indenterer,\\
    og fordi at Emacs er min ven
  \end{SBVerse}

  \begin{SBVerse}
    Se min kode – den er let at læse,\\
    alt hvad jeg skriver, det er ligesom den.\\
    Det er fordi jeg altid kommenterer,\\
    og fordi /* er min ven
  \end{SBVerse}

  \begin{SBVerse}
    Se min kode – den vil kompilere,\\
    alt hvad jeg skriver oversættes nemt.\\
    Det er fordi jeg skriver simpel kode,\\
    og fordi gcc den er min ven
  \end{SBVerse}

  \begin{SBVerse}
    Se min kode – er i mange filer,\\
    alt hvad jeg skriver, det kan findes nemt.\\
    Det er fordi jeg altid fragmenterer,\\
    og fordi at make den er min ven
  \end{SBVerse}

  \begin{SBVerse}
    Se min kode – den er uden fejl i,\\
    alt hvad jeg skriver, det er lissom den.\\
    Det er fordi, jeg altid er forsigtig,\\
    og fordi GDB den er min ven.
  \end{SBVerse}

  \begin{SBVerse}
    Se min kode – den kan let genskabes,\\
    alt hvad jeg ændrer, rettes let igen.\\
    Det er fordi, jeg altid tager backup,\\
    og fordi CVS den er min ven.
  \end{SBVerse}

  \begin{SBVerse}
    Se min kode - den er fri som fuglen,\\
    alt hvad jeg ejer, det er frit som den.\\
    Det er fordi, jeg elsker Open Software,\\
    og fordi GPL den er min ven.
  \end{SBVerse}

  \begin{SBVerse}
    Se min kode – den er årtotusindsikret,\\
    alt hvad jeg skriver, det er årtotusindsikret som den.\\
    Det er fordi jeg passer på, der aldrig kommer overflow,\\
    og fordi alt andet ville være fjollet, simpelthen.
  \end{SBVerse}

  \begin{SBVerse}
    Se min kode – den er tem’lig fjollet,\\
    alt hvad jeg skriver, det er lissom den.\\
    Det er fordi jeg ofte går i selvsving,\\
    og fordi revyen er min ven.
  \end{SBVerse}

  \begin{SBVerse}
    Semikolon, sangen den er slut nu;
  \end{SBVerse}
\end{song}
\begin{song}{Jeg har fundet mig en bug}
  {} % Bruges ikke, lad stå blank
  {Jeg har fundet mig en myg, Astrid Kjærgaard} % Titel, Kunstner - eks.: "Jutlandia, Kim Larsen". Hvis sangen er på sin egen melodi, brug da \SBOrgMel.
  {} % Navnet på forfatteren. Undlad kaldenavne. Brug gerne TBF. Brug "&" frem for "og". Hvis forfatter er ukendt, lad da stå tom.
  {} % Eks. "Fysikrevy, 2010" eller "2010"
  {\NotCCLIed} % Lad stå som den er

  \begin{SBVerse}
    Jeg har fundet mig en bug\\
    den er stor og væm’lig.\\
    GDB er ikke nok\\
    dér er buggen nemlig\\\medskip
    Jeg har brugt den hele nat\\
    på at stoppe huller\\
    Og er efterhånden træt\\
    af etter og nuller
  \end{SBVerse}

  \begin{SBVerse}
    Jeg har voldsom kaffetrang\\
    men der er ej mere\\
    Koden den vil ik’ engang\\
    næsten kompilere\\\medskip
    Der er noget voldsomt galt\\
    på så mang’ niveauer\\
    Helt basalt er det fatalt\\
    at jeg sidder og sover
  \end{SBVerse}

  \begin{SBVerse}
    De fandt ham den næste dag\\
    med tastetryk i panden.\\
    Plud’slig vågned’ han og sa’:\\
    "Fejlen var en anden!"\\\medskip
    Husk det, gæve datalog\\
    Ofte skal du bare\\
    ha’ det lidt på afstand,\\
    så ser du meget klarer’!
  \end{SBVerse}
\end{song}
\begin{song}{Linieskriverdriversangen}
  {} % Bruges ikke, lad stå blank
  {Melodi} % Titel, Kunstner - eks.: "Jutlandia, Kim Larsen". Hvis sangen er på sin egen melodi, brug da \SBOrgMel.
  {Forfatter} % Navnet på forfatteren. Undlad kaldenavne. Brug gerne TBF. Brug "&" frem for "og". Hvis forfatter er ukendt, lad da stå tom.
  {Anledning og år} % Eks. "Fysikrevy, 2010" eller "2010"
  {\NotCCLIed} % Lad stå som den er

  \begin{SBVerse}
    % Skriv vers her
  \end{SBVerse}

  \begin{SBChorus}
    % Skriv omkvæd her
  \end{SBChorus}

  \begin{SBSection*}
    % Skriv sektioner her. Hvis du ønsker lidt mellemrum for at give luft i et langt afsnit el.lign., brug da \\\medskip
  \end{SBSection*}
\end{song}
\begin{song}{Server'n er crashed}
  {} % Bruges ikke, lad stå blank
  {I Want It That Way, Backstreet Boys} % Titel, Kunstner - eks.: "Jutlandia, Kim Larsen". Hvis sangen er på sin egen melodi, brug da \SBOrgMel.
  {} % Navnet på forfatteren. Undlad kaldenavne. Brug gerne TBF. Brug "&" frem for "og". Hvis forfatter er ukendt, lad da stå tom.
  {DIKUrevy, 2011} % Eks. "Fysikrevy, 2010" eller "2010"
  {\NotCCLIed} % Lad stå som den er

  \begin{SBVerse}
    Jeg bli'r helt ked, men\\
    Nu er den nede\\
    Den var belastet\\
    Nu er den crashed.
  \end{SBVerse}

  \begin{SBVerse}
    Min prompt den sejler.\\
    Systemet fejler\\
    Der er sgu knas med\\
    den server, der crashed.
  \end{SBVerse}

  \begin{SBChorus}
    For åh nej, vi sku' ha købt et nyt RAID\\
    For åh nej, jeg var sgu' ikke beredt\\
    før vi så undtagelsen den kasted.\\
    Server'n er crashed
  \end{SBChorus}

  \begin{SBVerse}
    Men kan den tvinges\\
    til at ku' pinges\\
    Nej, den er - helt trashed.\\
    Ja, server'n er crashed.
  \end{SBVerse}

  \begin{SBChorus}
    For åh nej\ldots
  \end{SBChorus}

  \begin{SBSection*}
    Tænkte, jeg fikser det bare i mor'n\\
    da den stod der og kasted' fejl\\
    Jeg burde ha' tjekket, men surfede porn\\
    Nu er den sgu' gåe't sin vej.
  \end{SBSection*}

  \begin{SBVerse}
    Det' ikk' så skid' rart\\
    Jeg ta'r en genstart\\
    Men nej, åh nej, åh nej, åh nej!\\
    \ldots
  \end{SBVerse}

  \begin{SBSection*}
    Det fucking neder'n!
  \end{SBSection*}

  \begin{SBSection*}
    Den fejler nu ved startup\\
    Vi har sgu ingen backup\\
    Disken er sikkert kvæstet\\
    Server'n er crashed
  \end{SBSection*}

  \begin{SBChorus}
    For åh nej\ldots
  \end{SBChorus}

  \begin{SBChorus}
    For åh nej\ldots
  \end{SBChorus}

  \begin{SBSection*}
    Server'n er crashed
  \end{SBSection*}
\end{song}
\begin{song}{Forever DIKU}
  {} % Bruges ikke, lad stå blank
  {Melodi} % Titel, Kunstner - eks.: "Jutlandia, Kim Larsen". Hvis sangen er på sin egen melodi, brug da \SBOrgMel.
  {Forfatter} % Navnet på forfatteren. Undlad kaldenavne. Brug gerne TBF. Brug "&" frem for "og". Hvis forfatter er ukendt, lad da stå tom.
  {Anledning og år} % Eks. "Fysikrevy, 2010" eller "2010"
  {\NotCCLIed} % Lad stå som den er

  \begin{SBVerse}
    % Skriv vers her
  \end{SBVerse}

  \begin{SBChorus}
    % Skriv omkvæd her
  \end{SBChorus}

  \begin{SBSection*}
    % Skriv sektioner her. Hvis du ønsker lidt mellemrum for at give luft i et langt afsnit el.lign., brug da \\\medskip
  \end{SBSection*}
\end{song}
\begin{song}{Han koder slam}
  {} % Bruges ikke, lad stå blank
  {Han får for lidt, Østkyst Hustlers} % Titel, Kunstner - eks.: "Jutlandia, Kim Larsen". Hvis sangen er på sin egen melodi, brug da \SBOrgMel.
  {} % Navnet på forfatteren. Undlad kaldenavne. Brug gerne TBF. Brug "&" frem for "og". Hvis forfatter er ukendt, lad da stå tom.
  {DIKUrevy, 1998} % Eks. "Fysikrevy, 2010" eller "2010"
  {\NotCCLIed} % Lad stå som den er

  \begin{SBVerse}
    Han roder med sin kode, for hans kode er for sej.\\
    Han ta'r Dat0 igen igen, han er et kæmpe kvaj.\\
    For funktionssprog er for kvinder, C++ det er for mænd.\\
    Han plejer' ta' en peger, holder typerne i spænd.
  \end{SBVerse}

  \begin{SBVerse}
    Og ud med kommentar'ne, for hans kod' er selvforklar'ne.\\
    Han koder som han koder, for at ligne ham der Bjarne.\\
    Og da han altid bruger goto, når koden er i udu,\\
    får han tit at vide at nog't såd'n det er misbrug.
  \end{SBVerse}

  \begin{SBVerse}
    Kun til Coca Cola, mand, for intet andet firma kan,\\
    få ham til at drikke deres søde sukkervand.\\
    Algoritmer for vatnisser. Han vil ha' Bugatti, så\\
    først ved terminalen. Der er slet ingen hvis'er.
  \end{SBVerse}

  \begin{SBVerse}
    Men hans kildekod', den er noget værre rod.\\
    Og "unsigned pointer temp stjerne", det virker ikke-nikke-\\
    Core-filen ligger der på klokkeslaget.\\
    Han føler sig så svag, han skal aflever' i dag!
  \end{SBVerse}

  \begin{SBChorus}
    Han koder slam! Han får sat koden på plads.\\
    Ingen klam analyse, bare en masse strabads.\\
    Han plejed' at kod' lidt kvajet, men nu koder han flot.\\
    Det syn's han selv, alligevel, så kør' det aldrig godt.\\
    Han koder slam! Han er det groveste skvat.\\
    Ingen klam analyse, bare kode i nat.\\
    Han koder slam! Koder slam!\\
    Han koder slam!
  \end{SBChorus}

  \begin{SBVerse}
    Han er en ualmind'lig stræbernørd, han syn's han er for hård.\\
    Han går i gang med sit speciale, mens han er på første år.\\
    Han læser datalogi, men er lidt skuffet fordi,\\
    at analyser og rapporter ka' han slet ikke li'.
  \end{SBVerse}

  \begin{SBVerse}
    Og han ta'r en masse kurser, han har næsen ned i bogen,\\
    han har helt utroligt svært ved at holde sig vågen.\\
    Og han stræber fremad, vil vær' professor en dag\\
    for multimedie programmør er ikke lige hans sag.
  \end{SBVerse}

  \begin{SBVerse}
    Men han har det problem, at han hader semantik,\\
    for der er induktionsbeviser, og han fatter det ikk'.\\
    Han burd' ku' nå en masse, med alle hans ressourcer,\\
    men det går bare ikke, han ta'r 10 forskellig' kurser.
  \end{SBVerse}

  \begin{SBVerse}
    Og han koder, gør han, men på alle hans fag,\\
    skal han både skriv' rapport og prøv' programmerne af.\\
    Man burd' ku' programmere, når man er en nørd som ham,\\
    men det ender jo uværg'ligt, med at blive noget slam.
  \end{SBVerse}

  \begin{SBChorus}
    Han koder slam!\ldots
  \end{SBChorus}

  \begin{SBVerse}
    Nu er han færdig med at kode, han mangler kun sin test\\
    men det er godt nok ikke sjovt, for han vil hellere til fest.\\
    Ta'r på cafe'n, der vil han bli', for han vil score en pi'!\\
    Han er jo såd'n' smart fyr, så det gør han bare li'.
  \end{SBVerse}

  \begin{SBVerse}
    Han fortæller vidt og bredt om sine datastrukturer\\
    Men pigen gaber højlydt og kigger på sit ur og\\
    han forklarer alt om C++, og klør sig i skridtet,\\
    illustrerer hvord'n GDB ka' debugge skidtet
  \end{SBVerse}

  \begin{SBVerse}
    Pigen hun ser godt ud, han er blevet helt forjættet.\\
    Hun ligner nemlig noget man ku' hente ned fra nettet.\\
    At hun er fysiker er okay, når bare han får sagt.\\
    At det fysiske som han vil ha' er fysisk nærkontakt.
  \end{SBVerse}

  \begin{SBVerse}
    Nu er hun sikkert snart mør, han siger "prøv nu at hør"\\
    Hvis hun hører mer' om data, så tror hun hun bli'r skør.\\
    Han siger sed og awk og lex og yacc så går hun sin vej!\\
    og stiv går han tilbage for at rette sine fejl.
  \end{SBVerse}

  \begin{SBChorus}
    Han koder slam!\ldots
  \end{SBChorus}
\end{song}
\begin{song}{Hjemmehackeriet}
  {} % Bruges ikke, lad stå blank
  {Hjemmebrænderiet} % Titel, Kunstner - eks.: "Jutlandia, Kim Larsen". Hvis sangen er på sin egen melodi, brug da \SBOrgMel.
  {} % Navnet på forfatteren. Undlad kaldenavne. Brug gerne TBF. Brug "&" frem for "og". Hvis forfatter er ukendt, lad da stå tom.
  {DIKUrevy, 2001} % Eks. "Fysikrevy, 2010" eller "2010"
  {\NotCCLIed} % Lad stå som den er

  \begin{SBVerse}
    Jeg bor her i Ishøj på syvende sal\\
    i en lejlighed der stort set er normal.\\
    En stue, et køkken, et bad med WC\\
    og et kammer hvor jeg har min hjemme-PC.
  \end{SBVerse}

  \begin{SBChorus}
    Jeg hacker, jeg cracker, jeg downloader spil,\\
    og jeg logger ind lig' præcis hvor jeg vil.\\
    Jeg kender dit password, jeg læser din post;\\
    for en hacker som mig er den slags hverdagskost.
  \end{SBChorus}

  \begin{SBVerse}
    Min fætter har hacket i Pentagons net.\\
    De tro'ed det var svært, men han syn's det var let.\\
    De fandt ham dog efter en længere jagt,\\
    så nu er han ansat som sikkerhedsvagt.
  \end{SBVerse}

  \begin{SBChorus}
    Jeg hacker, jeg cracker, jeg downloader spil,\\
    og jeg logger ind lig' præcis hvor jeg vil.\\
    Jeg kender dit password, jeg læser din post;\\
    for en hacker som mig er den slags hverdagskost.
  \end{SBChorus}

  % \begin{SBChorus}
  %   Jeg hacker, jeg cracker,\ldots
  % \end{SBChorus}

  \begin{SBVerse}
    Jeg laved' en virus som hed "I Love You".\\
    Jeg indrømmer dog, jeg fortryder det nu.\\
    Da jeg gik i banken, min løn for at få,\\
    havde virusen sat der's computer i stå.
  \end{SBVerse}

  \begin{SBChorus}
    Jeg hacker, jeg cracker, jeg downloader spil,\\
    og jeg logger ind lig' præcis hvor jeg vil.\\
    Jeg kender dit password, jeg læser din post;\\
    for en hacker som mig er den slags hverdagskost.
  \end{SBChorus}

  % \begin{SBChorus}
  %   Jeg hacker, jeg cracker,\ldots
  % \end{SBChorus}

  \begin{SBVerse}
    Hvis du sku' få lyst til at hacke lidt selv,\\
    jeg ønsker dig al mulig lykke og held.\\
    Det giver dig magt som om du var en gud,\\
    og du kan endda få din pizza bragt ud.
  \end{SBVerse}

  \begin{SBChorus}
    Jeg hacker, jeg cracker, jeg downloader spil,\\
    og jeg logger ind lig' præcis hvor jeg vil.\\
    Jeg kender dit password, jeg læser din post;\\
    for en hacker som mig er den slags hverdagskost.
  \end{SBChorus}

  % \begin{SBChorus}
  %   Jeg hacker, jeg cracker,\ldots
  % \end{SBChorus}
\end{song}
\begin{song}{Programmørens 8-bit drikkesang}
  {} % Bruges ikke, lad stå blank
  {99 Bottles of Beer} % Titel, Kunstner - eks.: "Jutlandia, Kim Larsen". Hvis sangen er på sin egen melodi, brug da \SBOrgMel.
  {} % Navnet på forfatteren. Undlad kaldenavne. Brug gerne TBF. Brug "&" frem for "og". Hvis forfatter er ukendt, lad da stå tom.
  {} % Eks. "Fysikrevy, 2010" eller "2010"
  {\NotCCLIed} % Lad stå som den er

  \begin{SBVerse}
    Der' 1 lille fejl i min kod',\\
    kun 1 lille fejl i min kod'.\\
    Jeg retter den lige, oversætter igen,\\
    så' der 3 små fejl i min kod'.
  \end{SBVerse}

  \begin{SBVerse}
    Der' 3 små fejl i min kod',\\
    kun 3 små fejl i min kod'.\\
    Jeg retter lige én, oversætter igen,\\
    så' der 5 små fejl i min kod'.
  \end{SBVerse}

  \begin{SBVerse}
    Der' 5 små fejl i min kod',\\
    kun 5 små fejl i min kod'.\\
    Jeg retter lige én, oversætter igen,\\
    så' der 9 små fejl i min kod'.
  \end{SBVerse}

  \begin{SBVerse}
    Der' 129 små fejl i min kod',\\
    129 små fejl i min kod'.\\
    Jeg retter lige én, oversætter igen,\\
    så' der 1 stor fejl i min kod'.
  \end{SBVerse}

  \begin{SBVerse}
    Der' 1 stor fejl i min kod',\\
    kun 1 stor fejl i min kod'.\\
    Jeg retter den lige, oversætter igen,\\
    så' der 3 stor' fejl i min kod'.
  \end{SBVerse}

  \begin{SBVerse}
    Der' 129 stor' fejl i min kod',\\
    129 stor' fejl i min kod.\\
    Jeg retter lige én, oversætter igen,\\
    så' der 1 gigantisk fejl.
  \end{SBVerse}

  \begin{SBVerse}
    Der' 1 gigantisk fejl,\\
    kun 1 gigantisk fejl.\\
    Jeg retter den lige, oversætter igen,\\
    så' der 3 gigantiske fejl.
  \end{SBVerse}

  \begin{SBVerse}
    Der' 129 gigantiske fejl,\\
    129 gigantiske fejl.\\
    Jeg retter lige én, oversætter igen,\\
    så' der 1 katastrofal fadæse.
  \end{SBVerse}

  \begin{SBVerse}
    Der' 1 katastrofal fadæse,\\
    kun 1 katastrofal fadæse.\\
    Jeg retter den lige, oversætter igen,\\
    så' der 3 katastrofale fadæser.
  \end{SBVerse}

  \begin{SBVerse}
    Der' 129 katastrofale fadæser,\\
    129 katastrofale fadæser.\\
    Jeg retter lige én, oversætter igen,\\
    og så udgiver Microsoft mit program!
  \end{SBVerse}
\end{song}
\begin{song}{HTML}
  {} % Bruges ikke, lad stå blank
  {YMCA, Village People} % Titel, Kunstner - eks.: "Jutlandia, Kim Larsen". Hvis sangen er på sin egen melodi, brug da \SBOrgMel.
  {Forfatter} % Navnet på forfatteren. Undlad kaldenavne. Brug gerne TBF. Brug "&" frem for "og". Hvis forfatter er ukendt, lad da stå tom.
  {DIKUrevy, 2000} % Eks. "Fysikrevy, 2010" eller "2010"
  {\NotCCLIed} % Lad stå som den er

  \begin{SBVerse}
    Færdig, jeg blev da-a-talog\\
    jeg blev færdig, og eksamen var go'\\
    jeg blev færdig, og sku' ha mig et job\\
    det sku' være no'et med kode
  \end{SBVerse}

  \begin{SBVerse}
    Jobbet, jeg ville ha' noget sjovt\\
    ja og poppet, så det var nu lidt flovt\\
    at jeg ikke, fik no'et med logik\\
    eller smarte algoritmer
  \end{SBVerse}

  \begin{SBChorus}
    For nu koder jeg i H-T-M-L\\
    ja nu koder jeg i H-T-M-L\\
    ja jeg fik mig et job med mange penge i\\
    og min hjerne den fik fri\\\medskip
    Så nu koder jeg i H-T-M-L\\
    ja nu koder jeg i H-T-M-L\\
    Det var ikke li' det jeg havde tænkt mig sku' ske\\
    jeg vil hellere kode 'C'
  \end{SBChorus}

  \begin{SBVerse}
    ML det var sagen for mig\\
    og Miranda, for jeg var jo for sej\\
    men på webben kan de ik' brug's til no'et\\
    de dur ik' på internettet
  \end{SBVerse}

  \begin{SBVerse}
    Kodet, bli'r der ik' meget af\\
    nej men møder, der er mange hver dag\\
    der er kunder, der vil kø-øbe alt\\
    bar' det' no'et med multimedier
  \end{SBVerse}

  \begin{SBChorus}
    For nu koder jeg i H-T-M-L\ldots
  \end{SBChorus}

  \begin{SBVerse}
    Penge, får jeg da mange af\\
    mange penge, flere tusind' hver dag\\
    men jeg gider snart ik' mere det her\\
    for jeg keder mig på jobbet
  \end{SBVerse}

  \begin{SBVerse}
    Hej ven, tag og hør lidt på mig\\
    jeg sagde hej ven, gør nu ikke som mig\\
    få et godt job, hvor du skal lav' noget sjovt\\
    og hold dig fra webbureauer
  \end{SBVerse}

  \begin{SBChorus}
    For nu koder jeg i H-T-M-L\ldots
  \end{SBChorus}
\end{song}
\begin{song}{Terminalsangen}
  {} % Bruges ikke, lad stå blank
  {Vuffelivov, Shu-bi-dua} % Titel, Kunstner - eks.: "Jutlandia, Kim Larsen". Hvis sangen er på sin egen melodi, brug da \SBOrgMel.
  {} % Navnet på forfatteren. Undlad kaldenavne. Brug gerne TBF. Brug "&" frem for "og". Hvis forfatter er ukendt, lad da stå tom.
  {DIKUrevy, 1978} % Eks. "Fysikrevy, 2010" eller "2010"
  {\NotCCLIed} % Lad stå som den er

  \begin{SBVerse}
Jeg har en skærm med mange taster\\
En for hvert symbol\\
Og bagved sidder lysintensiteten\\
Den ledning har mange tråde\\
En til hver sin bit\\
og en ekstra en til pariteten\\
Når man har venner og kærester, så er man normal\\
Men de ta'r tiden fra mig og min terminal
  \end{SBVerse}

  \begin{SBVerse}
Jeg er koblet via DIXI\\
Når DIXI ellers vil\\
Og der er plads på centrets multiplekser\\
Når jeg har lyst så kan jeg sidder\\
Og lege natten lang\\
Med RECKUs mange programmelkomplekser\\
Når man har venner og kærester så er man normal\\
Men de ta'r tiden fra mig og min terminal
  \end{SBVerse}

  \begin{SBVerse}
Jeg kør' på en maskine\\
Der klarer tusind jobs\\
Selvom deta'r syv lange og syv bredde\\
CAU'en har den to af\\
Og det er vældig smart\\
En til hvis den anden sku' vær' nede\\
Når man har venner og kærester så er man normal\\
Men de ta'r tiden fra mig og min terminal
  \end{SBVerse}

  \begin{SBVerse}
Jeg spiller skak og kryds og bolle\\
Den hele lange nat\\
Det er nu trist man ingen kender\\
For selvom den er dejlig\\
Så er den datamat\\
Nu kun et surrogat for menn'ske-venner\\
Når man har venner og kærester så er man normal\\
Og har det bedre end mig med min terminal
  \end{SBVerse}
\end{song}

\onecolumn
\chapter*{Kemi, Biologi og Nano}
\twocolumn
\begin{song}{The New Periodic Table Song}
  {} % Bruges ikke, lad stå blank
  {Orphée aux enfers, Jaques Offenbach (Can Can)} % Titel, Kunstner - eks.: "Jutlandia, Kim Larsen". Hvis sangen er på sin egen melodi, brug da \SBOrgMel.
  {ASAPScience} % Navnet på forfatteren. Undlad aliasser. Brug "&" frem for "og". Hvis forfatter er ukendt, lad da stå tom.
  {2015} % Eks. "Fysikrevy 2010" eller "2010"
  {\NotCCLIed} % Lad stå som den er

  \begin{SBVerse}
    There's Hydrogen and Helium, then\\
    Lithium, Beryllium,\\
    Boron, Carbon everywhere,\\
    Nitrogen all through the air\\
    With Oxygen so you can breathe, and\\
    Fluorine for your pretty teeth,\\
    Neon to light up the signs,\\
    Sodium for salty times
  \end{SBVerse}
     
  \begin{SBVerse}
    Magnesium, Aluminium, Silicon,\\
    Phosphorus, then Sulfur, Chlorine and Argon\\
    Potassium, and Calcium so you'll grow strong,\\
    Scandium, Titanium, Vanadium and\\
    Chromium and Manganese
  \end{SBVerse}
     
  \begin{SBChorus}
    This is the Periodic Table,\\
    Noble gas is stable,\\
    Halogens and Alkali react agressively,\\
    each period will see new\\
    outer shells while elec-\\
    trons are added moving to the right\\
  \end{SBChorus}
     
  \begin{SBVerse}
    Iron is the 26th, then\\
    Cobalt, Nickel coins you get,\\
    Copper, Zinc and Gallium,\\
    Germanium and Arsenic\\
    Selenium and Bromine film,\\
    while Krypton helps light up your room,\\
    Rubidium and Strontium,\\
    then Yttrium, Zirconium
  \end{SBVerse}
     
  \begin{SBVerse}
    Niobium, Molybdenum, Technetium,\\
    Ruthenium, Rhodium, Palladium,\\
    Silver-ware then Cadmium and Indium,\\
    Tin-cans, Antimony, then Tellurium and\\
    Iodine and Xenon and then Caesium and...
  \end{SBVerse}
     
  \begin{SBVerse}
    Barium is 56 and\\
    this is where the table splits\\
    Where Lanthanides have just begun:\\
    Lanthanum, Cerium and Praseodymium\\
    Neodymium's next too\\
    Promethium, then 62's\\
    Samarium, Europium,\\
    Gadolinium and Terbium,\\
    Dysprosium, Holmium, Erbium, Thulium\\
    Ytterbium, Lutetium
  \end{SBVerse}
     
  \begin{SBVerse}
    Hafnium, Tantalum, Tungsten then we're on to\\
    Rhenium, Osmium and Iridium,\\
    Platinum, Gold to make you rich till you grow old\\
    Mercury to tell you when it's really cold
  \end{SBVerse}
     
  \begin{SBVerse}
    Thallium and Lead, then Bismuth for your tummy,\\
    Polonium, Astatine would not be yummy,\\
    Radon, Francium will last a little time,\\
    Radium then Actinides at 89
  \end{SBVerse}
     
  \begin{SBChorus}
    This is the Periodic Table\ldots
  \end{SBChorus}
     
  \begin{SBVerse}
    Actinium, Thorium, Protactinium,\\
    Uranium, Neptunium, Plutonium,\\
    Americium, Curium, Berkelium,\\
    Californium, Einsteinium, Fermium,\\
    Mendelevium, Nobelium, Lawrencium,\\
    Rutherfordium, Dubnium, Seaborgium,\\
    Bohrium, Hassium then Meitnerium,\\
    Darmstadtium, Roentgenium, Copernicium
  \end{SBVerse}
     
  \begin{SBVerse}
    Ununtrium, Flerovium,\\
    Ununpentium, Livermorium,\\
    Ununseptium, Ununoctium,\\
    And then we're done!
  \end{SBVerse}
\end{song}
\begin{song}{Kemisk elskovsvise}
  {} % Bruges ikke, lad stå blank
  {Santa Lucia} % Titel, Kunstner - eks.: "Jutlandia, Kim Larsen". Hvis sangen er på sin egen melodi, brug da \SBOrgMel.
  {Forfatter} % Navnet på forfatteren. Undlad kaldenavne. Brug gerne TBF. Brug "&" frem for "og". Hvis forfatter er ukendt, lad da stå tom.
  {Anledning og år} % Eks. "Fysikrevy, 2010" eller "2010"
  {\NotCCLIed} % Lad stå som den er

  \begin{SBVerse}
Oh Pige, vær mig huld,\\
fattig på gods og Au,\\
står her din riddersmand - \\
går gennem ild og H$_2$O,\\
for dig jeg ofrer alt,\\
du er mig livets NaCl,\\
smiler du til verdens vrimmel,\\
Al$_2$O$_3$ og himmel.
  \end{SBVerse}

  \begin{SBVerse}
Sødeste lille skalk,\\
tag fra mig længsels CaO,\\
sig blot et kærligt ord,\\
håbet i hjertet B.\\
I dine øjne fandt,\\
jeg livets Cx,\\
helt til jeg mit øje lukker,\\
for dig jeg C$_{12}$H$_{22}$O$_{11}$.
  \end{SBVerse}

  \begin{SBVerse}
Bi du længer står,\\
får jeg mit banesår,\\
er da mit ønske galt,\\
skal håbet Na$_3$SbS$_4$$\cdot$9H$_2$O.\\
Må jeg for NaOH og H$_2$O\\
vandre i ensom stand,\\
til en P vist jeg haster,\\
og ned mig kaster.
  \end{SBVerse}
\end{song}
\begin{song}{Kemisk julevise}
  {} % Bruges ikke, lad stå blank
  {Højt fra træets grønne top} % Titel, Kunstner - eks.: "Jutlandia, Kim Larsen". Hvis sangen er på sin egen melodi, brug da \SBOrgMel.
  {} % Navnet på forfatteren. Undlad kaldenavne. Brug gerne TBF. Brug "&" frem for "og". Hvis forfatter er ukendt, lad da stå tom.
  {} % Eks. "Fysikrevy, 2010" eller "2010"
  {\NotCCLIed} % Lad stå som den er

  \begin{SBVerse}
    Lehninger og Holleman\\
    kan vi uden skelen\\
    alle ved, at H$_2$O er vand,\\
    læg jer nu i Se.\\
    Jule-Tin-een falder blidt,\\
    snart er jorden hvid som CaCO$_3$,\\
    smuk-R$_2$CO-er spiller,\\
    spurven slår R-CN-ler.
  \end{SBVerse}

  \begin{SBVerse}
    Nu hvor vi om dette B'd,\\
    alle blevet m-C$_2$H$_5$OC$_2$H$_5$,\\
    sk-R-NH$_2$ sang i lystigt kor\\
    gøre maven letter.\\
    Vi en m-CH$_3$CO-led' ned,\\
    ROH gi'r h-Fe-en fred.\\
    Titan gla-Ag-i tømte\\
    Na$_2$CO$_3$-H$_2$O forsømte.
  \end{SBVerse}

  \begin{SBVerse}
    R$_1$COOR$_2$ hun har ingen SCN$_2$ -\\
    ...ser rundt med CnH$_2$n -\\
    ...der gaven ka-NO$_2$\\
    N$_2$ til en kjole.\\
    Højt Mn-tes uafbrudt\\
    og med øjet slår Bi.\\
    Højt en m-R-CH(OR)2'er\\
    hoved-C$_{10}$H$_{16}$ maler.
  \end{SBVerse}

  \begin{SBVerse}
    Aluminium Sn-g det er nu spist op,\\
    og man det Ti-er,\\
    som en W i sin krop\\
    mad for vore ganer.\\
    Jern-sten ku' vi alle Lithium\\
    den var så HgCl$_2$ vi\\
    Ni-et og C$_{12}$H$_{22}$O$_{11}$\\
    før vi lyset slukker.
  \end{SBVerse}
\end{song}
\begin{song}{10 små biorus}{}
  {10 små cyklister}
  {}
  {Biorevy 2015}
  {\NotCCLIed}

  \begin{SBVerse}
  10 nye studerende kom til bio C.\\
  En mødte ikke op, og så var der 9\\
  \end{SBVerse}

  \begin{SBChorus}
  Der var 1, der var 2, der var 3, der var 4,\\
  der var 5 på bio C\\
  Der var 6, der var 7, der var 8, der var 9,\\
  der var 10 på bio C
  \end{SBChorus}

  \begin{SBVerse}
  Bo sku' på hyttetur, ville gerne nå det,\\
  men han sov over sig og så var der otte
  \end{SBVerse}

  \begin{SBChorus}
  Der var 1, der var 2, der var 3, der var 4,\\
  der var 5 på bio C\\
  Der var 6, der var 7, der var 8, der var 9,\\
  der var 10 på bio C
  \end{SBChorus}

  % \begin{SBChorus}
  % Der var en, der var to\ldots
  % \end{SBChorus}

  \begin{SBVerse}
  En af de studerende var eksamens-tyv,\\
  men han blev opdaget og så var der syv
  \end{SBVerse}

  \begin{SBChorus}
  Der var 1, der var 2, der var 3, der var 4,\\
  der var 5 på bio C\\
  Der var 6, der var 7, der var 8, der var 9,\\
  der var 10 på bio C
  \end{SBChorus}

  % \begin{SBChorus}
  % Der var en, der var to\ldots
  % \end{SBChorus}

  \begin{SBVerse}
  Gry drak sig pisse stiv nærmest per refleks,\\
  men der var mødepligt og så var der seks
  \end{SBVerse}

  \begin{SBChorus}
  Der var 1, der var 2, der var 3, der var 4,\\
  der var 5 på bio C\\
  Der var 6, der var 7, der var 8, der var 9,\\
  der var 10 på bio C
  \end{SBChorus}

  % \begin{SBChorus}
  % Der var en, der var to\ldots
  % \end{SBChorus}

  \begin{SBVerse}
  Anna fra Aabenraa ville gerne hjem,\\
  men toget kørte galt og så var der fem
  \end{SBVerse}

  \begin{SBChorus}
  Der var 1, der var 2, der var 3, der var 4,\\
  der var 5 på bio C\\
  Der var 6, der var 7, der var 8, der var 9,\\
  der var 10 på bio C
  \end{SBChorus}

  % \begin{SBChorus}
  % Der var en, der var to\ldots
  % \end{SBChorus}

  \begin{SBVerse}
  En var en tur i zoo, hvor han så en tig’r.\\
  Han ville nøgle den og så var der fir’
  \end{SBVerse}

  \begin{SBChorus}
  Der var 1, der var 2, der var 3, der var 4,\\
  der var 5 på bio C\\
  Der var 6, der var 7, der var 8, der var 9,\\
  der var 10 på bio C
  \end{SBChorus}

  % \begin{SBChorus}
  % Der var en, der var to\ldots
  % \end{SBChorus}

  \begin{SBVerse}
  Ida tænkte: studydrugs - sikk’ en god idé!\\
  Hun tog for mang' af dem, og så var der tre
  \end{SBVerse}

  \begin{SBChorus}
  Der var 1, der var 2, der var 3, der var 4,\\
  der var 5 på bio C\\
  Der var 6, der var 7, der var 8, der var 9,\\
  der var 10 på bio C
  \end{SBChorus}

  % \begin{SBChorus}
  % Der var en, der var to\ldots
  % \end{SBChorus}

  \begin{SBVerse}
  Gummistøvler er et must for en biolog.\\
  Ib fik kviksand i sin sko, og så var der to
  \end{SBVerse}

  \begin{SBChorus}
  Der var 1, der var 2, der var 3, der var 4,\\
  der var 5 på bio C\\
  Der var 6, der var 7, der var 8, der var 9,\\
  der var 10 på bio C
  \end{SBChorus}

  % \begin{SBChorus}
  % Der var en, der var to\ldots
  % \end{SBChorus}

  \begin{SBVerse}
  To små studer'nde stod under mistelten.\\
  Bærrene er giftige, og så var der en
  \end{SBVerse}

  \begin{SBChorus}
  Der var 1, der var 2, der var 3, der var 4,\\
  der var 5 på bio C\\
  Der var 6, der var 7, der var 8, der var 9,\\
  der var 10 på bio C
  \end{SBChorus}

  % \begin{SBChorus}
  % Der var en, der var to\ldots
  % \end{SBChorus}

  \begin{SBVerse}
  En lille biorus blev ægte biolog\\
  Bestod alle fagene så han er sgu go'!
  \end{SBVerse}

  \begin{SBChorus}
  Der var 1, der var 2, der var 3, der var 4,\\
  der var 5, som dropped ud.\\
  Der var 6, der var 7, der var 8, der var 9,\\
  men den tiende han\ldots
  \end{SBChorus}

  han blev færdig, så han kom ud i arbejdsløsheden og havde glemt at melde sig ind i en A-kasse og kunne desværre ikke få dagpenge, og derfor skred konen og hunden, han røg på flasken og nu er han dranker med en slatten PIK!

\end{song}
\begin{song}{Spas i waders}
  {} % Bruges ikke, lad stå blank
  {Space Invaders, Hit n Hide} % Titel, Kunstner - eks.: "Jutlandia, Kim Larsen". Hvis sangen er på sin egen melodi, brug da \SBOrgMel.
  {} % Navnet på forfatteren. Undlad kaldenavne. Brug gerne TBF. Brug "&" frem for "og". Hvis forfatter er ukendt, lad da stå tom.
  {Biorevy, 2012} % Eks. "Fysikrevy, 2010" eller "2010"
  {\NotCCLIed} % Lad stå som den er

  \begin{SBChorus}
    Spas i waders i en sø\\
    Fanger Myggelarver, Super biolog\\
    Spas i waders i en sø\\
    Se på makrofytter, mål pH og lys\\
    Må afsted, kom nu med
  \end{SBChorus}

  \begin{SBVerse}
    Du er feltbiolog med super seje waders på\\
    Render og ta'r prøver dagen lang\\
    Jeg er feltbiolog med super seje waders på\\
    Renser danske søer for plankton
  \end{SBVerse}

  \begin{SBChorus}
    Spas i waders i en sø\\
    Fanger Myggelarver, Super biolog\\
    Spas i waders i en sø\\
    Se på makrofytter, mål pH og lys
  \end{SBChorus}

  \begin{SBChorus}
    Spas i waders i en sø\\
    Fanger Myggelarver, Super biolog\\
    Spas i waders i en sø\\
    Se på makrofytter, mål pH og lys\\
    Må afsted, kom nu med
  \end{SBChorus}

%   \begin{SBChorus}
% Spas i waders i en sø\ldots
%   \end{SBChorus}

%   \begin{SBChorus}
% Spas i waders i en sø\ldots\\
% Må afsted, kom nu med
%   \end{SBChorus}

  \begin{SBVerse}
Du er feltbiolog udstyret med planktonnet\\
Tæller og nøgler fisk og zooplankton\\
Ude i søerne, der har jeg net med småfisk i\\
At nøgle og måle, det kan jeg li'
  \end{SBVerse}

  \begin{SBChorus}
Spas i waders i en sø\\
Fanger Myggelarver, Super biolog\\
Spas i waders i en sø\\
Se på makrofytter, mål pH og lys
  \end{SBChorus}

  \begin{SBChorus}
Spas i waders i en sø\\
Fanger Myggelarver, Super biolog\\
Spas i waders i en sø\\
Se på makrofytter, mål pH og lys
  \end{SBChorus}

%   \begin{SBChorus}
% Spas i waders i en sø\ldots
%   \end{SBChorus}

%   \begin{SBChorus}
% Spas i waders i en sø\ldots
%   \end{SBChorus}

  \begin{SBSection*}
Du ta'r afsted op til Hillerød\\
Jeg vil med, se på livet i søen\\
 -- Vi finder en ny plankton art\\
Du og jeg tilsammen kan vi alt
 -- Vi kan alt
  \end{SBSection*}

  \begin{SBChorus}
Spas i waders i en sø\\
Fanger Myggelarver, Super biolog\\
Spas i waders i en sø\\
Se på makrofytter, mål pH og lys\
  \end{SBChorus}

  \begin{SBChorus}
Spas i waders i en sø\\
Fanger Myggelarver, Super biolog\\
Spas i waders i en sø\\
Se på makrofytter, mål pH og lys\\
Lom nu med
  \end{SBChorus}

%   \begin{SBChorus}
% Spas i waders i en sø\ldots
%   \end{SBChorus}

%   \begin{SBChorus}
% Spas i waders i en sø\ldots
% Kom nu med!
%   \end{SBChorus}
\end{song}
\begin{song}{Nano}
  {} % Bruges ikke, lad stå blank
  {Melodi} % Titel, Kunstner - eks.: "Jutlandia, Kim Larsen". Hvis sangen er på sin egen melodi, brug da \SBOrgMel.
  {Forfatter} % Navnet på forfatteren. Undlad kaldenavne. Brug gerne TBF. Brug "&" frem for "og". Hvis forfatter er ukendt, lad da stå tom.
  {Anledning og år} % Eks. "Fysikrevy, 2010" eller "2010"
  {\NotCCLIed} % Lad stå som den er

  \begin{SBVerse}
Nano, Nano.\\
Har du set en Nano?\\
Fremtidens genier\\
kan stå samlet på en tier
  \end{SBVerse}

  \begin{SBVerse}
Nano, Na-na-na-Nano.\\
Er det ikke sandt? Jo!\\
Venlige mikrober.\\
De er så små, at man ikke\\
kan se dem - selv i mikroskoper.
  \end{SBVerse}

  \begin{SBChorus}
Nano. Vi elsker jer\\
I gør det hele letter’.\\
For I kan bygge fremtidens tabletter\\
(Na-na-na-no-na-no-na-no-na-no)\\
med atomer og nano-pincetter.
  \end{SBChorus}

  \begin{SBVerse}
Nano, Nano.\\
Søde lille Nano.\\
Bittesmå profeter\\
på $10^-9$ meter.
  \end{SBVerse}

  \begin{SBVerse}
Nano. Na-na-na-Nano.\\
Vogt dig for en Nano.\\
Fremtidens spioner.\\
De infiltrerer hvad som helst\\
ved hjælp af simple diffusioner.
  \end{SBVerse}

  \begin{SBChorus}
Nano. Vi elsker jeres\\
nuttede studiner.\\
For de kan lave bittesmå maskiner\\
(Na-na-na-no-na-no-na-no-na-no)\\
som kan vaccinere selv vacciner.
  \end{SBChorus}

  \begin{SBVerse}
Nano, åh nano!\\
Ingen kan som Nano\\
nanorere sproget:\\
Med nano-adaptive nano-\\
gloser kører Nanotoget.
  \end{SBVerse}

  \begin{SBChorus}
Nano, Vi elsker jer!\\
Mikro er blot et minde.\\
Og selvom I er ret svære at finde\\
(Be-Be-Besenbacher-Besenbacher)\\
ved vi, Nano-tiden den er inde.
  \end{SBChorus}
\end{song}
\begin{song}{Laborant}
  {} % Bruges ikke, lad stå blank
  {Nøddepatruljen, Disney} % Titel, Kunstner - eks.: "Jutlandia, Kim Larsen". Hvis sangen er på sin egen melodi, brug da \SBOrgMel.
  {Forfatter} % Navnet på forfatteren. Undlad kaldenavne. Brug gerne TBF. Brug "&" frem for "og". Hvis forfatter er ukendt, lad da stå tom.
  {MBK-revyen, 2013} % Eks. "Fysikrevy, 2010" eller "2010"
  {\NotCCLIed} % Lad stå som den er

  \begin{SBVerse}
    Gi'r geler problemer,
    og sejler dit projekt?
    Dit array er gay,
    din protokol er væk.
    Ja så kommer de og redder dig.
    Specialet blir' en leg!
  \end{SBVerse}

  \begin{SBChorus}
    La-la-la-la-bo-rant!
    De kan mixe
    La-la-la-la-bo-rant!
    Uden at kikse!
    For hvis du er I nød så kommer de,
    bar' husk at spørg' før de tager fri!
  \end{SBChorus}

  \begin{SBVerse}
    De blander buffer
    i hver koncentration
    Men kommer aldrig
    på en publikation!
    Oprens mit plasmid, bestil enzym,
    de er et sørg'ligt syn!
  \end{SBVerse}

  \begin{SBChorus}
    La-la-la-la-bo-rant
    De' en gave
    La-la-la-la-bo-ra-
    -torieslave!
    For de er svar på hver professors bøn,
    det' godt de får så lidt i løn!
  \end{SBChorus}

  \begin{SBChorus}
    La-la-la-la-bo-rant
    De' en gave
    La-la-la-la-bo-ra-
    -torieslave!
    For de er svar på hver professors bøn,
    det' godt de får så lidt i løn!
  \end{SBChorus}

  % \begin{SBChorus}
  %   La-la-la-la-bo-rant\ldots
  % \end{SBChorus}

  \begin{SBSection*}
    La-la-la-la-bo-rant!
  \end{SBSection*}
\end{song}
\begin{song}{Selektionssangen}
  {} % Bruges ikke, lad stå blank
  {Hodja fra Pjort, Sebastian} % Titel, Kunstner - eks.: "Jutlandia, Kim Larsen". Hvis sangen er på sin egen melodi, brug da \SBOrgMel.
  {} % Navnet på forfatteren. Undlad kaldenavne. Brug gerne TBF. Brug "&" frem for "og". Hvis forfatter er ukendt, lad da stå tom.
  {BioRevy, 2011} % Eks. "Fysikrevy, 2010" eller "2010"
  {\NotCCLIed} % Lad stå som den er

  \begin{SBVerse}
    Jeg så en hjort, jeg så en hjort,\\
    men den løb længere og længere bort,\\
    indtil jeg ramte den temmelig hårdt,\\
    kørende i min Ford.\\
    Jeg ved jo ik', hvad jeg ellers sku' ha gjort.
  \end{SBVerse}

  \begin{SBVerse}
    Jeg så en mår, jeg så en mår.\\
    Den var så nuttet og dækket med hår.\\
    Pelsen var filtret lidt li'som et får,\\
    men det jeg ik' forstår,\\
    hvordan min kam ku' gi så mange sår.
  \end{SBVerse}

  \begin{SBChorus}
    Naturens love\\
    er bare så sjove.\\
    Evolutionen,\\
    vi sidder på tronen\\
    med scepter\\
    og styr' selektionen!
  \end{SBChorus}

  \begin{SBVerse}
    Jeg så en kat, jeg så en kat.\\
    Aldrig har jeg hørt så voldsomt et splat.\\
    Pludselig lå der en kæmpestor klat\\
    lige der hvor jeg sat.\\
    Hvad sku jeg gøre, for helved' det var nat!
  \end{SBVerse}

  \begin{SBVerse}
    Jeg så en stær, jeg så en stær\\
    og da det kribled' i fingre og tær,\\
    måtte jeg simpelthen komme den nær,\\
    kæle den med mit sværd.\\
    Men det er mærk'ligt nu pipper den ej mer.
  \end{SBVerse}

  \begin{SBChorus}
    Naturens love\\
    er bare så sjove.\\
    Evolutionen,\\
    vi sidder på tronen\\
    med scepter\\
    og styr' selektionen!
  \end{SBChorus}

  % \begin{SBChorus}
  %   Naturens love\ldots
  % \end{SBChorus}

  \begin{SBVerse}
    Se den kanin, se den kanin.\\
    Den var så nuttet, så blød og så fin\\
    indtil jeg glemte den i min kamin.\\
    Jeg føler mig til grin.\\
    Hvorfor skulle den også dyppes i palmin?
  \end{SBVerse}

  \begin{SBVerse}
    Se den pingvin, se den pingvin.\\
    Jeg tror habitten er større end min!\\
    Geværet frem, tømmer mit magasin\\
    på det sort-hvide svin.\\
    Nu er der end'lig frisk mad i vor kantin'.
  \end{SBVerse}
\end{song}
\endsong
\begin{song}{Når jeg kloner min kat}
  {} % Bruges ikke, lad stå blank
  {When You're Looking Like That, Westlife} % Titel, Kunstner - eks.: "Jutlandia, Kim Larsen". Hvis sangen er på sin egen melodi, brug da \SBOrgMel.
  {Forfatter} % Navnet på forfatteren. Undlad kaldenavne. Brug gerne TBF. Brug "&" frem for "og". Hvis forfatter er ukendt, lad da stå tom.
  {BirRevy, 2012} % Eks. "Fysikrevy, 2010" eller "2010"
  {\NotCCLIed} % Lad stå som den er

  \begin{SBSection*}
    Når jeg kloner min kat
  \end{SBSection*}

  \begin{SBVerse}
    Med en ny teknik kan os på bio skabe liv\\
    Den ku' bruges ondt og klone Henrik Busch\\
    Til vi har en hær\\
    Det er ikke det, vi gør!\\
    Bruger teknikken til at redde arter, men\\
    Sker der noget skidt med dyret vi elsker, så\\
    Må mammutten vente\\
    Ikke tid til at grundforske
  \end{SBVerse}

  \begin{SBChorus}
    Ska' aldrig mere sig' farvel\\
    Når jeg kloner min kat\\
    Vil altid være perfekt, nuttet, sød\\
    For altid vær' min skat\\
    Aldrig købe mig en ny\\
    Aldrig købe et erstatningskæledyr\\
    Ska' aldrig mere sig farvel\\
    Når jeg kloner min kat
  \end{SBChorus}

  \begin{SBVerse}
    Jurassic Park var tyd'ligvis en succes\\
    Dinosaurus på jord ku' være fedt\\
    Men sig mig, hvordan gør man det?\\
    Indsæt gen og skab superarter\\
    Nem vej til at skabe über seje dyr\\
    Men når katten forsvinder, har vi et hyr\\
    Så mammutten venter\\
    ikke tid til at grundforske
  \end{SBVerse}

  \begin{SBChorus}
    Ska' aldrig mere sig' farvel\\
    Når jeg kloner min kat\\
    Vil altid være perfekt, nuttet, sød\\
    For altid vær' min skat\\
    Aldrig købe mig en ny\\
    Aldrig købe et erstatningskæledyr\\
    Ska' aldrig mere sig farvel\\
    Når jeg kloner min kat
  \end{SBChorus}

%   \begin{SBChorus}
% Ska' aldrig mere sig' farvel\ldots
%   \end{SBChorus}

  \begin{SBSection*}
    Skal jeg så spørge jer nu\\
    Hvilket studie har magt?\\
    Biologer har livets kraft\\
    Så jeg kloner min kat
  \end{SBSection*}

  \begin{SBChorus}
    Aldrig mere sig' farvel\\
    Når jeg kloner min kat\\
    Vil altid være perfekt, nuttet, sød\\
    For altid vær' min skat\\
    Aldrig købe mig en ny\\
    Aldrig købet et erstatningskæledyr\\
    Ska' aldrig mere sig farvel\\
    Når jeg kloner min kat
  \end{SBChorus}

  \begin{SBChorus}
    Ska' aldrig mere sig' farvel\\
    Når jeg kloner min kat\\
    Vil altid være perfekt, nuttet, sød\\
    For altid vær' min skat\\
    Aldrig købe mig en ny\\
    Aldrig købe et erstatningskæledyr\\
    Ska' aldrig mere sig farvel\\
    Når jeg kloner min kat
  \end{SBChorus}

  \begin{SBChorus}
    Ska' aldrig mere sig' farvel\\
    Når jeg kloner min kat\\
    Vil altid være perfekt, nuttet, sød\\
    For altid vær' min skat\\
    Aldrig købe mig en ny\\
    Aldrig købe et erstatningskæledyr\\
    Ska' aldrig mere sig farvel\\
    Når jeg kloner min kat
  \end{SBChorus}

%   \begin{SBChorus}
% Ska' aldrig mere sig' farvel\ldots
%   \end{SBChorus}

%   \begin{SBChorus}
% Ska' aldrig mere sig' farvel\ldots
%   \end{SBChorus}
\end{song}
\begin{song}{Pest}
  {} % Bruges ikke, lad stå blank
  {Melodi} % Titel, Kunstner - eks.: "Jutlandia, Kim Larsen". Hvis sangen er på sin egen melodi, brug da \SBOrgMel.
  {Forfatter} % Navnet på forfatteren. Undlad kaldenavne. Brug gerne TBF. Brug "&" frem for "og". Hvis forfatter er ukendt, lad da stå tom.
  {Anledning og år} % Eks. "Fysikrevy, 2010" eller "2010"
  {\NotCCLIed} % Lad stå som den er

  \begin{SBVerse}
    % Skriv vers her
  \end{SBVerse}

  \begin{SBChorus}
    % Skriv omkvæd her
  \end{SBChorus}

  \begin{SBSection*}
    % Skriv sektioner her. Hvis du ønsker lidt mellemrum for at give luft i et langt afsnit el.lign., brug da \\\medskip
  \end{SBSection*}
\end{song}

\onecolumn
\chapter*{Alkohol og druk}
\twocolumn
\begin{song}{Busted}
  {} % Bruges ikke, lad stå blank
  {Melodi} % Titel, Kunstner - eks.: "Jutlandia, Kim Larsen". Hvis sangen er på sin egen melodi, brug da \SBOrgMel.
  {Forfatter} % Navnet på forfatteren. Undlad kaldenavne. Brug gerne TBF. Brug "&" frem for "og". Hvis forfatter er ukendt, lad da stå tom.
  {Anledning og år} % Eks. "Fysikrevy, 2010" eller "2010"
  {\NotCCLIed} % Lad stå som den er

  \begin{SBVerse}
    % Skriv vers her
  \end{SBVerse}

  \begin{SBChorus}
    % Skriv omkvæd her
  \end{SBChorus}

  \begin{SBSection*}
    % Skriv sektioner her. Hvis du ønsker lidt mellemrum for at give luft i et langt afsnit el.lign., brug da \\\medskip
  \end{SBSection*}
\end{song}
\begin{song}{Der er et ølrigt land}
  {} % Bruges ikke, lad stå blank
  {Melodi} % Titel, Kunstner - eks.: "Jutlandia, Kim Larsen". Hvis sangen er på sin egen melodi, brug da \SBOrgMel.
  {Forfatter} % Navnet på forfatteren. Undlad kaldenavne. Brug gerne TBF. Brug "&" frem for "og". Hvis forfatter er ukendt, lad da stå tom.
  {Anledning og år} % Eks. "Fysikrevy, 2010" eller "2010"
  {\NotCCLIed} % Lad stå som den er

  \begin{SBVerse}
    % Skriv vers her
  \end{SBVerse}

  \begin{SBChorus}
    % Skriv omkvæd her
  \end{SBChorus}

  \begin{SBSection*}
    % Skriv sektioner her. Hvis du ønsker lidt mellemrum for at give luft i et langt afsnit el.lign., brug da \\\medskip
  \end{SBSection*}
\end{song}

% \beginsong{Der er et ølrigt land}[sr={Melodi: Der er et yndigt land}
% ,
% by={}
% ,
% cr={}]
% \beginverse
% Der er et ølrigt land,
% det står med nød og næppe
% blandt alt det pokkers vand
% blandt alt det pokkers vand.
% Det bugter sig i bar og kro.
% Det hedder gamle Danmark,
% og her er øllen go'
% ja, hver en øl er go'.

% \endverse
% \beginverse
% Her drak i fordums tid
% hver tillakkede kæmper
% sin mjød af fad med flid
% sin mjød af fad med flid.
% Så prøved' han, ej uden mén
% at finde sine bene,
% men faldt ved hver en sten
% ja, hver en bautasten.

% \endverse
% \beginverse
% Den øl endnu er skøn,
% og gid den aldrig vælter.
% Lad baj'ren stå så grøn
% lad baj'ren stå så grøn.
% De ædle sorters skønne øer
% med sutter, sulde svende
% og svimle danske møer
% ja, svimle danske møer.

% \endverse
% \beginverse
% Hil druk og fædreland.
% Hil hver en kølig bajer.
% Vi drikker dem vi kan
% vi drikker dem vi kan.
% Vort gamle Danmark -- SKÅL! -- bestå
% så længe øllet skummer,
% og næsen den bli'r blå
% med røde prikker på.

% \endverse
% \endsong


\onecolumn
\chapter*{Bare gode sange}
\twocolumn
\begin{song}{Puma (Stork)}{}
  {\SBOrgMel}
  {Roben og Knud}
  {2001}
  {\NotCCLIed}

  \begin{SBVerse}
    Jeg vil gerne ha' en stork\\
    Men dens næb sku' være kort\\
    \SBRepeat{Jeg synes en stork er grim med langt næb}
  \end{SBVerse}

  \begin{SBVerse}
    Jeg vil gerne ha' en stork\\
    Men dens hals sku' være kort\\
    \SBRepeat{Jeg synes en stork er grim med lang hals}
  \end{SBVerse}

  \begin{SBSection*}
    Og med et kort næb\\
    Og med en kort hals\\
    Måske med lidt gullig pels\\
    Så ville det ligne en puma
  \end{SBSection*}

  \begin{SBChorus}
    \SBRepeat{Måske ik' den sejeste puma verden har set,\\
    Med dog en puma, og hurra for det!}
  \end{SBChorus}

  \begin{SBVerse}
    Jeg vil gerne ha' en stork\\
    Men to ben er bare ikke nok\\
    \SBRepeat{Jeg synes en stork er grim med to ben}
  \end{SBVerse}

  \begin{SBSection*}
    Og med et kort næb\\
    Og med en kort hals\\
    Måske med lidt gullig pels\\
    Så ville det ligne en puma
  \end{SBSection*}

  \begin{SBChorus}
    Måske ik' den sejeste puma verden har set\ldots
  \end{SBChorus}

  \begin{SBSection*}
    Og med et kort næb\\
    Og med en kort hals\\
    Måske med lidt gullig pels\\
    Så ville det ligne en puma
  \end{SBSection*}

  \begin{SBChorus}
    Måske ik' den sejeste puma verden har set\ldots
  \end{SBChorus}

  \begin{SBSection*}
    \SBRepeat{\SBRepeat{\SBRepeat{Hurra for det}}}
  \end{SBSection*}

\end{song}
\begin{song}{All Star}
  {} % Bruges ikke, lad stå blank
  {Melodi} % Titel, Kunstner - eks.: "Jutlandia, Kim Larsen". Hvis sangen er på sin egen melodi, brug da \SBOrgMel.
  {Forfatter} % Navnet på forfatteren. Undlad kaldenavne. Brug gerne TBF. Brug "&" frem for "og". Hvis forfatter er ukendt, lad da stå tom.
  {Anledning og år} % Eks. "Fysikrevy, 2010" eller "2010"
  {\NotCCLIed} % Lad stå som den er

  \begin{SBVerse}
    % Skriv vers her
  \end{SBVerse}

  \begin{SBChorus}
    % Skriv omkvæd her
  \end{SBChorus}

  \begin{SBSection*}
    % Skriv sektioner her. Hvis du ønsker lidt mellemrum for at give luft i et langt afsnit el.lign., brug da \\\medskip
  \end{SBSection*}
\end{song}
\begin{song}{Never Gonna Give You Up}
  {} % Bruges ikke, lad stå blank
  {Melodi} % Titel, Kunstner - eks.: "Jutlandia, Kim Larsen". Hvis sangen er på sin egen melodi, brug da \SBOrgMel.
  {Forfatter} % Navnet på forfatteren. Undlad kaldenavne. Brug gerne TBF. Brug "&" frem for "og". Hvis forfatter er ukendt, lad da stå tom.
  {Anledning og år} % Eks. "Fysikrevy, 2010" eller "2010"
  {\NotCCLIed} % Lad stå som den er

  \begin{SBVerse}
    We're no strangers to love\\
    You know the rules and so do I\\
    A full commitment's what I'm thinking of\\
    You wouldn't get this from any other guy\\\medskip
    I just wanna tell you how I'm feeling\\
    Gotta make you understand
  \end{SBVerse}

  \begin{SBChorus}
    Never gonna give you up\\
    Never gonna let you down\\
    Never gonna run around and desert you\\
    Never gonna make you cry\\
    Never gonna say goodbye\\
    Never gonna tell a lie and hurt you
  \end{SBChorus}

  \begin{SBVerse}
    We've known each other for so long\\
    Your heart's been aching but you're too shy to say it\\
    Inside we both know what's been going on\\
    We know the game and we're gonna play it\\\medskip
    And if you ask me how I'm feeling\\
    Don't tell me you're too blind to see
  \end{SBVerse}

  \begin{SBChorus}
    Never gonna give you up\ldots
  \end{SBChorus}

  \begin{SBChorus}
    Never gonna give you up\ldots
  \end{SBChorus}

  \begin{SBSection*}
    Never gonna give, never gonna give\\
    \emph{(Give you up)}\\
    Never gonna give, never gonna give\\
    \emph{(Give you up)}
  \end{SBSection*}

  \begin{SBVerse}
    We've known each other for so long\\
    Your heart's been aching but you're too shy to say it\\
    Inside we both know what's been going on\\
    We know the game and we're gonna play it\\\medskip
    And if you ask me how I'm feeling\\
    Don't tell me you're too blind to see
  \end{SBVerse}

  \begin{SBChorus}
    Never gonna give you up\ldots
  \end{SBChorus}

  \begin{SBChorus}
    Never gonna give you up\ldots
  \end{SBChorus}
\end{song}
\begin{song}{En elefant kom marcherende}
  {} % Bruges ikke, lad stå blank
  {Melodi} % Titel, Kunstner - eks.: "Jutlandia, Kim Larsen". Hvis sangen er på sin egen melodi, brug da \SBOrgMel.
  {Forfatter} % Navnet på forfatteren. Undlad kaldenavne. Brug gerne TBF. Brug "&" frem for "og". Hvis forfatter er ukendt, lad da stå tom.
  {Anledning og år} % Eks. "Fysikrevy, 2010" eller "2010"
  {\NotCCLIed} % Lad stå som den er

  \begin{SBVerse}
    % Skriv vers her
  \end{SBVerse}

  \begin{SBChorus}
    % Skriv omkvæd her
  \end{SBChorus}

  \begin{SBSection*}
    % Skriv sektioner her. Hvis du ønsker lidt mellemrum for at give luft i et langt afsnit el.lign., brug da \\\medskip
  \end{SBSection*}
\end{song}
\begin{song}{Den rekursive skovsang}
  {} % Bruges ikke, lad stå blank
  {\SBOrgMel} % Titel, Kunstner - eks.: "Jutlandia, Kim Larsen". Hvis sangen er på sin egen melodi, brug da \SBOrgMel.
  {} % Navnet på forfatteren. Undlad aliasser. Brug "&" frem for "og". Hvis forfatter er ukendt, lad da stå tom.
  {} % Eks. "Fysikrevy 2010" eller "2010"
  {\NotCCLIed} % Lad stå som den er

  \begin{SBVerse}
    Langt ude i skoven lå en lille skov.\\
    Aldrig så jeg så dejlig en skov.\\
    Skoven ligger langt ude i skoven
  \end{SBVerse}
  \renewcommand{\theSBVerseCnt}{$n>0$}
  
  \begin{SBVerse}
    Og i den lille skov, der lå en lille skov.\\
    Aldrig så jeg så dejlig en skov.\\
    \{Skoven i skoven,\}$^{n}$\\
    skoven ligger langt ude i skoven
  \end{SBVerse}
\end{song}
\begin{song}{Jeg ved en lærkerede}
  {} % Bruges ikke, lad stå blank
  {\SBOrgMel} % Titel, Kunstner - eks.: "Jutlandia, Kim Larsen". Hvis sangen er på sin egen melodi, brug da \SBOrgMel.
  {Harald Bergstedt, Carl Nielsen, slettelak} % Navnet på forfatteren. Undlad aliasser. Brug "&" frem for "og". Hvis forfatter er ukendt, lad da stå tom.
  {} % Eks. "Fysikrevy 2010" eller "2010"
  {\NotCCLIed} % Lad stå som den er

  \begin{SBVerse}
    Jeg ved en lærkerede,\\
    jeg siger ikke mer';
  \end{SBVerse}
\end{song}



% 
\begin{song}{What A Mighty God We Serve}{C}
  {\SBOrgMel}
  {}
  {Isaiah~9:6}
  {\NotCCLIed}

  \renewcommand{\RevDate}{February~11,~1993}
  \SBRef{Hosanna! Music Book~I}{\#93}

  \begin{SBOpGroup}
    \Ch{C}{What} a mighty God we serve,
    
    What a mighty God we \Ch{G7}{serve},
    
    \Ch{C}{An}gels bow before Him,
    
    \Ch{C}{Hea}ven and earth adore Him,
    
    \Ch{C}{What} a mighty \Ch{G7}{God} we \Ch{C}{serve!}\Ch{[}{}\Ch{F}{} \Ch{C}{}\Ch{]}{}
  \end{SBOpGroup}

  \begin{SBVerse}
    O \Ch{C}{Zion,} O \Ch{F}{Zion,} that \Ch{G7}{bring}est good \Ch{C}{tid}ings,

    Get thee \Ch{F}{up} into the \Ch{G7}{High} Moun\Ch{C}{tains}

    Je\Ch{C}{ru}salem, Je\Ch{F}{ru}salem, that \Ch{G7}{bring}est good \Ch{C}{tid}ings

    Lift up thy \Ch{F}{voice} with \Ch{G7}{all} thy \Ch{C}{strength}

    Lift it \Ch{F}{up,} be not afraid;

    Lift it \Ch{C}{up,} be not afraid

    Say \Ch{Am}{unto} the \Ch{C}{ci}ties of \Ch{G7}{Judah,}

    ``Behold your \Ch{C}{God,}\Ch{C7}{} Behold your \Ch{F}{God,}

    Be\Ch{C}{hold} \Ch{G7}{your} \Ch{C}{God!''}
  \end{SBVerse}

  \begin{SBExtraKeys}{
  \CBPageBrk
  \CSColBrk
    \STitle{What A Mighty God We Serve}{D}

    \begin{SBOpGroup}
      \Ch{D}{What} a mighty God we serve,
        
      What a mighty God we \Ch{A7}{serve},
      
      \Ch{D}{An}gels bow before Him,
      
      \Ch{D}{Hea}ven and earth adore Him,
      
      \Ch{D}{What} a mighty \Ch{A7}{God} we \Ch{D}{serve!}\Ch{[}{}\Ch{G}{} \Ch{D}{}\Ch{]}{}
    \end{SBOpGroup}

    \begin{SBVerse}
      O \Ch{D}{Zion,} O \Ch{G}{Zion,} that \Ch{A7}{bring}est good \Ch{D}{tid}ings,
                
      Get thee \Ch{G}{up} to into the \Ch{A7}{High} Moun\Ch{D}{tains}

      Je\Ch{D}{ru}salem, Je\Ch{G}{ru}salem, that \Ch{A7}{bring}est good \Ch{D}{tid}ings

      Lift up thy \Ch{G}{voice} with \Ch{A7}{all} thy \Ch{D}{strength}

      Lift it \Ch{G}{up} be not afraid,

      Lift it \Ch{D}{up} be not afraid

      Say \Ch{Bm}{unto} the \Ch{D}{ci}ties of \Ch{A7}{Judah,}

      ``Behold your \Ch{D}{God,}\Ch{D7}{} Behold your \Ch{G}{God,}

      Be\Ch{D}{hold} \Ch{A7}{your} \Ch{D}{God!''}
    \end{SBVerse}
  }\end{SBExtraKeys}
\end{song}


\begin{song}{Finally, My Brethren}{D}
  {}
  {}
  {Ephesians~6:10}
  {\NotCCLIed}

  \renewcommand{\RevDate}{February~11,~1993}
  %\SBRef{}{}

  \begin{SBVerse}
    \Ch{D}{Fi}nally, my brethren, be \Ch{C}{strong} in the \Ch{D}{Lord,}

    \Ch{D}{Fi}nally, my brethren, be \Ch{C}{strong} in the \Ch{D}{Lord,}

    And in the \Ch{G}{pow}er \Ch{A}{of} His \Ch{D}{might,}

    And in the \Ch{G}{pow}er \Ch{A}{of} His \Ch{D}{might.}

    For it is \Ch{G}{God} at \Ch{A}{work} with\Ch{D}{in} you,

    Both to \Ch{G}{will} \Ch{A}{and} to \Ch{D}{do,}

    \Ch{G}{All} his \Ch{A}{glo}rious \Ch{D}{plea}sure.

    His \Ch{G}{\em truth} shall \Ch{A}{sur}ely en\Ch{D}{dure.}
  \end{SBVerse}

  \begin{SBVerse}
    {\em \ldots strength \ldots}
  \end{SBVerse}

  \begin{SBVerse}
    {\em \ldots mercy \ldots}
  \end{SBVerse}

  \begin{SBVerse}
    {\em \ldots grace \ldots}
  \end{SBVerse}

  \begin{SBVerse}
    \Ch{D}{Fi}nally my brethren be \Ch{C}{strong} in the \Ch{D}{Lord,}
    
    \Ch{D}{Fi}nally my brethren be \Ch{C}{strong} in the \Ch{D}{Lord,}

    And in the \Ch{G}{pow}er \Ch{A}{of} His \Ch{D}{might,}

    And in the \Ch{G}{pow}er \Ch{A}{of} His \Ch{D}{might,}

    And in the \Ch{G}{pow}er \Ch{A}{of} His \Ch{D}{might.}
  \end{SBVerse}
\end{song}


\begin{song}[C]{The Lord Of Covenant}{Am}
  {}
  {}
  {}
  {\NotCCLIed}

  \renewcommand{\RevDate}{February~11,~1993}
  %\SBRef{}{}
  \FLineIdx{Lift your hands, for He is Holy}

  \begin{SBOpGroup}
    \Ch{Am}{Lift} your hands, for \Ch{E}{He} is Holy.
    
    \Ch{Dm7}{Worship} Him in \Ch{E}{spir}it and \Ch{Am}{truth.}
    
    For His face, is \Ch{E}{like} the sun,
    
    \Ch{Dm7}{The} Lord of \Ch{E}{Cove}\Ch{Am}{nant.}
  \end{SBOpGroup}

  \begin{SBVerse}
    \Ch{Dm}{Clap} your hands and \Ch{G}{shout} in victo\Ch{C}{ry,}
    
    For the Lamb was \Ch{E}{slain} but lives.

    \Ch{Dm7}{Dance} for joy and \Ch{G}{give} Him thanks.

    {\SBLyricNoteFont (Men)} For the \Ch{C}{free}dom \ldots

    {\SBLyricNoteFont (Women)} For the freedom His \Ch{E}{life} gives.

    {\SBLyricNoteFont (Men)} For the \Ch{F}{free}dom \ldots

    {\SBLyricNoteFont (Women)} For the freedom His \Ch{E}{life} gives.
  \end{SBVerse}

  \begin{SBVerse}
    \Ch{Dm}{Come} you people, \Ch{G}{lift} your voices, \Ch{C}{}

    Praise Him with your \Ch{E}{hearts} as one.

    \Ch{Dm7}{There} is One who \Ch{G}{turn}ed the key.

    {\SBLyricNoteFont (Men)} And the \Ch{C}{door} \ldots

    {\SBLyricNoteFont (Women)} To His throne is \Ch{E}{open.}

    {\SBLyricNoteFont (Men)} Yes, the \Ch{F}{door} \ldots

    {\SBLyricNoteFont (Women)} To His throne is \Ch{E}{open.}
  \end{SBVerse}

  \CSColBrk
  \begin{SBVerse}
    \Ch{Dm}{He} creates Je\Ch{G}{rusa}lem for re\Ch{C}{joic}ing,

    For re\Ch{E}{joic}ing,

    \Ch{Dm7}{And} her people \Ch{G}{for} gladness.

    {\SBLyricNoteFont (Men)} \Ch{C}{He} \ldots

    {\SBLyricNoteFont (Women)} He is glad in His \Ch{E}{peo}ple.

    {\SBLyricNoteFont (Men)} Yes, \Ch{F}{He} \ldots

    {\SBLyricNoteFont (Women)} He is glad in His \Ch{E}{peo}ple.
  \end{SBVerse}
\end{song}


\begin{song}{We Will Rejoice}{D}
  {}
  {}
  {Song Of Solomon~1:4}
  {\NotCCLIed}

  \renewcommand{\RevDate}{February~11,~1993}
  \SBRef{Hosanna! Music Book~II}{\#143}

  \begin{SBOpGroup}
    \Ch{D}{We} will rejoice in \Ch{G}{You} and be glad,
  
    \Ch{D}{We} will extol Your \Ch{A7}{love} more than wine
        
    \Ch{D}{Draw} me after \Ch{D7}{You} and let us \Ch{G}{run} together,
    
    \Ch{D}{We} will rejoice in \Ch{G}{You} and \Ch{A7}{be} \Ch{D}{glad}
  \end{SBOpGroup}

  \begin{SBChorus}
    \Ch{D}{Lift} up the light of Thy \Ch{G}{coun}tenance,
    
    U\Ch{D}{pon} us O \Ch{A7}{Lord}

    \Ch{D}{Lift} up the light of Thy \Ch{G}{coun}tenance,

    U\Ch{D}{pon} \Ch{A7}{us} O \Ch{D}{Lord} \Ch{G}{} \Ch{D}{}
  \end{SBChorus}
\end{song}


\begin{song}{Praise Him In His Sanctuary}{Em}
  {}
  {}
  {Psalm~150}
  {\NotCCLIed}

  \renewcommand{\RevDate}{February~11,~1993}
  %\SBRef{}{}

  \begin{SBVerse}
    Praise Him \Ch{Em}{in} His sanctuary

    Praise Him \Ch{Em}{in} the skies above

    Praise Him \Ch{G}{for} the acts of \Ch{D}{pow}er that He \Ch{Em}{does}

    Praise Him \Ch{Em}{for} surpassing greatness

    With the \Ch{Em}{trum}pet, harp and lyre

    With the \Ch{G}{tam}bourines and \Ch{Am}{dan}cing, praise Him \Ch{Em}{now}
  \end{SBVerse}

  \begin{SBOpGroup}
    Come and \Ch{G}{praise} \Ch{D}{Him} \Ch{Bm}{for} the Lord is \Ch{Em}{good}
  
    And His mercy is \Ch{B7}{ever}\Ch{Em}{lasting}
    
    Come and \Ch{G}{praise} \Ch{D}{Him} \Ch{Bm}{for} the Lord is \Ch{Em}{good}
    
    And His mercy is \Ch{B7}{ever}\Ch{Em}{lasting}
  \end{SBOpGroup}

  \begin{SBVerse}
    Praise Him \Ch{Em}{with} the clashing cymbals

    Let them \Ch{Em}{hear} it far and near

    With the \Ch{G}{strings} and flutes we'll \Ch{D}{praise} the Lord our \Ch{Em}{God}

    Who with \Ch{Em}{majes}ty is reigning

    He has \Ch{Em}{power} over all

    Give Him \Ch{G}{glory} and be \Ch{Am}{thankful} for His \Ch{Em}{love}
  \end{SBVerse}
\end{song}


\begin{song}{Rise Up}{Em}
  {}
  {}
  {}
  {\NotCCLIed}

  \renewcommand{\RevDate}{February~11,~1993}
  %\SBRef{}{}

  \begin{SBOpGroup}
    Rise \Ch{Em}{up,} rise up, we are the \Ch{Am}{soldiers} of the \Ch{Em}{cross}
    
    We are the \Ch{Am}{ones} who are to glorify the \Ch{B7}{King}
    
    Cre\Ch{Em}{ation} groans for the \Ch{Am}{Sons} of God to \Ch{Em}{come}
    
    Mani\Ch{Am}{fest}ing all the glory of the \Ch{B7}{King}
    
    So \Ch{Am}{rise} and shine, for your \Ch{Em}{light} has come,
    
    And the \Ch{Am}{glor}y of the Lord is u\Ch{Em}{pon} you
    
    Lift \Ch{Am}{up} your eyes round a\Ch{Em}{bout} and see
    
    All the \Ch{Am}{nations} are falling at your \Ch{B7}{feet}
  \end{SBOpGroup}

  \begin{SBChorus}
    So let's \Ch{Em}{glor}ify the Lord,

    Let's \Ch{Am}{glor}ify the \Ch{Em}{King}

    Let's shout and sing as we \Ch{Am}{take} the victo\Ch{B7}{ry}

    So let's \Ch{Em}{glor}ify the Lord,

    Let's \Ch{Am}{glor}ify the \Ch{Em}{King}

    Let's \Ch{Am}{shout} and \Ch{B7}{take} the victo\Ch{Em}{ry!}
  \end{SBChorus}
\end{song}


\begin{song}{We Are Marching In Messiah's Band}{Am}
  {}
  {}
  {}
  {\NotCCLIed}

  \renewcommand{\RevDate}{February~11,~1993}
  %\SBRef{}{}

  \begin{SBVerse}
    We are \Ch{Am}{march}ing, in Messiah's band;

    The keys of \Ch{Am}{vict'ry} in His mighty hand;

    Let us \Ch{Dm}{go} on to take the promised \Ch{Am}{land.}

    Raise the \Ch{Am}{anthem,} sing the victory song;

    Praise the \Ch{Am}{Lord} for the battle's won;

    No \Ch{Dm}{wea}pon formed against Him shall \Ch{Am}{stand.}
  \end{SBVerse}

  \begin{SBOpGroup}
    For the \Ch{F}{Cap}tain of the Host is \Ch{C}{Je}sus;
    
    We are \Ch{F}{fol}lowing in His \Ch{C}{foot}steps.
    
    No \Ch{F}{foe} can stand a\Ch{Am}{gainst} us in the \Ch{F}{fray.} \Ch{G}{}
  \end{SBOpGroup}
\end{song}


\begin{song}{A Song Of Love}{G}
  {}
  {}
  {}
  {\NotCCLIed}

  \renewcommand{\RevDate}{February~11,~1993}
  %\SBRef{}{}

  \begin{SBOpGroup}
    A \Ch{G}{Song} of \Ch{Em}{Love,} I \Ch{Am7}{sing} to \Ch{D}{You}
    
    A \Ch{G}{Song} of \Ch{Em}{ado}ration, \Ch{Am7}{to} the \Ch{D}{King}
    
    You \Ch{Am7}{are} the \Ch{D}{Lord,} the \Ch{G}{Ma}ker \Ch{G/F#}{of} the \Ch{Em}{universe}
    
    You \Ch{Am7}{are} the \Ch{D}{King} of all \Ch{G}{Kings}
  \end{SBOpGroup}
\end{song}


\begin{song}{Hail To The King}{D}
  {1989 Christopher Rath}
  {Christopher Rath}
  {}
  {\NotCCLIed}

  \renewcommand{\RevDate}{November~22,~1993}
  \SBRef{New Wine Covenant Church Song Book}{\#70}
  \FLineIdx{I praise You, Lord, for You are my King}

  \begin{SBVerse}
    I \Ch{D}{praise} You, Lord, for \Ch{C}{You} are \Ch{G}{my} \Ch{D}{King}
    
    I \Ch{D}{praise} You, Lord, for \Ch{C}{You} are \Ch{G}{my} \Ch{D}{King}
  \end{SBVerse}

  \begin{SBOpGroup}
    All \Ch{C}{hail} to\Ch{G}{} the \Ch{D}{King,}
    
    All \Ch{C}{hail} to\Ch{G}{} King Je\Ch{D}{\SBen sus}
    
    All \Ch{C}{hail} to\Ch{G}{} the \Ch{D}{King} of Kings,
    
    All \Ch{C}{hail} to\Ch{G}{} the \Ch{D}{King}
  \end{SBOpGroup}

  \begin{SBVerse}
    We \Ch{D}{praise} You, Lord, For \Ch{C}{You} are \Ch{G}{our} \Ch{D}{King}

    We \Ch{D}{praise} You, Lord, For \Ch{C}{You} are \Ch{G}{our} \Ch{D}{King}
  \end{SBVerse}

  \begin{SBVerse}
    I \Ch{D}{wor}ship You, my \Ch{C}{Lord} and \Ch{G}{my} \Ch{D}{God}

    I \Ch{D}{wor}ship You, my \Ch{C}{Lord} and \Ch{G}{my} \Ch{D}{God}
  \end{SBVerse}
\end{song}


\WBPageBrk
\begin{song}{The Horse And Rider}{C}
  {1950 Mills Music (music only)}
  {Anonymous and I.~Miron \&~J.~Grossman}
  {Exodus~15:1--2}
  {\NotCCLIed}

  \renewcommand{\RevDate}{February~11,~1993}
  \SBRef{Songs Of Praise}{\#214}
  \FLineIdx{I will sing unto the Lord, for He has}

  \ifChordBk
    {\SBLyricNoteFont This song can be sung as a round.  Each part of the
      round is indicated below as a verse.}
  \fi

  \begin{SBVerse}
    \Ch{C}{I} will sing unto the Lord, for \Ch{F}{He} has triumphed \Ch{Dm}{glorious}ly
                
    The \Ch{G7}{horse} and rider thrown into the \Ch{C}{sea}

    \Ch{C}{I} will sing unto the Lord, for \Ch{F}{He} has triumphed \Ch{Dm}{glorious}ly

    The \Ch{G7}{horse} and rider thrown into the \Ch{C}{sea}
  \end{SBVerse}
        
  \begin{SBVerse}
    The \Ch{C}{Lord,} my God, my \Ch{F}{strength,} my \Ch{Dm}{song,}

    Is \Ch{G7}{now} become my victo\Ch{C}{ry}

    The \Ch{C}{Lord,} my God, my \Ch{F}{strength,} my \Ch{Dm}{song,}

    Is \Ch{G7}{now} become my victo\Ch{C}{ry}
  \end{SBVerse}
  
  \begin{SBVerse}
    The \Ch{C}{Lord} is God and \Ch{F}{I} will \Ch{Dm}{praise} Him,

    My \Ch{G7}{Father's} God, and \Ch{C}{I} will ex\Ch{G7}{alt} Him!

    The \Ch{C}{Lord} is God and \Ch{F}{I} will \Ch{Dm}{praise} Him,

    My \Ch{G7}{Father's} God, and \Ch{C}{I} will ex\Ch{C}{alt} Him!
  \end{SBVerse}
\end{song}


\begin{song}{Nothing But The Blood}{D}
  {\SBOrgMel}
  {Robert Lowry}
  {Ephesians~1:7}
  {\NotCCLIed}

  \renewcommand{\RevDate}{February~11,~1993}
  \SBRef{Hosanna! Music Book~I}{\#65}
  \FLineIdx{What can wash away my sin}

  \begin{SBVerse}
    \Ch{D}{What} can wash a\Ch{F#m}{way} my \Ch{Bm}{sin?}

    \Ch{G}{Noth}ing but the \Ch{D}{blood} of \Chr{A7}{Je}\Ch{D}{sus.}

    \Ch{D}{What} can make me \Ch{F#m}{whole} a\Ch{Bm}{gain?}

    \Ch{G}{Noth}ing but the \Ch{D}{blood} of \Chr{A7}{Je}\Ch{D}{sus.}
  \end{SBVerse}

  \begin{SBOpGroup}
    \Ch{D}{Oh,} Precious \Ch{F#m}{is} the \Ch{Bm}{flow}
  
    \Ch{G}{That} \Ch{Em}{makes} me \Ch{A}{white} as \Ch{D}{snow.}\Ch{A7}{}
    
    \Ch{D}{No} other \Ch{F#m}{fount} I \Ch{Bm}{know,}
    
    \Ch{G}{Noth}ing but the \Ch{D}{blood} of \Chr{A7}{Je}\Ch{D}{sus.}
  \end{SBOpGroup}

  \begin{SBVerse}
    \Ch{D}{This} is all my \Ch{F#m}{righteous}\Ch{Bm}{ness,}

    \Ch{G}{Noth}ing but the \Ch{D}{blood} of \Chr{A7}{Je}\Ch{D}{sus.}

    \Ch{D}{This} is all my \Ch{F#m}{hope} and \Ch{Bm}{peace,}

    \Ch{G}{Noth}ing but the \Ch{D}{blood} of \Chr{A7}{Je}\Ch{D}{sus.}
  \end{SBVerse}
\end{song}


\begin{song}{Far Above All Rule And Authority}{C}
  {}
  {}
  {Ephesians~1:20--21;~2:6}
  {\NotCCLIed}

  \renewcommand{\RevDate}{February~11,~1993}
  %\SBRef{}{}

  \begin{SBOpGroup}
    Far a\Ch{C}{bove} all rule and au\Ch{F}{thor}ity, and \Ch{C}{po}wer and do\Ch{G7}{min}\Ch{C}{ion}
    
    Far a\Ch{C}{bove} all rule and au\Ch{F}{thor}ity, and \Ch{C}{po}wer and do\Ch{G7}{min}\Ch{C}{ion}
    
    And every \Ch{F}{name} that is \Ch{G7}{nam}ed, not \Ch{C}{on}ly in this age
    
    But \Ch{F}{al}so in the \Ch{G7}{one} to \Ch{C}{come}
    
    And every \Ch{F}{name} that is \Ch{G7}{nam}ed, not \Ch{C}{on}ly in this age
    
    But \Ch{F}{al}so in the \Ch{G7}{one} to \Ch{C}{come}
    
    Christ is \Ch{C}{seat}\Ch{G}{ed} at the \Ch{C}{Fa}ther's right hand,
    
    In \Chr{F}{hea}\Chr{G7}{}venly \Ch{C}{pla}ces.
    
    We are \Ch{C}{seat}ed with \Ch{G}{Him,} at the \Ch{C}{Fa}ther's right hand,
    
    In \Chr{F}{hea}\Chr{G7}{}venly \Ch{C}{pla}ces.
  \end{SBOpGroup}
\end{song}


\begin{song}{Sing Hallelujah}{D}
  {}
  {}
  {}
  {\NotCCLIed}

  \renewcommand{\RevDate}{February~11,~1993}
  \SBRef{Scripture In Song Volume~II}{\#167}

  \begin{SBOpGroup}
    \Ch{D}{Sing} Halle\Ch{G}{lu}jah, \Ch{D}{sing} halle\Ch{A7}{lujah,}
    
    \Ch{D}{sing} halle\Ch{G}{lu}jah, there's \Ch{D}{joy} \Ch{A7}{in} the \Ch{D}{Lord}
  \end{SBOpGroup}

  \begin{SBVerse}
    There \Ch{D}{is} a king in \Ch{G}{Zion,}

    There \Ch{D}{is} a king in \Ch{A7}{Zion,}

    There \Ch{D}{is} a king in \Ch{G}{Zion},

    The \Ch{D}{Ci}ty \Ch{A7}{of} our \Ch{D}{God!}
  \end{SBVerse}

  \begin{SBVerse}
    There's \Ch{D}{oil} and wine in \Ch{G}{Zion,}

    There's \Ch{D}{oil} and wine in \Ch{A7}{Zion,}

    There's \Ch{D}{oil} and wine in \Ch{G}{Zion,}

    The \Ch{D}{Ci}ty \Ch{A7}{of} our \Ch{D}{God!}
  \end{SBVerse}

  \begin{SBVerse}
    \Ch{D}{Let's} go up to \Ch{G}{Zion,}

    \Ch{D}{Let's} go up to \Ch{A7}{Zion,}

    \Ch{D}{Let's} go up to \Ch{G}{Zion},

    The \Ch{D}{Ci}ty \Ch{A7}{of} our \Ch{D}{God!}
  \end{SBVerse}

  \begin{SBVerse}
    The \Ch{D}{church} is built on \Ch{G}{Zion,}

    The \Ch{D}{church} is built on \Ch{A7}{Zion}

    The \Ch{D}{church} is built on \Ch{G}{Zion,}

    The \Ch{D}{City} \Ch{A7}{of} our \Ch{D}{God!}
  \end{SBVerse}
\end{song}


\begin{song}{Zephaniah 3:17}{G}
  {}
  {}
  {}
  {\NotCCLIed}

  \renewcommand{\RevDate}{February~11,~1993}
  %\SBRef{}{}
  \FLineIdx{The Lord your God is in your midst}

  \begin{SBOpGroup}
    \Ch{G}{} The Lord your \Ch{D}{God} is \Ch{Em}{in} your midst,
    
    \Ch{G}{} The Lord of \Ch{D}{lords,} who sa\Ch{Em}{ves,}
    
    \Ch{G}{} He will ex\Ch{D}{ult} over \Ch{Em}{you} with joy,
    
    \Ch{C}{} He will renew you \Ch{D}{in} His love,
    
    \Ch{C}{} He will rejoice over \Ch{D}{you,}
    
    With shouts of \Ch{G}{joy,} \Ch{D}{} \Ch{Em}{} with shouts of \Ch{G}{joy,} \Ch{D}{} \Ch{Em}{}
    
    With shouts of \Ch{C}{joy,} \Ch{D}{} with shouts of \Ch{C}{joy,} \Ch{D}{}
    
    With shouts of \Ch{G}{joy!}
  \end{SBOpGroup}
\end{song}


\begin{song}{Come And Let Us Go}{D}
  {}
  {}
  {Micah 4:2}
  {\NotCCLIed}

  \renewcommand{\RevDate}{February~11,~1993}
  %\SBRef{}{}

  \begin{SBOpGroup}
    \Ch{D}{Come} and let us \Ch{Bm}{go} to the \Ch{Em}{moun}tain of the \Ch{A7}{Lord,}
    
    And \Ch{Em}{to} the \Ch{A7}{house} of our \Ch{D}{God.}
    
    \Ch{D}{Come} and let us \Ch{Bm}{go} to the \Ch{Em}{moun}tain of the \Ch{A7}{Lord,}
    
    And \Ch{Em}{to} the \Ch{A7}{house} of our \Ch{D}{God.}
    
    And \Ch{G}{He} will \Ch{A}{teach} us of His \Ch{D}{ways,} \Ch{D7}{}
    
    And \Ch{G}{we} will \Ch{A}{walk} in His \Ch{D}{paths.} \Ch{D7}{}
    
    And the \Ch{G}{law} shall go \Ch{A}{forth} from \Ch{D}{Zi}\Ch{Bm}{on,}
    
    And the \Ch{Em}{Word} of the \Ch{A}{Lord} from Jerusa\Ch{D}{lem.}
  \end{SBOpGroup}
\end{song}


\begin{song}{Behold, I Am The Lord}{D}
  {}
  {}
  {}
  {\NotCCLIed}

  \renewcommand{\RevDate}{February~11,~1993}
  \SBRef{Sounds Of Zion}{Volume~III}

  \begin{SBOpGroup}
    Be\Ch{D}{hold,} I am the \Ch{F#m}{Lord,} the \Ch{G}{God} of all \Ch{A}{flesh.}
    
    Is there \Ch{D}{any}thing,\Ch{D7}{} is there \Ch{G}{any}thing,\Ch{Em}{} \Ch{Gm}{too} \Ch{D}{hard} \Ch{A}{for} \Ch{D}{Me?}
    
    Is there \Ch{D}{any}thing, \Ch{G}{any}thing, \Ch{D}{any}thing, too hard \Ch{G}{for} \Ch{A}{Me?}
    
    \Ch{A7}{Is} there \Ch{D}{any}thing, \Ch{G}{any}thing, \Ch{D}{any}thing,\Ch{Bm7}{} \Ch{Gm}{too} \Ch{D/A}{hard} \Ch{A}{for} \Ch{D}{Me?}
  \end{SBOpGroup}
\end{song}


\begin{song}{Stand Up And Bless}{D}
  {}
  {}
  {Nehemiah 9:5}
  {\NotCCLIed}

  \renewcommand{\RevDate}{February~11,~1993}
  \SBRef{Hosanna! Music Book~IV}{\#332}

  \begin{SBOpGroup}
    Stand \Ch{D}{up} and bless the \Ch{Em}{Lord} your God
    
    From ever\Ch{A}{last}ing to ever\Ch{D}{last}ing
    
    Stand \Ch{D}{up} and bless the \Ch{Em}{Lord} your God
    
    From ever\Ch{A}{last}ing to ever\Ch{D}{last}ing
  \end{SBOpGroup}

\WBPageBrk
  \begin{SBChorus}
    And \Ch{G}{ble}ssed be your glorious \Ch{D}{name,} O Lord

    Which is e\Ch{Em}{xalted}\Ch{A}{} above all \Ch{D}{ble}ssing and \Ch{D7}{praise}

    And \Ch{G}{ble}ssed be your glorious \Ch{D}{name,} O Lord

    Which is e\Ch{Em}{xalted,}\Ch{A}{} which is e\Chr{G/D}{xal}\Ch{D}{ted}
  \end{SBChorus}
\end{song}


\begin{song}{Jesus Is The Lord}{G}
  {}
  {}
  {}
  {\NotCCLIed}

  \renewcommand{\RevDate}{February~11,~1993}
  %\SBRef{}{}

  \begin{SBOpGroup}
    \Ch{G}{Je}sus is the \Ch{G7}{Lord}
    
    \Ch{C}{Je}sus the Lord \Ch{Am}{reigns}
    
    \Ch{D}{We} shall take the \Ch{D7}{king}doms of this \Ch{G}{world} \Ch{C}{in} His \Ch{G}{name} \Ch{D7}{}
    
    \Ch{G}{Ev}ery tribe and \Ch{G7}{nation}
    
    \Ch{C}{Ev}ery situ\Ch{Am}{ation}
    
    \Ch{D}{Must} declare that \Ch{D7}{Jesus} is the \Ch{G}{Lord} \Ch{G7}{}
  \end{SBOpGroup}

  \begin{SBChorus}
    For the \Ch{C}{Lord} \Ch{G}{our} \Ch{D}{God} has de\Ch{G}{liver}ed Him from \Ch{Em}{death}

    And es\Ch{Am}{tablish}ed \Ch{D}{Je}sus as \Ch{G}{Lord} \Ch{G7}{}

    He has \Ch{C}{gi}ven \Ch{G}{Him} the \Ch{D}{po}wer over \Ch{G}{all} that He has \Ch{Em}{made}

    For our \Ch{Am}{God} has \Ch{D}{made} His Christ the \Ch{G}{Lord}
  \end{SBChorus}
\end{song}


\begin{song}{I Will Sing Praises}{G}
  {}
  {}
  {}
  {\NotCCLIed}

  \renewcommand{\RevDate}{February~11,~1993}
  %\SBRef{}{}

  \begin{SBOpGroup}
    \Ch{G}{} I will sing \Ch{C}{prais}es,
    
    \Ch{G}{} I will sing \Ch{C}{prais}es,
    
    \Ch{G}{} I will sing \Ch{C}{prais}es \Ch{D}{to} my \Ch{G}{God}
    
    \Ch{G}{} I will sing \Ch{C}{prais}es,
    
    \Ch{G}{} I will sing \Ch{C}{prais}es,
    
    \Ch{G}{} I will sing \Ch{C}{prais}es \Ch{D}{to} my \Ch{G}{God}
    
    \Ch{G}{} And when He \Ch{C}{comes} a\Ch{D}{gain,}
    
    \Ch{G}{} The rocks will not \Ch{C}{have} to \Ch{D}{cry} out
    
    \Ch{G}{} His people will \Ch{C}{shout} the mighty \Ch{D}{name}
    
    Of the King of \Ch{Em}{Glor}y
    
    \Ch{G}{} When the mountains \Ch{C}{sing} forth
    
    \Ch{G}{} And all the trees \Ch{C}{clap} their hands
    
    \Ch{G}{} His people will \Ch{C}{shout} the name
    
    Of the \Ch{D}{beau}ty of holiness:
    
    \Ch{D7}{Jesus} Christ is His \Ch{G}{name!}
  \end{SBOpGroup}
\end{song}


\begin{song}{Let God Be Magnified}{G}
  {}
  {}
  {}
  {\NotCCLIed}

  \renewcommand{\RevDate}{February~11,~1993}
  %\SBRef{}{}

  \begin{SBOpGroup}
    Let \Ch{G}{all} those that seek Thee rejoice and be glad
    
    In \Ch{Am}{Thee}, In \Ch{D7}{Thee}
    
    And \Ch{G}{let} such as love thy salvation, say
    
    Continual\Ch{Am}{ly,} continual\Ch{D7}{ly,}
    
    \Ch{C}{}``Let God be magnified, \Ch{G}{}let God be magnified,
        
    \Ch{Am}{Let} God be magni\Ch{D7}{fied.''}
    
    \Ch{C}{}``Let God be magnified, \Ch{G}{}let God be magnified,
    
    \Ch{Am}{Let} God be \Ch{D7}{mag}ni\Ch{G}{fied.''}\Ch{C}{} \Ch{G}{}
  \end{SBOpGroup}
\end{song}


\begin{song}{There's A Light Shining Forth}{D}
  {}
  {}
  {}
  {\NotCCLIed}

  \renewcommand{\RevDate}{February~11,~1993}
  %\SBRef{}{}

  \begin{SBOpGroup}
    There's a light shining \Ch{D}{forth,}
    
    I can see it on the ho\Ch{Em}{rizon;} \Ch{A7}{}
    
    It's the army of \Ch{Em}{God,} \Ch{A}{} preparing for \Ch{D}{war;}
    
    Coming conquering vic\Ch{F#m}{torious,} \Ch{D7}{}
    
    O'er the army of \Ch{G}{sa}tan; \Ch{Em}{}
    
    Nothing shall \Ch{D}{stand,}\Ch{A7}{} before the army of \Ch{D}{God.} \Ch{G}{} \Ch{D}{}
  \end{SBOpGroup}
\end{song}


\begin{song}{How Great Is Our God}{D}
  {\SBOrgMel}
  {}
  {}
  {\NotCCLIed}

  \renewcommand{\RevDate}{February~11,~1993}
  \SBRef{Worship Him~II}{\#65}

  \begin{SBOpGroup}
    \Ch{D}{How} great is our \Ch{A7}{God!}  How great is His \Ch{D}{name}\Ch{G}{} \Ch{D}{}
    
    How great is our \Ch{A7}{God!}  Forever the \Ch{D}{same}\Ch{G}{} \Ch{D}{}
    
    He rolled back the \Ch{D}{wa}te\Ch{D7}{rs,} of the mighty Red \Ch{G}{Sea}
    
    And He said, ``I'll never \Ch{D}{leave} you,\Ch{A7}{}
    
    Put your trust in \Ch{D}{Me!''} \Ch{[}{}\Ch{G}{} \Ch{D}{}\Ch{]}{}
  \end{SBOpGroup}

  \begin{SBExtraKeys}{
    \STitle{How Great Is Our God}{E}

    \begin{SBOpGroup}
      \Ch{E}{How} great is our \Ch{B7}{God!}  How great is His \Ch{E}{name}\Ch{A}{} \Ch{E}{}
      
      How great is our \Ch{B7}{God!}  Forever the \Ch{E}{same}\Ch{A}{} \Ch{E}{}
      
      He rolled back the \Ch{E}{wa}te\Ch{E7}{rs,} of the mighty Red \Ch{A}{Sea}
      
      And He said, ``I'll never \Ch{E}{leave} you,\Ch{B7}{}
      
      Put your trust in \Ch{E}{Me!''} \Ch{[}{}\Ch{A}{} \Ch{E}{}\Ch{]}{}
    \end{SBOpGroup}
  }\end{SBExtraKeys}
\end{song}


\begin{song}{Lift Jesus Higher}{E}
  {}
  {}
  {}
  {\NotCCLIed}

  \renewcommand{\RevDate}{February~11,~1993}
  %\SBRef{}{}

  \begin{SBOpGroup}
    Lift Jesus \Ch{E}{high}\Ch{A}{er.} \Ch{E}{} Lift Jesus \Ch{E}{high}\Ch{A}{er.} \Ch{E}{}
    
    Lift Him up for the world to \Ch{B7}{see.}
    
    He said, ``If \Ch{E}{I} \Ch{E7}{} be lifted \Ch{A}{up} from the earth,
    
    I will \Ch{E}{draw} all \Ch{B7}{men} unto \Ch{E}{me}.''
  \end{SBOpGroup}
\end{song}


\WBPageBrk
\begin{song}{Garment Of Praise}{C}
  {}
  {}
  {Isaiah~61:3}
  {\NotCCLIed}

  \renewcommand{\RevDate}{February~11,~1993}
  %\SBRef{}{}
  \FLineIdx{I have put on my garment of praise}

  \begin{SBOpGroup}
    I have \Ch{C}{put} on my garment of \Ch{G7}{praise,}
    
    I have \Ch{G7}{put} on my garment of \Ch{C}{prai}\Ch{C7}{se}
    
    And the \Ch{F}{spirit} of heaviness, is \Ch{C}{gone} from \Ch{Am}{me.}
    
    I have \Ch{C}{put} on my \Ch{G7}{gar}ment of \Ch{C}{praise!} \ChX{[}{}\ChX{F}{} \ChX{C}{}\ChX{]}{}
  \end{SBOpGroup}

  \begin{SBExtraKeys}{
    \STitle{Garment Of Praise}{D}

    \begin{SBOpGroup}
      I have \Ch{D}{put} on my garment of \Ch{A7}{praise,}
      
      I have \Ch{A7}{put} on my garment of \Ch{D}{prai}\Ch{D7}{se}
      
      And the \Ch{G}{spirit} of heaviness, is \Ch{D}{gone} from \Ch{Bm}{me.}
      
      I have \Ch{D}{put} on my \Ch{A7}{gar}ment of \Ch{D}{praise!} \Ch{[}{}\Ch{G}{} \Ch{D}{}\Ch{]}{}
    \end{SBOpGroup}
  }\end{SBExtraKeys}
\end{song}


\begin{song}{I Was Glad}{C}
  {}
  {}
  {Psalm 122:1}
  {\NotCCLIed}

  \renewcommand{\RevDate}{February~11,~1993}
  %\SBRef{}{}

  \begin{SBOpGroup}
    I was \Ch{C}{glad,} very glad, when they \Ch{F}{said} unto me
    
    Let us \Ch{G7}{go,} into the house of the \Ch{C}{Lord,} today
    
    There is singing, there is \Ch{C7}{danc}ing, there is \Ch{F}{vic}tory
    
    In the \Ch{G7}{house} of the Lord, to\Ch{C}{day!}
  \end{SBOpGroup}
\end{song}


\begin{song}{Oh, The Blood}{C}
  {\SBOrgMel}
  {}
  {Psalm 51:7}
  {\NotCCLIed}

  \renewcommand{\RevDate}{February~11,~1993}
  \SBRef{Hosanna! Music Book~I}{\#68}

  \begin{SBOpGroup}
    \Ch{C}{Oh} the blood \Ch{F}{of} \Ch{C}{Jesus.}
  
    \Ch{G7}{Oh} the blood of \Ch{C}{Jesus.}
    
    Oh the blood \Ch{F}{of} \Chr{C}{Je}\Ch{Am}{sus.}
    
    It \Ch{Dm}{wash}es \Ch{G7}{white} as \Ch{C}{snow.}
  \end{SBOpGroup}

  \begin{SBExtraKeys}{
    \STitle{Oh, The Blood}{D}

    \begin{SBOpGroup}
      \Ch{D}{Oh} the blood \Ch{G}{of} \Ch{D}{Jesus.}
  
      \Ch{A7}{Oh} the blood of \Ch{D}{Jesus.}
      
      Oh the blood \Ch{G}{of} \Chr{D}{Je}\Ch{Bm}{sus.}
      
      It \Ch{Em}{wash}es \Ch{A7}{white} as \Ch{D}{snow.}
    \end{SBOpGroup}

    %%%
    \STitle{Oh, The Blood}{E}

    \begin{SBOpGroup}
      \Ch{E}{Oh} the blood \Ch{A}{of} \Ch{E}{Jesus.}
      
      \Ch{B7}{Oh} the blood of \Ch{E}{Jesus.}
      
      Oh the blood \Ch{A}{of} \Chr{E}{Je}\Ch{C#m}{sus.}
      
      It \Ch{F#m}{wash}es \Ch{B7}{white} as \Ch{E}{snow.}
    \end{SBOpGroup}

\CBPageBrk
    %%%
    \STitle{Oh, The Blood}{F}

    \begin{SBOpGroup}
      \Ch{Eb}{Oh} the blood \Ch{Ab}{of} \Ch{Eb}{Jesus.}
  
      \Ch{Bb7}{Oh} the blood of \Ch{Eb}{Jesus.}
      
      Oh the blood \Ch{Ab}{of} \Chr{Eb}{Je}\Ch{Cm}{sus.}
      
      It \Ch{Fm}{wash}es \Ch{Bb7}{white} as \Ch{Eb}{snow.}
    \end{SBOpGroup}
  }\end{SBExtraKeys}
\end{song}


\begin{song}{You Alone}{C}
  {1992 Glory Alleluia Music}
  {Christopher Rath}
  {Revelation 4:8}
  {Used by Permission}

  \renewcommand{\RevDate}{February~11,~1993}
  \FLineIdx{Holy, Holy, You alone are Holy}

  \begin{SBVerse}
    \Ch{C}{Ho}\Ch{Em}{ly,} \Ch{F}{Ho}\Ch{C}{ly,} \Ch{F}{} You alone are \Ch{C}{Ho}\Ch{G}{ly.}

    \Ch{C}{Ho}\Ch{Em}{ly,} \Ch{F}{Ho}\Ch{C}{ly,} \Ch{F}{} You alone are \Chr{Am}{Ho}\Ch{G}{ly.}

    Al\Ch{F}{migh}ty God\Ch{C}{} I \Ch{E}{wor}ship You\Ch{Am}{}

    For \Ch{F}{You} a\Ch{G}{lone,} \Ch{F}{You} a\Ch{G}{lone} are \Chr{Dm}{Ho}\Chr{G}{}\Ch{C}{ly.}  \Ch{F}{}
  \end{SBVerse}

  \begin{SBVerse}
    \Ch{C}{Wor}\Ch{Em}{thy,} \Ch{F}{Wor}\Ch{C}{thy,} \Ch{F}{}You alone are \Ch{C}{Wor}\Ch{G}{thy.}

    \Ch{C}{Wor}\Ch{Em}{thy,} \Ch{F}{Wor}\Ch{C}{thy,} \Ch{F}{}You alone are \Ch{Am}{Wor}\Ch{G}{thy.}

    Who \Ch{F}{was,} who is,\Ch{C}{} and \Ch{E}{is} to come.\Ch{Am}{}

    \Ch{F}{You} a\Ch{G}{lone,} \Ch{F}{You} a\Ch{G}{lone} are \Chr{Dm}{Wor}\Chr{G}{}\Ch{C}{thy.} \Ch{[}{}\Ch{F}{}\Ch{]}{} \Ch{[{$^{Mod.}$}}{}\Ch{A7}{}\Ch{]}{}
  \end{SBVerse}

  \begin{SBExtraKeys}{%
    \STitle{You Alone}{D}

    \begin{SBVerse}
      \Ch{D}{Ho}\Ch{F#m}{ly,} \Ch{G}{Ho}\Ch{D}{ly,} \Ch{G}{} You alone are \Ch{D}{Ho}\Ch{A}{ly.}

      \Ch{D}{Ho}\Ch{F#m}{ly,} \Ch{G}{Ho}\Ch{D}{ly,} \Ch{G}{} You alone are \Chr{Bm}{Ho}\Ch{A}{ly.}

      Al\Ch{G}{migh}ty God\Ch{D}{} I \Ch{F#}{wor}ship You\Ch{Bm}{}

      For \Ch{G}{You} a\Ch{A}{lone,} \Ch{G}{You} a\Ch{A}{lone} are \Chr{Em}{Ho}\Chr{A}{}\Ch{D}{ly.}  \Ch{G}{}
    \end{SBVerse}
  }\end{SBExtraKeys}

\CBPageBrk
  \begin{xlatn}{Tu Es Saint}
    {}
    {Translation by Jocelyne Sarrazin}
    \renewcommand{\RevDate}{8~April,~1998}

    \begin{SBVerse}
      \Ch{C}{Tu} es \Ch{Em}{Saint,} \Ch{F}{Tu} es \Ch{C}{Saint,}\Ch{F}{} Tu es Saint O \Ch{C}{Sei}\Ch{G}{gneur.}

      \Ch{C}{Tu} es \Ch{Em}{Saint,} \Ch{F}{Tu} es \Ch{C}{Saint,}\Ch{F}{} Tu es Saint O \Chr{Am}{Sei}\Ch{G}{gneur.}

      Dieu \Ch{F}{Tout-}Puissant\Ch{C}{} soit \Ch{E}{glo}rifi\'e,\Ch{Am}{}

      Car \Ch{F}{Tu} es tr\'es \Ch{G}{Saint,} \Ch{F}{Tu} es tr\'es \Ch{G}{Saint,} O oui \Ch{Dm}{Saint}\Ch{G}{} O Sei\Ch{C}{gneur.}\Ch{F}{}
    \end{SBVerse}

    \begin{SBVerse}
      \Ch{C}{Tu} es \Ch{Em}{Digne,} \Ch{F}{Tu} es \Ch{C}{Digne,} Tu es Digne O \Ch{C}{Sei}\Ch{G}{gneur.}

      \Ch{C}{Tu} es \Ch{Em}{Digne,} \Ch{F}{Tu} es \Ch{C}{Digne,} Tu es Digne O \Chr{Am}{Sei}\Ch{G}{gneur.}

      Toi \Ch{F}{qui} \'etais,\Ch{C}{} qui \Ch{E}{es} et vien,\Ch{Am}{}

      \Ch{F}{\'{A}} Toi l'hon\Ch{G}{neur,} \Ch{F}{\`a} Toi la \Ch{G}{gloire,} \`a Toi \Ch{Dm}{seul}\Ch{G}{} O Sei\Ch{C}{gneur.}\Ch{F}{}
    \end{SBVerse}
  \end{xlatn}
\end{song}


\end{document}
\bye
%
%%%
% Document ends.
%%%

      \begin{minipage}{.8\textwidth}
        \printcontents{}{1}{}
      \end{minipage}%
    \renewcommand{\SBThechapter}{#1}
    \clearpage
}

\titleformat{\chapter}
[display]
{}
{%\vspace*{\fill}
 % \titlerule[1pt]%
 % \vspace{1pt}%
 % \titlerule
 % \vspace{1pc}%
 \chaptertitlename}
{}
{\Huge}



%%%%%%%%%%%%%%%%%%%%%%%%%%%%%%%%%%%%%%%%%%%%%%%%%%%%%%%%%%
%%%%%%%%%%%%%%%%%%%%%%%%%%%%%%%%%%%%%%%%%%%%%%%%%%%%%%%%%%
%%                                                      %%
%%           D O C U M E N T   B E G I N S              %%
%%                                                      %%
%%%%%%%%%%%%%%%%%%%%%%%%%%%%%%%%%%%%%%%%%%%%%%%%%%%%%%%%%%
%%%%%%%%%%%%%%%%%%%%%%%%%%%%%%%%%%%%%%%%%%%%%%%%%%%%%%%%%%
\begin{document}

%%%
% Uncomment "\maketitle" statement to make a title page.
%%%
\maketitle
\mainmatter
\ifWordBk
  \twocolumn
\fi
%%%
% Turn on and define fancy page heading/footing definition.
%%%
\pagestyle{fancy}

\ifChordBk
  % It's a words & chords songbook...
  \addtolength{\headwidth}{\marginparsep}
  \addtolength{\headwidth}{\marginparwidth}
  \renewcommand{\headrulewidth}{0.4pt}
  \renewcommand{\footrulewidth}{0.4pt}
  \fancyhead[LE,RO]{\LHeadFont\emph{\leftmark\/}\SBContinueMark}
  \fancyhead[CE,CO]{\CHeadFont\thepage}
  \fancyhead[RE,LO]{\RHeadFont \chaptermark}
\else\ifOverhead
  % It's an overhead...
  \renewcommand{\footrulewidth}{0pt}
  \renewcommand{\headrulewidth}{0pt}
  \fancyhead[LE,RO]{}
  \fancyhead[CE,CO]{}
  \fancyhead[RE,LO]{}
\else\ifWordBk
  % It's a words only songbook...
  \addtolength{\headwidth}{\marginparsep}
  \addtolength{\headwidth}{\marginparwidth}
  \renewcommand{\headrulewidth}{0.4pt}
  \renewcommand{\footrulewidth}{0.4pt}
  \fancyhead[LE,RO]{\LHeadFont UNF Sangbogen}
  \fancyhead[CE,CO]{\CHeadFont\thepage}
  \fancyhead[RE,LO]{\RHeadFont \SBThechapter}
\fi\fi\fi

\fancyfoot[LE,RO]{\LFootFont ScienceCamps 2018}
\ifSongEject
  \fancyfoot[CE,CO]{\CFootFont Last Revised:  \RevDate}
\else
  \fancyfoot[CE,CO]{\CFootFont}
\fi
\fancyfoot[RE,LO]{\RFootFont TODO: Et eller andet smart kan stå her}


%%%
% Songbook begins.
%%%

\onecolumn
\SBChapter{Os mod de andre}
\twocolumn
\begin{song}{Vi kan ikke li'}
  {} % Bruges ikke, lad stå blank
  {Vores buschauffør kan ikke køre bus} % Titel, Kunstner - eks.: "Jutlandia, Kim Larsen". Hvis sangen er på sin egen melodi, brug da \SBOrgMel.
  {} % Navnet på forfatteren. Undlad aliasser. Brug "&" frem for "og". Hvis forfatter er ukendt, lad da stå tom.
  {} % Eks. "Fysikrevy 2010" eller "2010"
  {\NotCCLIed} % Lad stå som den er

 \begin{SBVerse}
    \SBRepeat{Vi kan ikke li' folk vi ik' kan li'}\\
    For vi kan sgu ikke li' dem,\\
    de skulle hellere tage at blive hjemme\\
    Vi kan ikke li' folk vi ik' kan li'
  \end{SBVerse}

  \begin{SBVerse}
    \SBRepeat{Vi kan ikke li' folk fra teologi}\\
    For de er nogle hængemuler,\\
    tror at helligånden puler\\
    Vi kan ikke li' folk fra teologi
  \end{SBVerse}

  \begin{SBVerse}
    \SBRepeat{Vi kan ikke li' folk fra biologi}\\
    For de går i strikketrøjer,\\
    og med dyrene sig fornøjer\\
    Vi kan ikke li' folk fra biologi
  \end{SBVerse}

  \begin{SBVerse}
    \SBRepeat{Vi kan ikke li' folk der har fysik}\\
    For de har sgu nogle testikler\\
    der er på størrelse med partikler\\
    Vi kan ikke li' folk der har fysik
  \end{SBVerse}

  \begin{SBVerse}
    \SBRepeat{Vi kan ikke li' folk fra musikvidenskab}\\
    For de er nogle lamme nødder,\\
    passer ik' på versefødder\\
    Vi kan ikke li' folk fra musikvidenskab
  \end{SBVerse}

  \begin{SBVerse}
    \SBRepeat{Vi kan ikke li' folk fra nano-tek}\\
    De går rundt med sure miner,\\
    laver bittesmå maskiner\\
    Vi kan ikke li' folk fra nano-tek
  \end{SBVerse}

  \begin{SBVerse}
    \SBRepeat{Vi kan ikke li' folk fra statistik}\\
    De vil hypoteser teste\\
    mens de hel're burde feste\\
    Vi kan ikke li' folk fra statistik
  \end{SBVerse}

  \begin{SBVerse}
    \SBRepeat{Vi kan ikke li' folk fra arkæologi}\\
    fordi de graver og de graver\\
    og på gamle mennesker rager\\
    Vi kan ikke li' folk fra arkæologi
  \end{SBVerse}

  \begin{SBVerse}
    \SBRepeat{Vi kan ikke li' folk fra filosofi}\\
    for de går bar' og funderer\\
    om de selv mon eksisterer\\
    Vi kan ikke li' folk fra filosofi
  \end{SBVerse}

  \begin{SBVerse}
    \SBRepeat{Vi kan ikke li' folk fra farmaci}\\
    for de går og triller piller\\
    og tabletter og pastiller\\
    Vi kan ikke li' folk fra farmaci
  \end{SBVerse}

  \begin{SBVerse}
    \SBRepeat{Vi kan ikke li' folk fra geologi}\\
    For de går og ser på stenene\\
    når de burde sprede benene\\
    Vi kan ikke li' folk fra geologi
  \end{SBVerse}

  \begin{SBVerse}
    \SBRepeat{Vi kan ikke li' folk fra astronomi}\\
    De er fjolser uden hjerner\\
    der går rundt og ser på stjerner\\
    Vi kan ikke li' folk fra astronomi
  \end{SBVerse}
%  \begin{SBVerse}
%    Vi kan ikke li' folk fra iNano\ldots\\
%    bagfra blir' de lidt for kække\\
%    synes Ångstrøm de er frække
%  \end{SBVerse}

  \begin{SBVerse}
    \SBRepeat{Vi kan ikke li' folk fra økonomi}\\
    For de regner på budgettet\\
    når de sidder på toilettet\\
    Vi kan ikke li' folk fra økonomi
  \end{SBVerse}

  \begin{SBVerse}
    \SBRepeat{Vi kan ikke li' folk fra lingvistik}\\
    Kulli waffli zarka gunku\\
    emfle birnan smöja dunku\\
    Vi kan ikke li' folk fra lingvistik
  \end{SBVerse}

  \begin{SBVerse}
    \SBRepeat{Vi kan ikke li' folk fra folk der har kemi}\\
    For de hedder Ion og Ester\\
    mens de luften stærkt forpester\\
    Vi kan ikke li' folk fra folk der har kemi
  \end{SBVerse}

 \begin{SBVerse}
    \SBRepeat{Vi kan ikke li' folk fra matematik}\\
    For de teoretiserer\\
    indtil $2$ plus $3$ bli'r $4$\\
    Vi kan ikke li' folk fra matematik
  \end{SBVerse}

% \begin{SBVerse}
%    Vi kan ikke li' folk der har IT\ldots\\
%    Uni skaffer studiejobbet\\
%    Det er næsten alt for snobbet
%  \end{SBVerse}

% \begin{SBVerse}
%    Vi kan ikke li' folk der har dat-mult\ldots\\
%    Datalogcirkusartister;\\
%    de er næsten humanister
%  \end{SBVerse}

% \begin{SBVerse}
%    Vi kan ikke li' folk der har tek-fys\ldots\\
%    for de har en bærbar PC\\
%    læser mails når de' på WC
%  \end{SBVerse}

% \begin{SBVerse}
%    Vi kan ikke li' folk fra statskundskab\ldots\\
%    for de gider kun at kom' for- di\\
%    de tror de møder Lomborg
%  \end{SBVerse}

% \begin{SBVerse}
%    Vi kan ikke li' folk fra farmaceutisk kemi\ldots\\
%    de fungerer som en buffer\\
%    der i alle emner skuffer
%  \end{SBVerse}

 \begin{SBVerse}
    \SBRepeat{Vi kan ikke li' folk fra folk fra medicin}\\
    For de går i hvide kitler,\\
    og det rimer jo på Hitler\\
    Vi kan ikke li' folk fra folk fra medicin
  \end{SBVerse}

 \begin{SBVerse}
    \SBRepeat{Vi kan ikke li' folk fra molekylær(biologi)}\\
    for de kloner små kaniner\\
    splejser dem med appelsiner\\
    Vi kan ikke li' folk fra molekylær(biologi)
  \end{SBVerse}

 \begin{SBVerse}
    \SBRepeat{Vi kan ikke li' folk fra psykologi}\\
    For de vader rundt i læder,\\
    og de hader deres fædre\\
    Vi kan ikke li' folk fra psykologi
  \end{SBVerse}

 \begin{SBVerse}
    \SBRepeat{Vi kan ikke li' folk der er jurister}\\
    for med hule paragraffer\\
    dagen lang sig selv de straffer\\
    Vi kan ikke li' folk der er jurister
  \end{SBVerse}

 \begin{SBVerse}
    \SBRepeat{Vi kan ikke li' folk der har historie}\\
    for de har kun gamle minder\\
    og et job de aldrig finder\\
    Vi kan ikke li' folk der har historie
  \end{SBVerse}

 \begin{SBVerse}
    \SBRepeat{Vi kan ikke li' folk der har idræt}\\
    For de ligner jo rødbeder,\\
    når de render rundt og sveder\\
    Vi kan ikke li' folk der har idræt
  \end{SBVerse}

 \begin{SBVerse}
    \SBRepeat{Vi kan ikke li' folk fra datalogi}$^+$
  \end{SBVerse}
\end{song}
\begin{song}{Vi er ikke humanister}
  {} % Bruges ikke, lad stå blank
  {Vi er ikke rigtig voksne, Bølle Bob} % Titel, Kunstner - eks.: "Jutlandia, Kim Larsen". Hvis sangen er på sin egen melodi, brug da \SBOrgMel.
  {Jeanette Pinderup} % Navnet på forfatteren. Undlad aliasser. Brug "&" frem for "og". Hvis forfatter er ukendt, lad da stå tom.
  {TÅGEKAMMERETs Julerevy, 2007} % Eks. "Fysikrevy 2010" eller "2010"
  {\NotCCLIed} % Lad stå som den er

  \begin{SBChorus}
    Vi er ikke humanister, vi har ikke samfundsfag,\\
    Medicin er også noget for de andre.\\
    Vi er ikke teologer, og går ik’ på Business School,\\
    og de andre synes ikke vi er cool.
  \end{SBChorus}

  \begin{SBVerse}
    Når jeg vil ud og danse hele natten, siger de:\\
    Du har ingen rytmesans! Alle tænker bare stands,\\
    men når de vil have drenge med til fester hører vi\\
    at de pludselig kan li’ datalogi.
  \end{SBVerse}

  \begin{SBVerse}
    I hverdagen der skal vi ikke sende dem et nik.\\
    Du er ikke lækker nok! Du er bare et lille pjok!\\
    Men når de skal bruge laser eller nano-mekanik,\\
    er de blevet meget glade for fysik.
  \end{SBVerse}

  \begin{SBChorus}
    Vi er ikke humanister\ldots
  \end{SBChorus}

  \begin{SBVerse}
    Hvis vi vil sige noget får vi bare deres blik:\\
    Hey - hvem tror du dog du er? Du er lige meget her!\\
    Men hvis de så har et spørgsmål til en vigtig statistik,\\
    ved de godt at svaret er på mat’matik.
  \end{SBVerse}

  \begin{SBChorus}
    Vi er ikke humanister\ldots
  \end{SBChorus}
\end{song}

\begin{song}{Er jeg humanist?}
  {} % Bruges ikke, lad stå blank
  {Har jeg bildæk?, Drengene fra Angora} % Titel, Kunstner - eks.: "Jutlandia, Kim Larsen". Hvis sangen er på sin egen melodi, brug da \SBOrgMel.
  {} % Navnet på forfatteren. Undlad kaldenavne. Brug gerne TBF. Brug "&" frem for "og". Hvis forfatter er ukendt, lad da stå tom.
  {Ingeniørrevyen, 2017} % Eks. "Fysikrevy, 2010" eller "2010"
  {\NotCCLIed} % Lad stå som den er

  \begin{SBVerse}
    Andet semester i eksamenstid'n.\\
    Vi skrev på vor's projekt, og der stod dias på menuen.\\
    Vi sku' fremlægge beregninger med $\pi$,\\
    og jeg ha'd lav'd en fucking sexet koreografi.\\\medskip
    Jeg viste dem det, men pluds'lig slog det mig,\\
    at der var fucking gay, og derfor sagde jeg:\\
    "Drenge, jeg tror jeg har et meget stort problem!\\
    Jeg tror ikke at jeg passer ind i jer's system"
  \end{SBVerse}

  \begin{SBChorus}
    Er jeg humanist? Helt ærligt, er jeg humanist?\\
    Nej, det' da helt normalt at ha' et ekstra følsimt twist.\\
    Men er jeg ik' ubrugelig? Går op i antropologi!\\
    Nej, du' da bare lidt blød i kanten, tag nu og klap i!\\
    \emph{Vi går i baren nu\ldots}
  \end{SBChorus}

  \begin{SBVerse}
    Senere i vores fredagsbar\\
    mødte jeg en pige, som jeg syn's var rar.\\
    Jeg sagde hun havde smukke øjne, så begyndte hun at grin'\\
    og sagde: "hey, hvad tror du selv, dit store arbejdsløse svin?"\\
    Senere, da vi sku' skriv' rapport,\\
    jeg lav'd en lækker forside med mange nuancer af sort.\\\medskip
    Men ku' mærke at de andre var vrede på mig\\
    selvon jeg hav'd medbragt speltkiks med økologisk butterdej.\\
    Jeg ku' ikke li' det, og spurgt' dem derfor ad\\
    hvorfor de var vred', så begyndte jeg at græd'
  \end{SBVerse}

  \begin{SBChorus}
    Er jeg humanist? Burd' jeg ik' få pisk?\\
    Nej, nogle gange er man bare en lille smule trist.\\
    Men bare se min striksweater! Jeg er en fucking trendsetter!\\
    Nej, du' da bar' en fimset fætter, tag nu og slap af.\\
  \end{SBChorus}

  \begin{SBSection*}
    Men altså, helt ærligt,\\
    synes I slet ikke jeg er humanist, eller hvad?\\
    Vi har jo sagt til dig at du ikke er humanist!\\
    Jamen bare se den forside jeg har lavet, den er helt... lyserød!\\
    Jamen du bliver jo ikke tilfreds før vi siger at du er humanist.\\
    Hva?!!\\
    \emph{Du bliver jo ikke tilfreds før vi siger at du er humanist!}
  \end{SBSection*}

  \begin{SBChorus}
    Du er humanist! En fucking ord-onanist!\\
    Du går op i følelser, farver og andet ubrugeligt pis!\\
    Du er humanist! En arbejdsløs artist!\\
    Du får jo aldrig noget job, bu-hu hvor er det trist.
  \end{SBChorus}

  \begin{SBChorus}
    I er ikke særligt søde! Skal jeg nu ha' stød?\\
    Nej det' ik' farligt, men krokodiller bli'r din død.\\
    Dit humanist-svin! Dit humanist-svin!\\
    Mangel på indtægt, det ligger faktisk til min slægt.\\
    Dit humanist-svin! Dit humanist-svin!\\
    Mangel på indtægt, det ligger faktisk til min slægt.\\
    Dit humanist-svin! Dit humanist-svin!
  \end{SBChorus}
\end{song}













\begin{song}{Humanist}
  {} % Bruges ikke, lad stå blank
  {Onani, Dario von Slutty} % Titel, Kunstner - eks.: "Jutlandia, Kim Larsen". Hvis sangen er på sin egen melodi, brug da \SBOrgMel.
  {} % Navnet på forfatteren. Undlad kaldenavne. Brug gerne TBF. Brug "&" frem for "og". Hvis forfatter er ukendt, lad da stå tom.
  {DIKUrevy, 2013} % Eks. "Fysikrevy, 2010" eller "2010"
  {\NotCCLIed} % Lad stå som den er

  \begin{SBChorus}
    Humanist - det' godt nok trist,\\
    humanist - det' godt nok trist,\\
    humanist - det' godt nok trist:\\
    Det ta'r dem 10 semestre, og kan ikke brug's til sidst.
  \end{SBChorus}

  \begin{SBVerse}
    Jeg er ham der Klaes, jeg har ECTS-behov,\\
    men kurserne jeg tager, de gi'r mig aldrig lov.\\
    De taler mig i søvne, her til forelæsningen,\\
    og når den så er ovre, har jeg det hele glemt igen.\\
    Hvorfor tage kurser, der altid dumper mig?\\
    Det er derfor, at jeg siger: "Der er en nem're vej!"
  \end{SBVerse}

  \begin{SBChorus}
    Humanist - det' godt nok trist,\\
    humanist - det' godt nok trist,\\
    humanist - det' godt nok trist:\\
    Det ta'r dem 10 semestre, og kan ikke brug's til sidst.
  \end{SBChorus}

  \begin{SBChorus}
    Humanist - det' godt nok trist,\\
    humanist - det' godt nok trist,\\
    humanist - det' godt nok trist:\\
    Det ta'r dem 10 semestre, og kan ikke brug's til sidst.
  \end{SBChorus}

  % \begin{SBChorus}
  %   Humanist - det' godt nok trist\ldots
  % \end{SBChorus}

  % \begin{SBChorus}
  %   Humanist - det' godt nok trist\ldots
  % \end{SBChorus}

  \begin{SBVerse}
    Se, de fleste går og drømmer om at arbejde med kode,\\
    men jeg vil bar' være færdig - arbejdsløshed er på mode!\\
    Jeg er ligeglad med penge, jeg har børneopsparing,\\
    og når den så løber tør, kan jeg pizza udbring'.\\
    Hvorfor skulle arbejd', og ikke ha' tid til leg?\\
    Det er derfor, at jeg siger: "Der er en nem're vej!"
  \end{SBVerse}

  \begin{SBChorus}
    Humanist - det' godt nok trist,\\
    humanist - det' godt nok trist,\\
    humanist - det' godt nok trist:\\
    Det ta'r dem 10 semestre, og kan ikke brug's til sidst.
  \end{SBChorus}

  \begin{SBChorus}
    Humanist - det' godt nok trist,\\
    humanist - det' godt nok trist,\\
    humanist - det' godt nok trist:\\
    Det ta'r dem 10 semestre, og kan ikke brug's til sidst.
  \end{SBChorus}

  % \begin{SBChorus}
  %   Humanist - det' godt nok trist\ldots
  % \end{SBChorus}

  % \begin{SBChorus}
  %   Humanist - det' godt nok trist\ldots
  % \end{SBChorus}

  \begin{SBVerse}
    Nu er det ovre, jeg er endelig blevet fri,\\
    er blevet kandidat i østrigsk eskimologi.\\
    Der findes intet job hvori denne grad kan bruges,\\
    så alle jobsamtaler sluttes af med at der buh'es.\\
    Her er det så, jeg fortryder jeg droppede ud.\\
    Hvis bare jeg kunne kode, så fremtiden lysere ud.
  \end{SBVerse}

  \begin{SBChorus}
    Datalog - han er jo god,\\
    datalog - han er jo god,\\
    datalog - får job, værsgo'!\\
    Det ta'r dem 20 semestre, men i det mindste kan de kode.
  \end{SBChorus}

  \begin{SBChorus}
    Datalog - han er jo god,\\
    datalog - han er jo god,\\
    datalog - får job, værsgo'!\\
    Det ta'r dem 20 semestre, men i det mindste kan de kode.
  \end{SBChorus}

  % \begin{SBChorus}
  %   Datalog - han er jo god\ldots
  % \end{SBChorus}
\end{song}
\begin{song}{Når en humanist adderer}
  {} % Bruges ikke, lad stå blank
  {Peberkagesangen, Dyrene i Hakkebakkeskoven} % Titel, Kunstner - eks.: "Jutlandia, Kim Larsen". Hvis sangen er på sin egen melodi, brug da \SBOrgMel.
  {\TKET{}} % Navnet på forfatteren. Undlad aliasser. Brug "&" frem for "og". Hvis forfatter er ukendt, lad da stå tom.
  {} % Eks. "Fysikrevy 2010" eller "2010"
  {\NotCCLIed} % Lad stå som den er

  \begin{SBVerse}
    Når en humanist adderer,\\
    tar’ hun først et $\pi$ i fjerde,\\
    ganger det med $n$ matricer,\\
    mens hun vælger en "sød" brøk.\\
    \medskip
    Læg det til determinanten,\\
    det er humanistisk fjanten -\\
    for hun når jo aldrig læng’re\\
    end at svaret det er $x$.
  \end{SBVerse}

  \begin{SBVerse}
    Når en biolog skal lære,\\
    hvordan man multiplicerer,\\
    skal eksempler observeres,\\
    hvor to køer bli’r til tre.\\
    \medskip
    Og gør man det en gang mere,\\
    Så bli’r tre kø’r jo til flere!\\
    Så hun kender kun’ et svar:\\
    at der er flere nu end før.
  \end{SBVerse}

  \begin{SBVerse}
    Geologer de kan regne,\\
    dem skal man ikke forklejne.\\
    De bestemmer snildt en alder,\\
    når de måler på en sten.\\
    \medskip
    Men har deres kompetencer\\
    nogen sociale nuancer?\\
    Man skal være i en grusgrav\\
    før det bliver relevant!
  \end{SBVerse}

  \begin{SBVerse}
    Men hos os fra \TKET{}\\
    giver regning ingen jamren,\\
    vores ligninger gi’r mening\\
    også med socialt aspekt.\\
    \medskip
    Vi beregner øl i kasser\\
    med potens og inderklasser,\\
    og vor \KASS har altid styr på,\\
    hvem der skylder hvad for hvad.
  \end{SBVerse}

  \begin{SBVerse}
    Så lektionen den må være:\\
    Hold mat/fys’erne i ære,\\
    de har styr på deres sager,\\
    og de drikker mange øl. \emph{(Skål!)}\\
    \medskip
    Udregningerne skal komme\\
    fra han-dyrene med vomme,\\
    der beviser de har været\\
    alt for meget på TK!
  \end{SBVerse}
\end{song}
\begin{song}{Jeg har set en rigtig .*}
  {} % Bruges ikke, lad stå blank
  {Jeg har set en rigtig negermand, Niels C. Andersen} % Titel, Kunstner - eks.: "Jutlandia, Kim Larsen". Hvis sangen er på sin egen melodi, brug da \SBOrgMel.
  {} % Navnet på forfatteren. Undlad aliasser. Brug "&" frem for "og". Hvis forfatter er ukendt, lad da stå tom.
  {DIKUrevy/Sommerfest, 2002} % Eks. "Fysikrevy 2010" eller "2010"
  {\NotCCLIed} % Lad stå som den er

  \begin{SBVerse}
    Jeg har set en rigtig fysiker.\\
    Han havd’ skæg og hår så langt som en musiker.\\
    Han var sikkert ikk’ en rigtig mand.\\
    Han ligned’ en der sku’ ha buksevand.\\\medskip
    Jeg spurgte ham: "Hvad er du da for en?\\
    Hvorfor har du så tynde blege stankelben?\\
    Kom du fra rummet? Hvem er mon din mor?"\\
    Så lo han blot og sagde disse ord:\\\medskip
    \emph{``Det er sandt, at I fandt en fysikmutant.\\
    Jeg har tragten med og hjernen sat i pant.''}
  \end{SBVerse}

  \begin{SBVerse}
    Jeg har set en rigtig smart jurist.\\
    Gik i jakkesæt og spilled sej, det var sgu’ trist.\\
    Det kan godt vær’ han får mange pi’r,\\
    men hans studiejob det er at ta’ kopier.\\\medskip
    Hans hobby er at mishandle små dyr.\\
    Før han vil hjælpe nogen kræver han gebyr.\\
    Men det er også klart han er sadist,\\
    det skal man være for at bli’ jurist.\\\medskip
    \emph{En jurist er så trist, han er AntiKrist.\\
    Gider nogen hente Max von XORcist?}
  \end{SBVerse}

  \begin{SBVerse}
    Jeg har set en rigtig humanist.\\
    Han var knokleskæv og sikkert også kommunist.\\
    Så jeg spurgte hvad hans pensum var,\\
    og han kigged’ op og gav mig dette svar:\\\medskip
    ``Til sommer skal jeg op i Anders And,\\
    et nummer af Fantomet og en Superman,\\
    og hvis en dag jeg vågner af min døs,\\
    så bli’r jeg færdig og er arbejdsløs.''\\\medskip
    \emph{Har meldt pas! Tigger hash! Lever kun på nas!\\
    Når vi styrer verden får I alle gas!}
  \end{SBVerse}

  \begin{SBVerse}
    Jeg har set en rigtig biolog,\\
    og hun rulled’ sig i mudder’t som en anden so.\\
    Hendes hår var fyldt med andemad.\\
    Hun var måske køn hvis bar’ hun fik et bad.\\\medskip
    Jeg spurgte: "Er du kvinde eller mand?\\
    Hvorfor har du det mudder nede i din spand?\\
    Har du en kær’ste, kan jeg bli’ din fyr?"\\
    "Nej tak," sa’ hun, "jeg boller kun med dyr."\\\medskip
    \emph{Biolog, er en so, sku’ man ellers tro.\\
    Hvis du ellers får hend’ ren, så er hun go’.}
  \end{SBVerse}

  \begin{SBVerse}
    Jeg er go’, for jeg er datalog.\\
    Jeg har jakkesæt \& slips \& nye gummisko.\\
    Mine bumser fik lidt kirurgi,\\
    og nu er det mig som pigerne kan li’.\\\medskip
    I Rungsted lever jeg i sus og dus\\
    med Porsche, fast forbindelse og kæmpe hus.\\
    Jeg lever højt på Internet-trafik,\\
    og tjener fedt på andre fjolsers klik.\\\medskip
    \emph{Jeg var fed, jeg var led, på succes jeg red\\
    indtil sidste år da aktierne gik ned.}
  \end{SBVerse}
\end{song}
\begin{song}{Din mor}
  {} % Bruges ikke, lad stå blank
  {Gaston, Disney} % Titel, Kunstner - eks.: "Jutlandia, Kim Larsen". Hvis sangen er på sin egen melodi, brug da \SBOrgMel.
  {Ole Søe Sørensen og Christian Bladt Brandt} % Navnet på forfatteren. Undlad aliasser. Brug "&" frem for "og". Hvis forfatter er ukendt, lad da stå tom.
  {\TKET{}, 2011} % Eks. "Fysikrevy 2010" eller "2010"
  {\NotCCLIed} % Lad stå som den er

  \begin{SBSection*}
    Hør, unge mand, hvilket sprog De dog har!\\
    Og hvad mon De bilder Dem ind?\\
    Stod det til mig, ja så ville De snart\\
    føle min hånd på din kind.
  \end{SBSection*}

  \begin{SBSection*}
    For opdrag’lse kræver en hånd der er fast,\\
    og lejlighedsvis ogs’ et bat!\\
    Jeg ved, at De nok ej kan lægges til last,\\
    for det er jo tydeligt at:
  \end{SBSection*}

  \begin{SBVerse}
    Ingen er som din mor så’n en mær som din mor.\\
    Ingen ifør’r sig kejserens klæ’r som din mor.\\
    For der er ingen skøge i Aalborg\\
    med et rygte så blakket som hun.\\
    Hun er skaldet og trind li’som Voldborg.\\
    Ud’n at sige for meget så siger jeg kun:
  \end{SBVerse}

  \begin{SBVerse}
    Ingen sug’r som din mor, ingen slug’r som din mor,\\
    ingen burde gå på slankekur som din mor.\\
    Der er ingen så fed her i kongeriget,\\
    ingen så stor som din mor.
  \end{SBVerse}

  \begin{SBVerse}
    Ingen ved som din mor hver en ed som din mor,\\
    ingen har en så udforsket sked’ som din mor.\\
    Hendes skød er som fadkoteletter;\\
    sejt og med klumper af brusk.\\
    Altså det har jeg hørt fra min fætter.\\
    Hun rider dig hårdt med en pisk som en kusk.
  \end{SBVerse}

  \begin{SBVerse}
    Ingen skrig’r som din mor på lidt lir som din mor.\\
    Ingen lær’ os om blomster og bier som din mor.\\
    Inspirerede Munch til at male Skriget!\\
    Hip hip hurra for din mor.
  \end{SBVerse}

  \begin{SBSection*}
    Som lille der fik jeg historier fortalt\\
    om lig’ne der rådner i jord.\\
    Og jeg ville langt hel’re døden ha’ valgt\\
    end at kaste et blik på din mor\ldots
  \end{SBSection*}

  \begin{SBVerse}
    Ingen har som din mor et pessar som din mor,\\
    der’ på stør’lse med et badekar som din mor.\\
    Hun er mægtig som bølgernes fråden,\\
    og har tusinde mænd i sin favn.\\
    Hun er løs, båd’ i kød’t og på tråden.\\
    Du har titusind søskende i hver en havn.
  \end{SBVerse}

  \begin{SBVerse}
    Ingen slår som din mor, og har hår som din mor,\\
    ingen sagde så meget i går som din mor!\\
    Alle helvedes pinsler er blid massage\\
    i forhold til – din mor!
  \end{SBVerse}
\end{song}
\begin{song}{Glad for Fysik}
  {} % Bruges ikke, lad stå blank
  {Quangs sang, Anders Matthesen} % Titel, Kunstner - eks.: "Jutlandia, Kim Larsen". Hvis sangen er på sin egen melodi, brug da \SBOrgMel.
  {} % Navnet på forfatteren. Undlad kaldenavne. Brug gerne TBF. Brug "&" frem for "og". Hvis forfatter er ukendt, lad da stå tom.
  {FysikRevy, 2013} % Eks. "Fysikrevy, 2010" eller "2010"
  {\NotCCLIed} % Lad stå som den er

  \begin{SBVerse}
	Her er en historie om en dreng der hedder Finn.\\
	Han er humanist på 7. år, og fatter ik' en pind.\\
	KUA det er mørkt og goldt, og Finn har studiegæld.\\
	Så han prøver sig med Logos, for det meste uden held.
  \end{SBVerse}

  \begin{SBVerse}
	Jeg har kendt en smart jurist, hvis navn jeg nu har glemt.\\
	Han er specialist i at slikke røv, og han har det ikke nemt.\\
	Han pukler hele natten, til den næste morgen gryr.\\
	Kopierer ting for rige folk og lever af geby'r.
  \end{SBVerse}

  \begin{SBChorus}
	Så hvordan kan du dog sige, at du ikke er tilfreds?\\
	At din regnelærer er dum og sur og dine lektier gi'r dig stress?\\
	Du har alt, hvad du skal bruge, du har Schuams og hævegrej.\\
	Og der er tusind humanister, der ville ønske de var dig.
  \end{SBChorus}

  \begin{SBVerse}
	Tom sidder og koder med kroppen på et ton.\\
	Han har slet ingen venner, og hans kær'ste er hans hånd.\\
	Han scorer sjældent piger, men synes Ponyer er fedt.\\
	Han er groet fast bag skærmen, hvor han koder Latin-1.
  \end{SBVerse}

  \begin{SBChorus}
	Så hvad driver dig til at sige at studiet er grumt?\\
	At du ikke gider Newton og synes termo, det er dumt?\\
	Du har alt, hvad du skal bruge, du har Schuams og hævegrej.\\
	Og utallige dataloger ville ønske de var dig.
  \end{SBChorus}
\end{song}

\onecolumn
\SBChapter{Diverse}
\twocolumn
\begin{song}{Gaffa og WD-40, Oh my}
  {} % Bruges ikke, lad stå blank
  {Happy, Pharell Williams} % Titel, Kunstner - eks.: "Jutlandia, Kim Larsen". Hvis sangen er på sin egen melodi, brug da \SBOrgMel.
  {Jonatan, Jonas Kielsholm, Diderik \& Diana} % Navnet på forfatteren. Undlad kaldenavne. Brug gerne TBF. Brug "&" frem for "og". Hvis forfatter er ukendt, lad da stå tom.
  {TÅGEKAMMERET, 2014} % Eks. "Fysikrevy, 2010" eller "2010"
  {\NotCCLIed} % Lad stå som den er

  \begin{SBVerse}
    Hvad ville du gøre hvis dit hus ik havde et tag?\\
    Dine arme virker ikke for de falder af.\\
    Jeg bruger det bedste værktøj, creme d’la creme.\\
    Uh, Hvis du lytter nu, bliver det hele så nemt.
  \end{SBVerse}

  \begin{SBChorus}
    \emph{Jeg bruger Gaffa!}
    Hvis din bil den mangler dæk og dit tv det går itu.\\
    \emph{Jeg bruger Gaffa!}
    Hvis din hund den løber væk, men det var egentlig ik’ det den sku’.\\
    \emph{Jeg bruger Gaffa!}
    Hvis du ik’ ka lukke kærstens mund eller har hårvækst du gerne vil fjern’.\\
    \emph{Jeg bruger Gaffa!}
    Når du pepper dit sexliv op men du har ikke nogen gode håndjern.
  \end{SBChorus}

  \begin{SBVerse}
    Gaffa er fint, hvis det er stationært.\\
    Skal det køre rundt, bliver det en smule svært.\\
    Bedstefars gigt, det klares i en fart,\\
    buksen ryger på med mit præparat.
  \end{SBVerse}

  \begin{SBChorus}
    \emph{WD-40!} Når dit køleskab er tomt, men din rugbrødsmad mangler smør.\\
    \emph{WD-40!} Når din kærest’ har hovedpin’, men du selv er i godt humør.\\
    \emph{WD-40!} Når din hamster piver lidt, og du sidder med tømmermænd.\\
    \emph{WD-40!} Mod slidte led og mod stramme låg, det hele løses med et klem.
  \end{SBChorus}

  \begin{SBSection*}
    Ingenting, der er bedre end\\
    Ingenting, der er bedre end\\
    \emph{(WD, Gaffa, 40, Gaffa, Gaffa, WD, gaffa, 40)}
  \end{SBSection*}

  \begin{SBChorus}
    \emph{Gaffa!} Hvis du er til forelæsning og øjnene de falder i.\\
    \emph{WD-40!} Yndlings-olie-bi-produkt, for alt det andet er blasfemi!\\
    \emph{Brug kun gaffa!} Hvis du har et åbent sår, og din læge ik’ kan fikse det.\\
    \emph{WD-40!} Dit Halting-problem løses og dit NP bliver lig med P.
  \end{SBChorus}
\end{song}
\begin{song}{Doom Doom}
  {} % Bruges ikke, lad stå blank
  {Melodi} % Titel, Kunstner - eks.: "Jutlandia, Kim Larsen". Hvis sangen er på sin egen melodi, brug da \SBOrgMel.
  {Forfatter} % Navnet på forfatteren. Undlad kaldenavne. Brug gerne TBF. Brug "&" frem for "og". Hvis forfatter er ukendt, lad da stå tom.
  {Anledning og år} % Eks. "Fysikrevy, 2010" eller "2010"
  {\NotCCLIed} % Lad stå som den er

  \begin{SBVerse}
    % Skriv vers her
  \end{SBVerse}

  \begin{SBChorus}
    % Skriv omkvæd her
  \end{SBChorus}

  \begin{SBSection*}
    % Skriv sektioner her. Hvis du ønsker lidt mellemrum for at give luft i et langt afsnit el.lign., brug da \\\medskip
  \end{SBSection*}
\end{song}
\begin{song}{Jeg' en nørd}
  {} % Bruges ikke, lad stå blank
  {Pokémon} % Titel, Kunstner - eks.: "Jutlandia, Kim Larsen". Hvis sangen er på sin egen melodi, brug da \SBOrgMel.
  {} % Navnet på forfatteren. Undlad kaldenavne. Brug gerne TBF. Brug "&" frem for "og". Hvis forfatter er ukendt, lad da stå tom.
  {FysikRevy, 2010} % Eks. "Fysikrevy, 2010" eller "2010"
  {\NotCCLIed} % Lad stå som den er

  \begin{SBVerse}
    AD\&D, det er min leg,\\
    og Go det er min sport.\\
    Nyt tøj, det bytter jeg\\
    til sjældne magickort.
  \end{SBVerse}

  \begin{SBVerse}
    World of Warcraft er så fedt.\\
    Jeg spiller natten lang.\\
    Back to the Future har jeg set\\
    for 85. gang!
  \end{SBVerse}

  \begin{SBChorus}
    Jeg' en nørd! Jeg er stolt af det.\\
    Jeg læser XKCD. Jeg' en nørd!\\
    Skriver i \LaTeX,\\
    og jeg tænder på $dx$.\\\medskip
    Jeg' en nørd! Nu er jeg ved\\
    at finde en at nørde med,\\
    som ka’ ta’ min uskyldighed.\\
    Jeg' en nørd, og jeg er stolt af det!\\\medskip
    Jeg er stolt af det!
  \end{SBChorus}

  \begin{SBVerse}
    Jeg er level 10 i sværd,\\
    så spil nu ikke smart.\\
    I min bæltetaske er\\
    der terninger parat.
  \end{SBVerse}

  \begin{SBVerse}
    Sig at det skal være os,\\
    og mærk min energi.\\
    Når jeg læser Hitchhikers’\\
    Guide to the Galaxy.
  \end{SBVerse}

  \begin{SBChorus}
    Jeg' en nørd! Jeg er stolt af det.\\
    Jeg læser XKCD. Jeg' en nørd!\\
    Skriver i \LaTeX,\\
    og jeg tænder på $dx$.\\\medskip
    Jeg' en nørd! Nu er jeg ved\\
    at finde en at nørde med,\\
    som ka’ ta’ min uskyldighed.\\
    Jeg' en nørd, og jeg er stolt af det!\\\medskip
    Jeg er stolt af det!\\\medskip
    \emph{Jeg' en nørd!}
  \end{SBChorus}

  \begin{SBSection*}
    % Skriv sektioner her. Hvis du ønsker lidt mellemrum for at give luft i et langt afsnit el.lign., brug da \\\medskip
  \end{SBSection*}
\end{song}

\begin{song}{Ode til bacon}
  {} % Bruges ikke, lad stå blank
  {I want it that way, Backstreet Boys} % Titel, Kunstner - eks.: "Jutlandia, Kim Larsen". Hvis sangen er på sin egen melodi, brug da \SBOrgMel.
  {Rasmus Fruergaard-Pedersen} % Navnet på forfatteren. Undlad aliasser. Brug "&" frem for "og". Hvis forfatter er ukendt, lad da stå tom.
  {TÅGEKAMMERETs Julerevy, 2005} % Eks. "Fysikrevy 2010" eller "2010"
  {\NotCCLIed} % Lad stå som den er

  \begin{SBVerse}
    Gik hen i Fø\TeX.\\
    Det var en refleks.\\
    Det var med en ræson.\\
    Jeg sku’ ha’ bacon.
  \end{SBVerse}

  \begin{SBVerse}
    Jeg så i disken,\\
    helt tom. Jeg følte\\
    på min vom – var som beton.\\
    Den skal ha’ bacon.
  \end{SBVerse}

  \begin{SBChorus}
    Hvorfor er der tomt i køledisken?\\
    Hvorfor er der ingen lækkerbidsken?\\
    Og hvor er den gris der’ skåret i facon?\\
    Jeg vil ha’ bacon.
  \end{SBChorus}

  \begin{SBVerse}
    Jeg så i vantro.\\
    (Vægt)-vogterne de stod og lo.\\
    Kom nu herhen smag vor karton.\\
    Nej, jeg vil ha’ bacon.
  \end{SBVerse}

  \begin{SBChorus}
    Hvorfor er der tomt i køledisken\ldots
  \end{SBChorus}

  \begin{SBSection*}
    Smagen kan give dig lysten til livet,\\
    selvom du er vegetar, Aaah!\\
    Du blir hvad du spiser, så blir jeg en gris, ja.\\
    Det’ hel’re end hundred’ år!
  \end{SBSection*}

  \begin{SBVerse}
    Stegt let med smør til.\\
    Svøbt om pølser på grill.\\
    Stegt sprødt, stegt sprødt, stegt sprødt, stegt sprødt\\
    \ldots
  \end{SBVerse}

  \begin{SBSection*}
    \emph{Svøb flæsk i bacon!}
  \end{SBSection*}

  \begin{SBSection*}
    Stegt i smør eller helt naturel.\\
    Føler mig så høj og kulturel.\\
    Jeg ku’ synge ud fra en balkon:\\
    "Jeg vil ha’ bacon."
  \end{SBSection*}

  \begin{SBChorus}
    Smag på lidt bacon svøbt om dejlig mørksej.\\
    Smag på lidt bacon fyldt i julepostej.\\
    Smag på lidt bacon strøget let med stærk dijon, dijon.\\
    Smag på lidt bacon.
  \end{SBChorus}

  \begin{SBChorus}
    Giv mig en bacon-krydder-frikadelle.\\
    Jeg har lyst til en saltet stegt fedtcelle.\\
    Hvem laver mig en saltet flæskesværs-bonbon?\\
    Jeg hylder bacon.
  \end{SBChorus}

  \begin{SBSection*}
    For jeg vil ha’ bacon.
  \end{SBSection*}
\end{song}
  \begin{song}{Fulbert og Beatrice}
  {} % Bruges ikke, lad stå blank
  {\SBOrgMel} % Titel, Kunstner - eks.: "Jutlandia, Kim Larsen". Hvis sangen er på sin egen melodi, brug da \SBOrgMel.
  {Jens Louis Petersen} % Navnet på forfatteren. Undlad aliasser. Brug "&" frem for "og". Hvis forfatter er ukendt, lad da stå tom.
  {1951} % Eks. "Fysikrevy 2010" eller "2010"
  {\NotCCLIed} % Lad stå som den er
  \fancyhead[CE,CO]{\CHeadFont10}

  \begin{SBVerse}
    I frankens rige, hvor floder rinde\\
    som sølverstrømme i lune dal,\\
    lå ridderborgen på bjergets tinde\\
    med slanke tårne og gylden sal.\\\medskip
    Og det var sommer med blomsterbrise\\
    og suk af elskov i urtegård.\\
    Og det var Fulbert og Beatrice,\\
    og Beatrice var sytten år.
  \end{SBVerse}

  \begin{SBVerse}
    De havde leget som børn på borgen,\\
    mens Fulbert endnu var gangerpilt.\\
    Men langvejs drog han en årle morgen,\\
    mod Saracenen han higed' vildt.\\\medskip
    Han spidded' tyrker som pattegrise,\\
    et tusind stykker blev lagt på bår,\\
    for Fulbert kæmped' for Beatrice,\\
    og Beatrice var sytten år.
  \end{SBVerse}

  \begin{SBVerse}
    Med gluttens farver på sølversaddel\\
    han havde stridt ved Jerusalem.\\
    Han kæmped' kækt uden frygt og dadel\\
    og gik til fods hele vejen hjem.\\\medskip
    Nu sad han atter på bænkens flise\\
    og viste stolt sine heltesår,\\
    som ganske henrykked' Beatrice\\
    for Beatrice var sytten år.
  \end{SBVerse}

  \begin{SBVerse}
    En kappe prydet med små opaler\\
    og smagfuldt ternet med tyrkens blod,\\
    en ring af guld og et par sandaler\\
    den ridder lagde for pigens fod.\\\medskip
    Og da hun øjnede hans caprise\\
    blev hjertet mygt i den væne mår.\\
    Af lykke dånede Beatrice,\\
    for Beatrice var sytten år.
  \end{SBVerse}

  \begin{SBVerse}
    Da banked' blodet i heltens tinding,\\
    thi ingen helte er gjort af træ.\\
    Til trods for plastre og knæforbinding\\
    sank ridder Fulbert med stil på knæ.\\\medskip
    Han kvad: ``Skønjomfru, oh skænk mig lise,\\
    thi du alene mit hjerte rår!''\\
    ``Min helt, min ridder,'' kvad Beatrice,\\
    for Beatrice var sytten år.
  \end{SBVerse}

  \begin{SBVerse}
    Og der blev bryllup i højen sale\\
    med guldpokaler og troubadour,\\
    og under sange og djærven tale\\
    blev Fulbert ført til sin jomfrus bur.\\\medskip
    Og følget hvisked' om øm kurtice\\
    og skæmtsom puslen blandt dun og vår.\\
    For det var Fulbert og Beatrice,\\
    og Beatrice var sytten år.
  \end{SBVerse}

  \begin{SBVerse}
    Men ridder Fulbert den samme aften\\
    af borgens sale blev båren død.\\
    Den megen krig havde tær't på kraften,\\
    og sejrens palmer den sidste brød.\\\medskip
    Oh bejler, lær da af denne vise:\\
    Ød ej din kraft under krigens kår.\\
    Nej, spar potensen til Beatrice,\\
    når Beatrice er sytten år.
  \end{SBVerse}
\end{song}
\begin{song}{$\alpha\beta$-sangen}{}
  {I en kælder sort som kul, Vilhelm Høm}
  {Tenna Schaldemose og Tine Nyeng}
  {\TKET{}s Majrevy, 2004}
  {\NotCCLIed}

  \begin{SBVerse}
    $\alpha\ \beta\ \gamma\ \delta$\hspace{-2.5pt}\protect\colorbox{white}{\phantom{$\delta$}}\\
    \hspace{-2.5pt}$\delta$\hspace{-1.25em}\colorbox{white}{\phantom{$\delta$}}$\ \varepsilon\ \zeta$\\
    $\eta\ \theta\ \iota\ \kappa$\hspace{-2.5pt}\protect\colorbox{white}{\phantom{$\kappa$}}\\
    \hspace{-3pt}$\kappa$\hspace{-1.3em}\colorbox{white}{\phantom{$\kappa$}}$\ \lambda\ \mu\ \nu\ \xi$\\
    $o\ \pi\ \rho\ \sigma$\\
    $\tau\ \upsilon\ \varphi\ \chi\ \psi$\\
    $\omega$ er sidste,\\
    $24$ på liste!
  \end{SBVerse}
\CBPageBrk
  \begin{xlatn}{alfa-beta-sangen}
    {}
    {}

    \begin{SBVerse}
      Alfa beta gamma del-\\
      ta epsilon zeta\\
      eta theta iota kap-\\
      pa lambda my ny ksi\\
      omikron pi rho sigma\\
      tau ypsilon phi chi psi\\
      omega er sidste,\\
      fireogtyve på liste!
    \end{SBVerse}
  \end{xlatn}
\end{song}
\fancyhead[CE,CO]{\CHeadFont\thepage}
\begin{song}{Sort snak}
  {} % Bruges ikke, lad stå blank
  {Storkespringvandet, Cæsar} % Titel, Kunstner - eks.: "Jutlandia, Kim Larsen". Hvis sangen er på sin egen melodi, brug da \SBOrgMel.
  {} % Navnet på forfatteren. Undlad kaldenavne. Brug gerne TBF. Brug "&" frem for "og". Hvis forfatter er ukendt, lad da stå tom.
  {} % Eks. "Fysikrevy, 2010" eller "2010"
  {\NotCCLIed} % Lad stå som den er

  \begin{SBVerse}
    $\aleph\ \aleph\ \aleph\ \aleph\ \chi$\\
    $\aleph\ \aleph\ \aleph\ \pi$\\
    $\aleph\ \pi\ \chi\ \aleph\ \pi\ \chi\ \pi$\\
    $\chi\ \chi\ \pi\ \aleph\ \lambda\ \aleph$
  \end{SBVerse}

  \begin{SBVerse}
    $\beta\ \beta\ \beta\ \beta\ \xi$\\
    $\beta\ \beta\ \beta\ \psi$\\
    $\beta\ \psi\ \xi\ \beta\ \psi\ \xi\ \psi$\\
    $\xi\ \xi\ \psi\ \beta\ \gamma\ \beta$
  \end{SBVerse}
\CBPageBrk
  \begin{xlatn}{Sort snak}
    {}
    {Oversat af SSM}

    \begin{SBVerse}
      Alef alef alef alef chi\\
      Alef alef alef pi\\
      Alef pi chi alef pi chi pi\\
      Chi chi pi alef lambda alef
    \end{SBVerse}
    \begin{SBVerse}
      Beta beta beta beta xi\\
      Beta beta beta psi\\
      Beta psi xi beta psi xi psi\\
      Xi xi psi beta gamma beta
    \end{SBVerse}
  \end{xlatn}
\end{song}
\begin{song}{En veganer}
  {} % Bruges ikke, lad stå blank
  {Indianer, Tøsedrengene} % Titel, Kunstner - eks.: "Jutlandia, Kim Larsen". Hvis sangen er på sin egen melodi, brug da \SBOrgMel.
  {} % Navnet på forfatteren. Undlad kaldenavne. Brug gerne TBF. Brug "&" frem for "og". Hvis forfatter er ukendt, lad da stå tom.
  {Birevy, 2015} % Eks. "Fysikrevy, 2010" eller "2010"
  {\NotCCLIed} % Lad stå som den er

  \begin{SBVerse}
    Står ved en pølsevogn,\\
    føler mig lidt som en klovn.\\
    Står og tåger ud i luften -\\
    hvordan endte jeg her?
  \end{SBVerse}

  \begin{SBVerse}
    Bacon, pølser, flæskesteg\\
    er ikke lige nogt’ for mig.\\
    Jeg vil gerne ud i skoven\\
    hvor jeg kommer fra.
  \end{SBVerse}

  \begin{SBChorus}
    En veganer.\\
    Spiser alt der gror,\\
    planter med lidt jord.\\
    En veganer\\
    kalder alle mig
  \end{SBChorus}

  \begin{SBVerse}
    Svampene ka’ jeg li.\\
    Ba-e-o-cy-stin gør mig fri,\\
    og min hjerne den ser farver,\\
    gør mig lykkelig
  \end{SBVerse}

  \begin{SBVerse}
    Min hjerne bliver helt brændt af\\
    af svampene jeg tog idag.\\
    Før jeg følte mig så fanget,\\
    svampe gjor’ mig fri
  \end{SBVerse}

  \begin{SBChorus}
    En veganer\\
    Spiser alt der gror\\
    svampe med lidt jord\\
    En veganer\\
    hvem kalder på mig?
  \end{SBChorus}

  \begin{SBChorus}
    En veganer\\
    Spiser alt der gror\\
    svampe med lidt jord\\
    En veganer\\
    nu er jeg for sej
  \end{SBChorus}

  \begin{SBVerse}
    I skovene hvor træerne gror\\
    begik jeg mit første mord.\\
    Fælder træer og lave buske,\\
    træblod i mit spor.
  \end{SBVerse}

  \begin{SBVerse}
    Er klar til at gå berserk.\\
    De mange planter jeg vil kværk'\\
    med min økse og min flamme,\\
    skoven brænder stærkt 
  \end{SBVerse}

  \begin{SBChorus}
    En veganer\\
    jager alt der gror\\
    på vor sarte jord\\
    En veganer\\
    planter dræber jeg
  \end{SBChorus}

  \begin{SBChorus}
    En veganer\\
    jager alt der gror\\
    på vor sarte jord\\
    En veganer\\
    buske brænder jeg
  \end{SBChorus}

  \begin{SBChorus}
    En veganer\\
    jager alt der gror\\
    på vor sarte jord\\
    En veganer\\
    blomster dræber jeg
  \end{SBChorus}

  \begin{SBChorus}
    En veganer\\
    jager alt der gror\\
    på vor sarte jord\\
    En veganer\\
    træer fælder jeg
  \end{SBChorus}

  \begin{SBChorus}
    En veganer\\
    jager alt der gror\\
    på vor sarte jord\\
    En veganer\\
    ukrudt dræber jeg
  \end{SBChorus}
\end{song}

\onecolumn
\SBChapter{UNF}
\twocolumn
\begin{song}{Minken Mink}
  {} % Bruges ikke, lad stå blank
  {Bjørnen Bjørn, Sigurd Barrett} % Titel, Kunstner - eks.: "Jutlandia, Kim Larsen". Hvis sangen er på sin egen melodi, brug da \SBOrgMel.
  {SSM \& SABH} % Navnet på forfatteren. Undlad aliasser. Brug "&" frem for "og". Hvis forfatter er ukendt, lad da stå tom.
  {UNF Computer Science Camp, 2015} % Eks. "Fysikrevy 2010" eller "2010"
  {\NotCCLIed} % Lad stå som den er

  \begin{SBSection*}
    \SBRepeat{Minken Mink er en mink.}\\
    Og den kan li' de fleste, den er så flink,\\
    For Minken Mink er en mink
  \end{SBSection*}

  \begin{SBSection*}
    Nogen tror at Minken er en ræv.\\
    Nogen tror at Minken er en husmår.\\
    Nogen tror at Minken er en kat.\\
    Men Minken Mink er en mink.\\
    Ja, Minken Mink er en mink!
  \end{SBSection*}
\end{song}
\begin{song}{ScienceCamps}
  {} % Bruges ikke, lad stå blank
  {Pokémon} % Titel, Kunstner - eks.: "Jutlandia, Kim Larsen". Hvis sangen er på sin egen melodi, brug da \SBOrgMel.
  {SSM, KBE, MGS, RES, MDE, DBR, HFA} % Navnet på forfatteren. Undlad aliasser. Brug "&" frem for "og". Hvis forfatter er ukendt, lad da stå tom.
  {UNF Revy, 2016} % Eks. "Fysikrevy 2010" eller "2010"
  {\NotCCLIed} % Lad stå som den er

  \begin{SBVerse}
    Jeg drager ud på livets vej,\\
    jeg har et enkelt mål\\
    ScienceCamps afholder jeg,\\
    min vilje er af stål!
  \end{SBVerse}

  \begin{SBVerse}
    Starter ud på Chemisty\\
    Og ta’r så til Fysik\\
    GDC og Medico,\\
    derefter Matematik
  \end{SBVerse}

  \begin{SBChorus}
    ScienceCamps, jeg kan klare dem!\\
    Multitasking er mit lod\\
    4 camps - åhh, jeg elsker det,\\
    bare sommer’n bli’r ved og ved\\\medskip
    4 camps, jeg ka’ klare det,\\
    det kræver bare koffein\\
    Jeg vil bare hjælpe til!\\
    ScienceCamps - kan jeg klare dem?\\
    Jeg kan klare dem!
  \end{SBChorus}

  \begin{SBVerse}
    Socialt ansvar på nanocamp,\\
    kasserer på SDC\\
    Fagligt team og så PR,\\
    koordinators vilje ske!
  \end{SBVerse}

  \begin{SBVerse}
    Første uge er ovre nu,\\
    Ser frem til kriminal\\
    Astro, Show og CSC\\
    Det bliver fænomenalt!
  \end{SBVerse}

  \begin{SBChorus}
    13 camps, jeg skal klare det!\\
    måske det er lidt hårdt\\
    20 camps, ååh, Discovery!\\
    Jeg skal smadre den rekord!\\\medskip
    40 camps, jeg har brug for hjælp!\\
    Jeg mangler stadig CSI!\\
    Robodays og Futureweek,\\
    Medicin, jeg skal bruge det!
  \end{SBChorus}

  \begin{SBVerse}
    Når presset det bliver alt for stort\\
    og jeg går psykisk ned\\
    Så tager jeg bare en ekstra camp,\\
    for jeg bliver ved og ved!
  \end{SBVerse}

  \begin{SBVerse}
    Min psykolog, han siger stop!\\
    - min læge ligeså\\
    Men jeg mangler Biotech,\\
    den må jeg lige på!
  \end{SBVerse}

  \begin{SBChorus}
    100 camps, alle skal jeg nå!\\
    Det er det jeg duer til\\
    1000 camps, åh Medicinal\\
    Robot og ISSC\\\medskip
    Alle camps, åh, jeg er lidt træt,\\
    men stemmerne bliver ved og ved\\
    Næste år, der står jeg klar!\\
    ScienceCamps, jeg kan klare dem,\\
    kan jeg klare dem? ScienceCamps!
  \end{SBChorus}
\end{song}
\begin{song}{Det var i 1944}
  {} % Bruges ikke, lad stå blank
  {Jutlandia, Kim Larsen} % Titel, Kunstner - eks.: "Jutlandia, Kim Larsen". Hvis sangen er på sin egen melodi, brug da \SBOrgMel.
  {DBR, MWR} % Navnet på forfatteren. Undlad kaldenavne. Brug gerne TBF. Brug "&" frem for "og". Hvis forfatter er ukendt, lad da stå tom.
  {UNF Revy, 2016} % Eks. "Fysikrevy, 2010" eller "2010"
  {\NotCCLIed} % Lad stå som den er

  \begin{SBVerse}
    Det var i 1944 eller cirka der omkring,\\
    da UNF blev stiftet\\
    Vi kommer fra foreningen, der hedder SNU,\\
    nen nu er de forældet\\
    Ung men’sker fra gym’asiet af\\
    uddeler naturvidenskab
  \end{SBVerse}

  \begin{SBChorus}
    Ja ja! Fra UNF af\\
    Vi kommer som altid med viden\\
    Newton med tyngdekraft og Bohr med fysik,\\
    og Bjerrum, han lav’d diagrammet
  \end{SBChorus}

  \begin{SBVerse}
    Vi holder foredrag og workshops, og vi tar’ på studietur\\
    Når vi formidler vor viden\\
    Rundt i hele landet, ja, med alle vor forsøg\\
    - og folk de falder på stribe\\
    Forskere, det skal vi alle vær’\\
    Come on students, vis os noget blær
  \end{SBVerse}

  \begin{SBChorus}
    Ja ja! Fra UNF af\ldots
  \end{SBChorus}

  \begin{SBVerse}
    Vi er fir' foreninger der dækker vores land\\
    men vi vil gern’ være flere\\
    Vi ha’d’ både Midtjylland og Sønderborg en gang\\
    Så’n er det ikke mere\\
    Arrangører på 16 år\\
    laver camps, som de ikke forstår
  \end{SBVerse}

  \begin{SBChorus}
    Ja ja! Fra UNF af\ldots
  \end{SBChorus}

  \begin{SBChorus}
    Ja ja! Fra UNF af\ldots
  \end{SBChorus}
\end{song}
\begin{song}{Din TBF}
  {} % Bruges ikke, lad stå blank
  {YMCA, Village People} % Titel, Kunstner - eks.: "Jutlandia, Kim Larsen". Hvis sangen er på sin egen melodi, brug da \SBOrgMel.
  {SSM, MWR, AAN} % Navnet på forfatteren. Undlad kaldenavne. Brug gerne TBF. Brug "&" frem for "og". Hvis forfatter er ukendt, lad da stå tom.
  {UNF Revy, 2016} % Eks. "Fysikrevy, 2010" eller "2010"
  {\NotCCLIed} % Lad stå som den er

  \begin{SBVerse}
    Du der - du skal vær’ arrangør\\
    Jeg sa’e du der - hvis du ellers da tør\\
    Jeg sa’e du der - hvis du vil være med\\
    til at gør’ en kæmpe forskel
  \end{SBVerse}

  \begin{SBVerse}
    Du der, hvis du skriver dit navn\\
    på den her er--klæring bliver dit savn\\
    for naturvi--denskab fuldstændig væk\\
    og du bliver en af os nu
  \end{SBVerse}

  \begin{SBChorus}
    For du skal bare ha' en TBF\\
    Så bliver du en del af UNF\\
    Så kan du logge ind på vor’s eget system\\
    Du kan rode rundt i vores ting\\\medskip
    Ja, du skal bare have en TBF\\
    Så bliver du en del af UNF\\
    Men pas på hvad du si’r, hvis du let bli’r genert\\
    For du bliver jo nok citeret
  \end{SBChorus}

  \begin{SBVerse}
    Du der, har du fanget det nu?\\
    Jeg si’r du der, er du klar til at brug’\\
    din tid på at sætte deadlines til folk\\
    som allig’vel bryder dem, og
  \end{SBVerse}

  \begin{SBVerse}
    har du nu helt styr på hvordan\\
    du kontakter forelæser så han\\
    siger ja tak så’n at vi kan få fyldt\\
    hele vor's sæsonprogram ud
  \end{SBVerse}

  \begin{SBChorus}
    For du har lige fået din TBF\\
    Så nu er du en del af UNF\\
    Referater er der, formularer med mer’,\\
    og din pizzafaktor er her\\\medskip
    For du har lige fået din TBF\\
    Så nu er du en del af UNF\\
    Hvis du får brug for hjælp, eller bliver i tvivl\\
    kan en QuickPoll redde dit livl
  \end{SBChorus}

  \begin{SBSection*}
    \emph{Din TBF!}
  \end{SBSection*}
\end{song}







\begin{song}{Hymne til Matematik Camp}
  {} % Bruges ikke, lad stå blank
  {I Danmark er jeg født, H. C. Andersen} % Titel, Kunstner - eks.: "Jutlandia, Kim Larsen". Hvis sangen er på sin egen melodi, brug da \SBOrgMel.
  {Søren Møller} % Navnet på forfatteren. Undlad kaldenavne. Brug gerne TBF. Brug "&" frem for "og". Hvis forfatter er ukendt, lad da stå tom.
  {Matematik Camp, 2014} % Eks. "Fysikrevy, 2010" eller "2010"
  {\NotCCLIed} % Lad stå som den er

  \begin{SBVerse}
    På campen har jeg lært, der har jeg hjemme\\
    Der er alt sandt, derfra min fremtid går.\\
    Du mat'matik, du er min egen stemme,\\
    så klart beviserne min hjerne når.\\\medskip
    Min tankes $\pi$ og $\rho$,\\
    hvor Euklids elementer\\
    står mellem algebra og rentesrenter.\\
    Dig agter jeg -- MatCamp mit origo.
  \end{SBVerse}

  \begin{SBVerse}
    Hvor lægges næste år mon sommerskolen?\\
    Mon ik' det er ved jydens Unisø.\\
    Et yndigt sted hvor alle knækker koden,\\
    så dejligt som på Danas egen ø.\\\medskip
    Min tankes $\pi$ og $\rho$,\\
    hvor aksiomer holder.\\
    Gauss gav os alt, han er sandhedens tolder.\\
    Dig agter jeg -- MatCamp mit origo.
  \end{SBVerse}

  \begin{SBVerse}
    Engang du herre var i kloges færden.\\
    Bød over dumhed - nu du kaldes led.\\
    Et lille fag som dog ved alt om verden,\\
    og hører vores sang og troskabsed.\\\medskip
    Min tankes $\pi$ og $\rho$\\
    funktioners nulpunkt finder.\\
    Gauss giv dig rødder, som vi gi'r dig minder.\\
    Dig agter jeg -- MatCamp mit origo.
  \end{SBVerse}

  \begin{SBVerse}
    Se divergens ved polyedrets kanter,\\
    der gi'r uend'ligt mange mindre skridt;\\
    og symmetri blandt vores kommutanter\\
    gør stadig majoranten lige skidt.\\\medskip
    Min tankes $\pi$ og $\rho$,\\
    hvor grænser frit bestemmes,\\
    og ingen sig ved udvælgelsen græmmes.\\
    Dig agter jeg -- MatCamp mit origo
  \end{SBVerse}

  \begin{SBVerse}
    Du camp hvor jeg har lært, hvor jeg har hjemme,\\
    hvor alt er sandt, hvorfra min fremtid går,\\
    hvor mat'matik er livets klare stemme,\\
    og klart beviserne min hjerne når\\\medskip
    Min tankes $\pi$ og $\rho$\\
    ved videnskabens kilde.\\
    I mat'matik min hjerne alt vil vide.\\
    Dig agter jeg -- MatCamp mit origo.
  \end{SBVerse}
\end{song}
\begin{song}{Hvad må man? (i UNF)}
  {} % Bruges ikke, lad stå blank
  {Fy fy skamme, Omsen og Momsen} % Titel, Kunstner - eks.: "Jutlandia, Kim Larsen". Hvis sangen er på sin egen melodi, brug da \SBOrgMel.
  {KBE, RTR, MIL, DBR, SSM, MDE, RES} % Navnet på forfatteren. Undlad kaldenavne. Brug gerne TBF. Brug "&" frem for "og". Hvis forfatter er ukendt, lad da stå tom.
  {UNF Revy, 2016} % Eks. "Fysikrevy, 2010" eller "2010"
  {\NotCCLIed} % Lad stå som den er

  \begin{SBVerse}
    Må man flyv’ på første klasse?\\
    Må man spise mad på NOMA?\\
    Må man hæld’ så meget sprut på delta’r\\
    at de går i koma?\\\medskip
    Må man gamble med vor’s penge \\
    Må man drukne sig i vodka?\\
    Må man bruge Times New Roman \\
    når man laver sig en tryksag?
  \end{SBVerse}

  \begin{SBChorus}
    Næ næ næ næ næ det må vi ikke,\\
    fy fy skamme skamme fy fy ah ah\\
    slemme slemme fy fy næ næ nix nix\\
    slut forbudt - men hva må vi så?
  \end{SBChorus}

  \begin{SBVerse}
    Må man ændre vores logo?\\
    Må man slet’ vor’s database?\\
    Må man pille på en deltager \\
    og gå til næste fase?\\\medskip
    Må man bade under sprinkleren\\
    som er i lab’ratoriet?\\
    Må man sparke til en forelæ-\\
    ser ned’ i auditoriet?
  \end{SBVerse}

  \begin{SBChorus}
    Næ næ næ næ næ det må vi ikke,\\
    fy fy skamme skamme fy fy ah ah\\
    slemme slemme fy fy næ næ nix nix\\
    slut forbudt - men hva må vi så?
  \end{SBChorus}

  % \begin{SBChorus}
  %   Næ næ næ næ næ det må vi ikke\ldots
  % \end{SBChorus}

  \begin{SBVerse}
    Må man sove til et foredrag?\\
    Må man ryge pot på brandvagt?\\
    Må man rode på vor’s lager\\
    eller arranger’ en andagt?\\\medskip
    Må man udsmide plakater\\
    eller grin’ af folk der bløder?\\
    Må man be’ om hem’lig afstemning\\
    til alle vores møder?
  \end{SBVerse}

  \begin{SBChorus}
    Næ næ næ næ næ det må vi ikke,\\
    fy fy skamme skamme fy fy ah ah\\
    slemme slemme fy fy næ næ nix nix\\
    slut forbudt - men hva må vi så?
  \end{SBChorus}

  % \begin{SBChorus}
  %   Næ næ næ næ næ det må vi ikke\ldots
  % \end{SBChorus}

  \begin{SBVerse}
    Må man melde sig til opvask?\\
    Må passe på sin kasse?\\
    Må man komm’ med gode input\\
    når vi planlægger en masse?\\\medskip
    Må man sige fra i tide?\\
    Må man stille op til DK?\\
    Må man finde sig en UNF’er\\
    som man ka’ læg’ an på?
  \end{SBVerse}

  \begin{SBChorus}
    Ja ja ja ja ja det må vi gerne\\
    næ sikke fint fint ja ja meget det blir\\
    stort stort klap klap fint fint flot flot\\
    det var godt - nå det må vi godt
  \end{SBChorus}
\end{song}
\begin{song}{Jeg elsker "science"}
  {} % Bruges ikke, lad stå blank
  {Melodi} % Titel, Kunstner - eks.: "Jutlandia, Kim Larsen". Hvis sangen er på sin egen melodi, brug da \SBOrgMel.
  {Forfatter} % Navnet på forfatteren. Undlad kaldenavne. Brug gerne TBF. Brug "&" frem for "og". Hvis forfatter er ukendt, lad da stå tom.
  {Anledning og år} % Eks. "Fysikrevy, 2010" eller "2010"
  {\NotCCLIed} % Lad stå som den er

  \begin{SBVerse}
    % Skriv vers her
  \end{SBVerse}

  \begin{SBChorus}
    % Skriv omkvæd her
  \end{SBChorus}

  \begin{SBSection*}
    % Skriv sektioner her. Hvis du ønsker lidt mellemrum for at give luft i et langt afsnit el.lign., brug da \\\medskip
  \end{SBSection*}
\end{song}

% Først så var jeg bange, jeg var skrækslagen
% Jeg tænkte, det lød lig’så skrækkeligt som et stræklagen.
% Og jeg sad søvnløs hele natten
% stirred’ tomt ind i min pejs
% Før hed det naturvidenskab,
% nu skal man sige “science”

% Syn’s det var dumt - noget rigtigt lort
% Men nu der ingen vej tilbage, “science” er det nye sort.
% Jeg sku’ ha sagt noget eller gjort noget, 
% åh gået i protest
% Men sket er sket, og science er jo dansk når det er bedst

% Det’ ikk’ så slemt, og egentlig nemt
% Syv bogstaver?
% Det enkelt og bekvemt
% Hvorfor skal man slæbe rundt på gamle danske ord?
% Alle snakker engelsk nu, 
% ligegyldigt hvor de bor.

% Jeg elsker science, jeg elsker science!
% Fysik, kemi, biologi 
% det er jo alt for nice.
% Matematik, geologi 
% og medicinsk teknologi, jeg elsker science
% Jeg elsker science!

% \begin{song}{Vi er brandvagter}
  {} % Bruges ikke, lad stå blank
  {Melodi} % Titel, Kunstner - eks.: "Jutlandia, Kim Larsen". Hvis sangen er på sin egen melodi, brug da \SBOrgMel.
  {Forfatter} % Navnet på forfatteren. Undlad kaldenavne. Brug gerne TBF. Brug "&" frem for "og". Hvis forfatter er ukendt, lad da stå tom.
  {Anledning og år} % Eks. "Fysikrevy, 2010" eller "2010"
  {\NotCCLIed} % Lad stå som den er

  \begin{SBVerse}
    % Skriv vers her
  \end{SBVerse}

  \begin{SBChorus}
    % Skriv omkvæd her
  \end{SBChorus}

  \begin{SBSection*}
    % Skriv sektioner her. Hvis du ønsker lidt mellemrum for at give luft i et langt afsnit el.lign., brug da \\\medskip
  \end{SBSection*}
\end{song}

% Festen er slut
% For os som passer på jer nu
% Vi sidder her med vores film
% Du må sove godt
% Sidder alen’
% Tænk på festen vi har haft
% Ser I kommer meget vakst
% Og går til ro

% Vi er brandvag-(t)er
% Ja for vi er
% i U-N-F
% UNF

% Vi holder øje
% Og lytter efter brandalarm
% Hjælper jer-hvis den går igang
% Men nu er den tam

% Vi er brandvag-(t)er
% Ja for vi er
% i U-N-F
% UNF
% Vi er brandvag-(t)er
% Ja for vi er
% i U-N-F
% UNF

% Klokken slår to
% Nu er I alle gået til ro
% Og vi sidder kun os to
% Og’ vi kigger på en sko

% Vi er brandvag-(t)er
% Ja for vi er
% i U-N-F
% UNF
% Vi er brandvag-(t)er Rebecca
% For I skal ikk’ Maria
% Oh Rebecca
% Slå’s helt ihjel
% Af en brand, af en brand
% I UN-F Rebecca
% Af en brand (maria)


\onecolumn
\SBChapter{Studieliv og \LaTeX}
\twocolumn
\begin{song}{Tål daj' i læseferien}
  {} % Bruges ikke, lad stå blank
  {12 Days of Christmas} % Titel, Kunstner - eks.: "Jutlandia, Kim Larsen". Hvis sangen er på sin egen melodi, brug da \SBOrgMel.
  {Mona Holm \& Marianne Bangsø} % Navnet på forfatteren. Undlad kaldenavne. Brug gerne TBF. Brug "&" frem for "og". Hvis forfatter er ukendt, lad da stå tom.
  {TÅGEKAMMERETs Julerevy, 1987} % Eks. "Fysikrevy, 2010" eller "2010"
  {\NotCCLIed} % Lad stå som den er

  \begin{SBVerse}
    På den første dag i læseferien sagde jeg til mig selv:\\
    Der er god tid, jeg pjækker i dag.
  \end{SBVerse}

  \begin{SBVerse}
    På den anden dag i læseferien sagde jeg til mig selv:\\
    Mat’matik er trist,\\
    der er god tid, jeg pjækker i dag.
  \end{SBVerse}

  \begin{SBVerse}
    På den tredje dag i læseferien sagde jeg til mig selv:\\
    Jeg er træt,\\
    mat’matik er trist,\\
    der er god tid, jeg pjækker i dag.
  \end{SBVerse}

  \begin{SBVerse}
    På den fjerde dag i læseferien sagde jeg til mig selv:\\
    Tror jeg går på druk -- \emph{Skål!}\\
    Jeg er træt,\\
    mat’matik er trist,\\
    der er god tid, jeg pjækker i dag.
  \end{SBVerse}

  \begin{SBVerse}
    På den femte dag i læseferien sagde jeg til mig selv:\\
    Åh, tømmermænd!\\
    Tror jeg går på druk -- \emph{Skål!}\\
    Jeg er træt,\\
    mat’matik er trist,\\
    der er god tid, jeg pjækker i dag.
  \end{SBVerse}

  \begin{SBVerse}
    På den sjette dag i læseferien sagde jeg til mig selv:\\
    Der er fest i aften,\\
    åh, tømmermænd!\\
    Tror jeg går på druk -- \emph{Skål!}\\
    Jeg er træt,\\
    mat’matik er trist,\\
    der er god tid, jeg pjækker i dag.
  \end{SBVerse}

  \begin{SBVerse}
    På den syvende dag i læseferien sagde jeg til mig selv:\\
    Hvor er min bog nu?\\
    Der er fest i aften,\\
    åh, tømmermænd!\\
    Tror jeg går på druk -- \emph{Skål!}\\
    Jeg er træt,\\
    mat’matik er trist,\\
    der er god tid, jeg pjækker i dag.
  \end{SBVerse}

  \begin{SBVerse}
    På den ottende dag i læseferien sagde jeg til mig selv:\\
    Der' film i TV,\\
    hvor er min bog nu?\\
    Der er fest i aften,\\
    åh, tømmermænd!\\
    Tror jeg går på druk -- \emph{Skål!}\\
    Jeg er træt,\\
    mat’matik er trist,\\
    der er god tid, jeg pjækker i dag.
  \end{SBVerse}

  \begin{SBVerse}
    På den niende dag i læseferien sagde jeg til mig selv:\\
    Savner mine venner,\\
    der' film i TV,\\
    hvor er min bog nu?\\
    Der er fest i aften,\\
    åh, tømmermænd!\\
    Tror jeg går på druk -- \emph{Skål!}\\
    Jeg er træt,\\
    mat’matik er trist,\\
    der er god tid, jeg pjækker i dag.
  \end{SBVerse}

  \begin{SBVerse}
    På den tiende dag i læseferien sagde jeg til mig selv:\\
    Mon jeg kan nå det?\\
    Savner mine venner,\\
    der' film i TV,\\
    hvor er min bog nu?\\
    Der er fest i aften,\\
    åh, tømmermænd!\\
    Tror jeg går på druk -- \emph{Skål!}\\
    Jeg er træt,\\
    mat’matik er trist,\\
    der er god tid, jeg pjækker i dag.
  \end{SBVerse}

  \begin{SBVerse}
    På den elvete dag i læseferien sagde jeg til mig selv:\\
    Nu skal der læses!\\
    Mon jeg kan nå det?\\
    Savner mine venner,\\
    der' film i TV,\\
    hvor er min bog nu?\\
    Der er fest i aften,\\
    åh, tømmermænd!\\
    Tror jeg går på druk -- \emph{Skål!}\\
    Jeg er træt,\\
    mat’matik er trist,\\
    der er god tid, jeg pjækker i dag.
  \end{SBVerse}

  \begin{SBVerse}
    På den tolvte dag i læseferien sagde jeg til mig selv:\\
    Gud, det' i morgen!\\
    Nu skal der læses!\\
    Mon jeg kan nå det?\\
    Savner mine venner,\\
    der' film i TV,\\
    hvor er min bog nu?\\
    Der er fest i aften,\\
    åh, tømmermænd!\\
    Tror jeg går på druk -- \emph{Skål!}\\
    Jeg er træt,\\
    mat’matik er trist,\\\medskip
    Så'n gik tiden -- jeg dumped' i dag!
  \end{SBVerse}

  \begin{SBSection*}
    % Skriv sektioner her. Hvis du ønsker lidt mellemrum for at give luft i et langt afsnit el.lign., brug da \\\medskip
  \end{SBSection*}
\end{song}
\begin{song}{Homo-sangen}
  {} % Bruges ikke, lad stå blank
  {\SBOrgMel} % Titel, Kunstner - eks.: "Jutlandia, Kim Larsen". Hvis sangen er på sin egen melodi, brug da \SBOrgMel.
  {Jan Midtgaard} % Navnet på forfatteren. Undlad aliasser. Brug "&" frem for "og". Hvis forfatter er ukendt, lad da stå tom.
  {TÅGEKAMMERETs Julerevy, 1999} % Eks. "Fysikrevy 2010" eller "2010"
  {\NotCCLIed} % Lad stå som den er

  \begin{SBVerse}
    Livets mange glæder man jo dele kan,\\
    men det er nu bedt, så'n mand til mand'\\
    Intet er skam større end den ægte kærlighed,\\
    spring ud af skabet, tag din læsemakker med!
  \end{SBVerse}

  \begin{SBChorus}
    Får du lyst til rigtig mand igen,\\
    er det for lidt med kun en pige-ven.\\
    Du får et helt nyt syn på lineær algebra \emph{(shi-bu-du-ah)}\\
    når du gi’r din læsemakker den bagfra!
  \end{SBChorus}

  \begin{SBVerse}
    Når du går og voldtag’r en and ved unisø’n\\
    ved du den er gal med den der kløen.\\
    Til TÅGEKAMMER-fester du scorer ikke spor,\\
    den sidste pige som du kyssed’ var din mor!
  \end{SBVerse}

  \begin{SBChorus}
    Får du lyst til rigtig mand igen,\ldots
  \end{SBChorus}

  \begin{SBSection*}
  Her på matematisk der er pigerne få,\\
  her går sexlivet bestemt i stå.
  \end{SBSection*}

  \begin{SBChorus}
    Du får et helt nyt syn på lineær algebra \emph{(shi-bu-du-ah)}\\
    når du gi’r din læsemakker,\\
    når du gi'r din forelæser,\\
    når du gi'r din koordinator den bagfra!
  \end{SBChorus}
\end{song}
\begin{song}{Jeg vil læse .*}
  {} % Bruges ikke, lad stå blank
  {Regnvejrsdag i november, Pia Raug} % Titel, Kunstner - eks.: "Jutlandia, Kim Larsen". Hvis sangen er på sin egen melodi, brug da \SBOrgMel.
  {} % Navnet på forfatteren. Undlad kaldenavne. Brug gerne TBF. Brug "&" frem for "og". Hvis forfatter er ukendt, lad da stå tom.
  {DIKUrevy, 2008} % Eks. "Fysikrevy, 2010" eller "2010"
  {\NotCCLIed} % Lad stå som den er

  \begin{SBVerse}
    Jeg vil læse retorik,\\
    for jeg fatter ik' en brik.\\
    Jeg vil lær' om sprogfinesser\\
    og gå op i petitesser,\\
    jeg bli'r nok en stor professor.\\
    Jeg vil læse retorik.
  \end{SBVerse}

  \begin{SBVerse}
    Jeg vil være teolog,\\
    selvom lønnen ik' er god.\\
    Tro på gud det er det sande,\\
    også når man træder vande,\\
    missionere i fjerne lande.\\
    Jeg vil være teolog.
  \end{SBVerse}

  \begin{SBVerse}
    Jeg vil læse medicin,\\
    tjene penge som et svin.\\
    Jeg vil skær i døde men'sker,\\
    men jeg er jo også svensker,\\
    klæde mig i latexhandsker.\\
    Jeg vil læse medicin.
  \end{SBVerse}

  \begin{SBVerse}
    Jeg vil læse på kemi,\\
    hårde stoffer kan jeg li'.\\
    Jeg er skæv fra ni til sytten,\\
    finsprit hælder jeg i bøtten,\\
    så jeg ender nok på støtten.\\
    Jeg vil læse på kemi.
  \end{SBVerse}

  \begin{SBVerse}
    Jeg vil læse cand.polit.,\\
    for mit men'skesyn er skidt.\\
    Jeg vil skær' på hospitaler,\\
    og på skoler uden kvaler,\\
    mens jeg holder falske taler.\\
    Jeg vil læse cand.polit.
  \end{SBVerse}

  \begin{SBVerse}
    Jeg vil læse på fysik,\\
    lære kvantemekanik.\\
    Jeg går op i Einsteins tanker,\\
    og jeg er en rigtig dranker,\\
    kæler for mit fadølsanker.\\
    Jeg vil læse på fysik.
  \end{SBVerse}

  \begin{SBVerse}
    Jeg vil være officér\\
    i den danske dronnings hær.\\
    Jeg vil skyde talibaner',\\
    vifte med de danske faner,\\
    sprede død i lange baner.\\
    Jeg vil være officér
  \end{SBVerse}

  \begin{SBVerse}
    Historie er mit liv,\\
    støver rundt i et arkiv.\\
    Læser i de gamle skrifter,\\
    selvom jeg får mange rifter,\\
    jeg ved alt om Neros drifter.\\
    For historie er mit liv.
  \end{SBVerse}

  \begin{SBVerse}
    Jeg vil være pædagog,\\
    rende rundt i maosko.\\
    Hjælpe børn i overtøjet,\\
    så de ikke bli'r tilmøjet,\\
    du kan tro de er fornøjet.\\
    Jeg vil være pædagog.
  \end{SBVerse}

  \begin{SBVerse}
    Jeg vil være aktuar,\\
    hvor min skæbne den er klar.\\
    Forsikringsvidenskab er sagen,\\
    jeg vil ta' en bid af kagen,\\
    mens jeg sylter kundeklagen.\\
    Jeg vil være aktuar.
  \end{SBVerse}

  \begin{SBVerse}
    Jeg vil muge ud hos køer,\\
    køre rundt med trillebør.\\
    Sidde i en kæmpe traktor,\\
    moms det er en ukendt faktor,\\
    nu skal grisen ned til slagter.\\
    Jeg vil muge ud hos køer.
  \end{SBVerse}

  \begin{SBVerse}
    Jeg vil læse på MI,\\
    for musik kan alle li'.\\
    Selvom jeg var sidst i klassen\\
    kan jeg stadig score kassen,\\
    blot jeg lær' at spille bassen.\\
    Men jeg kan ikke holde takten.
  \end{SBVerse}

  \begin{SBVerse}
    Jeg vil læse æstetik\\
    i dansenes mystik.\\
    Svanens død den kan jeg lide,\\
    skønt jeg ej er en sylfide,\\
    vil jeg over scenen glide.\\
    Jeg vil læse æstetik.
  \end{SBVerse}

  \begin{SBVerse}
    Jeg vil læse nano tech,\\
    for min hjerne drak jeg væk\\
    sidte fredag på Caféen?,\\
    da jeg rigtig sku' gå te'en,\\
    og jeg vælted om i sneen.\\
    Jeg vil læse nano tech.
  \end{SBVerse}

  \begin{SBVerse}
    Jeg vil være datalog,\\
    og min fremtid den er god.\\
    gcc det er en gave,\\
    jeg blir nok en kodeslave,\\
    jeg vil gerne prøv' at snave.\\
    Jeg vil være datalog.
  \end{SBVerse}
\end{song}
\begin{song}{De tog stadig folk ind}
  {} % Bruges ikke, lad stå blank
  {Melodi} % Titel, Kunstner - eks.: "Jutlandia, Kim Larsen". Hvis sangen er på sin egen melodi, brug da \SBOrgMel.
  {Forfatter} % Navnet på forfatteren. Undlad kaldenavne. Brug gerne TBF. Brug "&" frem for "og". Hvis forfatter er ukendt, lad da stå tom.
  {Anledning og år} % Eks. "Fysikrevy, 2010" eller "2010"
  {\NotCCLIed} % Lad stå som den er

  \begin{SBVerse}
    % Skriv vers her
  \end{SBVerse}

  \begin{SBChorus}
    % Skriv omkvæd her
  \end{SBChorus}

  \begin{SBSection*}
    % Skriv sektioner her. Hvis du ønsker lidt mellemrum for at give luft i et langt afsnit el.lign., brug da \\\medskip
  \end{SBSection*}
\end{song}
\begin{song}{Eksamensangst}
  {} % Bruges ikke, lad stå blank
  {Min store kærlighed, Anders Matthesen} % Titel, Kunstner - eks.: "Jutlandia, Kim Larsen". Hvis sangen er på sin egen melodi, brug da \SBOrgMel.
  {} % Navnet på forfatteren. Undlad kaldenavne. Brug gerne TBF. Brug "&" frem for "og". Hvis forfatter er ukendt, lad da stå tom.
  {FysikRevy, 2014} % Eks. "Fysikrevy, 2010" eller "2010"
  {\NotCCLIed} % Lad stå som den er

  \begin{SBVerse}
    Uret tikker, hjertet det slår,\\
    jeg mærker pulsen inde i mit øre.\\
    Jeg får det værre som tiden går,\\
    men jeg har lovet at jeg nok forsøger.\\
    Jeg har dumpet alle mine fag.\\
    Jeg håber i dag er min lykkedag.
  \end{SBVerse}

  \begin{SBVerse}
    Solen falder blidt på min kind,\\
    og jeg mærker trykket i min blære.\\
    Jeg kan ikke nå at lette det nu;\\
    nu må det briste eller bære.\\
    Døren knirker, jeg gi'r et lille hvin.\\
    Jeg håber snart at turen er min.
  \end{SBVerse}

  \begin{SBChorus}
    Bare jeg nu kan få det sagt,\\
    det jeg har øvet og planlagt,\\
    det jeg har terpet ud og ind,\\
    eller går klappen ned igen?
  \end{SBChorus}

  \begin{SBVerse}
    Hånden ryster, halsen er tør.\\
    Jeg sveder kraftigt under mine arme.\\
    Jeg stirrer tomt lige ind i på den dør\\
    der står imellem mig og helvedets varme.\\
    De andre gør det gang på gang på gang,\\
    for mig er det stadig ulidelig tvang.
  \end{SBVerse}

  \begin{SBChorus}
    Bare jeg nu kan få det sagt,\\
    det jeg har øvet og planlagt,\\
    det jeg har terpet ud og ind,\\
    eller går klappen ned igen?
  \end{SBChorus}

  % \begin{SBChorus}
  %   Bare jeg nu kan få det sagt,\ldots
  % \end{SBChorus}

  \begin{SBSection*}
    Bare jeg ku forklare,\\
    bare censor ku' prøv' at forstå,\\
    at mundtlig eksamen ik' er sag'n\\
    når man bli'r så nervøs at man går i stå.
  \end{SBSection*}

  \begin{SBChorus}
    Bare jeg nu kan få det sagt,\\
    det jeg har øvet og planlagt,\\
    det jeg har terpet ud og ind,\\
    eller går klappen ned igen?
  \end{SBChorus}

  \begin{SBChorus}
    Bare jeg nu kan få det sagt,\\
    det jeg har øvet og planlagt,\\
    det jeg har terpet ud og ind,\\
    eller går klappen ned igen?
  \end{SBChorus}

  % \begin{SBChorus}
  %   Bare jeg nu kan få det sagt,\ldots
  % \end{SBChorus}

  % \begin{SBChorus}
  %   Bare jeg nu kan få det sagt,\ldots
  % \end{SBChorus}
\end{song}
\begin{song}{Fuck mit liv}
  {} % Bruges ikke, lad stå blank
  {Melodi} % Titel, Kunstner - eks.: "Jutlandia, Kim Larsen". Hvis sangen er på sin egen melodi, brug da \SBOrgMel.
  {Forfatter} % Navnet på forfatteren. Undlad kaldenavne. Brug gerne TBF. Brug "&" frem for "og". Hvis forfatter er ukendt, lad da stå tom.
  {Anledning og år} % Eks. "Fysikrevy, 2010" eller "2010"
  {\NotCCLIed} % Lad stå som den er

  \begin{SBVerse}
    % Skriv vers her
  \end{SBVerse}

  \begin{SBChorus}
    % Skriv omkvæd her
  \end{SBChorus}

  \begin{SBSection*}
    % Skriv sektioner her. Hvis du ønsker lidt mellemrum for at give luft i et langt afsnit el.lign., brug da \\\medskip
  \end{SBSection*}
\end{song}
\begin{song}{Hvad skal vi \TeX'e i nat?}
  {} % Bruges ikke, lad stå blank
  {Hvor skal vi sove i nat?, Laban} % Titel, Kunstner - eks.: "Jutlandia, Kim Larsen". Hvis sangen er på sin egen melodi, brug da \SBOrgMel.
  {} % Navnet på forfatteren. Undlad kaldenavne. Brug gerne TBF. Brug "&" frem for "og". Hvis forfatter er ukendt, lad da stå tom.
  {DIKUrevy, 2007} % Eks. "Fysikrevy, 2010" eller "2010"
  {\NotCCLIed} % Lad stå som den er

  \begin{SBVerse}
    Vi fik rapporten, og tænkte straks det samme;\\
    går den, så går den, og drak os bankelamme.\\
    Vi' alt for kloge, vi' nogle seje gutter.\\
    Selv i en tåge, ta'r den kun fem minutter.
  \end{SBVerse}

  \begin{SBVerse}
    Så fandt vi siden, hvor alting blev beskrevet.\\
    Vi mangled' viden, for at få alt det lavet.\\
    Vi hader livet under rapportregimer.\\
    Vi er fortvivlet, vi har kun fjorten timer.
  \end{SBVerse}

  \begin{SBChorus}
    Hvad skal vi \TeX'e i nat? Vi har jo intet lavet.\\
    Hvad skal vi \TeX'e i nat? Vi er slet ikke færdig'.\\
    Vi er jo fuldstændig sat. Vi emmed' af arrogancen,\\
    og mistede chancen for at gøre det godt.
  \end{SBChorus}

  \begin{SBChorus}
    Hvad skal vi \TeX'e i nat? Vi ku' jo lave genbrug.\\
    Hvad skal vi \TeX'e i nat? Så printer jeg den ud nu.\\
    Vi ku' jo drikke en sjat. Vi ta'r en overspringshandling,\\
    det gi'r en forvandling, så vi tror det er flot.
  \end{SBChorus}

  \begin{SBVerse}
    Vi lærte lektien, vi læser alle sider.\\
    Så nu er vi mænd der ikke går og lider.\\
    Vi starter tidligt og knokler meget længe.\\
    Vi får lidt slidgigt -- ser sjældent vore senge.
  \end{SBVerse}

  \begin{SBVerse}
    Men hov hvad er det, er det det gale pensum?\\
    Ja, så forstår jeg der var så tavst på forum.\\
    Vi starter forfra, vi føler os som tåber,\\
    og i kantinen er det kun os der råber.
  \end{SBVerse}

  \begin{SBChorus}
    Hvad skal vi \TeX'e i nat? Vi har jo intet lavet.\\
    Hvad skal vi \TeX'e i nat? Vi er slet ikke færdig'.\\
    Vi er jo fuldstændig sat. Vi emmed' af arrogancen,\\
    og mistede chancen for at gøre det godt.
  \end{SBChorus}

  \begin{SBChorus}
    Hvad skal vi \TeX'e i nat? Vi ku' jo lave genbrug.\\
    Hvad skal vi \TeX'e i nat? Så printer jeg den ud nu.\\
    Vi ku' jo drikke en sjat. Vi ta'r en overspringshandling,\\
    det gi'r en forvandling, så vi tror det er flot.
  \end{SBChorus}
\end{song}
\begin{song}{Brug \LaTeX}
  {} % Bruges ikke, lad stå blank
  {Spørg om hjælp, Anders Matthesen}
  {Morten Jensen}
  {TÅGEKAMMERET, 2015}
  {\NotCCLIed}

  \begin{SBChorus}
    Brug \LaTeX, når du skriver ind.\\
    For \TeX{} er smart, og \TeX{} er smukt, det luner i mit sind.\\
    Du må ikke bruge Word, for det er noget bøvl,\\
    og hvis du gør alligevel, så får du satme høvl!
  \end{SBChorus}

  \begin{SBVerse}
    Jeg fik engang en aflevering fra en sød student.\\
    Han havde potentiale til at blive verdenskendt;\\
    men han afleverede et dokument i Word,\\
    og det betød, hans bachelor den ej blev gennemført.
  \end{SBVerse}

  \begin{SBVerse}
    Og Lone, hun var også ganske dygtig, da hun kom.\\
    Hun var klog, og hun var sød, dog sku’ man brug’ kondom,\\
    og hendes løse læber så jeg ikk’ som et problem;\\
    men da hun tegned’ cosinus i Paint, fik jeg eksem!
  \end{SBVerse}

  \begin{SBChorus}
    Med \LaTeX, alt er som en leg;\\
    men bru’r du Maple, Word og Paint, så vil jeg straffe dig.\\
    Personalet her på stedet ved godt, hvad de skal,\\
    når nogen ikke bru’r \LaTeX, så gi’r de mig et kald.
  \end{SBChorus}

  \begin{SBVerse}
    Jørgen, han var også fin, trods han var humanist.\\
    Han vidste alt om didaktik, ja det var ganske vist;\\
    men hvis man bruger PowerPoint til slides, bli’r man til grin.\\
    De fandt ham snart med tasken fuld af ryge-heroin.
  \end{SBVerse}

  \begin{SBChorus}
    Brug \LaTeX. Lad nu vær’ at piv’,\\
    og kan du ikk’, så få det lært, for det kan red’ dit liv.\\
    Og ellers ku’ det være, at du fik besøg af mig\\
    Så ender du måske i grøften på din livets vej.
  \end{SBChorus}
\end{song}
\begin{song}{\TeX'e-sangen}
  {} % Bruges ikke, lad stå blank
  {Melodi} % Titel, Kunstner - eks.: "Jutlandia, Kim Larsen". Hvis sangen er på sin egen melodi, brug da \SBOrgMel.
  {Forfatter} % Navnet på forfatteren. Undlad kaldenavne. Brug gerne TBF. Brug "&" frem for "og". Hvis forfatter er ukendt, lad da stå tom.
  {Anledning og år} % Eks. "Fysikrevy, 2010" eller "2010"
  {\NotCCLIed} % Lad stå som den er

  \begin{SBVerse}
    % Skriv vers her
  \end{SBVerse}

  \begin{SBChorus}
    % Skriv omkvæd her
  \end{SBChorus}

  \begin{SBSection*}
    % Skriv sektioner her. Hvis du ønsker lidt mellemrum for at give luft i et langt afsnit el.lign., brug da \\\medskip
  \end{SBSection*}
\end{song}

\onecolumn
\SBChapter{Fysik}
\twocolumn
\begin{song}{Våbenfysik}{}
  {Jutlandia, Kim Larsen}
  {}
  {FysikRevy 2003}
  {\NotCCLIed}

  \begin{SBVerse}
	Det var i 1945 og nu ville man ha’ fred,\\
	men der var krig i Japan.\\
	Der blev samlet uran nok og det blev kastet ned,\\
	for der var krig i Japan.\\
	De fik at se hvad fysikken formår,\\
	og sjove børn de næste mange år.
  \end{SBVerse}

  \begin{SBChorus}
	Hey-Ho for våbenfysik!\\
	Vi blæser på alle traktater.\\
	Bomber her, bomber der, bomber for fred!\\
	Hvad skal man med diplomater?
  \end{SBChorus}

  \begin{SBVerse}
	Vi flyver gennem natten og gi’r din fjende smæk.\\
	Han ser os ikke komme.\\
	Vi stiller ingen spørgsmål, og så snart vi har din check,\\
	så er krigen omme.\\
	Hvis du gi’r, fyrer vi den sgu af.\\
	Det er det, de unge vil ha’.
  \end{SBVerse}

  \begin{SBChorus}
	Hey-Ho for våbenfysik!\\
	Vi blæser på alle traktater.\\
	Bomber her, bomber der, bomber for fred!\\
	Hvad skal man med diplomater?
  \end{SBChorus}

  % \begin{SBChorus}
  % 	Hey-Ho for våbenfysik\ldots
  % \end{SBChorus}

  \begin{SBVerse}
	Vores salgskontor har åbent hele døgnet – bare ring;\\
	du skal ikke tøve.\\
	Vacuum, brint og EMP og andre sjove ting,\\
	dem vil vi gerne prøve.\\
	Her er ugens tilbudskatalog:\\
	Start tre krige og betal for to!
  \end{SBVerse}

  \begin{SBChorus}
	Hey-Ho for våbenfysik!\\
	Vi blæser på alle traktater.\\
	Bomber her, bomber der, bomber for fred!\\
	Hvad skal man med diplomater?
  \end{SBChorus}

  % \begin{SBChorus}
  % 	Hey-Ho for våbenfysik\ldots
  % \end{SBChorus}
  
  \begin{SBChorus}
	Hey-Ho for våbenfysik!\\
	Vi blæser på alle traktater.\\
	Bomber her, bomber der, bomber for fred!\\
	Vi nakker de slyngelstater!
  \end{SBChorus}

\end{song}
\begin{song}{Fasebal i Kvanteland}
  {} % Bruges ikke, lad stå blank
  {Julebal i Nisseland, Far til fire i byen} % Titel, Kunstner - eks.: "Jutlandia, Kim Larsen"
  {} % Navnet på forfatteren. Undlad aliasser. Brug "&" frem for "og".
  {FysikRevy 2004} % Eks. "Fysikrevy 2010" eller "2010"
  {\NotCCLIed} % Lad stå som den er

  \begin{SBVerse}
    Sikk' en dejlig energi,\\
    Jeg kan li' entropi\\
    Cæsium atomet står\\
    For at tiden går
  \end{SBVerse}

  \begin{SBChorus}
    Til fasebal,\\
    \hspace{1em}til fasebal i Kvanteland\\
    Elektroner\\
    \hspace{1em}flyver frit omkring\\
    Skru' feltet op\\
    \hspace{1em}og styr dem som en kvantemand\\
    Se nu bare,\\
    \hspace{1em}de flyver rundt i ring
  \end{SBChorus}

  \begin{SBSection*}
    Atomer absorberer\\
    \hspace{1em}det lys, vi sender ind\\
    Et spejl reflekterer:\\
    \hspace{1em}Pas på, du ik' bli'r blind!\\
    Til fasebal,\\
    \hspace{1em}til fase-fase-fase-fase-fasebal\\
    Bose-Einstein\\
    \hspace{1em}er lysets karneval
  \end{SBSection*}

  \begin{SBVerse}
    Her er koldt uhadada\\
    Vakuum ska' vi ha'\\
    Det kan ikke være svært\\
    Brug det, du har lært
  \end{SBVerse}

  \begin{SBChorus}
    Til fasebal,\\
    \hspace{1em}til fasebal i Kvanteland\\
    Stoffet flyder\\
    \hspace{1em}i superfludium\\
    Og hver boson,\\
    \hspace{1em}den gør jo som sin nabomand,\\
    De står sammen\\
    \hspace{1em}og rummet bliver til skum\\
  \end{SBChorus}

  \begin{SBSection*}
    Atomer absorberer\\
    \hspace{1em}det lys, vi sender ind\\
    Et spejl reflekterer:\\
    \hspace{1em}Pas på, du ik' bli'r blind!\\
    Til fasebal,\\
    \hspace{1em}til fase-fase-fase-fase-fasebal\\
    Bose-Einstein\\
    \hspace{1em}er lysets karneval
  \end{SBSection*}

  \begin{SBSection*}
    Bose-Einstein\\
    \hspace{1em}er lysets karneval\\
    Bose-Einstein\\
    \hspace{1em}er lysets karneval\\
    Bose-Einstein\\
    \hspace{1em}er lysets karneval
  \end{SBSection*}
\end{song}
\begin{song}{ASTRID}
  {} % Bruges ikke, lad stå blank
  {What Makes You Beautiful, One Direction} % Titel, Kunstner - eks.: "Jutlandia, Kim Larsen". Hvis sangen er på sin egen melodi, brug da \SBOrgMel.
  {Sabrina Tang} % Navnet på forfatteren. Undlad kaldenavne. Brug gerne TBF. Brug "&" frem for "og". Hvis forfatter er ukendt, lad da stå tom.
  {TÅGEKAMMERET, 2013} % Eks. "Fysikrevy, 2010" eller "2010"
  {\NotCCLIed} % Lad stå som den er

  \begin{SBVerse}
    Der er en pig’, som jeg kan li’.\\
    Jeg går i stå, hver gang jeg går forbi\\
    lokalet hvor jeg ved, hun står;\\
    min drømmepige på 27 år.\\\medskip
    Alle fysikfyre kender hende,\\
    Alle ved, hvem hun er.
  \end{SBVerse}

  \begin{SBChorus}
    Hun underviser i atomar kollision,\\
    ved alt om ion-beams, synkrotonradiation\\
    og spektroskopiforsøg med en elektron.\\
    Jeg bli’r hø-ø-øj, høj på energifysik.\\\medskip
    Er på en sær måde underskøn,\\
    og hun formår at ku’ vær’ både rund og køn.\\
    Og hele bygningen giver et kæmpe drøn\\
    når hun gå-å-år i gang med kvantemekanik.\\
    Åh-å-åh, ASTRID, hun er så unik.
  \end{SBChorus}

  \begin{SBVerse}
    Men det kan ske, at FFB\\
    de får serveret for meg’t LFP.\\
    Et øludbrud – hun står for skud.\\
    Hvis hun bli’r ramt flipper alle folk ud.\\\medskip
    Hun kører løs lig’som en maskine,\\
    Kan holde natten lang.
  \end{SBVerse}

  \begin{SBChorus}
    Hvis bare jeg ku’ Proton-Enhanced-Nuklear\\
    Induktion-Spektroskopere dig, var jeg klar!\\
    Har noget Large Hadron Collider ikke har.\\
    I kan nok forstå; ASTRID er min drømmetøs.\\\medskip
    Din kærlighed er så adækvat.\\
    Og selv om jeg sku’ gå hen og bliv’ kandidat,\\
    så vil du altid forblive min egen skat.\\
    Du er så-å-å pisse-sexet og famøs,\\
    Åh-å-åh, ASTRID gør mig så nervøs.
  \end{SBChorus}

  \begin{SBSection*}
    ASTRID, når vi to er sam’n er jeg så tilpas,\\
    og ASTRID2, hun vil aldrig ku’ ta’ din plads.\\
    Stod det til mig, gav vi hende til BIOGAS.\\
    Fordi åh-åh-åh ASTRID er min ring af guld!
  \end{SBSection*}

  \begin{SBChorus}
    Befinder mig i en excitations-tilstand,\\
    der’ ingen and’n synkrotonstrålingslagerring\\
    der øger min entropi lig’som ASTRID kan.\\
    Fordi åh-å-åh hun kan permutere mig.\\\medskip
    I ISAs kælder ka’ vi ta’ ned,\\
    og du må godt ta’ din søster ELISA med,\\
    hvis hun da ellers ka’ hold’ på en hem’lighed\\
    Fordi åh-å-åh, min verden graviterer\\
    nå-å-år, min verden oscillerer\\
    nå-å-år ASTRID kollidér’ med mig!
  \end{SBChorus}
\end{song}
\begin{song}{Jeg er fysiker}
  {} % Bruges ikke, lad stå blank
  {The Lumberjack Song, Monty Python} % Titel, Kunstner - eks.: "Jutlandia, Kim Larsen". Hvis sangen er på sin egen melodi, brug da \SBOrgMel.
  {} % Navnet på forfatteren. Undlad kaldenavne. Brug gerne TBF. Brug "&" frem for "og". Hvis forfatter er ukendt, lad da stå tom.
  {Matematikrevyen, 2007} % Eks. "Fysikrevy, 2010" eller "2010"
  {\NotCCLIed} % Lad stå som den er

  \begin{SBChorus}
    Jeg er fysiker og jeg er glad.\\
    Jeg kan min Schaums bog uden ad!\\\medskip
    \emph{Han er fysiker og han er glad.\\
    Han kan sin Schaums bog uden ad!}
  \end{SBChorus}

  \begin{SBVerse}
    Ampere her, promille der,\\
    jeg gør et Maple-plot.\\
    Jeg plotter i den farve,\\
    min kone si’r er flot!\\\medskip
    \emph{Ampere her, promille der,\\
    han gør et Maple-plot.\\
    Han plotter i den farve,\\
    hans kone si’r er flot!}
  \end{SBVerse}

  \begin{SBChorus}
    \emph{Han er fysiker og han er glad.\\
    Han kan sin Schaums bog uden ad!}
  \end{SBChorus}

  \begin{SBVerse}
    Eksperimentér, analyser,\\
    og spis en masse slik!\\
    Jeg læser en artikel,\\
    og leger med min rubikskube\\\medskip
    \emph{Eksperimentér, analyser,\\
    og spis en masse slik!\\
    Han læser en artikel,\\
    og leger med sin rubikskube}
  \end{SBVerse}

  \begin{SBChorus}
    Jeg er fysiker, og jeg er fin.\\
    Jeg kan formlen for min urin!\\
    \emph{Han er fysiker og han er fin.\\
    Han kan formlen for sin urin!}
  \end{SBChorus}

  \begin{SBVerse}
    Jeg rækkeudvikler på alt\\
    fordi det har jeg lært.\\
    Hvis blot jeg læste mat'matik\\
    som ham der Erik Kjær!\\\medskip
    \emph{Han rækkeudvikler på alt\\
    fordi det har han lært.\\
    Hvis blot han læste mat'matik\\
    som ham der Erik Kjær!}
  \end{SBVerse}
\end{song}
\begin{song}{$\chi$-fitter}
  {} % Bruges ikke, lad stå blank
  {Kickflipper, Razz} % Titel, Kunstner - eks.: "Jutlandia, Kim Larsen". Hvis sangen er på sin egen melodi, brug da \SBOrgMel.
  {} % Navnet på forfatteren. Undlad kaldenavne. Brug gerne TBF. Brug "&" frem for "og". Hvis forfatter er ukendt, lad da stå tom.
  {FysikRevy, 2011} % Eks. "Fysikrevy, 2010" eller "2010"
  {\NotCCLIed} % Lad stå som den er

  \begin{SBChorus}
    $\chi$fitter vildt \emph{(Yeah!)}\\
    $\chi$fitter snildt \emph{(Yeah!)}\\
    $\chi$fitter højt \emph{(Øv!)}\\
    Ikke for at blære, men vi $\chi$fitter yeah!
  \end{SBChorus}

  \begin{SBVerse}
    Mig og alle gutterne sætter attributterne\\
    På fittet som I ser, de bedste startværdier\\
    Du bliver altid mobbet, hvis dataen er flobbet\\
    Vil fittet konvergere, når løkken itererer\\\medskip
    Usikkerheden stiger\\
    på mine fitværdier\\
    Mit $\chi$-kvadrat er stort,\\
    det' noget lort \emph{(Det' noget lort!)}
  \end{SBVerse}

  \begin{SBChorus}
    $\chi$fitter vildt \emph{(Yeah!)}\\
    $\chi$fitter snildt \emph{(Yeah!)}\\
    $\chi$fitter højt \emph{(Øv!)}\\
    Ikke for at blære, men vi...\\\medskip
    $\chi$fitter vildt \emph{(Yeah!)}\\
    $\chi$fitter snildt \emph{(Yeah!)}\\
    $\chi$fitter højt \emph{(Øv!)}\\
    Ikke for at blære, men vi $\chi$fitter yeah!
  \end{SBChorus}

  \begin{SBVerse}
    P-værdi fortæller om vores fitmodeller\\
    Beskriver data godt, om plottet det bliver flot\\
    $\chi$-kvadratet minimer', rutinen sørger for det sker\\
    Syns' du jeg er dum, tjek mit parameterrum\\\medskip
    Usikkerheden stiger\\
    på mine fitværdier\\
    Mit $\chi$-kvadrat er stort,\\
    det' noget lort \emph{(Det' noget lort!)}\\\medskip
    Usikkerheden stiger\\
    på mine fitværdier\\
    Mit $\chi$-kvadrat er stort,\\
    ligesom din mor \emph{(Større end din mor!)}
  \end{SBVerse}

  \begin{SBChorus}
    $\chi$fitter vildt\ldots
  \end{SBChorus}

  \begin{SBSection*}
    Åh, jeg er bare så træt af at fitte.
  \end{SBSection*}

  \begin{SBChorus}
    $\chi$fitter vildt \emph{(Yeah!)}\\
    $\chi$fitter snildt \emph{(Yeah!)}\\
    $\chi$fitter højt \emph{(Øv!)}\\
    Ikke for at blære, men vi $\chi$fitter yeah!
  \end{SBChorus}

  \begin{SBChorus}
    $\chi$fitter vildt \emph{(Yeah!)}\\
    $\chi$fitter snildt \emph{(Yeah!)}\\
    $\chi$fitter højt \emph{(Øv!)}\\
    Ikke for at blære, men vi $\chi$fitter yeah!
  \end{SBChorus}

  % \begin{SBChorus}
  %   $\chi$fitter vildt\ldots
  % \end{SBChorus}

  % \begin{SBChorus}
  %   $\chi$fitter vildt\ldots
  % \end{SBChorus}
\end{song}
\begin{song}{Steve Hawking}
  {} % Bruges ikke, lad stå blank
  {Melodi} % Titel, Kunstner - eks.: "Jutlandia, Kim Larsen". Hvis sangen er på sin egen melodi, brug da \SBOrgMel.
  {Forfatter} % Navnet på forfatteren. Undlad kaldenavne. Brug gerne TBF. Brug "&" frem for "og". Hvis forfatter er ukendt, lad da stå tom.
  {Anledning og år} % Eks. "Fysikrevy, 2010" eller "2010"
  {\NotCCLIed} % Lad stå som den er

  \begin{SBVerse}
    % Skriv vers her
  \end{SBVerse}

  \begin{SBChorus}
    % Skriv omkvæd her
  \end{SBChorus}

  \begin{SBSection*}
    % Skriv sektioner her. Hvis du ønsker lidt mellemrum for at give luft i et langt afsnit el.lign., brug da \\\medskip
  \end{SBSection*}
\end{song}
\begin{song}{Kun fysik}
  {} % Bruges ikke, lad stå blank
  {Rigtige mænd, Disney} % Titel, Kunstner - eks.: "Jutlandia, Kim Larsen". Hvis sangen er på sin egen melodi, brug da \SBOrgMel.
  {} % Navnet på forfatteren. Undlad kaldenavne. Brug gerne TBF. Brug "&" frem for "og". Hvis forfatter er ukendt, lad da stå tom.
  {FysikRevy, 2009} % Eks. "Fysikrevy, 2010" eller "2010"
  {\NotCCLIed} % Lad stå som den er

  \begin{SBVerse}
    Danmarks statsminister\\
    vil ha' lærde mænd.\\
    Ikke humanister,\\
    der er nok af dem!\\
    Modesnak det er slet intet værd!\\
    Og fodboldfacts kan inden brug'!\\
    Det' fysik! Kun fysik!\\
    I skal ku'!
  \end{SBVerse}

  \begin{SBVerse}
    Hurtig're end Pauli\\
    Skal I regne kvant.\\
    Altid skarp og saglig,\\
    Søg hvad der er sandt.\\
    Brug af Schaums og TI-89\\
    Det er for svagt, så hør mig nu!\\
    Det' fysik!\\
    Kun fysik!\\
    I skal ku'!
  \end{SBVerse}

  \begin{SBSection*}
    Ebbe Sand laver aldrig fejl!\\
    Skal jeg virk'lig pertubere?\\
    Det med mat'matikken faldt mig aldrig nemt.\\
    Der flækkede jeg igen en negl!\\
    Mon Beckham han kan transformere?\\
    Ku' jeg dividere, var det ik' så slemt.
  \end{SBSection*}

  \begin{SBChorus}
    \emph{Kun fysik!} I skal ku' regne på kvantefelter!\\
    \emph{Kun fysik!} Og integrere enhver funktion!\\
    \emph{Kun fysik!} Og I skal læse til hjernen smelter!\\
    Før I kan få en forskningsmillion!
  \end{SBChorus}

  \begin{SBVerse}
    Eksamen om en uge.\\
    Det vil aldrig gå.\\
    Natten må I bruge,\\
    Hvis I vil bestå.\\
    I' en dum, ubrug'lig, ynk'lig flok.\\
    Så drop ud, I dumper nu!\\
    Det' fysik!\\
    Kun fysik!\\
    I skal ku'!
  \end{SBVerse}

  \begin{SBChorus}
    \emph{Kun fysik!}\ldots
  \end{SBChorus}

  \begin{SBChorus}
    \emph{Kun fysik!}\ldots
  \end{SBChorus}
\end{song}

\onecolumn
\SBChapter{Matematik}
\twocolumn
\begin{song}{Den kanoniske matematikersang}
  {} % Bruges ikke, lad stå blank
  {Bamses fødselsdag}
  {Esben Bistrup Halvorsen og Rasmus Resen Amossen}
  {}
  {\NotCCLIed}

  \begin{SBVerse}
	Som mat'matikstuderende,\\
	så er jeg svær at narre.\\
	Når andre kalder mig for nørd,\\
	så må jeg bare svare:
  \end{SBVerse}

  \begin{SBChorus}
	Hip hurra for algebra,\\
	Euler, Gauss og Galois\\
	og for rum med en kompakt\\
	deformationsretrakt.
  \end{SBChorus}

  \begin{SBVerse}
	En dag jeg sa' til kæresten:\\
	"nu vil jeg dyrke grupper".\\
	Hun kiggede forbavset op,\\
	"det ikke super duper!".\\\medskip
	"Jamen det er ej med dig,\\
	gutterne de er på vej".\\
	"Får du ikke nok af mig?!"\\
	 -så rejste hun sin vej.\\
  \end{SBVerse}

  \begin{SBVerse}
	Nu var jeg blevet singelton,\\
	og sagde til min moder:\\
	"Jeg er disjunkt med kæresten\\
	som Peking med Nyboder.\\\medskip
	Inklusionen er nu vendt,\\
	jeg er blevet transcendent".\\
	"Er du trans, din klamme tøs?!"\\
	-så blev jeg arveløs.
  \end{SBVerse}

  \begin{SBVerse}
	Da arven nu var faldet bort,\\
	jeg måtte til at spare.\\
	Jeg talte med min vicevært\\
	og kunne ham forklare:\\\medskip
	"Pengemængden er kompakt,\\
	jeg vil ha' en ny kontrakt!".\\
	"Fint med mig, den kommer her:\\
	Du bor her ikke mer'!".
  \end{SBVerse}

  \begin{SBVerse}
	Nu var jeg efterhånden ved\\
	at få lidt hovedpine.\\
	Jeg tog til hospitalet og\\
	fik hjælp af en blondine.\\\medskip
	"Hovedsmerten går for vidt,\\
	den sku' deles op i snit!".\\
	"Hvidt og snit, og så nå'et, ik'?"\\
	-der røg mit overblik.
  \end{SBVerse}

  \begin{SBVerse}
	Jeg har nu få't det hvide snit\\
	og boligen er røget,\\
	men selvom både arv og kær'-\\
	ste fløj, er jeg fornøjet!
  \end{SBVerse}

  \begin{SBChorus}
	Hip hurra for algebra,\\
	Euler, Gauss og Galois\\
	og for rum med en kompakt\\
	deformationsretrakt.
  \end{SBChorus}
\end{song}





\begin{song}{Mat'matik}
  {} % Bruges ikke, lad stå blank
  {Bubbi-bjørnene, Disney} % Titel, Kunstner - eks.: "Jutlandia, Kim Larsen". Hvis sangen er på sin egen melodi, brug da \SBOrgMel.
  {} % Navnet på forfatteren. Undlad aliasser. Brug "&" frem for "og". Hvis forfatter er ukendt, lad da stå tom.
  {FysikRevy, 2008} % Eks. "Fysikrevy 2010" eller "2010"
  {\NotCCLIed} % Lad stå som den er

  \begin{SBVerse}
    Vektorfunktioner i tre dimensioner\\
    skal opereres med curl og divergens.\\
    $\nabla$\textsuperscript{2} gir koldsved på panden;\\
    planintegraler er en pestilens.
  \end{SBVerse}

  \begin{SBChorus}
    Mat'matik\\
    strider ofte mod enhver logik;\\
    men har du elektrodynamik,\\
    så skal du ku' mat'matik.
  \end{SBChorus}

  \begin{SBVerse}
    Kan en hermitisk og injektiv matrix\\
    ha' negativ trace men en nul-determinant?\\
    Solovej siger dens egenværdier\\
    vil være reelle, men er det mon sandt?
  \end{SBVerse}

  \begin{SBChorus}
    Mat'matik\\
    strider ofte mod enhver logik;\\
    men hvis du har kvantemekanik,\\
    så skal du ku' mat'matik.
  \end{SBChorus}

  \begin{SBVerse}
    $T_a$ vil gi' $SU(3)$-symmetri,\\
    og Lagrange-operator'n er Gauge-invariant.\\
    Når operator'ne virker på kvarkerne,\\
    får de en farve og smagen af kvant
  \end{SBVerse}

  \begin{SBChorus}
    Mat'matik\\
    strider ofte mod enhver logik;\\
    men i kvantekromodynamik,\\
    så skal du ku' mat'matik.
  \end{SBChorus}

  \begin{SBSection*}
    Så skal du ku' mat'matik!
  \end{SBSection*}
\end{song}
\begin{song}{Matrixrepræsentationsteoremet}
  {} % Bruges ikke, lad stå blank
  {I will survive, Gloria Gaynor} % Titel, Kunstner - eks.: "Jutlandia, Kim Larsen". Hvis sangen er på sin egen melodi, brug da \SBOrgMel.
  {Christian Bladt Brandt} % Navnet på forfatteren. Undlad aliasser. Brug "&" frem for "og". Hvis forfatter er ukendt, lad da stå tom.
  {\TKET{}, 2011} % Eks. "Fysikrevy 2010" eller "2010"
  {\NotCCLIed} % Lad stå som den er

  \begin{SBVerse}
    Hvis man skal transformere, altså lineært,\\
    mellem to vektorrum, så er det altså elementært.\\
    Det kan jo snildt repræsenteres ved en matrix, som I ser,\\
    og hvis I ikke helt kan se det, så beviser jeg det her:
  \end{SBVerse}

  \begin{SBVerse}
    Vi ser på $V$, et vektorrum,\\
    der har ordnet basis kaldet $E$, og $n$ som dimension.\\
    Med dimensionen $m$ og basen $F$ der har vi $W$,\\
    det vektorrum den lineær’ transformation går over i.
  \end{SBVerse}

  \begin{SBVerse}
    Så har vi $x$, hvad er nu det?\\
    En koordinatvektor i basen $E$ for vektor’n $v$ i $V$,\\
    og for vektorerne i $W$ der kan vi definer’\\
    en koordinatvektor i basen $F$, der kaldes $y$, oh yeah!
  \end{SBVerse}

  \begin{SBChorus}
    Så er vi klar\\
    til at bevise\\
    denne sætning, der er vigtig,\\
    men først skal vi lige ha’ formuleret helt præcist\\
    hvad vi gern’ vil ha’ bevist.\\
    Det kommer her,\\
    det kommer her:
  \end{SBChorus}

  \begin{SBVerse}
    Nu vil vi vise $Ax=y$ hvis og kun hvis\\
    $L$ på $v$ den ser så’n her ud i vores $F$-basis.\\
    Her la’r vi $A$ være en matrix der har søjler givet ve’\\
    transformationen anvendt på basisvektorerne fra $E$.
  \end{SBVerse}

  \begin{SBVerse}
    Så vi ta’r $L$, anvendt på $v$,\\
    og starter med at bruge linearitet på det.\\
    Og hvis vi husker på hvordan vor matrix $A$ var definer’t,\\
    så er det næste udtryk, som I ser, vist ikke helt forkert!
  \end{SBVerse}

  \begin{SBVerse}
    Det sidste skridt er trivielt!\\
    Vi bytter rundt på de to summer, ikke nog’t specielt.\\
    Nu ses det at det udtryk der’ i parentes, det er $y_i$,\\
    så $y$ lig $A$ på $x$, og jeg ka’ si':\\
    Q.E.D.
  \end{SBVerse}
\end{song}
\begin{song}{Er du dus med Grassmans Lemma}
  {} % Bruges ikke, lad stå blank
  {Melodi} % Titel, Kunstner - eks.: "Jutlandia, Kim Larsen". Hvis sangen er på sin egen melodi, brug da \SBOrgMel.
  {Forfatter} % Navnet på forfatteren. Undlad kaldenavne. Brug gerne TBF. Brug "&" frem for "og". Hvis forfatter er ukendt, lad da stå tom.
  {Anledning og år} % Eks. "Fysikrevy, 2010" eller "2010"
  {\NotCCLIed} % Lad stå som den er

  \begin{SBVerse}
    Er du dus med Grassmans lemma\\
    og Holgers gule bog\\
    og Borel Heines sætning\\
    så er du rigtig klog\\
    Kan du smile til et $\delta$\\
    og vinke til $dx$\\
    så har du fundet ud af det\\
    som er mere værd end – MAT A\\\medskip
    En vektor, en matrix,\\
    en n’te grads funktion,\\
    en basis, en række,\\
    en dårlig mi«knas»kro«knas»fon
  \end{SBVerse}

  \begin{SBVerse}
    Har du set når Holger summer\\
    på A og B og C\\
    så ved du hvad der kommer\\
    på tavle nr. tre.\\
    Har du forberedt dig hjemme\\
    det hjælper ikke spor\\
    for lige meget hvad han si’r\\
    så forstår du ikk’ et ord\\\medskip
    Et lemma, et $\gamma$,\\
    en summe over $X$,\\
    et primtal, en faktor,\\
    en ugeseddel – syv
  \end{SBVerse}

  \begin{SBVerse}
    Syn’s du at Leif han mumler\\
    at frikvarter er rart?\\
    Snakker du med humanister\\
    for at føle dig lidt smart?\\
    Er du træt af Leifes sweater\\
    af konkav og konveks?\\
    Så bare vent til algebra,\\
    så får du gruppe – teori\\\medskip
    Supremum og limes,\\
    et simpelt korollar,\\
    en normal fordeling,\\
    en stinkende cigar.
  \end{SBVerse}

  \begin{SBVerse}
    Er du dus med ham Igna– Ignat– Michael?\\
    Kan du programmere C\\
    Får du ondt når EMS han skriger\\
    om $O(\log n+p)$?\\
    Kan du smile til en binær\\
    og vinke til en hex?\\
    Jeg tror, at EMS blev gift\\
    på grund af børn’ne, ikke – sex
  \end{SBVerse}
\end{song}





\begin{song}{Her på mat'matik}
  {} % Bruges ikke, lad stå blank
  {Fætter Mikkel} % Titel, Kunstner - eks.: "Jutlandia, Kim Larsen". Hvis sangen er på sin egen melodi, brug da \SBOrgMel.
  {} % Navnet på forfatteren. Undlad kaldenavne. Brug gerne TBF. Brug "&" frem for "og". Hvis forfatter er ukendt, lad da stå tom.
  {Matematikrevyen, 2015} % Eks. "Fysikrevy, 2010" eller "2010"
  {\NotCCLIed} % Lad stå som den er

  \begin{SBVerse}
    Her på mat'matik skal man være kvik\\
    Sætte pris på elegance\\
    Ha' intuition, også ambition\\
    Gribe fat i hver en chance
  \end{SBVerse}

  \begin{SBChorus}
    For vi elsker ren logik \emph{(klap-klap)}\\
    algebra og statistik \emph{(klap-klap)}\\
    Alt der er komplekst: analyse, vækst,\\
    det er her vi har det bedst \emph{(klap-klap)}
  \end{SBChorus}

  \begin{SBVerse}
    Fra det første år, kurserne består\\
    Fællesskabet ej forglemme\\
    Der bli'r undervist, masser kage spist\\
    Det er her hvor vi har hjemme
  \end{SBVerse}

  \begin{SBChorus}
    For vi elsker ren logik\ldots
  \end{SBChorus}

  \begin{SBSection*}
    % Skriv sektioner her. Hvis du ønsker lidt mellemrum for at give luft i et langt afsnit el.lign., brug da \\\medskip
  \end{SBSection*}
\end{song}
\begin{song}{Jeg er en matematiker fra HCØ}
  {} % Bruges ikke, lad stå blank
  {Melodi} % Titel, Kunstner - eks.: "Jutlandia, Kim Larsen". Hvis sangen er på sin egen melodi, brug da \SBOrgMel.
  {Forfatter} % Navnet på forfatteren. Undlad kaldenavne. Brug gerne TBF. Brug "&" frem for "og". Hvis forfatter er ukendt, lad da stå tom.
  {Anledning og år} % Eks. "Fysikrevy, 2010" eller "2010"
  {\NotCCLIed} % Lad stå som den er

  \begin{SBVerse}
    % Skriv vers her
  \end{SBVerse}

  \begin{SBChorus}
    % Skriv omkvæd her
  \end{SBChorus}

  \begin{SBSection*}
    % Skriv sektioner her. Hvis du ønsker lidt mellemrum for at give luft i et langt afsnit el.lign., brug da \\\medskip
  \end{SBSection*}
\end{song}
\begin{song}{På mat'matik}
  {} % Bruges ikke, lad stå blank
  {Melodi} % Titel, Kunstner - eks.: "Jutlandia, Kim Larsen". Hvis sangen er på sin egen melodi, brug da \SBOrgMel.
  {Forfatter} % Navnet på forfatteren. Undlad kaldenavne. Brug gerne TBF. Brug "&" frem for "og". Hvis forfatter er ukendt, lad da stå tom.
  {Anledning og år} % Eks. "Fysikrevy, 2010" eller "2010"
  {\NotCCLIed} % Lad stå som den er

  \begin{SBVerse}
    % Skriv vers her
  \end{SBVerse}

  \begin{SBChorus}
    % Skriv omkvæd her
  \end{SBChorus}

  \begin{SBSection*}
    % Skriv sektioner her. Hvis du ønsker lidt mellemrum for at give luft i et langt afsnit el.lign., brug da \\\medskip
  \end{SBSection*}
\end{song}
\begin{song}{Matematikkens Historie 2}
  {} % Bruges ikke, lad stå blank
  {Major General's Song, Gilbert and Sullivan} % Titel, Kunstner - eks.: "Jutlandia, Kim Larsen". Hvis sangen er på sin egen melodi, brug da \SBOrgMel.
  {} % Navnet på forfatteren. Undlad kaldenavne. Brug gerne TBF. Brug "&" frem for "og". Hvis forfatter er ukendt, lad da stå tom.
  {Matematikrevy, 2012} % Eks. "Fysikrevy, 2010" eller "2010"
  {\NotCCLIed} % Lad stå som den er

  \begin{SBVerse}
    Jeg er godt eksempel på en klassisk matematiker\\
    Jeg har jo haft mit studie som fuldstændig analytiker\\
    I matematikhistorien der kan jeg finde skjulested\\
    Men jeg er ikke stor nok til at kunne sammenlignes med:\\
    Ramanujan, og Trachtenberg, Von Neumann og Kolmogorov\\
    Og Grothendieck, og De Moivre, Hippocrates, Lyapunov\\
    Og Weierstrass, Pythagoras, Lobaechevsky og så Cauchy\\
    Samt Cavalieri, Minkowski, Fibonacci og Jacobi 
  \end{SBVerse}

  \begin{SBVerse}
    Husk Wessel, Russell og Borel, og Aristoteles, Pascal\\
    Zermelo, Fraenkel, Gödel, Boole, og Goldbach, Hardy, og Abel\\
    Og Lindelöf, og Dirichlet, og Dedekind, og Sørensen\\
    Og Noether, Cantor, Mandelbrot, og Littlewood og Pedersen 
  \end{SBVerse}

  \begin{SBVerse}
    Og Hipparchos, og Möbius, Eudoxus, Apollonius\\
    Archimedes, og Bernoulli, Diophantos, Frobenius\\
    Og Poincaré, Galileo, Al-Khwârizmi og Fubini\\
    Brahmagupta, og Chebyshev, Caratheodory, Seki,\\
    Og Lie, Fourier og Bháscara, Iwasawa, Aryabhatta\\
    Og Liouville, og Fischer, og så Leibniz, Chern, og Atiyah\\
    Og Lebesgue, og Erdõs, Euclid, og Banach, Tarski og Sylow\\
    Og Hadamard, og Sylvester, og Weyl, Landau, og så Markov
  \end{SBVerse}

  \begin{SBVerse}
    Husk Wessel, Russell og Borel, og Aristoteles, Pascal\\
    Zermelo, Fraenkel, Gödel, Boole, og Goldbach, Hardy, og Abel\\
    Og Lindelöf, og Dirichlet, og Dedekind, og Sørensen\\
    Og Noether, Cantor, Mandelbrot, og Littlewood og Pedersen 
  \end{SBVerse}

  \begin{SBVerse}
    Wiles, Lagrange og Archytas, De Morgan, Babbage og Laplace\\
    Napier. Hilbert, Fermat, Lambert, Eilers, og l'Hôpital, Thales\\
    Og Kepler, Taylor, Escher, Lehrer, Brouwer, Nash og så Descartes\\
    Og Darboux, Jensen, Hansen, Grassmann, Jordan, Klein og Galois
  \end{SBVerse}

  \begin{SBVerse}
    Vi mangler tre af matematikkens helt centrale søjler\\
    Det' nogen af de største det er Riemann, Gauss og Euler
  \end{SBVerse}
\end{song}
\begin{song}{Primtal}
  {} % Bruges ikke, lad stå blank
  {Candy, Robbie Williams} % Titel, Kunstner - eks.: "Jutlandia, Kim Larsen". Hvis sangen er på sin egen melodi, brug da \SBOrgMel.
  {} % Navnet på forfatteren. Undlad kaldenavne. Brug gerne TBF. Brug "&" frem for "og". Hvis forfatter er ukendt, lad da stå tom.
  {Matematikrevyen, 2014} % Eks. "Fysikrevy, 2010" eller "2010"
  {\NotCCLIed} % Lad stå som den er

  \begin{SBVerse}
    Lad os definere de tal der fascinerer.\\
    For primtal skal der gælde faktorer trivielle.\\
    Elegant og simpelt, det kan man let forstå.\\
    Men trods det er det fulde billed’ umuligt at opnå.\\\medskip
    \emph{Men hør nu:}\\
    Euler og Euklid de viste,\\
    blandt primtal er der slet intet sidste.\\
    Givet $n$ kan primtal skrives\\
    helt entydigt, vi har uniqueness.
  \end{SBVerse}

  \begin{SBChorus}
    Skriv nu $p$ og $q$.\\
    Der’ så meg’t der ikk’ er vist endnu.\\
    Uden dem går teori itu,\\
    for alt er smukt ved primtal.\\
    Skriv nu $p$ og $q$.\\
    De er overalt, det vides jo.\\
    Også selvom det er svært at tro,\\
    for alt er smukt ved primtal.
  \end{SBChorus}

  \begin{SBVerse}
    Deres heltalsringe dem kan man let frembringe.\\
    For vilkårligt $p$ er der meget at indse.\\
    Man kan se tendenser og smukke kongruenser.\\
    Perfekt at anerkende, forbundet med Mersenne.\\\medskip
    Goldbach havde en formodning.\\
    Og kryptologer brug’r dem i kodning.\\
    Dankortkøb forløber sikkert fordi\\
    vi slipper primtal fri. Hvad gjord’ vi uden dem?\\
    Så kom nu!
  \end{SBVerse}

  \begin{SBChorus}
    Skriv nu $p$ og $q$.\ldots
  \end{SBChorus}

  \begin{SBVerse}
    $\pi(x)$ er asymptotisk.\\
    Distribueringen dog ej logisk.\\
    Sylows sætning hjælper os til at se\\
    grupper af orden $p$. Hvad gjord’ vi uden dem?\\
    \SBRepeat{\SBRepeat{\SBRepeat{Hvad gjord’ vi uden dem?}}}
  \end{SBVerse}

  \begin{SBChorus}
    Skriv nu $p$ og $q$.\ldots
  \end{SBChorus}

  \begin{SBChorus}
    Skriv nu $p$ og $q$.\ldots
  \end{SBChorus}
\end{song}
\begin{song}{Integralsangen}
  {} % Bruges ikke, lad stå blank
  {Melodi} % Titel, Kunstner - eks.: "Jutlandia, Kim Larsen". Hvis sangen er på sin egen melodi, brug da \SBOrgMel.
  {Forfatter} % Navnet på forfatteren. Undlad kaldenavne. Brug gerne TBF. Brug "&" frem for "og". Hvis forfatter er ukendt, lad da stå tom.
  {Anledning og år} % Eks. "Fysikrevy, 2010" eller "2010"
  {\NotCCLIed} % Lad stå som den er

  \begin{SBVerse}
    % Skriv vers her
  \end{SBVerse}

  \begin{SBChorus}
    % Skriv omkvæd her
  \end{SBChorus}

  \begin{SBSection*}
    % Skriv sektioner her. Hvis du ønsker lidt mellemrum for at give luft i et langt afsnit el.lign., brug da \\\medskip
  \end{SBSection*}
\end{song}

\onecolumn
\SBChapter{Datalogi}
\twocolumn
\begin{song}{Se min kode}
  {} % Bruges ikke, lad stå blank
  {Se min kjole, Gunnar Nyborg-Jensen} % Titel, Kunstner - eks.: "Jutlandia, Kim Larsen". Hvis sangen er på sin egen melodi, brug da \SBOrgMel.
  {Jacob Johannsen og Erik Søe Sørensen} % Navnet på forfatteren. Undlad kaldenavne. Brug gerne TBF. Brug "&" frem for "og". Hvis forfatter er ukendt, lad da stå tom.
  {\TKET{}s Julerevy, 2003} % Eks. "Fysikrevy, 2010" eller "2010"
  {\NotCCLIed} % Lad stå som den er

  \begin{SBVerse}
    Se min kode - den er struktureret,\\
    alt hvad jeg skriver, det er smukt som den.\\
    Det er fordi jeg altid indenterer,\\
    og fordi at Emacs er min ven
  \end{SBVerse}

  \begin{SBVerse}
    Se min kode – den er let at læse,\\
    alt hvad jeg skriver, det er ligesom den.\\
    Det er fordi jeg altid kommenterer,\\
    og fordi /* er min ven
  \end{SBVerse}

  \begin{SBVerse}
    Se min kode – den vil kompilere,\\
    alt hvad jeg skriver oversættes nemt.\\
    Det er fordi jeg skriver simpel kode,\\
    og fordi gcc den er min ven
  \end{SBVerse}

  \begin{SBVerse}
    Se min kode – er i mange filer,\\
    alt hvad jeg skriver, det kan findes nemt.\\
    Det er fordi jeg altid fragmenterer,\\
    og fordi at make den er min ven
  \end{SBVerse}

  \begin{SBVerse}
    Se min kode – den er uden fejl i,\\
    alt hvad jeg skriver, det er lissom den.\\
    Det er fordi, jeg altid er forsigtig,\\
    og fordi GDB den er min ven.
  \end{SBVerse}

  \begin{SBVerse}
    Se min kode – den kan let genskabes,\\
    alt hvad jeg ændrer, rettes let igen.\\
    Det er fordi, jeg altid tager backup,\\
    og fordi CVS den er min ven.
  \end{SBVerse}

  \begin{SBVerse}
    Se min kode - den er fri som fuglen,\\
    alt hvad jeg ejer, det er frit som den.\\
    Det er fordi, jeg elsker Open Software,\\
    og fordi GPL den er min ven.
  \end{SBVerse}

  \begin{SBVerse}
    Se min kode – den er årtotusindsikret,\\
    alt hvad jeg skriver, det er årtotusindsikret som den.\\
    Det er fordi jeg passer på, der aldrig kommer overflow,\\
    og fordi alt andet ville være fjollet, simpelthen.
  \end{SBVerse}

  \begin{SBVerse}
    Se min kode – den er tem’lig fjollet,\\
    alt hvad jeg skriver, det er lissom den.\\
    Det er fordi jeg ofte går i selvsving,\\
    og fordi revyen er min ven.
  \end{SBVerse}

  \begin{SBVerse}
    Semikolon, sangen den er slut nu;
  \end{SBVerse}
\end{song}
\begin{song}{Jeg har fundet mig en bug}
  {} % Bruges ikke, lad stå blank
  {Jeg har fundet mig en myg, Astrid Kjærgaard} % Titel, Kunstner - eks.: "Jutlandia, Kim Larsen". Hvis sangen er på sin egen melodi, brug da \SBOrgMel.
  {} % Navnet på forfatteren. Undlad kaldenavne. Brug gerne TBF. Brug "&" frem for "og". Hvis forfatter er ukendt, lad da stå tom.
  {} % Eks. "Fysikrevy, 2010" eller "2010"
  {\NotCCLIed} % Lad stå som den er

  \begin{SBVerse}
    Jeg har fundet mig en bug\\
    den er stor og væm’lig.\\
    GDB er ikke nok\\
    dér er buggen nemlig\\\medskip
    Jeg har brugt den hele nat\\
    på at stoppe huller\\
    Og er efterhånden træt\\
    af etter og nuller
  \end{SBVerse}

  \begin{SBVerse}
    Jeg har voldsom kaffetrang\\
    men der er ej mere\\
    Koden den vil ik’ engang\\
    næsten kompilere\\\medskip
    Der er noget voldsomt galt\\
    på så mang’ niveauer\\
    Helt basalt er det fatalt\\
    at jeg sidder og sover
  \end{SBVerse}

  \begin{SBVerse}
    De fandt ham den næste dag\\
    med tastetryk i panden.\\
    Plud’slig vågned’ han og sa’:\\
    "Fejlen var en anden!"\\\medskip
    Husk det, gæve datalog\\
    Ofte skal du bare\\
    ha’ det lidt på afstand,\\
    så ser du meget klarer’!
  \end{SBVerse}
\end{song}
\begin{song}{Linieskriverdriversangen}
  {} % Bruges ikke, lad stå blank
  {Melodi} % Titel, Kunstner - eks.: "Jutlandia, Kim Larsen". Hvis sangen er på sin egen melodi, brug da \SBOrgMel.
  {Forfatter} % Navnet på forfatteren. Undlad kaldenavne. Brug gerne TBF. Brug "&" frem for "og". Hvis forfatter er ukendt, lad da stå tom.
  {Anledning og år} % Eks. "Fysikrevy, 2010" eller "2010"
  {\NotCCLIed} % Lad stå som den er

  \begin{SBVerse}
    % Skriv vers her
  \end{SBVerse}

  \begin{SBChorus}
    % Skriv omkvæd her
  \end{SBChorus}

  \begin{SBSection*}
    % Skriv sektioner her. Hvis du ønsker lidt mellemrum for at give luft i et langt afsnit el.lign., brug da \\\medskip
  \end{SBSection*}
\end{song}
\begin{song}{Server'n er crashed}
  {} % Bruges ikke, lad stå blank
  {I Want It That Way, Backstreet Boys} % Titel, Kunstner - eks.: "Jutlandia, Kim Larsen". Hvis sangen er på sin egen melodi, brug da \SBOrgMel.
  {} % Navnet på forfatteren. Undlad kaldenavne. Brug gerne TBF. Brug "&" frem for "og". Hvis forfatter er ukendt, lad da stå tom.
  {DIKUrevy, 2011} % Eks. "Fysikrevy, 2010" eller "2010"
  {\NotCCLIed} % Lad stå som den er

  \begin{SBVerse}
    Jeg bli'r helt ked, men\\
    Nu er den nede\\
    Den var belastet\\
    Nu er den crashed.
  \end{SBVerse}

  \begin{SBVerse}
    Min prompt den sejler.\\
    Systemet fejler\\
    Der er sgu knas med\\
    den server, der crashed.
  \end{SBVerse}

  \begin{SBChorus}
    For åh nej, vi sku' ha købt et nyt RAID\\
    For åh nej, jeg var sgu' ikke beredt\\
    før vi så undtagelsen den kasted.\\
    Server'n er crashed
  \end{SBChorus}

  \begin{SBVerse}
    Men kan den tvinges\\
    til at ku' pinges\\
    Nej, den er - helt trashed.\\
    Ja, server'n er crashed.
  \end{SBVerse}

  \begin{SBChorus}
    For åh nej\ldots
  \end{SBChorus}

  \begin{SBSection*}
    Tænkte, jeg fikser det bare i mor'n\\
    da den stod der og kasted' fejl\\
    Jeg burde ha' tjekket, men surfede porn\\
    Nu er den sgu' gåe't sin vej.
  \end{SBSection*}

  \begin{SBVerse}
    Det' ikk' så skid' rart\\
    Jeg ta'r en genstart\\
    Men nej, åh nej, åh nej, åh nej!\\
    \ldots
  \end{SBVerse}

  \begin{SBSection*}
    Det fucking neder'n!
  \end{SBSection*}

  \begin{SBSection*}
    Den fejler nu ved startup\\
    Vi har sgu ingen backup\\
    Disken er sikkert kvæstet\\
    Server'n er crashed
  \end{SBSection*}

  \begin{SBChorus}
    For åh nej\ldots
  \end{SBChorus}

  \begin{SBChorus}
    For åh nej\ldots
  \end{SBChorus}

  \begin{SBSection*}
    Server'n er crashed
  \end{SBSection*}
\end{song}
\begin{song}{Forever DIKU}
  {} % Bruges ikke, lad stå blank
  {Melodi} % Titel, Kunstner - eks.: "Jutlandia, Kim Larsen". Hvis sangen er på sin egen melodi, brug da \SBOrgMel.
  {Forfatter} % Navnet på forfatteren. Undlad kaldenavne. Brug gerne TBF. Brug "&" frem for "og". Hvis forfatter er ukendt, lad da stå tom.
  {Anledning og år} % Eks. "Fysikrevy, 2010" eller "2010"
  {\NotCCLIed} % Lad stå som den er

  \begin{SBVerse}
    % Skriv vers her
  \end{SBVerse}

  \begin{SBChorus}
    % Skriv omkvæd her
  \end{SBChorus}

  \begin{SBSection*}
    % Skriv sektioner her. Hvis du ønsker lidt mellemrum for at give luft i et langt afsnit el.lign., brug da \\\medskip
  \end{SBSection*}
\end{song}
\begin{song}{Han koder slam}
  {} % Bruges ikke, lad stå blank
  {Han får for lidt, Østkyst Hustlers} % Titel, Kunstner - eks.: "Jutlandia, Kim Larsen". Hvis sangen er på sin egen melodi, brug da \SBOrgMel.
  {} % Navnet på forfatteren. Undlad kaldenavne. Brug gerne TBF. Brug "&" frem for "og". Hvis forfatter er ukendt, lad da stå tom.
  {DIKUrevy, 1998} % Eks. "Fysikrevy, 2010" eller "2010"
  {\NotCCLIed} % Lad stå som den er

  \begin{SBVerse}
    Han roder med sin kode, for hans kode er for sej.\\
    Han ta'r Dat0 igen igen, han er et kæmpe kvaj.\\
    For funktionssprog er for kvinder, C++ det er for mænd.\\
    Han plejer' ta' en peger, holder typerne i spænd.
  \end{SBVerse}

  \begin{SBVerse}
    Og ud med kommentar'ne, for hans kod' er selvforklar'ne.\\
    Han koder som han koder, for at ligne ham der Bjarne.\\
    Og da han altid bruger goto, når koden er i udu,\\
    får han tit at vide at nog't såd'n det er misbrug.
  \end{SBVerse}

  \begin{SBVerse}
    Kun til Coca Cola, mand, for intet andet firma kan,\\
    få ham til at drikke deres søde sukkervand.\\
    Algoritmer for vatnisser. Han vil ha' Bugatti, så\\
    først ved terminalen. Der er slet ingen hvis'er.
  \end{SBVerse}

  \begin{SBVerse}
    Men hans kildekod', den er noget værre rod.\\
    Og "unsigned pointer temp stjerne", det virker ikke-nikke-\\
    Core-filen ligger der på klokkeslaget.\\
    Han føler sig så svag, han skal aflever' i dag!
  \end{SBVerse}

  \begin{SBChorus}
    Han koder slam! Han får sat koden på plads.\\
    Ingen klam analyse, bare en masse strabads.\\
    Han plejed' at kod' lidt kvajet, men nu koder han flot.\\
    Det syn's han selv, alligevel, så kør' det aldrig godt.\\
    Han koder slam! Han er det groveste skvat.\\
    Ingen klam analyse, bare kode i nat.\\
    Han koder slam! Koder slam!\\
    Han koder slam!
  \end{SBChorus}

  \begin{SBVerse}
    Han er en ualmind'lig stræbernørd, han syn's han er for hård.\\
    Han går i gang med sit speciale, mens han er på første år.\\
    Han læser datalogi, men er lidt skuffet fordi,\\
    at analyser og rapporter ka' han slet ikke li'.
  \end{SBVerse}

  \begin{SBVerse}
    Og han ta'r en masse kurser, han har næsen ned i bogen,\\
    han har helt utroligt svært ved at holde sig vågen.\\
    Og han stræber fremad, vil vær' professor en dag\\
    for multimedie programmør er ikke lige hans sag.
  \end{SBVerse}

  \begin{SBVerse}
    Men han har det problem, at han hader semantik,\\
    for der er induktionsbeviser, og han fatter det ikk'.\\
    Han burd' ku' nå en masse, med alle hans ressourcer,\\
    men det går bare ikke, han ta'r 10 forskellig' kurser.
  \end{SBVerse}

  \begin{SBVerse}
    Og han koder, gør han, men på alle hans fag,\\
    skal han både skriv' rapport og prøv' programmerne af.\\
    Man burd' ku' programmere, når man er en nørd som ham,\\
    men det ender jo uværg'ligt, med at blive noget slam.
  \end{SBVerse}

  \begin{SBChorus}
    Han koder slam!\ldots
  \end{SBChorus}

  \begin{SBVerse}
    Nu er han færdig med at kode, han mangler kun sin test\\
    men det er godt nok ikke sjovt, for han vil hellere til fest.\\
    Ta'r på cafe'n, der vil han bli', for han vil score en pi'!\\
    Han er jo såd'n' smart fyr, så det gør han bare li'.
  \end{SBVerse}

  \begin{SBVerse}
    Han fortæller vidt og bredt om sine datastrukturer\\
    Men pigen gaber højlydt og kigger på sit ur og\\
    han forklarer alt om C++, og klør sig i skridtet,\\
    illustrerer hvord'n GDB ka' debugge skidtet
  \end{SBVerse}

  \begin{SBVerse}
    Pigen hun ser godt ud, han er blevet helt forjættet.\\
    Hun ligner nemlig noget man ku' hente ned fra nettet.\\
    At hun er fysiker er okay, når bare han får sagt.\\
    At det fysiske som han vil ha' er fysisk nærkontakt.
  \end{SBVerse}

  \begin{SBVerse}
    Nu er hun sikkert snart mør, han siger "prøv nu at hør"\\
    Hvis hun hører mer' om data, så tror hun hun bli'r skør.\\
    Han siger sed og awk og lex og yacc så går hun sin vej!\\
    og stiv går han tilbage for at rette sine fejl.
  \end{SBVerse}

  \begin{SBChorus}
    Han koder slam!\ldots
  \end{SBChorus}
\end{song}
\begin{song}{Hjemmehackeriet}
  {} % Bruges ikke, lad stå blank
  {Hjemmebrænderiet} % Titel, Kunstner - eks.: "Jutlandia, Kim Larsen". Hvis sangen er på sin egen melodi, brug da \SBOrgMel.
  {} % Navnet på forfatteren. Undlad kaldenavne. Brug gerne TBF. Brug "&" frem for "og". Hvis forfatter er ukendt, lad da stå tom.
  {DIKUrevy, 2001} % Eks. "Fysikrevy, 2010" eller "2010"
  {\NotCCLIed} % Lad stå som den er

  \begin{SBVerse}
    Jeg bor her i Ishøj på syvende sal\\
    i en lejlighed der stort set er normal.\\
    En stue, et køkken, et bad med WC\\
    og et kammer hvor jeg har min hjemme-PC.
  \end{SBVerse}

  \begin{SBChorus}
    Jeg hacker, jeg cracker, jeg downloader spil,\\
    og jeg logger ind lig' præcis hvor jeg vil.\\
    Jeg kender dit password, jeg læser din post;\\
    for en hacker som mig er den slags hverdagskost.
  \end{SBChorus}

  \begin{SBVerse}
    Min fætter har hacket i Pentagons net.\\
    De tro'ed det var svært, men han syn's det var let.\\
    De fandt ham dog efter en længere jagt,\\
    så nu er han ansat som sikkerhedsvagt.
  \end{SBVerse}

  \begin{SBChorus}
    Jeg hacker, jeg cracker, jeg downloader spil,\\
    og jeg logger ind lig' præcis hvor jeg vil.\\
    Jeg kender dit password, jeg læser din post;\\
    for en hacker som mig er den slags hverdagskost.
  \end{SBChorus}

  % \begin{SBChorus}
  %   Jeg hacker, jeg cracker,\ldots
  % \end{SBChorus}

  \begin{SBVerse}
    Jeg laved' en virus som hed "I Love You".\\
    Jeg indrømmer dog, jeg fortryder det nu.\\
    Da jeg gik i banken, min løn for at få,\\
    havde virusen sat der's computer i stå.
  \end{SBVerse}

  \begin{SBChorus}
    Jeg hacker, jeg cracker, jeg downloader spil,\\
    og jeg logger ind lig' præcis hvor jeg vil.\\
    Jeg kender dit password, jeg læser din post;\\
    for en hacker som mig er den slags hverdagskost.
  \end{SBChorus}

  % \begin{SBChorus}
  %   Jeg hacker, jeg cracker,\ldots
  % \end{SBChorus}

  \begin{SBVerse}
    Hvis du sku' få lyst til at hacke lidt selv,\\
    jeg ønsker dig al mulig lykke og held.\\
    Det giver dig magt som om du var en gud,\\
    og du kan endda få din pizza bragt ud.
  \end{SBVerse}

  \begin{SBChorus}
    Jeg hacker, jeg cracker, jeg downloader spil,\\
    og jeg logger ind lig' præcis hvor jeg vil.\\
    Jeg kender dit password, jeg læser din post;\\
    for en hacker som mig er den slags hverdagskost.
  \end{SBChorus}

  % \begin{SBChorus}
  %   Jeg hacker, jeg cracker,\ldots
  % \end{SBChorus}
\end{song}
\begin{song}{Programmørens 8-bit drikkesang}
  {} % Bruges ikke, lad stå blank
  {99 Bottles of Beer} % Titel, Kunstner - eks.: "Jutlandia, Kim Larsen". Hvis sangen er på sin egen melodi, brug da \SBOrgMel.
  {} % Navnet på forfatteren. Undlad kaldenavne. Brug gerne TBF. Brug "&" frem for "og". Hvis forfatter er ukendt, lad da stå tom.
  {} % Eks. "Fysikrevy, 2010" eller "2010"
  {\NotCCLIed} % Lad stå som den er

  \begin{SBVerse}
    Der' 1 lille fejl i min kod',\\
    kun 1 lille fejl i min kod'.\\
    Jeg retter den lige, oversætter igen,\\
    så' der 3 små fejl i min kod'.
  \end{SBVerse}

  \begin{SBVerse}
    Der' 3 små fejl i min kod',\\
    kun 3 små fejl i min kod'.\\
    Jeg retter lige én, oversætter igen,\\
    så' der 5 små fejl i min kod'.
  \end{SBVerse}

  \begin{SBVerse}
    Der' 5 små fejl i min kod',\\
    kun 5 små fejl i min kod'.\\
    Jeg retter lige én, oversætter igen,\\
    så' der 9 små fejl i min kod'.
  \end{SBVerse}

  \begin{SBVerse}
    Der' 129 små fejl i min kod',\\
    129 små fejl i min kod'.\\
    Jeg retter lige én, oversætter igen,\\
    så' der 1 stor fejl i min kod'.
  \end{SBVerse}

  \begin{SBVerse}
    Der' 1 stor fejl i min kod',\\
    kun 1 stor fejl i min kod'.\\
    Jeg retter den lige, oversætter igen,\\
    så' der 3 stor' fejl i min kod'.
  \end{SBVerse}

  \begin{SBVerse}
    Der' 129 stor' fejl i min kod',\\
    129 stor' fejl i min kod.\\
    Jeg retter lige én, oversætter igen,\\
    så' der 1 gigantisk fejl.
  \end{SBVerse}

  \begin{SBVerse}
    Der' 1 gigantisk fejl,\\
    kun 1 gigantisk fejl.\\
    Jeg retter den lige, oversætter igen,\\
    så' der 3 gigantiske fejl.
  \end{SBVerse}

  \begin{SBVerse}
    Der' 129 gigantiske fejl,\\
    129 gigantiske fejl.\\
    Jeg retter lige én, oversætter igen,\\
    så' der 1 katastrofal fadæse.
  \end{SBVerse}

  \begin{SBVerse}
    Der' 1 katastrofal fadæse,\\
    kun 1 katastrofal fadæse.\\
    Jeg retter den lige, oversætter igen,\\
    så' der 3 katastrofale fadæser.
  \end{SBVerse}

  \begin{SBVerse}
    Der' 129 katastrofale fadæser,\\
    129 katastrofale fadæser.\\
    Jeg retter lige én, oversætter igen,\\
    og så udgiver Microsoft mit program!
  \end{SBVerse}
\end{song}
\begin{song}{HTML}
  {} % Bruges ikke, lad stå blank
  {YMCA, Village People} % Titel, Kunstner - eks.: "Jutlandia, Kim Larsen". Hvis sangen er på sin egen melodi, brug da \SBOrgMel.
  {Forfatter} % Navnet på forfatteren. Undlad kaldenavne. Brug gerne TBF. Brug "&" frem for "og". Hvis forfatter er ukendt, lad da stå tom.
  {DIKUrevy, 2000} % Eks. "Fysikrevy, 2010" eller "2010"
  {\NotCCLIed} % Lad stå som den er

  \begin{SBVerse}
    Færdig, jeg blev da-a-talog\\
    jeg blev færdig, og eksamen var go'\\
    jeg blev færdig, og sku' ha mig et job\\
    det sku' være no'et med kode
  \end{SBVerse}

  \begin{SBVerse}
    Jobbet, jeg ville ha' noget sjovt\\
    ja og poppet, så det var nu lidt flovt\\
    at jeg ikke, fik no'et med logik\\
    eller smarte algoritmer
  \end{SBVerse}

  \begin{SBChorus}
    For nu koder jeg i H-T-M-L\\
    ja nu koder jeg i H-T-M-L\\
    ja jeg fik mig et job med mange penge i\\
    og min hjerne den fik fri\\\medskip
    Så nu koder jeg i H-T-M-L\\
    ja nu koder jeg i H-T-M-L\\
    Det var ikke li' det jeg havde tænkt mig sku' ske\\
    jeg vil hellere kode 'C'
  \end{SBChorus}

  \begin{SBVerse}
    ML det var sagen for mig\\
    og Miranda, for jeg var jo for sej\\
    men på webben kan de ik' brug's til no'et\\
    de dur ik' på internettet
  \end{SBVerse}

  \begin{SBVerse}
    Kodet, bli'r der ik' meget af\\
    nej men møder, der er mange hver dag\\
    der er kunder, der vil kø-øbe alt\\
    bar' det' no'et med multimedier
  \end{SBVerse}

  \begin{SBChorus}
    For nu koder jeg i H-T-M-L\ldots
  \end{SBChorus}

  \begin{SBVerse}
    Penge, får jeg da mange af\\
    mange penge, flere tusind' hver dag\\
    men jeg gider snart ik' mere det her\\
    for jeg keder mig på jobbet
  \end{SBVerse}

  \begin{SBVerse}
    Hej ven, tag og hør lidt på mig\\
    jeg sagde hej ven, gør nu ikke som mig\\
    få et godt job, hvor du skal lav' noget sjovt\\
    og hold dig fra webbureauer
  \end{SBVerse}

  \begin{SBChorus}
    For nu koder jeg i H-T-M-L\ldots
  \end{SBChorus}
\end{song}
\begin{song}{Terminalsangen}
  {} % Bruges ikke, lad stå blank
  {Vuffelivov, Shu-bi-dua} % Titel, Kunstner - eks.: "Jutlandia, Kim Larsen". Hvis sangen er på sin egen melodi, brug da \SBOrgMel.
  {} % Navnet på forfatteren. Undlad kaldenavne. Brug gerne TBF. Brug "&" frem for "og". Hvis forfatter er ukendt, lad da stå tom.
  {DIKUrevy, 1978} % Eks. "Fysikrevy, 2010" eller "2010"
  {\NotCCLIed} % Lad stå som den er

  \begin{SBVerse}
Jeg har en skærm med mange taster\\
En for hvert symbol\\
Og bagved sidder lysintensiteten\\
Den ledning har mange tråde\\
En til hver sin bit\\
og en ekstra en til pariteten\\
Når man har venner og kærester, så er man normal\\
Men de ta'r tiden fra mig og min terminal
  \end{SBVerse}

  \begin{SBVerse}
Jeg er koblet via DIXI\\
Når DIXI ellers vil\\
Og der er plads på centrets multiplekser\\
Når jeg har lyst så kan jeg sidder\\
Og lege natten lang\\
Med RECKUs mange programmelkomplekser\\
Når man har venner og kærester så er man normal\\
Men de ta'r tiden fra mig og min terminal
  \end{SBVerse}

  \begin{SBVerse}
Jeg kør' på en maskine\\
Der klarer tusind jobs\\
Selvom deta'r syv lange og syv bredde\\
CAU'en har den to af\\
Og det er vældig smart\\
En til hvis den anden sku' vær' nede\\
Når man har venner og kærester så er man normal\\
Men de ta'r tiden fra mig og min terminal
  \end{SBVerse}

  \begin{SBVerse}
Jeg spiller skak og kryds og bolle\\
Den hele lange nat\\
Det er nu trist man ingen kender\\
For selvom den er dejlig\\
Så er den datamat\\
Nu kun et surrogat for menn'ske-venner\\
Når man har venner og kærester så er man normal\\
Og har det bedre end mig med min terminal
  \end{SBVerse}
\end{song}

\onecolumn
\SBChapter{Kemi, Biologi og Nano}
\twocolumn
\begin{song}{The New Periodic Table Song}
  {} % Bruges ikke, lad stå blank
  {Orphée aux enfers, Jaques Offenbach (Can Can)} % Titel, Kunstner - eks.: "Jutlandia, Kim Larsen". Hvis sangen er på sin egen melodi, brug da \SBOrgMel.
  {ASAPScience} % Navnet på forfatteren. Undlad aliasser. Brug "&" frem for "og". Hvis forfatter er ukendt, lad da stå tom.
  {2015} % Eks. "Fysikrevy 2010" eller "2010"
  {\NotCCLIed} % Lad stå som den er

  \begin{SBVerse}
    There's Hydrogen and Helium, then\\
    Lithium, Beryllium,\\
    Boron, Carbon everywhere,\\
    Nitrogen all through the air\\
    With Oxygen so you can breathe, and\\
    Fluorine for your pretty teeth,\\
    Neon to light up the signs,\\
    Sodium for salty times
  \end{SBVerse}
     
  \begin{SBVerse}
    Magnesium, Aluminium, Silicon,\\
    Phosphorus, then Sulfur, Chlorine and Argon\\
    Potassium, and Calcium so you'll grow strong,\\
    Scandium, Titanium, Vanadium and\\
    Chromium and Manganese
  \end{SBVerse}
     
  \begin{SBChorus}
    This is the Periodic Table,\\
    Noble gas is stable,\\
    Halogens and Alkali react agressively,\\
    each period will see new\\
    outer shells while elec-\\
    trons are added moving to the right\\
  \end{SBChorus}
     
  \begin{SBVerse}
    Iron is the 26th, then\\
    Cobalt, Nickel coins you get,\\
    Copper, Zinc and Gallium,\\
    Germanium and Arsenic\\
    Selenium and Bromine film,\\
    while Krypton helps light up your room,\\
    Rubidium and Strontium,\\
    then Yttrium, Zirconium
  \end{SBVerse}
     
  \begin{SBVerse}
    Niobium, Molybdenum, Technetium,\\
    Ruthenium, Rhodium, Palladium,\\
    Silver-ware then Cadmium and Indium,\\
    Tin-cans, Antimony, then Tellurium and\\
    Iodine and Xenon and then Caesium and...
  \end{SBVerse}
     
  \begin{SBVerse}
    Barium is 56 and\\
    this is where the table splits\\
    Where Lanthanides have just begun:\\
    Lanthanum, Cerium and Praseodymium\\
    Neodymium's next too\\
    Promethium, then 62's\\
    Samarium, Europium,\\
    Gadolinium and Terbium,\\
    Dysprosium, Holmium, Erbium, Thulium\\
    Ytterbium, Lutetium
  \end{SBVerse}
     
  \begin{SBVerse}
    Hafnium, Tantalum, Tungsten then we're on to\\
    Rhenium, Osmium and Iridium,\\
    Platinum, Gold to make you rich till you grow old\\
    Mercury to tell you when it's really cold
  \end{SBVerse}
     
  \begin{SBVerse}
    Thallium and Lead, then Bismuth for your tummy,\\
    Polonium, Astatine would not be yummy,\\
    Radon, Francium will last a little time,\\
    Radium then Actinides at 89
  \end{SBVerse}
     
  \begin{SBChorus}
    This is the Periodic Table\ldots
  \end{SBChorus}
     
  \begin{SBVerse}
    Actinium, Thorium, Protactinium,\\
    Uranium, Neptunium, Plutonium,\\
    Americium, Curium, Berkelium,\\
    Californium, Einsteinium, Fermium,\\
    Mendelevium, Nobelium, Lawrencium,\\
    Rutherfordium, Dubnium, Seaborgium,\\
    Bohrium, Hassium then Meitnerium,\\
    Darmstadtium, Roentgenium, Copernicium
  \end{SBVerse}
     
  \begin{SBVerse}
    Ununtrium, Flerovium,\\
    Ununpentium, Livermorium,\\
    Ununseptium, Ununoctium,\\
    And then we're done!
  \end{SBVerse}
\end{song}
\begin{song}{Kemisk elskovsvise}
  {} % Bruges ikke, lad stå blank
  {Santa Lucia} % Titel, Kunstner - eks.: "Jutlandia, Kim Larsen". Hvis sangen er på sin egen melodi, brug da \SBOrgMel.
  {Forfatter} % Navnet på forfatteren. Undlad kaldenavne. Brug gerne TBF. Brug "&" frem for "og". Hvis forfatter er ukendt, lad da stå tom.
  {Anledning og år} % Eks. "Fysikrevy, 2010" eller "2010"
  {\NotCCLIed} % Lad stå som den er

  \begin{SBVerse}
Oh Pige, vær mig huld,\\
fattig på gods og Au,\\
står her din riddersmand - \\
går gennem ild og H$_2$O,\\
for dig jeg ofrer alt,\\
du er mig livets NaCl,\\
smiler du til verdens vrimmel,\\
Al$_2$O$_3$ og himmel.
  \end{SBVerse}

  \begin{SBVerse}
Sødeste lille skalk,\\
tag fra mig længsels CaO,\\
sig blot et kærligt ord,\\
håbet i hjertet B.\\
I dine øjne fandt,\\
jeg livets Cx,\\
helt til jeg mit øje lukker,\\
for dig jeg C$_{12}$H$_{22}$O$_{11}$.
  \end{SBVerse}

  \begin{SBVerse}
Bi du længer står,\\
får jeg mit banesår,\\
er da mit ønske galt,\\
skal håbet Na$_3$SbS$_4$$\cdot$9H$_2$O.\\
Må jeg for NaOH og H$_2$O\\
vandre i ensom stand,\\
til en P vist jeg haster,\\
og ned mig kaster.
  \end{SBVerse}
\end{song}
\begin{song}{Kemisk julevise}
  {} % Bruges ikke, lad stå blank
  {Højt fra træets grønne top} % Titel, Kunstner - eks.: "Jutlandia, Kim Larsen". Hvis sangen er på sin egen melodi, brug da \SBOrgMel.
  {} % Navnet på forfatteren. Undlad kaldenavne. Brug gerne TBF. Brug "&" frem for "og". Hvis forfatter er ukendt, lad da stå tom.
  {} % Eks. "Fysikrevy, 2010" eller "2010"
  {\NotCCLIed} % Lad stå som den er

  \begin{SBVerse}
    Lehninger og Holleman\\
    kan vi uden skelen\\
    alle ved, at H$_2$O er vand,\\
    læg jer nu i Se.\\
    Jule-Tin-een falder blidt,\\
    snart er jorden hvid som CaCO$_3$,\\
    smuk-R$_2$CO-er spiller,\\
    spurven slår R-CN-ler.
  \end{SBVerse}

  \begin{SBVerse}
    Nu hvor vi om dette B'd,\\
    alle blevet m-C$_2$H$_5$OC$_2$H$_5$,\\
    sk-R-NH$_2$ sang i lystigt kor\\
    gøre maven letter.\\
    Vi en m-CH$_3$CO-led' ned,\\
    ROH gi'r h-Fe-en fred.\\
    Titan gla-Ag-i tømte\\
    Na$_2$CO$_3$-H$_2$O forsømte.
  \end{SBVerse}

  \begin{SBVerse}
    R$_1$COOR$_2$ hun har ingen SCN$_2$ -\\
    ...ser rundt med CnH$_2$n -\\
    ...der gaven ka-NO$_2$\\
    N$_2$ til en kjole.\\
    Højt Mn-tes uafbrudt\\
    og med øjet slår Bi.\\
    Højt en m-R-CH(OR)2'er\\
    hoved-C$_{10}$H$_{16}$ maler.
  \end{SBVerse}

  \begin{SBVerse}
    Aluminium Sn-g det er nu spist op,\\
    og man det Ti-er,\\
    som en W i sin krop\\
    mad for vore ganer.\\
    Jern-sten ku' vi alle Lithium\\
    den var så HgCl$_2$ vi\\
    Ni-et og C$_{12}$H$_{22}$O$_{11}$\\
    før vi lyset slukker.
  \end{SBVerse}
\end{song}
\begin{song}{10 små biorus}{}
  {10 små cyklister}
  {}
  {Biorevy 2015}
  {\NotCCLIed}

  \begin{SBVerse}
  10 nye studerende kom til bio C.\\
  En mødte ikke op, og så var der 9\\
  \end{SBVerse}

  \begin{SBChorus}
  Der var 1, der var 2, der var 3, der var 4,\\
  der var 5 på bio C\\
  Der var 6, der var 7, der var 8, der var 9,\\
  der var 10 på bio C
  \end{SBChorus}

  \begin{SBVerse}
  Bo sku' på hyttetur, ville gerne nå det,\\
  men han sov over sig og så var der otte
  \end{SBVerse}

  \begin{SBChorus}
  Der var 1, der var 2, der var 3, der var 4,\\
  der var 5 på bio C\\
  Der var 6, der var 7, der var 8, der var 9,\\
  der var 10 på bio C
  \end{SBChorus}

  % \begin{SBChorus}
  % Der var en, der var to\ldots
  % \end{SBChorus}

  \begin{SBVerse}
  En af de studerende var eksamens-tyv,\\
  men han blev opdaget og så var der syv
  \end{SBVerse}

  \begin{SBChorus}
  Der var 1, der var 2, der var 3, der var 4,\\
  der var 5 på bio C\\
  Der var 6, der var 7, der var 8, der var 9,\\
  der var 10 på bio C
  \end{SBChorus}

  % \begin{SBChorus}
  % Der var en, der var to\ldots
  % \end{SBChorus}

  \begin{SBVerse}
  Gry drak sig pisse stiv nærmest per refleks,\\
  men der var mødepligt og så var der seks
  \end{SBVerse}

  \begin{SBChorus}
  Der var 1, der var 2, der var 3, der var 4,\\
  der var 5 på bio C\\
  Der var 6, der var 7, der var 8, der var 9,\\
  der var 10 på bio C
  \end{SBChorus}

  % \begin{SBChorus}
  % Der var en, der var to\ldots
  % \end{SBChorus}

  \begin{SBVerse}
  Anna fra Aabenraa ville gerne hjem,\\
  men toget kørte galt og så var der fem
  \end{SBVerse}

  \begin{SBChorus}
  Der var 1, der var 2, der var 3, der var 4,\\
  der var 5 på bio C\\
  Der var 6, der var 7, der var 8, der var 9,\\
  der var 10 på bio C
  \end{SBChorus}

  % \begin{SBChorus}
  % Der var en, der var to\ldots
  % \end{SBChorus}

  \begin{SBVerse}
  En var en tur i zoo, hvor han så en tig’r.\\
  Han ville nøgle den og så var der fir’
  \end{SBVerse}

  \begin{SBChorus}
  Der var 1, der var 2, der var 3, der var 4,\\
  der var 5 på bio C\\
  Der var 6, der var 7, der var 8, der var 9,\\
  der var 10 på bio C
  \end{SBChorus}

  % \begin{SBChorus}
  % Der var en, der var to\ldots
  % \end{SBChorus}

  \begin{SBVerse}
  Ida tænkte: studydrugs - sikk’ en god idé!\\
  Hun tog for mang' af dem, og så var der tre
  \end{SBVerse}

  \begin{SBChorus}
  Der var 1, der var 2, der var 3, der var 4,\\
  der var 5 på bio C\\
  Der var 6, der var 7, der var 8, der var 9,\\
  der var 10 på bio C
  \end{SBChorus}

  % \begin{SBChorus}
  % Der var en, der var to\ldots
  % \end{SBChorus}

  \begin{SBVerse}
  Gummistøvler er et must for en biolog.\\
  Ib fik kviksand i sin sko, og så var der to
  \end{SBVerse}

  \begin{SBChorus}
  Der var 1, der var 2, der var 3, der var 4,\\
  der var 5 på bio C\\
  Der var 6, der var 7, der var 8, der var 9,\\
  der var 10 på bio C
  \end{SBChorus}

  % \begin{SBChorus}
  % Der var en, der var to\ldots
  % \end{SBChorus}

  \begin{SBVerse}
  To små studer'nde stod under mistelten.\\
  Bærrene er giftige, og så var der en
  \end{SBVerse}

  \begin{SBChorus}
  Der var 1, der var 2, der var 3, der var 4,\\
  der var 5 på bio C\\
  Der var 6, der var 7, der var 8, der var 9,\\
  der var 10 på bio C
  \end{SBChorus}

  % \begin{SBChorus}
  % Der var en, der var to\ldots
  % \end{SBChorus}

  \begin{SBVerse}
  En lille biorus blev ægte biolog\\
  Bestod alle fagene så han er sgu go'!
  \end{SBVerse}

  \begin{SBChorus}
  Der var 1, der var 2, der var 3, der var 4,\\
  der var 5, som dropped ud.\\
  Der var 6, der var 7, der var 8, der var 9,\\
  men den tiende han\ldots
  \end{SBChorus}

  han blev færdig, så han kom ud i arbejdsløsheden og havde glemt at melde sig ind i en A-kasse og kunne desværre ikke få dagpenge, og derfor skred konen og hunden, han røg på flasken og nu er han dranker med en slatten PIK!

\end{song}
\begin{song}{Spas i waders}
  {} % Bruges ikke, lad stå blank
  {Space Invaders, Hit n Hide} % Titel, Kunstner - eks.: "Jutlandia, Kim Larsen". Hvis sangen er på sin egen melodi, brug da \SBOrgMel.
  {} % Navnet på forfatteren. Undlad kaldenavne. Brug gerne TBF. Brug "&" frem for "og". Hvis forfatter er ukendt, lad da stå tom.
  {Biorevy, 2012} % Eks. "Fysikrevy, 2010" eller "2010"
  {\NotCCLIed} % Lad stå som den er

  \begin{SBChorus}
    Spas i waders i en sø\\
    Fanger Myggelarver, Super biolog\\
    Spas i waders i en sø\\
    Se på makrofytter, mål pH og lys\\
    Må afsted, kom nu med
  \end{SBChorus}

  \begin{SBVerse}
    Du er feltbiolog med super seje waders på\\
    Render og ta'r prøver dagen lang\\
    Jeg er feltbiolog med super seje waders på\\
    Renser danske søer for plankton
  \end{SBVerse}

  \begin{SBChorus}
    Spas i waders i en sø\\
    Fanger Myggelarver, Super biolog\\
    Spas i waders i en sø\\
    Se på makrofytter, mål pH og lys
  \end{SBChorus}

  \begin{SBChorus}
    Spas i waders i en sø\\
    Fanger Myggelarver, Super biolog\\
    Spas i waders i en sø\\
    Se på makrofytter, mål pH og lys\\
    Må afsted, kom nu med
  \end{SBChorus}

%   \begin{SBChorus}
% Spas i waders i en sø\ldots
%   \end{SBChorus}

%   \begin{SBChorus}
% Spas i waders i en sø\ldots\\
% Må afsted, kom nu med
%   \end{SBChorus}

  \begin{SBVerse}
Du er feltbiolog udstyret med planktonnet\\
Tæller og nøgler fisk og zooplankton\\
Ude i søerne, der har jeg net med småfisk i\\
At nøgle og måle, det kan jeg li'
  \end{SBVerse}

  \begin{SBChorus}
Spas i waders i en sø\\
Fanger Myggelarver, Super biolog\\
Spas i waders i en sø\\
Se på makrofytter, mål pH og lys
  \end{SBChorus}

  \begin{SBChorus}
Spas i waders i en sø\\
Fanger Myggelarver, Super biolog\\
Spas i waders i en sø\\
Se på makrofytter, mål pH og lys
  \end{SBChorus}

%   \begin{SBChorus}
% Spas i waders i en sø\ldots
%   \end{SBChorus}

%   \begin{SBChorus}
% Spas i waders i en sø\ldots
%   \end{SBChorus}

  \begin{SBSection*}
Du ta'r afsted op til Hillerød\\
Jeg vil med, se på livet i søen\\
 -- Vi finder en ny plankton art\\
Du og jeg tilsammen kan vi alt
 -- Vi kan alt
  \end{SBSection*}

  \begin{SBChorus}
Spas i waders i en sø\\
Fanger Myggelarver, Super biolog\\
Spas i waders i en sø\\
Se på makrofytter, mål pH og lys\
  \end{SBChorus}

  \begin{SBChorus}
Spas i waders i en sø\\
Fanger Myggelarver, Super biolog\\
Spas i waders i en sø\\
Se på makrofytter, mål pH og lys\\
Lom nu med
  \end{SBChorus}

%   \begin{SBChorus}
% Spas i waders i en sø\ldots
%   \end{SBChorus}

%   \begin{SBChorus}
% Spas i waders i en sø\ldots
% Kom nu med!
%   \end{SBChorus}
\end{song}
\begin{song}{Nano}
  {} % Bruges ikke, lad stå blank
  {Melodi} % Titel, Kunstner - eks.: "Jutlandia, Kim Larsen". Hvis sangen er på sin egen melodi, brug da \SBOrgMel.
  {Forfatter} % Navnet på forfatteren. Undlad kaldenavne. Brug gerne TBF. Brug "&" frem for "og". Hvis forfatter er ukendt, lad da stå tom.
  {Anledning og år} % Eks. "Fysikrevy, 2010" eller "2010"
  {\NotCCLIed} % Lad stå som den er

  \begin{SBVerse}
Nano, Nano.\\
Har du set en Nano?\\
Fremtidens genier\\
kan stå samlet på en tier
  \end{SBVerse}

  \begin{SBVerse}
Nano, Na-na-na-Nano.\\
Er det ikke sandt? Jo!\\
Venlige mikrober.\\
De er så små, at man ikke\\
kan se dem - selv i mikroskoper.
  \end{SBVerse}

  \begin{SBChorus}
Nano. Vi elsker jer\\
I gør det hele letter’.\\
For I kan bygge fremtidens tabletter\\
(Na-na-na-no-na-no-na-no-na-no)\\
med atomer og nano-pincetter.
  \end{SBChorus}

  \begin{SBVerse}
Nano, Nano.\\
Søde lille Nano.\\
Bittesmå profeter\\
på $10^-9$ meter.
  \end{SBVerse}

  \begin{SBVerse}
Nano. Na-na-na-Nano.\\
Vogt dig for en Nano.\\
Fremtidens spioner.\\
De infiltrerer hvad som helst\\
ved hjælp af simple diffusioner.
  \end{SBVerse}

  \begin{SBChorus}
Nano. Vi elsker jeres\\
nuttede studiner.\\
For de kan lave bittesmå maskiner\\
(Na-na-na-no-na-no-na-no-na-no)\\
som kan vaccinere selv vacciner.
  \end{SBChorus}

  \begin{SBVerse}
Nano, åh nano!\\
Ingen kan som Nano\\
nanorere sproget:\\
Med nano-adaptive nano-\\
gloser kører Nanotoget.
  \end{SBVerse}

  \begin{SBChorus}
Nano, Vi elsker jer!\\
Mikro er blot et minde.\\
Og selvom I er ret svære at finde\\
(Be-Be-Besenbacher-Besenbacher)\\
ved vi, Nano-tiden den er inde.
  \end{SBChorus}
\end{song}
\begin{song}{Laborant}
  {} % Bruges ikke, lad stå blank
  {Nøddepatruljen, Disney} % Titel, Kunstner - eks.: "Jutlandia, Kim Larsen". Hvis sangen er på sin egen melodi, brug da \SBOrgMel.
  {Forfatter} % Navnet på forfatteren. Undlad kaldenavne. Brug gerne TBF. Brug "&" frem for "og". Hvis forfatter er ukendt, lad da stå tom.
  {MBK-revyen, 2013} % Eks. "Fysikrevy, 2010" eller "2010"
  {\NotCCLIed} % Lad stå som den er

  \begin{SBVerse}
    Gi'r geler problemer,
    og sejler dit projekt?
    Dit array er gay,
    din protokol er væk.
    Ja så kommer de og redder dig.
    Specialet blir' en leg!
  \end{SBVerse}

  \begin{SBChorus}
    La-la-la-la-bo-rant!
    De kan mixe
    La-la-la-la-bo-rant!
    Uden at kikse!
    For hvis du er I nød så kommer de,
    bar' husk at spørg' før de tager fri!
  \end{SBChorus}

  \begin{SBVerse}
    De blander buffer
    i hver koncentration
    Men kommer aldrig
    på en publikation!
    Oprens mit plasmid, bestil enzym,
    de er et sørg'ligt syn!
  \end{SBVerse}

  \begin{SBChorus}
    La-la-la-la-bo-rant
    De' en gave
    La-la-la-la-bo-ra-
    -torieslave!
    For de er svar på hver professors bøn,
    det' godt de får så lidt i løn!
  \end{SBChorus}

  \begin{SBChorus}
    La-la-la-la-bo-rant
    De' en gave
    La-la-la-la-bo-ra-
    -torieslave!
    For de er svar på hver professors bøn,
    det' godt de får så lidt i løn!
  \end{SBChorus}

  % \begin{SBChorus}
  %   La-la-la-la-bo-rant\ldots
  % \end{SBChorus}

  \begin{SBSection*}
    La-la-la-la-bo-rant!
  \end{SBSection*}
\end{song}
\begin{song}{Selektionssangen}
  {} % Bruges ikke, lad stå blank
  {Hodja fra Pjort, Sebastian} % Titel, Kunstner - eks.: "Jutlandia, Kim Larsen". Hvis sangen er på sin egen melodi, brug da \SBOrgMel.
  {} % Navnet på forfatteren. Undlad kaldenavne. Brug gerne TBF. Brug "&" frem for "og". Hvis forfatter er ukendt, lad da stå tom.
  {BioRevy, 2011} % Eks. "Fysikrevy, 2010" eller "2010"
  {\NotCCLIed} % Lad stå som den er

  \begin{SBVerse}
    Jeg så en hjort, jeg så en hjort,\\
    men den løb længere og længere bort,\\
    indtil jeg ramte den temmelig hårdt,\\
    kørende i min Ford.\\
    Jeg ved jo ik', hvad jeg ellers sku' ha gjort.
  \end{SBVerse}

  \begin{SBVerse}
    Jeg så en mår, jeg så en mår.\\
    Den var så nuttet og dækket med hår.\\
    Pelsen var filtret lidt li'som et får,\\
    men det jeg ik' forstår,\\
    hvordan min kam ku' gi så mange sår.
  \end{SBVerse}

  \begin{SBChorus}
    Naturens love\\
    er bare så sjove.\\
    Evolutionen,\\
    vi sidder på tronen\\
    med scepter\\
    og styr' selektionen!
  \end{SBChorus}

  \begin{SBVerse}
    Jeg så en kat, jeg så en kat.\\
    Aldrig har jeg hørt så voldsomt et splat.\\
    Pludselig lå der en kæmpestor klat\\
    lige der hvor jeg sat.\\
    Hvad sku jeg gøre, for helved' det var nat!
  \end{SBVerse}

  \begin{SBVerse}
    Jeg så en stær, jeg så en stær\\
    og da det kribled' i fingre og tær,\\
    måtte jeg simpelthen komme den nær,\\
    kæle den med mit sværd.\\
    Men det er mærk'ligt nu pipper den ej mer.
  \end{SBVerse}

  \begin{SBChorus}
    Naturens love\\
    er bare så sjove.\\
    Evolutionen,\\
    vi sidder på tronen\\
    med scepter\\
    og styr' selektionen!
  \end{SBChorus}

  % \begin{SBChorus}
  %   Naturens love\ldots
  % \end{SBChorus}

  \begin{SBVerse}
    Se den kanin, se den kanin.\\
    Den var så nuttet, så blød og så fin\\
    indtil jeg glemte den i min kamin.\\
    Jeg føler mig til grin.\\
    Hvorfor skulle den også dyppes i palmin?
  \end{SBVerse}

  \begin{SBVerse}
    Se den pingvin, se den pingvin.\\
    Jeg tror habitten er større end min!\\
    Geværet frem, tømmer mit magasin\\
    på det sort-hvide svin.\\
    Nu er der end'lig frisk mad i vor kantin'.
  \end{SBVerse}
\end{song}
\endsong
\begin{song}{Når jeg kloner min kat}
  {} % Bruges ikke, lad stå blank
  {When You're Looking Like That, Westlife} % Titel, Kunstner - eks.: "Jutlandia, Kim Larsen". Hvis sangen er på sin egen melodi, brug da \SBOrgMel.
  {Forfatter} % Navnet på forfatteren. Undlad kaldenavne. Brug gerne TBF. Brug "&" frem for "og". Hvis forfatter er ukendt, lad da stå tom.
  {BirRevy, 2012} % Eks. "Fysikrevy, 2010" eller "2010"
  {\NotCCLIed} % Lad stå som den er

  \begin{SBSection*}
    Når jeg kloner min kat
  \end{SBSection*}

  \begin{SBVerse}
    Med en ny teknik kan os på bio skabe liv\\
    Den ku' bruges ondt og klone Henrik Busch\\
    Til vi har en hær\\
    Det er ikke det, vi gør!\\
    Bruger teknikken til at redde arter, men\\
    Sker der noget skidt med dyret vi elsker, så\\
    Må mammutten vente\\
    Ikke tid til at grundforske
  \end{SBVerse}

  \begin{SBChorus}
    Ska' aldrig mere sig' farvel\\
    Når jeg kloner min kat\\
    Vil altid være perfekt, nuttet, sød\\
    For altid vær' min skat\\
    Aldrig købe mig en ny\\
    Aldrig købe et erstatningskæledyr\\
    Ska' aldrig mere sig farvel\\
    Når jeg kloner min kat
  \end{SBChorus}

  \begin{SBVerse}
    Jurassic Park var tyd'ligvis en succes\\
    Dinosaurus på jord ku' være fedt\\
    Men sig mig, hvordan gør man det?\\
    Indsæt gen og skab superarter\\
    Nem vej til at skabe über seje dyr\\
    Men når katten forsvinder, har vi et hyr\\
    Så mammutten venter\\
    ikke tid til at grundforske
  \end{SBVerse}

  \begin{SBChorus}
    Ska' aldrig mere sig' farvel\\
    Når jeg kloner min kat\\
    Vil altid være perfekt, nuttet, sød\\
    For altid vær' min skat\\
    Aldrig købe mig en ny\\
    Aldrig købe et erstatningskæledyr\\
    Ska' aldrig mere sig farvel\\
    Når jeg kloner min kat
  \end{SBChorus}

%   \begin{SBChorus}
% Ska' aldrig mere sig' farvel\ldots
%   \end{SBChorus}

  \begin{SBSection*}
    Skal jeg så spørge jer nu\\
    Hvilket studie har magt?\\
    Biologer har livets kraft\\
    Så jeg kloner min kat
  \end{SBSection*}

  \begin{SBChorus}
    Aldrig mere sig' farvel\\
    Når jeg kloner min kat\\
    Vil altid være perfekt, nuttet, sød\\
    For altid vær' min skat\\
    Aldrig købe mig en ny\\
    Aldrig købet et erstatningskæledyr\\
    Ska' aldrig mere sig farvel\\
    Når jeg kloner min kat
  \end{SBChorus}

  \begin{SBChorus}
    Ska' aldrig mere sig' farvel\\
    Når jeg kloner min kat\\
    Vil altid være perfekt, nuttet, sød\\
    For altid vær' min skat\\
    Aldrig købe mig en ny\\
    Aldrig købe et erstatningskæledyr\\
    Ska' aldrig mere sig farvel\\
    Når jeg kloner min kat
  \end{SBChorus}

  \begin{SBChorus}
    Ska' aldrig mere sig' farvel\\
    Når jeg kloner min kat\\
    Vil altid være perfekt, nuttet, sød\\
    For altid vær' min skat\\
    Aldrig købe mig en ny\\
    Aldrig købe et erstatningskæledyr\\
    Ska' aldrig mere sig farvel\\
    Når jeg kloner min kat
  \end{SBChorus}

%   \begin{SBChorus}
% Ska' aldrig mere sig' farvel\ldots
%   \end{SBChorus}

%   \begin{SBChorus}
% Ska' aldrig mere sig' farvel\ldots
%   \end{SBChorus}
\end{song}
\begin{song}{Pest}
  {} % Bruges ikke, lad stå blank
  {Melodi} % Titel, Kunstner - eks.: "Jutlandia, Kim Larsen". Hvis sangen er på sin egen melodi, brug da \SBOrgMel.
  {Forfatter} % Navnet på forfatteren. Undlad kaldenavne. Brug gerne TBF. Brug "&" frem for "og". Hvis forfatter er ukendt, lad da stå tom.
  {Anledning og år} % Eks. "Fysikrevy, 2010" eller "2010"
  {\NotCCLIed} % Lad stå som den er

  \begin{SBVerse}
    % Skriv vers her
  \end{SBVerse}

  \begin{SBChorus}
    % Skriv omkvæd her
  \end{SBChorus}

  \begin{SBSection*}
    % Skriv sektioner her. Hvis du ønsker lidt mellemrum for at give luft i et langt afsnit el.lign., brug da \\\medskip
  \end{SBSection*}
\end{song}

\onecolumn
\SBChapter{Alkohol og druk}
\twocolumn
\begin{song}{Busted}
  {} % Bruges ikke, lad stå blank
  {Melodi} % Titel, Kunstner - eks.: "Jutlandia, Kim Larsen". Hvis sangen er på sin egen melodi, brug da \SBOrgMel.
  {Forfatter} % Navnet på forfatteren. Undlad kaldenavne. Brug gerne TBF. Brug "&" frem for "og". Hvis forfatter er ukendt, lad da stå tom.
  {Anledning og år} % Eks. "Fysikrevy, 2010" eller "2010"
  {\NotCCLIed} % Lad stå som den er

  \begin{SBVerse}
    % Skriv vers her
  \end{SBVerse}

  \begin{SBChorus}
    % Skriv omkvæd her
  \end{SBChorus}

  \begin{SBSection*}
    % Skriv sektioner her. Hvis du ønsker lidt mellemrum for at give luft i et langt afsnit el.lign., brug da \\\medskip
  \end{SBSection*}
\end{song}
\begin{song}{Der er et ølrigt land}
  {} % Bruges ikke, lad stå blank
  {Melodi} % Titel, Kunstner - eks.: "Jutlandia, Kim Larsen". Hvis sangen er på sin egen melodi, brug da \SBOrgMel.
  {Forfatter} % Navnet på forfatteren. Undlad kaldenavne. Brug gerne TBF. Brug "&" frem for "og". Hvis forfatter er ukendt, lad da stå tom.
  {Anledning og år} % Eks. "Fysikrevy, 2010" eller "2010"
  {\NotCCLIed} % Lad stå som den er

  \begin{SBVerse}
    % Skriv vers her
  \end{SBVerse}

  \begin{SBChorus}
    % Skriv omkvæd her
  \end{SBChorus}

  \begin{SBSection*}
    % Skriv sektioner her. Hvis du ønsker lidt mellemrum for at give luft i et langt afsnit el.lign., brug da \\\medskip
  \end{SBSection*}
\end{song}

% \beginsong{Der er et ølrigt land}[sr={Melodi: Der er et yndigt land}
% ,
% by={}
% ,
% cr={}]
% \beginverse
% Der er et ølrigt land,
% det står med nød og næppe
% blandt alt det pokkers vand
% blandt alt det pokkers vand.
% Det bugter sig i bar og kro.
% Det hedder gamle Danmark,
% og her er øllen go'
% ja, hver en øl er go'.

% \endverse
% \beginverse
% Her drak i fordums tid
% hver tillakkede kæmper
% sin mjød af fad med flid
% sin mjød af fad med flid.
% Så prøved' han, ej uden mén
% at finde sine bene,
% men faldt ved hver en sten
% ja, hver en bautasten.

% \endverse
% \beginverse
% Den øl endnu er skøn,
% og gid den aldrig vælter.
% Lad baj'ren stå så grøn
% lad baj'ren stå så grøn.
% De ædle sorters skønne øer
% med sutter, sulde svende
% og svimle danske møer
% ja, svimle danske møer.

% \endverse
% \beginverse
% Hil druk og fædreland.
% Hil hver en kølig bajer.
% Vi drikker dem vi kan
% vi drikker dem vi kan.
% Vort gamle Danmark -- SKÅL! -- bestå
% så længe øllet skummer,
% og næsen den bli'r blå
% med røde prikker på.

% \endverse
% \endsong


\onecolumn
\SBChapter{Almindelige sange}
\twocolumn
\begin{song}{Puma (Stork)}{}
  {\SBOrgMel}
  {Roben og Knud}
  {2001}
  {\NotCCLIed}

  \begin{SBVerse}
    Jeg vil gerne ha' en stork\\
    Men dens næb sku' være kort\\
    \SBRepeat{Jeg synes en stork er grim med langt næb}
  \end{SBVerse}

  \begin{SBVerse}
    Jeg vil gerne ha' en stork\\
    Men dens hals sku' være kort\\
    \SBRepeat{Jeg synes en stork er grim med lang hals}
  \end{SBVerse}

  \begin{SBSection*}
    Og med et kort næb\\
    Og med en kort hals\\
    Måske med lidt gullig pels\\
    Så ville det ligne en puma
  \end{SBSection*}

  \begin{SBChorus}
    \SBRepeat{Måske ik' den sejeste puma verden har set,\\
    Med dog en puma, og hurra for det!}
  \end{SBChorus}

  \begin{SBVerse}
    Jeg vil gerne ha' en stork\\
    Men to ben er bare ikke nok\\
    \SBRepeat{Jeg synes en stork er grim med to ben}
  \end{SBVerse}

  \begin{SBSection*}
    Og med et kort næb\\
    Og med en kort hals\\
    Måske med lidt gullig pels\\
    Så ville det ligne en puma
  \end{SBSection*}

  \begin{SBChorus}
    Måske ik' den sejeste puma verden har set\ldots
  \end{SBChorus}

  \begin{SBSection*}
    Og med et kort næb\\
    Og med en kort hals\\
    Måske med lidt gullig pels\\
    Så ville det ligne en puma
  \end{SBSection*}

  \begin{SBChorus}
    Måske ik' den sejeste puma verden har set\ldots
  \end{SBChorus}

  \begin{SBSection*}
    \SBRepeat{\SBRepeat{\SBRepeat{Hurra for det}}}
  \end{SBSection*}

\end{song}
% \begin{song}{All Star}
  {} % Bruges ikke, lad stå blank
  {Melodi} % Titel, Kunstner - eks.: "Jutlandia, Kim Larsen". Hvis sangen er på sin egen melodi, brug da \SBOrgMel.
  {Forfatter} % Navnet på forfatteren. Undlad kaldenavne. Brug gerne TBF. Brug "&" frem for "og". Hvis forfatter er ukendt, lad da stå tom.
  {Anledning og år} % Eks. "Fysikrevy, 2010" eller "2010"
  {\NotCCLIed} % Lad stå som den er

  \begin{SBVerse}
    % Skriv vers her
  \end{SBVerse}

  \begin{SBChorus}
    % Skriv omkvæd her
  \end{SBChorus}

  \begin{SBSection*}
    % Skriv sektioner her. Hvis du ønsker lidt mellemrum for at give luft i et langt afsnit el.lign., brug da \\\medskip
  \end{SBSection*}
\end{song}
% \begin{song}{Never Gonna Give You Up}
  {} % Bruges ikke, lad stå blank
  {Melodi} % Titel, Kunstner - eks.: "Jutlandia, Kim Larsen". Hvis sangen er på sin egen melodi, brug da \SBOrgMel.
  {Forfatter} % Navnet på forfatteren. Undlad kaldenavne. Brug gerne TBF. Brug "&" frem for "og". Hvis forfatter er ukendt, lad da stå tom.
  {Anledning og år} % Eks. "Fysikrevy, 2010" eller "2010"
  {\NotCCLIed} % Lad stå som den er

  \begin{SBVerse}
    We're no strangers to love\\
    You know the rules and so do I\\
    A full commitment's what I'm thinking of\\
    You wouldn't get this from any other guy\\\medskip
    I just wanna tell you how I'm feeling\\
    Gotta make you understand
  \end{SBVerse}

  \begin{SBChorus}
    Never gonna give you up\\
    Never gonna let you down\\
    Never gonna run around and desert you\\
    Never gonna make you cry\\
    Never gonna say goodbye\\
    Never gonna tell a lie and hurt you
  \end{SBChorus}

  \begin{SBVerse}
    We've known each other for so long\\
    Your heart's been aching but you're too shy to say it\\
    Inside we both know what's been going on\\
    We know the game and we're gonna play it\\\medskip
    And if you ask me how I'm feeling\\
    Don't tell me you're too blind to see
  \end{SBVerse}

  \begin{SBChorus}
    Never gonna give you up\ldots
  \end{SBChorus}

  \begin{SBChorus}
    Never gonna give you up\ldots
  \end{SBChorus}

  \begin{SBSection*}
    Never gonna give, never gonna give\\
    \emph{(Give you up)}\\
    Never gonna give, never gonna give\\
    \emph{(Give you up)}
  \end{SBSection*}

  \begin{SBVerse}
    We've known each other for so long\\
    Your heart's been aching but you're too shy to say it\\
    Inside we both know what's been going on\\
    We know the game and we're gonna play it\\\medskip
    And if you ask me how I'm feeling\\
    Don't tell me you're too blind to see
  \end{SBVerse}

  \begin{SBChorus}
    Never gonna give you up\ldots
  \end{SBChorus}

  \begin{SBChorus}
    Never gonna give you up\ldots
  \end{SBChorus}
\end{song}
\begin{song}{En elefant kom marcherende}
  {} % Bruges ikke, lad stå blank
  {Melodi} % Titel, Kunstner - eks.: "Jutlandia, Kim Larsen". Hvis sangen er på sin egen melodi, brug da \SBOrgMel.
  {Forfatter} % Navnet på forfatteren. Undlad kaldenavne. Brug gerne TBF. Brug "&" frem for "og". Hvis forfatter er ukendt, lad da stå tom.
  {Anledning og år} % Eks. "Fysikrevy, 2010" eller "2010"
  {\NotCCLIed} % Lad stå som den er

  \begin{SBVerse}
    % Skriv vers her
  \end{SBVerse}

  \begin{SBChorus}
    % Skriv omkvæd her
  \end{SBChorus}

  \begin{SBSection*}
    % Skriv sektioner her. Hvis du ønsker lidt mellemrum for at give luft i et langt afsnit el.lign., brug da \\\medskip
  \end{SBSection*}
\end{song}
\begin{song}{Den rekursive skovsang}
  {} % Bruges ikke, lad stå blank
  {\SBOrgMel} % Titel, Kunstner - eks.: "Jutlandia, Kim Larsen". Hvis sangen er på sin egen melodi, brug da \SBOrgMel.
  {} % Navnet på forfatteren. Undlad aliasser. Brug "&" frem for "og". Hvis forfatter er ukendt, lad da stå tom.
  {} % Eks. "Fysikrevy 2010" eller "2010"
  {\NotCCLIed} % Lad stå som den er

  \begin{SBVerse}
    Langt ude i skoven lå en lille skov.\\
    Aldrig så jeg så dejlig en skov.\\
    Skoven ligger langt ude i skoven
  \end{SBVerse}
  \renewcommand{\theSBVerseCnt}{$n>0$}
  
  \begin{SBVerse}
    Og i den lille skov, der lå en lille skov.\\
    Aldrig så jeg så dejlig en skov.\\
    \{Skoven i skoven,\}$^{n}$\\
    skoven ligger langt ude i skoven
  \end{SBVerse}
\end{song}
\begin{song}{Jeg ved en lærkerede}
  {} % Bruges ikke, lad stå blank
  {\SBOrgMel} % Titel, Kunstner - eks.: "Jutlandia, Kim Larsen". Hvis sangen er på sin egen melodi, brug da \SBOrgMel.
  {Harald Bergstedt, Carl Nielsen, slettelak} % Navnet på forfatteren. Undlad aliasser. Brug "&" frem for "og". Hvis forfatter er ukendt, lad da stå tom.
  {} % Eks. "Fysikrevy 2010" eller "2010"
  {\NotCCLIed} % Lad stå som den er

  \begin{SBVerse}
    Jeg ved en lærkerede,\\
    jeg siger ikke mer';
  \end{SBVerse}
\end{song}



% 
\begin{song}{What A Mighty God We Serve}{C}
  {\SBOrgMel}
  {}
  {Isaiah~9:6}
  {\NotCCLIed}

  \renewcommand{\RevDate}{February~11,~1993}
  \SBRef{Hosanna! Music Book~I}{\#93}

  \begin{SBOpGroup}
    \Ch{C}{What} a mighty God we serve,
    
    What a mighty God we \Ch{G7}{serve},
    
    \Ch{C}{An}gels bow before Him,
    
    \Ch{C}{Hea}ven and earth adore Him,
    
    \Ch{C}{What} a mighty \Ch{G7}{God} we \Ch{C}{serve!}\Ch{[}{}\Ch{F}{} \Ch{C}{}\Ch{]}{}
  \end{SBOpGroup}

  \begin{SBVerse}
    O \Ch{C}{Zion,} O \Ch{F}{Zion,} that \Ch{G7}{bring}est good \Ch{C}{tid}ings,

    Get thee \Ch{F}{up} into the \Ch{G7}{High} Moun\Ch{C}{tains}

    Je\Ch{C}{ru}salem, Je\Ch{F}{ru}salem, that \Ch{G7}{bring}est good \Ch{C}{tid}ings

    Lift up thy \Ch{F}{voice} with \Ch{G7}{all} thy \Ch{C}{strength}

    Lift it \Ch{F}{up,} be not afraid;

    Lift it \Ch{C}{up,} be not afraid

    Say \Ch{Am}{unto} the \Ch{C}{ci}ties of \Ch{G7}{Judah,}

    ``Behold your \Ch{C}{God,}\Ch{C7}{} Behold your \Ch{F}{God,}

    Be\Ch{C}{hold} \Ch{G7}{your} \Ch{C}{God!''}
  \end{SBVerse}

  \begin{SBExtraKeys}{
  \CBPageBrk
  \CSColBrk
    \STitle{What A Mighty God We Serve}{D}

    \begin{SBOpGroup}
      \Ch{D}{What} a mighty God we serve,
        
      What a mighty God we \Ch{A7}{serve},
      
      \Ch{D}{An}gels bow before Him,
      
      \Ch{D}{Hea}ven and earth adore Him,
      
      \Ch{D}{What} a mighty \Ch{A7}{God} we \Ch{D}{serve!}\Ch{[}{}\Ch{G}{} \Ch{D}{}\Ch{]}{}
    \end{SBOpGroup}

    \begin{SBVerse}
      O \Ch{D}{Zion,} O \Ch{G}{Zion,} that \Ch{A7}{bring}est good \Ch{D}{tid}ings,
                
      Get thee \Ch{G}{up} to into the \Ch{A7}{High} Moun\Ch{D}{tains}

      Je\Ch{D}{ru}salem, Je\Ch{G}{ru}salem, that \Ch{A7}{bring}est good \Ch{D}{tid}ings

      Lift up thy \Ch{G}{voice} with \Ch{A7}{all} thy \Ch{D}{strength}

      Lift it \Ch{G}{up} be not afraid,

      Lift it \Ch{D}{up} be not afraid

      Say \Ch{Bm}{unto} the \Ch{D}{ci}ties of \Ch{A7}{Judah,}

      ``Behold your \Ch{D}{God,}\Ch{D7}{} Behold your \Ch{G}{God,}

      Be\Ch{D}{hold} \Ch{A7}{your} \Ch{D}{God!''}
    \end{SBVerse}
  }\end{SBExtraKeys}
\end{song}


\begin{song}{Finally, My Brethren}{D}
  {}
  {}
  {Ephesians~6:10}
  {\NotCCLIed}

  \renewcommand{\RevDate}{February~11,~1993}
  %\SBRef{}{}

  \begin{SBVerse}
    \Ch{D}{Fi}nally, my brethren, be \Ch{C}{strong} in the \Ch{D}{Lord,}

    \Ch{D}{Fi}nally, my brethren, be \Ch{C}{strong} in the \Ch{D}{Lord,}

    And in the \Ch{G}{pow}er \Ch{A}{of} His \Ch{D}{might,}

    And in the \Ch{G}{pow}er \Ch{A}{of} His \Ch{D}{might.}

    For it is \Ch{G}{God} at \Ch{A}{work} with\Ch{D}{in} you,

    Both to \Ch{G}{will} \Ch{A}{and} to \Ch{D}{do,}

    \Ch{G}{All} his \Ch{A}{glo}rious \Ch{D}{plea}sure.

    His \Ch{G}{\em truth} shall \Ch{A}{sur}ely en\Ch{D}{dure.}
  \end{SBVerse}

  \begin{SBVerse}
    {\em \ldots strength \ldots}
  \end{SBVerse}

  \begin{SBVerse}
    {\em \ldots mercy \ldots}
  \end{SBVerse}

  \begin{SBVerse}
    {\em \ldots grace \ldots}
  \end{SBVerse}

  \begin{SBVerse}
    \Ch{D}{Fi}nally my brethren be \Ch{C}{strong} in the \Ch{D}{Lord,}
    
    \Ch{D}{Fi}nally my brethren be \Ch{C}{strong} in the \Ch{D}{Lord,}

    And in the \Ch{G}{pow}er \Ch{A}{of} His \Ch{D}{might,}

    And in the \Ch{G}{pow}er \Ch{A}{of} His \Ch{D}{might,}

    And in the \Ch{G}{pow}er \Ch{A}{of} His \Ch{D}{might.}
  \end{SBVerse}
\end{song}


\begin{song}[C]{The Lord Of Covenant}{Am}
  {}
  {}
  {}
  {\NotCCLIed}

  \renewcommand{\RevDate}{February~11,~1993}
  %\SBRef{}{}
  \FLineIdx{Lift your hands, for He is Holy}

  \begin{SBOpGroup}
    \Ch{Am}{Lift} your hands, for \Ch{E}{He} is Holy.
    
    \Ch{Dm7}{Worship} Him in \Ch{E}{spir}it and \Ch{Am}{truth.}
    
    For His face, is \Ch{E}{like} the sun,
    
    \Ch{Dm7}{The} Lord of \Ch{E}{Cove}\Ch{Am}{nant.}
  \end{SBOpGroup}

  \begin{SBVerse}
    \Ch{Dm}{Clap} your hands and \Ch{G}{shout} in victo\Ch{C}{ry,}
    
    For the Lamb was \Ch{E}{slain} but lives.

    \Ch{Dm7}{Dance} for joy and \Ch{G}{give} Him thanks.

    {\SBLyricNoteFont (Men)} For the \Ch{C}{free}dom \ldots

    {\SBLyricNoteFont (Women)} For the freedom His \Ch{E}{life} gives.

    {\SBLyricNoteFont (Men)} For the \Ch{F}{free}dom \ldots

    {\SBLyricNoteFont (Women)} For the freedom His \Ch{E}{life} gives.
  \end{SBVerse}

  \begin{SBVerse}
    \Ch{Dm}{Come} you people, \Ch{G}{lift} your voices, \Ch{C}{}

    Praise Him with your \Ch{E}{hearts} as one.

    \Ch{Dm7}{There} is One who \Ch{G}{turn}ed the key.

    {\SBLyricNoteFont (Men)} And the \Ch{C}{door} \ldots

    {\SBLyricNoteFont (Women)} To His throne is \Ch{E}{open.}

    {\SBLyricNoteFont (Men)} Yes, the \Ch{F}{door} \ldots

    {\SBLyricNoteFont (Women)} To His throne is \Ch{E}{open.}
  \end{SBVerse}

  \CSColBrk
  \begin{SBVerse}
    \Ch{Dm}{He} creates Je\Ch{G}{rusa}lem for re\Ch{C}{joic}ing,

    For re\Ch{E}{joic}ing,

    \Ch{Dm7}{And} her people \Ch{G}{for} gladness.

    {\SBLyricNoteFont (Men)} \Ch{C}{He} \ldots

    {\SBLyricNoteFont (Women)} He is glad in His \Ch{E}{peo}ple.

    {\SBLyricNoteFont (Men)} Yes, \Ch{F}{He} \ldots

    {\SBLyricNoteFont (Women)} He is glad in His \Ch{E}{peo}ple.
  \end{SBVerse}
\end{song}


\begin{song}{We Will Rejoice}{D}
  {}
  {}
  {Song Of Solomon~1:4}
  {\NotCCLIed}

  \renewcommand{\RevDate}{February~11,~1993}
  \SBRef{Hosanna! Music Book~II}{\#143}

  \begin{SBOpGroup}
    \Ch{D}{We} will rejoice in \Ch{G}{You} and be glad,
  
    \Ch{D}{We} will extol Your \Ch{A7}{love} more than wine
        
    \Ch{D}{Draw} me after \Ch{D7}{You} and let us \Ch{G}{run} together,
    
    \Ch{D}{We} will rejoice in \Ch{G}{You} and \Ch{A7}{be} \Ch{D}{glad}
  \end{SBOpGroup}

  \begin{SBChorus}
    \Ch{D}{Lift} up the light of Thy \Ch{G}{coun}tenance,
    
    U\Ch{D}{pon} us O \Ch{A7}{Lord}

    \Ch{D}{Lift} up the light of Thy \Ch{G}{coun}tenance,

    U\Ch{D}{pon} \Ch{A7}{us} O \Ch{D}{Lord} \Ch{G}{} \Ch{D}{}
  \end{SBChorus}
\end{song}


\begin{song}{Praise Him In His Sanctuary}{Em}
  {}
  {}
  {Psalm~150}
  {\NotCCLIed}

  \renewcommand{\RevDate}{February~11,~1993}
  %\SBRef{}{}

  \begin{SBVerse}
    Praise Him \Ch{Em}{in} His sanctuary

    Praise Him \Ch{Em}{in} the skies above

    Praise Him \Ch{G}{for} the acts of \Ch{D}{pow}er that He \Ch{Em}{does}

    Praise Him \Ch{Em}{for} surpassing greatness

    With the \Ch{Em}{trum}pet, harp and lyre

    With the \Ch{G}{tam}bourines and \Ch{Am}{dan}cing, praise Him \Ch{Em}{now}
  \end{SBVerse}

  \begin{SBOpGroup}
    Come and \Ch{G}{praise} \Ch{D}{Him} \Ch{Bm}{for} the Lord is \Ch{Em}{good}
  
    And His mercy is \Ch{B7}{ever}\Ch{Em}{lasting}
    
    Come and \Ch{G}{praise} \Ch{D}{Him} \Ch{Bm}{for} the Lord is \Ch{Em}{good}
    
    And His mercy is \Ch{B7}{ever}\Ch{Em}{lasting}
  \end{SBOpGroup}

  \begin{SBVerse}
    Praise Him \Ch{Em}{with} the clashing cymbals

    Let them \Ch{Em}{hear} it far and near

    With the \Ch{G}{strings} and flutes we'll \Ch{D}{praise} the Lord our \Ch{Em}{God}

    Who with \Ch{Em}{majes}ty is reigning

    He has \Ch{Em}{power} over all

    Give Him \Ch{G}{glory} and be \Ch{Am}{thankful} for His \Ch{Em}{love}
  \end{SBVerse}
\end{song}


\begin{song}{Rise Up}{Em}
  {}
  {}
  {}
  {\NotCCLIed}

  \renewcommand{\RevDate}{February~11,~1993}
  %\SBRef{}{}

  \begin{SBOpGroup}
    Rise \Ch{Em}{up,} rise up, we are the \Ch{Am}{soldiers} of the \Ch{Em}{cross}
    
    We are the \Ch{Am}{ones} who are to glorify the \Ch{B7}{King}
    
    Cre\Ch{Em}{ation} groans for the \Ch{Am}{Sons} of God to \Ch{Em}{come}
    
    Mani\Ch{Am}{fest}ing all the glory of the \Ch{B7}{King}
    
    So \Ch{Am}{rise} and shine, for your \Ch{Em}{light} has come,
    
    And the \Ch{Am}{glor}y of the Lord is u\Ch{Em}{pon} you
    
    Lift \Ch{Am}{up} your eyes round a\Ch{Em}{bout} and see
    
    All the \Ch{Am}{nations} are falling at your \Ch{B7}{feet}
  \end{SBOpGroup}

  \begin{SBChorus}
    So let's \Ch{Em}{glor}ify the Lord,

    Let's \Ch{Am}{glor}ify the \Ch{Em}{King}

    Let's shout and sing as we \Ch{Am}{take} the victo\Ch{B7}{ry}

    So let's \Ch{Em}{glor}ify the Lord,

    Let's \Ch{Am}{glor}ify the \Ch{Em}{King}

    Let's \Ch{Am}{shout} and \Ch{B7}{take} the victo\Ch{Em}{ry!}
  \end{SBChorus}
\end{song}


\begin{song}{We Are Marching In Messiah's Band}{Am}
  {}
  {}
  {}
  {\NotCCLIed}

  \renewcommand{\RevDate}{February~11,~1993}
  %\SBRef{}{}

  \begin{SBVerse}
    We are \Ch{Am}{march}ing, in Messiah's band;

    The keys of \Ch{Am}{vict'ry} in His mighty hand;

    Let us \Ch{Dm}{go} on to take the promised \Ch{Am}{land.}

    Raise the \Ch{Am}{anthem,} sing the victory song;

    Praise the \Ch{Am}{Lord} for the battle's won;

    No \Ch{Dm}{wea}pon formed against Him shall \Ch{Am}{stand.}
  \end{SBVerse}

  \begin{SBOpGroup}
    For the \Ch{F}{Cap}tain of the Host is \Ch{C}{Je}sus;
    
    We are \Ch{F}{fol}lowing in His \Ch{C}{foot}steps.
    
    No \Ch{F}{foe} can stand a\Ch{Am}{gainst} us in the \Ch{F}{fray.} \Ch{G}{}
  \end{SBOpGroup}
\end{song}


\begin{song}{A Song Of Love}{G}
  {}
  {}
  {}
  {\NotCCLIed}

  \renewcommand{\RevDate}{February~11,~1993}
  %\SBRef{}{}

  \begin{SBOpGroup}
    A \Ch{G}{Song} of \Ch{Em}{Love,} I \Ch{Am7}{sing} to \Ch{D}{You}
    
    A \Ch{G}{Song} of \Ch{Em}{ado}ration, \Ch{Am7}{to} the \Ch{D}{King}
    
    You \Ch{Am7}{are} the \Ch{D}{Lord,} the \Ch{G}{Ma}ker \Ch{G/F#}{of} the \Ch{Em}{universe}
    
    You \Ch{Am7}{are} the \Ch{D}{King} of all \Ch{G}{Kings}
  \end{SBOpGroup}
\end{song}


\begin{song}{Hail To The King}{D}
  {1989 Christopher Rath}
  {Christopher Rath}
  {}
  {\NotCCLIed}

  \renewcommand{\RevDate}{November~22,~1993}
  \SBRef{New Wine Covenant Church Song Book}{\#70}
  \FLineIdx{I praise You, Lord, for You are my King}

  \begin{SBVerse}
    I \Ch{D}{praise} You, Lord, for \Ch{C}{You} are \Ch{G}{my} \Ch{D}{King}
    
    I \Ch{D}{praise} You, Lord, for \Ch{C}{You} are \Ch{G}{my} \Ch{D}{King}
  \end{SBVerse}

  \begin{SBOpGroup}
    All \Ch{C}{hail} to\Ch{G}{} the \Ch{D}{King,}
    
    All \Ch{C}{hail} to\Ch{G}{} King Je\Ch{D}{\SBen sus}
    
    All \Ch{C}{hail} to\Ch{G}{} the \Ch{D}{King} of Kings,
    
    All \Ch{C}{hail} to\Ch{G}{} the \Ch{D}{King}
  \end{SBOpGroup}

  \begin{SBVerse}
    We \Ch{D}{praise} You, Lord, For \Ch{C}{You} are \Ch{G}{our} \Ch{D}{King}

    We \Ch{D}{praise} You, Lord, For \Ch{C}{You} are \Ch{G}{our} \Ch{D}{King}
  \end{SBVerse}

  \begin{SBVerse}
    I \Ch{D}{wor}ship You, my \Ch{C}{Lord} and \Ch{G}{my} \Ch{D}{God}

    I \Ch{D}{wor}ship You, my \Ch{C}{Lord} and \Ch{G}{my} \Ch{D}{God}
  \end{SBVerse}
\end{song}


\WBPageBrk
\begin{song}{The Horse And Rider}{C}
  {1950 Mills Music (music only)}
  {Anonymous and I.~Miron \&~J.~Grossman}
  {Exodus~15:1--2}
  {\NotCCLIed}

  \renewcommand{\RevDate}{February~11,~1993}
  \SBRef{Songs Of Praise}{\#214}
  \FLineIdx{I will sing unto the Lord, for He has}

  \ifChordBk
    {\SBLyricNoteFont This song can be sung as a round.  Each part of the
      round is indicated below as a verse.}
  \fi

  \begin{SBVerse}
    \Ch{C}{I} will sing unto the Lord, for \Ch{F}{He} has triumphed \Ch{Dm}{glorious}ly
                
    The \Ch{G7}{horse} and rider thrown into the \Ch{C}{sea}

    \Ch{C}{I} will sing unto the Lord, for \Ch{F}{He} has triumphed \Ch{Dm}{glorious}ly

    The \Ch{G7}{horse} and rider thrown into the \Ch{C}{sea}
  \end{SBVerse}
        
  \begin{SBVerse}
    The \Ch{C}{Lord,} my God, my \Ch{F}{strength,} my \Ch{Dm}{song,}

    Is \Ch{G7}{now} become my victo\Ch{C}{ry}

    The \Ch{C}{Lord,} my God, my \Ch{F}{strength,} my \Ch{Dm}{song,}

    Is \Ch{G7}{now} become my victo\Ch{C}{ry}
  \end{SBVerse}
  
  \begin{SBVerse}
    The \Ch{C}{Lord} is God and \Ch{F}{I} will \Ch{Dm}{praise} Him,

    My \Ch{G7}{Father's} God, and \Ch{C}{I} will ex\Ch{G7}{alt} Him!

    The \Ch{C}{Lord} is God and \Ch{F}{I} will \Ch{Dm}{praise} Him,

    My \Ch{G7}{Father's} God, and \Ch{C}{I} will ex\Ch{C}{alt} Him!
  \end{SBVerse}
\end{song}


\begin{song}{Nothing But The Blood}{D}
  {\SBOrgMel}
  {Robert Lowry}
  {Ephesians~1:7}
  {\NotCCLIed}

  \renewcommand{\RevDate}{February~11,~1993}
  \SBRef{Hosanna! Music Book~I}{\#65}
  \FLineIdx{What can wash away my sin}

  \begin{SBVerse}
    \Ch{D}{What} can wash a\Ch{F#m}{way} my \Ch{Bm}{sin?}

    \Ch{G}{Noth}ing but the \Ch{D}{blood} of \Chr{A7}{Je}\Ch{D}{sus.}

    \Ch{D}{What} can make me \Ch{F#m}{whole} a\Ch{Bm}{gain?}

    \Ch{G}{Noth}ing but the \Ch{D}{blood} of \Chr{A7}{Je}\Ch{D}{sus.}
  \end{SBVerse}

  \begin{SBOpGroup}
    \Ch{D}{Oh,} Precious \Ch{F#m}{is} the \Ch{Bm}{flow}
  
    \Ch{G}{That} \Ch{Em}{makes} me \Ch{A}{white} as \Ch{D}{snow.}\Ch{A7}{}
    
    \Ch{D}{No} other \Ch{F#m}{fount} I \Ch{Bm}{know,}
    
    \Ch{G}{Noth}ing but the \Ch{D}{blood} of \Chr{A7}{Je}\Ch{D}{sus.}
  \end{SBOpGroup}

  \begin{SBVerse}
    \Ch{D}{This} is all my \Ch{F#m}{righteous}\Ch{Bm}{ness,}

    \Ch{G}{Noth}ing but the \Ch{D}{blood} of \Chr{A7}{Je}\Ch{D}{sus.}

    \Ch{D}{This} is all my \Ch{F#m}{hope} and \Ch{Bm}{peace,}

    \Ch{G}{Noth}ing but the \Ch{D}{blood} of \Chr{A7}{Je}\Ch{D}{sus.}
  \end{SBVerse}
\end{song}


\begin{song}{Far Above All Rule And Authority}{C}
  {}
  {}
  {Ephesians~1:20--21;~2:6}
  {\NotCCLIed}

  \renewcommand{\RevDate}{February~11,~1993}
  %\SBRef{}{}

  \begin{SBOpGroup}
    Far a\Ch{C}{bove} all rule and au\Ch{F}{thor}ity, and \Ch{C}{po}wer and do\Ch{G7}{min}\Ch{C}{ion}
    
    Far a\Ch{C}{bove} all rule and au\Ch{F}{thor}ity, and \Ch{C}{po}wer and do\Ch{G7}{min}\Ch{C}{ion}
    
    And every \Ch{F}{name} that is \Ch{G7}{nam}ed, not \Ch{C}{on}ly in this age
    
    But \Ch{F}{al}so in the \Ch{G7}{one} to \Ch{C}{come}
    
    And every \Ch{F}{name} that is \Ch{G7}{nam}ed, not \Ch{C}{on}ly in this age
    
    But \Ch{F}{al}so in the \Ch{G7}{one} to \Ch{C}{come}
    
    Christ is \Ch{C}{seat}\Ch{G}{ed} at the \Ch{C}{Fa}ther's right hand,
    
    In \Chr{F}{hea}\Chr{G7}{}venly \Ch{C}{pla}ces.
    
    We are \Ch{C}{seat}ed with \Ch{G}{Him,} at the \Ch{C}{Fa}ther's right hand,
    
    In \Chr{F}{hea}\Chr{G7}{}venly \Ch{C}{pla}ces.
  \end{SBOpGroup}
\end{song}


\begin{song}{Sing Hallelujah}{D}
  {}
  {}
  {}
  {\NotCCLIed}

  \renewcommand{\RevDate}{February~11,~1993}
  \SBRef{Scripture In Song Volume~II}{\#167}

  \begin{SBOpGroup}
    \Ch{D}{Sing} Halle\Ch{G}{lu}jah, \Ch{D}{sing} halle\Ch{A7}{lujah,}
    
    \Ch{D}{sing} halle\Ch{G}{lu}jah, there's \Ch{D}{joy} \Ch{A7}{in} the \Ch{D}{Lord}
  \end{SBOpGroup}

  \begin{SBVerse}
    There \Ch{D}{is} a king in \Ch{G}{Zion,}

    There \Ch{D}{is} a king in \Ch{A7}{Zion,}

    There \Ch{D}{is} a king in \Ch{G}{Zion},

    The \Ch{D}{Ci}ty \Ch{A7}{of} our \Ch{D}{God!}
  \end{SBVerse}

  \begin{SBVerse}
    There's \Ch{D}{oil} and wine in \Ch{G}{Zion,}

    There's \Ch{D}{oil} and wine in \Ch{A7}{Zion,}

    There's \Ch{D}{oil} and wine in \Ch{G}{Zion,}

    The \Ch{D}{Ci}ty \Ch{A7}{of} our \Ch{D}{God!}
  \end{SBVerse}

  \begin{SBVerse}
    \Ch{D}{Let's} go up to \Ch{G}{Zion,}

    \Ch{D}{Let's} go up to \Ch{A7}{Zion,}

    \Ch{D}{Let's} go up to \Ch{G}{Zion},

    The \Ch{D}{Ci}ty \Ch{A7}{of} our \Ch{D}{God!}
  \end{SBVerse}

  \begin{SBVerse}
    The \Ch{D}{church} is built on \Ch{G}{Zion,}

    The \Ch{D}{church} is built on \Ch{A7}{Zion}

    The \Ch{D}{church} is built on \Ch{G}{Zion,}

    The \Ch{D}{City} \Ch{A7}{of} our \Ch{D}{God!}
  \end{SBVerse}
\end{song}


\begin{song}{Zephaniah 3:17}{G}
  {}
  {}
  {}
  {\NotCCLIed}

  \renewcommand{\RevDate}{February~11,~1993}
  %\SBRef{}{}
  \FLineIdx{The Lord your God is in your midst}

  \begin{SBOpGroup}
    \Ch{G}{} The Lord your \Ch{D}{God} is \Ch{Em}{in} your midst,
    
    \Ch{G}{} The Lord of \Ch{D}{lords,} who sa\Ch{Em}{ves,}
    
    \Ch{G}{} He will ex\Ch{D}{ult} over \Ch{Em}{you} with joy,
    
    \Ch{C}{} He will renew you \Ch{D}{in} His love,
    
    \Ch{C}{} He will rejoice over \Ch{D}{you,}
    
    With shouts of \Ch{G}{joy,} \Ch{D}{} \Ch{Em}{} with shouts of \Ch{G}{joy,} \Ch{D}{} \Ch{Em}{}
    
    With shouts of \Ch{C}{joy,} \Ch{D}{} with shouts of \Ch{C}{joy,} \Ch{D}{}
    
    With shouts of \Ch{G}{joy!}
  \end{SBOpGroup}
\end{song}


\begin{song}{Come And Let Us Go}{D}
  {}
  {}
  {Micah 4:2}
  {\NotCCLIed}

  \renewcommand{\RevDate}{February~11,~1993}
  %\SBRef{}{}

  \begin{SBOpGroup}
    \Ch{D}{Come} and let us \Ch{Bm}{go} to the \Ch{Em}{moun}tain of the \Ch{A7}{Lord,}
    
    And \Ch{Em}{to} the \Ch{A7}{house} of our \Ch{D}{God.}
    
    \Ch{D}{Come} and let us \Ch{Bm}{go} to the \Ch{Em}{moun}tain of the \Ch{A7}{Lord,}
    
    And \Ch{Em}{to} the \Ch{A7}{house} of our \Ch{D}{God.}
    
    And \Ch{G}{He} will \Ch{A}{teach} us of His \Ch{D}{ways,} \Ch{D7}{}
    
    And \Ch{G}{we} will \Ch{A}{walk} in His \Ch{D}{paths.} \Ch{D7}{}
    
    And the \Ch{G}{law} shall go \Ch{A}{forth} from \Ch{D}{Zi}\Ch{Bm}{on,}
    
    And the \Ch{Em}{Word} of the \Ch{A}{Lord} from Jerusa\Ch{D}{lem.}
  \end{SBOpGroup}
\end{song}


\begin{song}{Behold, I Am The Lord}{D}
  {}
  {}
  {}
  {\NotCCLIed}

  \renewcommand{\RevDate}{February~11,~1993}
  \SBRef{Sounds Of Zion}{Volume~III}

  \begin{SBOpGroup}
    Be\Ch{D}{hold,} I am the \Ch{F#m}{Lord,} the \Ch{G}{God} of all \Ch{A}{flesh.}
    
    Is there \Ch{D}{any}thing,\Ch{D7}{} is there \Ch{G}{any}thing,\Ch{Em}{} \Ch{Gm}{too} \Ch{D}{hard} \Ch{A}{for} \Ch{D}{Me?}
    
    Is there \Ch{D}{any}thing, \Ch{G}{any}thing, \Ch{D}{any}thing, too hard \Ch{G}{for} \Ch{A}{Me?}
    
    \Ch{A7}{Is} there \Ch{D}{any}thing, \Ch{G}{any}thing, \Ch{D}{any}thing,\Ch{Bm7}{} \Ch{Gm}{too} \Ch{D/A}{hard} \Ch{A}{for} \Ch{D}{Me?}
  \end{SBOpGroup}
\end{song}


\begin{song}{Stand Up And Bless}{D}
  {}
  {}
  {Nehemiah 9:5}
  {\NotCCLIed}

  \renewcommand{\RevDate}{February~11,~1993}
  \SBRef{Hosanna! Music Book~IV}{\#332}

  \begin{SBOpGroup}
    Stand \Ch{D}{up} and bless the \Ch{Em}{Lord} your God
    
    From ever\Ch{A}{last}ing to ever\Ch{D}{last}ing
    
    Stand \Ch{D}{up} and bless the \Ch{Em}{Lord} your God
    
    From ever\Ch{A}{last}ing to ever\Ch{D}{last}ing
  \end{SBOpGroup}

\WBPageBrk
  \begin{SBChorus}
    And \Ch{G}{ble}ssed be your glorious \Ch{D}{name,} O Lord

    Which is e\Ch{Em}{xalted}\Ch{A}{} above all \Ch{D}{ble}ssing and \Ch{D7}{praise}

    And \Ch{G}{ble}ssed be your glorious \Ch{D}{name,} O Lord

    Which is e\Ch{Em}{xalted,}\Ch{A}{} which is e\Chr{G/D}{xal}\Ch{D}{ted}
  \end{SBChorus}
\end{song}


\begin{song}{Jesus Is The Lord}{G}
  {}
  {}
  {}
  {\NotCCLIed}

  \renewcommand{\RevDate}{February~11,~1993}
  %\SBRef{}{}

  \begin{SBOpGroup}
    \Ch{G}{Je}sus is the \Ch{G7}{Lord}
    
    \Ch{C}{Je}sus the Lord \Ch{Am}{reigns}
    
    \Ch{D}{We} shall take the \Ch{D7}{king}doms of this \Ch{G}{world} \Ch{C}{in} His \Ch{G}{name} \Ch{D7}{}
    
    \Ch{G}{Ev}ery tribe and \Ch{G7}{nation}
    
    \Ch{C}{Ev}ery situ\Ch{Am}{ation}
    
    \Ch{D}{Must} declare that \Ch{D7}{Jesus} is the \Ch{G}{Lord} \Ch{G7}{}
  \end{SBOpGroup}

  \begin{SBChorus}
    For the \Ch{C}{Lord} \Ch{G}{our} \Ch{D}{God} has de\Ch{G}{liver}ed Him from \Ch{Em}{death}

    And es\Ch{Am}{tablish}ed \Ch{D}{Je}sus as \Ch{G}{Lord} \Ch{G7}{}

    He has \Ch{C}{gi}ven \Ch{G}{Him} the \Ch{D}{po}wer over \Ch{G}{all} that He has \Ch{Em}{made}

    For our \Ch{Am}{God} has \Ch{D}{made} His Christ the \Ch{G}{Lord}
  \end{SBChorus}
\end{song}


\begin{song}{I Will Sing Praises}{G}
  {}
  {}
  {}
  {\NotCCLIed}

  \renewcommand{\RevDate}{February~11,~1993}
  %\SBRef{}{}

  \begin{SBOpGroup}
    \Ch{G}{} I will sing \Ch{C}{prais}es,
    
    \Ch{G}{} I will sing \Ch{C}{prais}es,
    
    \Ch{G}{} I will sing \Ch{C}{prais}es \Ch{D}{to} my \Ch{G}{God}
    
    \Ch{G}{} I will sing \Ch{C}{prais}es,
    
    \Ch{G}{} I will sing \Ch{C}{prais}es,
    
    \Ch{G}{} I will sing \Ch{C}{prais}es \Ch{D}{to} my \Ch{G}{God}
    
    \Ch{G}{} And when He \Ch{C}{comes} a\Ch{D}{gain,}
    
    \Ch{G}{} The rocks will not \Ch{C}{have} to \Ch{D}{cry} out
    
    \Ch{G}{} His people will \Ch{C}{shout} the mighty \Ch{D}{name}
    
    Of the King of \Ch{Em}{Glor}y
    
    \Ch{G}{} When the mountains \Ch{C}{sing} forth
    
    \Ch{G}{} And all the trees \Ch{C}{clap} their hands
    
    \Ch{G}{} His people will \Ch{C}{shout} the name
    
    Of the \Ch{D}{beau}ty of holiness:
    
    \Ch{D7}{Jesus} Christ is His \Ch{G}{name!}
  \end{SBOpGroup}
\end{song}


\begin{song}{Let God Be Magnified}{G}
  {}
  {}
  {}
  {\NotCCLIed}

  \renewcommand{\RevDate}{February~11,~1993}
  %\SBRef{}{}

  \begin{SBOpGroup}
    Let \Ch{G}{all} those that seek Thee rejoice and be glad
    
    In \Ch{Am}{Thee}, In \Ch{D7}{Thee}
    
    And \Ch{G}{let} such as love thy salvation, say
    
    Continual\Ch{Am}{ly,} continual\Ch{D7}{ly,}
    
    \Ch{C}{}``Let God be magnified, \Ch{G}{}let God be magnified,
        
    \Ch{Am}{Let} God be magni\Ch{D7}{fied.''}
    
    \Ch{C}{}``Let God be magnified, \Ch{G}{}let God be magnified,
    
    \Ch{Am}{Let} God be \Ch{D7}{mag}ni\Ch{G}{fied.''}\Ch{C}{} \Ch{G}{}
  \end{SBOpGroup}
\end{song}


\begin{song}{There's A Light Shining Forth}{D}
  {}
  {}
  {}
  {\NotCCLIed}

  \renewcommand{\RevDate}{February~11,~1993}
  %\SBRef{}{}

  \begin{SBOpGroup}
    There's a light shining \Ch{D}{forth,}
    
    I can see it on the ho\Ch{Em}{rizon;} \Ch{A7}{}
    
    It's the army of \Ch{Em}{God,} \Ch{A}{} preparing for \Ch{D}{war;}
    
    Coming conquering vic\Ch{F#m}{torious,} \Ch{D7}{}
    
    O'er the army of \Ch{G}{sa}tan; \Ch{Em}{}
    
    Nothing shall \Ch{D}{stand,}\Ch{A7}{} before the army of \Ch{D}{God.} \Ch{G}{} \Ch{D}{}
  \end{SBOpGroup}
\end{song}


\begin{song}{How Great Is Our God}{D}
  {\SBOrgMel}
  {}
  {}
  {\NotCCLIed}

  \renewcommand{\RevDate}{February~11,~1993}
  \SBRef{Worship Him~II}{\#65}

  \begin{SBOpGroup}
    \Ch{D}{How} great is our \Ch{A7}{God!}  How great is His \Ch{D}{name}\Ch{G}{} \Ch{D}{}
    
    How great is our \Ch{A7}{God!}  Forever the \Ch{D}{same}\Ch{G}{} \Ch{D}{}
    
    He rolled back the \Ch{D}{wa}te\Ch{D7}{rs,} of the mighty Red \Ch{G}{Sea}
    
    And He said, ``I'll never \Ch{D}{leave} you,\Ch{A7}{}
    
    Put your trust in \Ch{D}{Me!''} \Ch{[}{}\Ch{G}{} \Ch{D}{}\Ch{]}{}
  \end{SBOpGroup}

  \begin{SBExtraKeys}{
    \STitle{How Great Is Our God}{E}

    \begin{SBOpGroup}
      \Ch{E}{How} great is our \Ch{B7}{God!}  How great is His \Ch{E}{name}\Ch{A}{} \Ch{E}{}
      
      How great is our \Ch{B7}{God!}  Forever the \Ch{E}{same}\Ch{A}{} \Ch{E}{}
      
      He rolled back the \Ch{E}{wa}te\Ch{E7}{rs,} of the mighty Red \Ch{A}{Sea}
      
      And He said, ``I'll never \Ch{E}{leave} you,\Ch{B7}{}
      
      Put your trust in \Ch{E}{Me!''} \Ch{[}{}\Ch{A}{} \Ch{E}{}\Ch{]}{}
    \end{SBOpGroup}
  }\end{SBExtraKeys}
\end{song}


\begin{song}{Lift Jesus Higher}{E}
  {}
  {}
  {}
  {\NotCCLIed}

  \renewcommand{\RevDate}{February~11,~1993}
  %\SBRef{}{}

  \begin{SBOpGroup}
    Lift Jesus \Ch{E}{high}\Ch{A}{er.} \Ch{E}{} Lift Jesus \Ch{E}{high}\Ch{A}{er.} \Ch{E}{}
    
    Lift Him up for the world to \Ch{B7}{see.}
    
    He said, ``If \Ch{E}{I} \Ch{E7}{} be lifted \Ch{A}{up} from the earth,
    
    I will \Ch{E}{draw} all \Ch{B7}{men} unto \Ch{E}{me}.''
  \end{SBOpGroup}
\end{song}


\WBPageBrk
\begin{song}{Garment Of Praise}{C}
  {}
  {}
  {Isaiah~61:3}
  {\NotCCLIed}

  \renewcommand{\RevDate}{February~11,~1993}
  %\SBRef{}{}
  \FLineIdx{I have put on my garment of praise}

  \begin{SBOpGroup}
    I have \Ch{C}{put} on my garment of \Ch{G7}{praise,}
    
    I have \Ch{G7}{put} on my garment of \Ch{C}{prai}\Ch{C7}{se}
    
    And the \Ch{F}{spirit} of heaviness, is \Ch{C}{gone} from \Ch{Am}{me.}
    
    I have \Ch{C}{put} on my \Ch{G7}{gar}ment of \Ch{C}{praise!} \ChX{[}{}\ChX{F}{} \ChX{C}{}\ChX{]}{}
  \end{SBOpGroup}

  \begin{SBExtraKeys}{
    \STitle{Garment Of Praise}{D}

    \begin{SBOpGroup}
      I have \Ch{D}{put} on my garment of \Ch{A7}{praise,}
      
      I have \Ch{A7}{put} on my garment of \Ch{D}{prai}\Ch{D7}{se}
      
      And the \Ch{G}{spirit} of heaviness, is \Ch{D}{gone} from \Ch{Bm}{me.}
      
      I have \Ch{D}{put} on my \Ch{A7}{gar}ment of \Ch{D}{praise!} \Ch{[}{}\Ch{G}{} \Ch{D}{}\Ch{]}{}
    \end{SBOpGroup}
  }\end{SBExtraKeys}
\end{song}


\begin{song}{I Was Glad}{C}
  {}
  {}
  {Psalm 122:1}
  {\NotCCLIed}

  \renewcommand{\RevDate}{February~11,~1993}
  %\SBRef{}{}

  \begin{SBOpGroup}
    I was \Ch{C}{glad,} very glad, when they \Ch{F}{said} unto me
    
    Let us \Ch{G7}{go,} into the house of the \Ch{C}{Lord,} today
    
    There is singing, there is \Ch{C7}{danc}ing, there is \Ch{F}{vic}tory
    
    In the \Ch{G7}{house} of the Lord, to\Ch{C}{day!}
  \end{SBOpGroup}
\end{song}


\begin{song}{Oh, The Blood}{C}
  {\SBOrgMel}
  {}
  {Psalm 51:7}
  {\NotCCLIed}

  \renewcommand{\RevDate}{February~11,~1993}
  \SBRef{Hosanna! Music Book~I}{\#68}

  \begin{SBOpGroup}
    \Ch{C}{Oh} the blood \Ch{F}{of} \Ch{C}{Jesus.}
  
    \Ch{G7}{Oh} the blood of \Ch{C}{Jesus.}
    
    Oh the blood \Ch{F}{of} \Chr{C}{Je}\Ch{Am}{sus.}
    
    It \Ch{Dm}{wash}es \Ch{G7}{white} as \Ch{C}{snow.}
  \end{SBOpGroup}

  \begin{SBExtraKeys}{
    \STitle{Oh, The Blood}{D}

    \begin{SBOpGroup}
      \Ch{D}{Oh} the blood \Ch{G}{of} \Ch{D}{Jesus.}
  
      \Ch{A7}{Oh} the blood of \Ch{D}{Jesus.}
      
      Oh the blood \Ch{G}{of} \Chr{D}{Je}\Ch{Bm}{sus.}
      
      It \Ch{Em}{wash}es \Ch{A7}{white} as \Ch{D}{snow.}
    \end{SBOpGroup}

    %%%
    \STitle{Oh, The Blood}{E}

    \begin{SBOpGroup}
      \Ch{E}{Oh} the blood \Ch{A}{of} \Ch{E}{Jesus.}
      
      \Ch{B7}{Oh} the blood of \Ch{E}{Jesus.}
      
      Oh the blood \Ch{A}{of} \Chr{E}{Je}\Ch{C#m}{sus.}
      
      It \Ch{F#m}{wash}es \Ch{B7}{white} as \Ch{E}{snow.}
    \end{SBOpGroup}

\CBPageBrk
    %%%
    \STitle{Oh, The Blood}{F}

    \begin{SBOpGroup}
      \Ch{Eb}{Oh} the blood \Ch{Ab}{of} \Ch{Eb}{Jesus.}
  
      \Ch{Bb7}{Oh} the blood of \Ch{Eb}{Jesus.}
      
      Oh the blood \Ch{Ab}{of} \Chr{Eb}{Je}\Ch{Cm}{sus.}
      
      It \Ch{Fm}{wash}es \Ch{Bb7}{white} as \Ch{Eb}{snow.}
    \end{SBOpGroup}
  }\end{SBExtraKeys}
\end{song}


\begin{song}{You Alone}{C}
  {1992 Glory Alleluia Music}
  {Christopher Rath}
  {Revelation 4:8}
  {Used by Permission}

  \renewcommand{\RevDate}{February~11,~1993}
  \FLineIdx{Holy, Holy, You alone are Holy}

  \begin{SBVerse}
    \Ch{C}{Ho}\Ch{Em}{ly,} \Ch{F}{Ho}\Ch{C}{ly,} \Ch{F}{} You alone are \Ch{C}{Ho}\Ch{G}{ly.}

    \Ch{C}{Ho}\Ch{Em}{ly,} \Ch{F}{Ho}\Ch{C}{ly,} \Ch{F}{} You alone are \Chr{Am}{Ho}\Ch{G}{ly.}

    Al\Ch{F}{migh}ty God\Ch{C}{} I \Ch{E}{wor}ship You\Ch{Am}{}

    For \Ch{F}{You} a\Ch{G}{lone,} \Ch{F}{You} a\Ch{G}{lone} are \Chr{Dm}{Ho}\Chr{G}{}\Ch{C}{ly.}  \Ch{F}{}
  \end{SBVerse}

  \begin{SBVerse}
    \Ch{C}{Wor}\Ch{Em}{thy,} \Ch{F}{Wor}\Ch{C}{thy,} \Ch{F}{}You alone are \Ch{C}{Wor}\Ch{G}{thy.}

    \Ch{C}{Wor}\Ch{Em}{thy,} \Ch{F}{Wor}\Ch{C}{thy,} \Ch{F}{}You alone are \Ch{Am}{Wor}\Ch{G}{thy.}

    Who \Ch{F}{was,} who is,\Ch{C}{} and \Ch{E}{is} to come.\Ch{Am}{}

    \Ch{F}{You} a\Ch{G}{lone,} \Ch{F}{You} a\Ch{G}{lone} are \Chr{Dm}{Wor}\Chr{G}{}\Ch{C}{thy.} \Ch{[}{}\Ch{F}{}\Ch{]}{} \Ch{[{$^{Mod.}$}}{}\Ch{A7}{}\Ch{]}{}
  \end{SBVerse}

  \begin{SBExtraKeys}{%
    \STitle{You Alone}{D}

    \begin{SBVerse}
      \Ch{D}{Ho}\Ch{F#m}{ly,} \Ch{G}{Ho}\Ch{D}{ly,} \Ch{G}{} You alone are \Ch{D}{Ho}\Ch{A}{ly.}

      \Ch{D}{Ho}\Ch{F#m}{ly,} \Ch{G}{Ho}\Ch{D}{ly,} \Ch{G}{} You alone are \Chr{Bm}{Ho}\Ch{A}{ly.}

      Al\Ch{G}{migh}ty God\Ch{D}{} I \Ch{F#}{wor}ship You\Ch{Bm}{}

      For \Ch{G}{You} a\Ch{A}{lone,} \Ch{G}{You} a\Ch{A}{lone} are \Chr{Em}{Ho}\Chr{A}{}\Ch{D}{ly.}  \Ch{G}{}
    \end{SBVerse}
  }\end{SBExtraKeys}

\CBPageBrk
  \begin{xlatn}{Tu Es Saint}
    {}
    {Translation by Jocelyne Sarrazin}
    \renewcommand{\RevDate}{8~April,~1998}

    \begin{SBVerse}
      \Ch{C}{Tu} es \Ch{Em}{Saint,} \Ch{F}{Tu} es \Ch{C}{Saint,}\Ch{F}{} Tu es Saint O \Ch{C}{Sei}\Ch{G}{gneur.}

      \Ch{C}{Tu} es \Ch{Em}{Saint,} \Ch{F}{Tu} es \Ch{C}{Saint,}\Ch{F}{} Tu es Saint O \Chr{Am}{Sei}\Ch{G}{gneur.}

      Dieu \Ch{F}{Tout-}Puissant\Ch{C}{} soit \Ch{E}{glo}rifi\'e,\Ch{Am}{}

      Car \Ch{F}{Tu} es tr\'es \Ch{G}{Saint,} \Ch{F}{Tu} es tr\'es \Ch{G}{Saint,} O oui \Ch{Dm}{Saint}\Ch{G}{} O Sei\Ch{C}{gneur.}\Ch{F}{}
    \end{SBVerse}

    \begin{SBVerse}
      \Ch{C}{Tu} es \Ch{Em}{Digne,} \Ch{F}{Tu} es \Ch{C}{Digne,} Tu es Digne O \Ch{C}{Sei}\Ch{G}{gneur.}

      \Ch{C}{Tu} es \Ch{Em}{Digne,} \Ch{F}{Tu} es \Ch{C}{Digne,} Tu es Digne O \Chr{Am}{Sei}\Ch{G}{gneur.}

      Toi \Ch{F}{qui} \'etais,\Ch{C}{} qui \Ch{E}{es} et vien,\Ch{Am}{}

      \Ch{F}{\'{A}} Toi l'hon\Ch{G}{neur,} \Ch{F}{\`a} Toi la \Ch{G}{gloire,} \`a Toi \Ch{Dm}{seul}\Ch{G}{} O Sei\Ch{C}{gneur.}\Ch{F}{}
    \end{SBVerse}
  \end{xlatn}
\end{song}


\end{document}
\bye
%
%%%
% Document ends.
%%%
