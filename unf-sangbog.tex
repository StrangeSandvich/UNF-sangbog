%%%%%% rcsid = @(#)$Id: sample-sb.tex,v 1.23 2010-04-12 18:04:11 rathc Exp $
%%%%%%
%%
%%      ===============================
%%      Sample Songbook (sample-sb.tex)
%%      ===============================
%%
%%      Version 4.5, 30 April, 2010
%%
%%      Copyright 1992--2010 Christopher Rath <christopher@rath.ca>
%%
%%      This package is free software; you can redistribute it and/or
%%      modify it under the terms of version 2.1 of the GNU Lesser
%%	General Public License as published by the Free Software 
%%	Foundation.
%%
%%      This package is distributed in the hope that it will be
%%      useful, but WITHOUT ANY WARRANTY; without even the implied
%%      warranty of MERCHANTABILITY or FITNESS FOR A PARTICULAR
%%      PURPOSE.  See the GNU Lesser General Public License for more
%%      details.
%%
%%      This file contains a subset of the songbook we distribute
%%      at our church.  To the best of my knowledge, all of the lyrics
%%      contained herein are freely distributable.  This file has been
%%      provided as a sample of what can be produced by the chordbk,
%%      wordbk, and overhead LaTeX styles.
%%
%%      NEEDED:  The fancyhdr LaTeX style is required to properly
%%              format this file.  If you don't have that then comment
%%              out the commands in the preamble which deal with the
%%              fancyhdr style.
%%
%%%%%%
%%%%%%
%%
%%      1. Chord notation.  Within this songbook the following
%%         conventions have been adopted:
%%
%%              "Minor" is entered as "m";
%%                      e.g. Cm7 for C minor 7th.
%%              "Major" is entered as "M";
%%                      e.g. CM7 for C major 7th.
%%
%%%%%%
%%%%%%
%%      ============
%%      Bibliography
%%      ============
%%
%%      Exalt Him!: Exalt Him!  Compiled by Tom Fettke.  (c)1989
%%                      Word Music.
%%
%%      Hosanna! Music Books: Hosanna! Music Books #1--#6.
%%                      (c)1987--92 Integrity Music, Inc.
%%
%%      Worship Him II: Worship Him II.  Compiled by Jesse Peterson
%%                      and Bruce Ballinger.  (c)1989 Tempo Music
%%                      Publications.
%%
%%      Worship Songs Of The Vineyard: Worship Songs Of The Vineyard
%%                      --- Volume 2.  (c)1989 Vineyard Ministries
%%                      International.
%%
%%%%%%
%%%%%%

%%%%%%%%%%%%%%%%%%%%%%%%%%%%%%%%%%%%%%%%%%%%%%%%%%%%%%%%%%
%%%%%%%%%%%%%%%%%%%%%%%%%%%%%%%%%%%%%%%%%%%%%%%%%%%%%%%%%%
%%                                                      %%
%%           P R E A M B L E   B E G I N S              %%
%%                                                      %%
%%%%%%%%%%%%%%%%%%%%%%%%%%%%%%%%%%%%%%%%%%%%%%%%%%%%%%%%%%
%%%%%%%%%%%%%%%%%%%%%%%%%%%%%%%%%%%%%%%%%%%%%%%%%%%%%%%%%%

\documentclass[a5paper]{book}
\usepackage{latexsym,
            fancyhdr,
            titlesec,
            amsmath,
            amssymb,
            multicol,
            amsthm,
            stmaryrd,
            amsthm,
            color,
            needspace,
            stackengine,
            wasysym}
\usepackage[utf8]{inputenc}
\usepackage[T1]{fontenc}
% \usepackage[chordbk]{songbook}                  %% Words & Chords edition.
%%\usepackage[compactallsongs,chordbk]{songbook}    %% Words & Chords edition.
\usepackage[wordbk]{songbook}                 %% Words Only edition.
%%\usepackage[overhead]{songbook}               %% Overhead Transparency edition.
\usepackage{titletoc}
\usepackage{tket}  % Draws "TÅGEKAMMERET" correctly

%%%
% Revision Date and Release Date definitions.
%
%       \RelDate - The last time this songbook was released.  Set this
%                  date each time a new release/update of the songbook
%                  is generated.
%       \RevDate - The last time a particular song was revised in any
%                  way.  This command will be renewed inside every
%                  song.
%%%
\newcommand{\RelDate}{31~August,~2003}
\newcommand{\RevDate}{\today}

%%%
% C.C.L.I. license number definition; for copyright licensing info.
% One of these macros will be manually inserted into the {SBMel}
% parameter of the {song} environment.
%
%       \CCLInumber - The actual copyright license number.  Don't
%               insert this command in the {SBMel} parameter, use one
%               of the others.
%       \CCLIed - Indicates a song falls under our CCLI license.
%       \NotCCLIed - Indicates a song doesn't fall under our CCLI
%               license.  Public Domain songs fall into this category.
%       \PGranted - We have received specific permission from the
%               copyright holder to use this song.
%       \PPending - We are in the process of obtaining permission to
%               use this song.
%%%
\newcommand{\CCLInumber}{Your CCLI Number}
\newcommand{\CCLIed}{{\SBMelInfoFont (CCLI \CCLInumber)}}
\newcommand{\NotCCLIed}{\relax}
\newcommand{\PGranted}{\relax}
\newcommand{\PPending}{{\SBMelInfoFont (Permission Pending)}}

%%%
% Title page information.
%%%
%\title{UNF Computer Science Camp 2019 Sangbog}
%\author{}
%\date{Revideret:  \RevDate}

%%%
% Redefine fonts from SongBook style that I don't like.
%%%
\font\myTinySF=cmss8 at 8pt
\renewcommand{\SBMelInfoFont}{\tiny\myTinySF}

%%%
% Define fonts to use in the headers and footers of the songbook.
%%%
\newcommand{\LHeadFont}{\normalsize}            % = cmr12  at 12pt
\newcommand{\CHeadFont}{\normalsize\rm}         % = cmr12  at 12pt
\newcommand{\RHeadFont}{\normalsize}            % = cmr12  at 12pt
\newcommand{\LFootFont}{\scriptsize}            % = cmr8   at  8pt
\newcommand{\CFootFont}{\tiny\myTinySF}         % = cmss8  at  8pt
\newcommand{\RFootFont}{\scriptsize}            % = cmr8   at  8pt

\def\repeat{%
  \stackanchor{.}{.}%
  \rule[-\dp\strutbox]{.3pt}{\normalbaselineskip}%
  \kern0.5pt%
  \rule[-\dp\strutbox]{1pt}{\normalbaselineskip}%
  \kern1pt%
}
\def\frepeat{%
  \kern1pt%
  \rule[-\dp\strutbox]{1pt}{\normalbaselineskip}%
  \kern0.5pt%
  \rule[-\dp\strutbox]{.3pt}{\normalbaselineskip}%
  \stackanchor{.}{.}%
}
% \newcommand{\SBRepeat}[1]{#1\\#1}
\newcommand{\SBRepeat}[1]{\frepeat #1\repeat}
\setcounter{SBSongCnt}{-1}
\renewcommand{\SBWAndMTag}{Forfatter:}
\renewcommand{\SBUnknownTag}{Ukendt}
\renewcommand{\SBChorusTag}{Ref.}
\renewcommand{\SBOrgMel}{Originalmelodi}
\renewcommand{\SpaceAfterChorus}   {\vspace{0ex plus1ex minus 0.5ex}}
\renewcommand{\SpaceAfterOpGroup}  {\vspace{0ex plus1ex minus 0.5ex}}
\renewcommand{\SpaceAfterSBBracket}{\vspace{0ex plus1ex minus 0.5ex}}
\renewcommand{\SpaceAfterSection}  {\vspace{0ex plus1ex minus 0.5ex}}
\renewcommand{\SpaceAfterSong}     {\vspace{0ex plus1ex minus 0.5ex}}
\renewcommand{\SpaceAfterVerse}    {\vspace{0ex plus1ex minus 0.5ex}}

% Tell LaTeX that \medskip is a good place to make a page break
\let\oldmedskip\medskip
\renewcommand{\medskip}{\oldmedskip\pagebreak[2]}

%%%
% Turn on/off index-file generation.  Uncomment the \makeindex line to
% turn index generation on;  comment it out to turn index generation
% off.
%%%
%\makeTitleIndex         %% Title and First Line Index.
%\makeTitleContents      %% Table of Contents.
%\makeKeyIndex           %% Index of song by key.
% \makeArtistIndex	%% Index of song by artist.
% \newcommand{\SBThechapter}[0]{}
% \newcommand{\SBChapter}[1]{
%     \startcontents
%     \chapter*{#1} 
%     % %%%%%% rcsid = @(#)$Id: sample-sb.tex,v 1.23 2010-04-12 18:04:11 rathc Exp $
%%%%%%
%%
%%      ===============================
%%      Sample Songbook (sample-sb.tex)
%%      ===============================
%%
%%      Version 4.5, 30 April, 2010
%%
%%      Copyright 1992--2010 Christopher Rath <christopher@rath.ca>
%%
%%      This package is free software; you can redistribute it and/or
%%      modify it under the terms of version 2.1 of the GNU Lesser
%%	General Public License as published by the Free Software 
%%	Foundation.
%%
%%      This package is distributed in the hope that it will be
%%      useful, but WITHOUT ANY WARRANTY; without even the implied
%%      warranty of MERCHANTABILITY or FITNESS FOR A PARTICULAR
%%      PURPOSE.  See the GNU Lesser General Public License for more
%%      details.
%%
%%      This file contains a subset of the songbook we distribute
%%      at our church.  To the best of my knowledge, all of the lyrics
%%      contained herein are freely distributable.  This file has been
%%      provided as a sample of what can be produced by the chordbk,
%%      wordbk, and overhead LaTeX styles.
%%
%%      NEEDED:  The fancyhdr LaTeX style is required to properly
%%              format this file.  If you don't have that then comment
%%              out the commands in the preamble which deal with the
%%              fancyhdr style.
%%
%%%%%%
%%%%%%
%%
%%      1. Chord notation.  Within this songbook the following
%%         conventions have been adopted:
%%
%%              "Minor" is entered as "m";
%%                      e.g. Cm7 for C minor 7th.
%%              "Major" is entered as "M";
%%                      e.g. CM7 for C major 7th.
%%
%%%%%%
%%%%%%
%%      ============
%%      Bibliography
%%      ============
%%
%%      Exalt Him!: Exalt Him!  Compiled by Tom Fettke.  (c)1989
%%                      Word Music.
%%
%%      Hosanna! Music Books: Hosanna! Music Books #1--#6.
%%                      (c)1987--92 Integrity Music, Inc.
%%
%%      Worship Him II: Worship Him II.  Compiled by Jesse Peterson
%%                      and Bruce Ballinger.  (c)1989 Tempo Music
%%                      Publications.
%%
%%      Worship Songs Of The Vineyard: Worship Songs Of The Vineyard
%%                      --- Volume 2.  (c)1989 Vineyard Ministries
%%                      International.
%%
%%%%%%
%%%%%%

%%%%%%%%%%%%%%%%%%%%%%%%%%%%%%%%%%%%%%%%%%%%%%%%%%%%%%%%%%
%%%%%%%%%%%%%%%%%%%%%%%%%%%%%%%%%%%%%%%%%%%%%%%%%%%%%%%%%%
%%                                                      %%
%%           P R E A M B L E   B E G I N S              %%
%%                                                      %%
%%%%%%%%%%%%%%%%%%%%%%%%%%%%%%%%%%%%%%%%%%%%%%%%%%%%%%%%%%
%%%%%%%%%%%%%%%%%%%%%%%%%%%%%%%%%%%%%%%%%%%%%%%%%%%%%%%%%%

\documentclass[a5paper]{book}
\usepackage{latexsym,
            fancyhdr,
            titlesec,
            amsmath,
            amssymb,
            multicol,
            amsthm,
            stmaryrd,
            amsthm,
            color,
            needspace,
            stackengine,
            wasysym}
\usepackage[utf8]{inputenc}
\usepackage[T1]{fontenc}
% \usepackage[chordbk]{songbook}                  %% Words & Chords edition.
%%\usepackage[compactallsongs,chordbk]{songbook}    %% Words & Chords edition.
\usepackage[wordbk]{songbook}                 %% Words Only edition.
%%\usepackage[overhead]{songbook}               %% Overhead Transparency edition.
\usepackage{titletoc}
\usepackage{tket}  % Draws "TÅGEKAMMERET" correctly

%%%
% Revision Date and Release Date definitions.
%
%       \RelDate - The last time this songbook was released.  Set this
%                  date each time a new release/update of the songbook
%                  is generated.
%       \RevDate - The last time a particular song was revised in any
%                  way.  This command will be renewed inside every
%                  song.
%%%
\newcommand{\RelDate}{31~August,~2003}
\newcommand{\RevDate}{\today}

%%%
% C.C.L.I. license number definition; for copyright licensing info.
% One of these macros will be manually inserted into the {SBMel}
% parameter of the {song} environment.
%
%       \CCLInumber - The actual copyright license number.  Don't
%               insert this command in the {SBMel} parameter, use one
%               of the others.
%       \CCLIed - Indicates a song falls under our CCLI license.
%       \NotCCLIed - Indicates a song doesn't fall under our CCLI
%               license.  Public Domain songs fall into this category.
%       \PGranted - We have received specific permission from the
%               copyright holder to use this song.
%       \PPending - We are in the process of obtaining permission to
%               use this song.
%%%
\newcommand{\CCLInumber}{Your CCLI Number}
\newcommand{\CCLIed}{{\SBMelInfoFont (CCLI \CCLInumber)}}
\newcommand{\NotCCLIed}{\relax}
\newcommand{\PGranted}{\relax}
\newcommand{\PPending}{{\SBMelInfoFont (Permission Pending)}}

%%%
% Title page information.
%%%
%\title{UNF Computer Science Camp 2019 Sangbog}
%\author{}
%\date{Revideret:  \RevDate}

%%%
% Redefine fonts from SongBook style that I don't like.
%%%
\font\myTinySF=cmss8 at 8pt
\renewcommand{\SBMelInfoFont}{\tiny\myTinySF}

%%%
% Define fonts to use in the headers and footers of the songbook.
%%%
\newcommand{\LHeadFont}{\normalsize}            % = cmr12  at 12pt
\newcommand{\CHeadFont}{\normalsize\rm}         % = cmr12  at 12pt
\newcommand{\RHeadFont}{\normalsize}            % = cmr12  at 12pt
\newcommand{\LFootFont}{\scriptsize}            % = cmr8   at  8pt
\newcommand{\CFootFont}{\tiny\myTinySF}         % = cmss8  at  8pt
\newcommand{\RFootFont}{\scriptsize}            % = cmr8   at  8pt

\def\repeat{%
  \stackanchor{.}{.}%
  \rule[-\dp\strutbox]{.3pt}{\normalbaselineskip}%
  \kern0.5pt%
  \rule[-\dp\strutbox]{1pt}{\normalbaselineskip}%
  \kern1pt%
}
\def\frepeat{%
  \kern1pt%
  \rule[-\dp\strutbox]{1pt}{\normalbaselineskip}%
  \kern0.5pt%
  \rule[-\dp\strutbox]{.3pt}{\normalbaselineskip}%
  \stackanchor{.}{.}%
}
% \newcommand{\SBRepeat}[1]{#1\\#1}
\newcommand{\SBRepeat}[1]{\frepeat #1\repeat}
\setcounter{SBSongCnt}{-1}
\renewcommand{\SBWAndMTag}{Forfatter:}
\renewcommand{\SBUnknownTag}{Ukendt}
\renewcommand{\SBChorusTag}{Ref.}
\renewcommand{\SBOrgMel}{Originalmelodi}
\renewcommand{\SpaceAfterChorus}   {\vspace{0ex plus1ex minus 0.5ex}}
\renewcommand{\SpaceAfterOpGroup}  {\vspace{0ex plus1ex minus 0.5ex}}
\renewcommand{\SpaceAfterSBBracket}{\vspace{0ex plus1ex minus 0.5ex}}
\renewcommand{\SpaceAfterSection}  {\vspace{0ex plus1ex minus 0.5ex}}
\renewcommand{\SpaceAfterSong}     {\vspace{0ex plus1ex minus 0.5ex}}
\renewcommand{\SpaceAfterVerse}    {\vspace{0ex plus1ex minus 0.5ex}}

% Tell LaTeX that \medskip is a good place to make a page break
\let\oldmedskip\medskip
\renewcommand{\medskip}{\oldmedskip\pagebreak[2]}

%%%
% Turn on/off index-file generation.  Uncomment the \makeindex line to
% turn index generation on;  comment it out to turn index generation
% off.
%%%
%\makeTitleIndex         %% Title and First Line Index.
%\makeTitleContents      %% Table of Contents.
%\makeKeyIndex           %% Index of song by key.
% \makeArtistIndex	%% Index of song by artist.
% \newcommand{\SBThechapter}[0]{}
% \newcommand{\SBChapter}[1]{
%     \startcontents
%     \chapter*{#1} 
%     % %%%%%% rcsid = @(#)$Id: sample-sb.tex,v 1.23 2010-04-12 18:04:11 rathc Exp $
%%%%%%
%%
%%      ===============================
%%      Sample Songbook (sample-sb.tex)
%%      ===============================
%%
%%      Version 4.5, 30 April, 2010
%%
%%      Copyright 1992--2010 Christopher Rath <christopher@rath.ca>
%%
%%      This package is free software; you can redistribute it and/or
%%      modify it under the terms of version 2.1 of the GNU Lesser
%%	General Public License as published by the Free Software 
%%	Foundation.
%%
%%      This package is distributed in the hope that it will be
%%      useful, but WITHOUT ANY WARRANTY; without even the implied
%%      warranty of MERCHANTABILITY or FITNESS FOR A PARTICULAR
%%      PURPOSE.  See the GNU Lesser General Public License for more
%%      details.
%%
%%      This file contains a subset of the songbook we distribute
%%      at our church.  To the best of my knowledge, all of the lyrics
%%      contained herein are freely distributable.  This file has been
%%      provided as a sample of what can be produced by the chordbk,
%%      wordbk, and overhead LaTeX styles.
%%
%%      NEEDED:  The fancyhdr LaTeX style is required to properly
%%              format this file.  If you don't have that then comment
%%              out the commands in the preamble which deal with the
%%              fancyhdr style.
%%
%%%%%%
%%%%%%
%%
%%      1. Chord notation.  Within this songbook the following
%%         conventions have been adopted:
%%
%%              "Minor" is entered as "m";
%%                      e.g. Cm7 for C minor 7th.
%%              "Major" is entered as "M";
%%                      e.g. CM7 for C major 7th.
%%
%%%%%%
%%%%%%
%%      ============
%%      Bibliography
%%      ============
%%
%%      Exalt Him!: Exalt Him!  Compiled by Tom Fettke.  (c)1989
%%                      Word Music.
%%
%%      Hosanna! Music Books: Hosanna! Music Books #1--#6.
%%                      (c)1987--92 Integrity Music, Inc.
%%
%%      Worship Him II: Worship Him II.  Compiled by Jesse Peterson
%%                      and Bruce Ballinger.  (c)1989 Tempo Music
%%                      Publications.
%%
%%      Worship Songs Of The Vineyard: Worship Songs Of The Vineyard
%%                      --- Volume 2.  (c)1989 Vineyard Ministries
%%                      International.
%%
%%%%%%
%%%%%%

%%%%%%%%%%%%%%%%%%%%%%%%%%%%%%%%%%%%%%%%%%%%%%%%%%%%%%%%%%
%%%%%%%%%%%%%%%%%%%%%%%%%%%%%%%%%%%%%%%%%%%%%%%%%%%%%%%%%%
%%                                                      %%
%%           P R E A M B L E   B E G I N S              %%
%%                                                      %%
%%%%%%%%%%%%%%%%%%%%%%%%%%%%%%%%%%%%%%%%%%%%%%%%%%%%%%%%%%
%%%%%%%%%%%%%%%%%%%%%%%%%%%%%%%%%%%%%%%%%%%%%%%%%%%%%%%%%%

\documentclass[a5paper]{book}
\usepackage{latexsym,
            fancyhdr,
            titlesec,
            amsmath,
            amssymb,
            multicol,
            amsthm,
            stmaryrd,
            amsthm,
            color,
            needspace,
            stackengine,
            wasysym}
\usepackage[utf8]{inputenc}
\usepackage[T1]{fontenc}
% \usepackage[chordbk]{songbook}                  %% Words & Chords edition.
%%\usepackage[compactallsongs,chordbk]{songbook}    %% Words & Chords edition.
\usepackage[wordbk]{songbook}                 %% Words Only edition.
%%\usepackage[overhead]{songbook}               %% Overhead Transparency edition.
\usepackage{titletoc}
\usepackage{tket}  % Draws "TÅGEKAMMERET" correctly

%%%
% Revision Date and Release Date definitions.
%
%       \RelDate - The last time this songbook was released.  Set this
%                  date each time a new release/update of the songbook
%                  is generated.
%       \RevDate - The last time a particular song was revised in any
%                  way.  This command will be renewed inside every
%                  song.
%%%
\newcommand{\RelDate}{31~August,~2003}
\newcommand{\RevDate}{\today}

%%%
% C.C.L.I. license number definition; for copyright licensing info.
% One of these macros will be manually inserted into the {SBMel}
% parameter of the {song} environment.
%
%       \CCLInumber - The actual copyright license number.  Don't
%               insert this command in the {SBMel} parameter, use one
%               of the others.
%       \CCLIed - Indicates a song falls under our CCLI license.
%       \NotCCLIed - Indicates a song doesn't fall under our CCLI
%               license.  Public Domain songs fall into this category.
%       \PGranted - We have received specific permission from the
%               copyright holder to use this song.
%       \PPending - We are in the process of obtaining permission to
%               use this song.
%%%
\newcommand{\CCLInumber}{Your CCLI Number}
\newcommand{\CCLIed}{{\SBMelInfoFont (CCLI \CCLInumber)}}
\newcommand{\NotCCLIed}{\relax}
\newcommand{\PGranted}{\relax}
\newcommand{\PPending}{{\SBMelInfoFont (Permission Pending)}}

%%%
% Title page information.
%%%
%\title{UNF Computer Science Camp 2019 Sangbog}
%\author{}
%\date{Revideret:  \RevDate}

%%%
% Redefine fonts from SongBook style that I don't like.
%%%
\font\myTinySF=cmss8 at 8pt
\renewcommand{\SBMelInfoFont}{\tiny\myTinySF}

%%%
% Define fonts to use in the headers and footers of the songbook.
%%%
\newcommand{\LHeadFont}{\normalsize}            % = cmr12  at 12pt
\newcommand{\CHeadFont}{\normalsize\rm}         % = cmr12  at 12pt
\newcommand{\RHeadFont}{\normalsize}            % = cmr12  at 12pt
\newcommand{\LFootFont}{\scriptsize}            % = cmr8   at  8pt
\newcommand{\CFootFont}{\tiny\myTinySF}         % = cmss8  at  8pt
\newcommand{\RFootFont}{\scriptsize}            % = cmr8   at  8pt

\def\repeat{%
  \stackanchor{.}{.}%
  \rule[-\dp\strutbox]{.3pt}{\normalbaselineskip}%
  \kern0.5pt%
  \rule[-\dp\strutbox]{1pt}{\normalbaselineskip}%
  \kern1pt%
}
\def\frepeat{%
  \kern1pt%
  \rule[-\dp\strutbox]{1pt}{\normalbaselineskip}%
  \kern0.5pt%
  \rule[-\dp\strutbox]{.3pt}{\normalbaselineskip}%
  \stackanchor{.}{.}%
}
% \newcommand{\SBRepeat}[1]{#1\\#1}
\newcommand{\SBRepeat}[1]{\frepeat #1\repeat}
\setcounter{SBSongCnt}{-1}
\renewcommand{\SBWAndMTag}{Forfatter:}
\renewcommand{\SBUnknownTag}{Ukendt}
\renewcommand{\SBChorusTag}{Ref.}
\renewcommand{\SBOrgMel}{Originalmelodi}
\renewcommand{\SpaceAfterChorus}   {\vspace{0ex plus1ex minus 0.5ex}}
\renewcommand{\SpaceAfterOpGroup}  {\vspace{0ex plus1ex minus 0.5ex}}
\renewcommand{\SpaceAfterSBBracket}{\vspace{0ex plus1ex minus 0.5ex}}
\renewcommand{\SpaceAfterSection}  {\vspace{0ex plus1ex minus 0.5ex}}
\renewcommand{\SpaceAfterSong}     {\vspace{0ex plus1ex minus 0.5ex}}
\renewcommand{\SpaceAfterVerse}    {\vspace{0ex plus1ex minus 0.5ex}}

% Tell LaTeX that \medskip is a good place to make a page break
\let\oldmedskip\medskip
\renewcommand{\medskip}{\oldmedskip\pagebreak[2]}

%%%
% Turn on/off index-file generation.  Uncomment the \makeindex line to
% turn index generation on;  comment it out to turn index generation
% off.
%%%
%\makeTitleIndex         %% Title and First Line Index.
%\makeTitleContents      %% Table of Contents.
%\makeKeyIndex           %% Index of song by key.
% \makeArtistIndex	%% Index of song by artist.
% \newcommand{\SBThechapter}[0]{}
% \newcommand{\SBChapter}[1]{
%     \startcontents
%     \chapter*{#1} 
%     % %%%%%% rcsid = @(#)$Id: sample-sb.tex,v 1.23 2010-04-12 18:04:11 rathc Exp $
%%%%%%
%%
%%      ===============================
%%      Sample Songbook (sample-sb.tex)
%%      ===============================
%%
%%      Version 4.5, 30 April, 2010
%%
%%      Copyright 1992--2010 Christopher Rath <christopher@rath.ca>
%%
%%      This package is free software; you can redistribute it and/or
%%      modify it under the terms of version 2.1 of the GNU Lesser
%%	General Public License as published by the Free Software 
%%	Foundation.
%%
%%      This package is distributed in the hope that it will be
%%      useful, but WITHOUT ANY WARRANTY; without even the implied
%%      warranty of MERCHANTABILITY or FITNESS FOR A PARTICULAR
%%      PURPOSE.  See the GNU Lesser General Public License for more
%%      details.
%%
%%      This file contains a subset of the songbook we distribute
%%      at our church.  To the best of my knowledge, all of the lyrics
%%      contained herein are freely distributable.  This file has been
%%      provided as a sample of what can be produced by the chordbk,
%%      wordbk, and overhead LaTeX styles.
%%
%%      NEEDED:  The fancyhdr LaTeX style is required to properly
%%              format this file.  If you don't have that then comment
%%              out the commands in the preamble which deal with the
%%              fancyhdr style.
%%
%%%%%%
%%%%%%
%%
%%      1. Chord notation.  Within this songbook the following
%%         conventions have been adopted:
%%
%%              "Minor" is entered as "m";
%%                      e.g. Cm7 for C minor 7th.
%%              "Major" is entered as "M";
%%                      e.g. CM7 for C major 7th.
%%
%%%%%%
%%%%%%
%%      ============
%%      Bibliography
%%      ============
%%
%%      Exalt Him!: Exalt Him!  Compiled by Tom Fettke.  (c)1989
%%                      Word Music.
%%
%%      Hosanna! Music Books: Hosanna! Music Books #1--#6.
%%                      (c)1987--92 Integrity Music, Inc.
%%
%%      Worship Him II: Worship Him II.  Compiled by Jesse Peterson
%%                      and Bruce Ballinger.  (c)1989 Tempo Music
%%                      Publications.
%%
%%      Worship Songs Of The Vineyard: Worship Songs Of The Vineyard
%%                      --- Volume 2.  (c)1989 Vineyard Ministries
%%                      International.
%%
%%%%%%
%%%%%%

%%%%%%%%%%%%%%%%%%%%%%%%%%%%%%%%%%%%%%%%%%%%%%%%%%%%%%%%%%
%%%%%%%%%%%%%%%%%%%%%%%%%%%%%%%%%%%%%%%%%%%%%%%%%%%%%%%%%%
%%                                                      %%
%%           P R E A M B L E   B E G I N S              %%
%%                                                      %%
%%%%%%%%%%%%%%%%%%%%%%%%%%%%%%%%%%%%%%%%%%%%%%%%%%%%%%%%%%
%%%%%%%%%%%%%%%%%%%%%%%%%%%%%%%%%%%%%%%%%%%%%%%%%%%%%%%%%%

\documentclass[a5paper]{book}
\usepackage{latexsym,
            fancyhdr,
            titlesec,
            amsmath,
            amssymb,
            multicol,
            amsthm,
            stmaryrd,
            amsthm,
            color,
            needspace,
            stackengine,
            wasysym}
\usepackage[utf8]{inputenc}
\usepackage[T1]{fontenc}
% \usepackage[chordbk]{songbook}                  %% Words & Chords edition.
%%\usepackage[compactallsongs,chordbk]{songbook}    %% Words & Chords edition.
\usepackage[wordbk]{songbook}                 %% Words Only edition.
%%\usepackage[overhead]{songbook}               %% Overhead Transparency edition.
\usepackage{titletoc}
\usepackage{tket}  % Draws "TÅGEKAMMERET" correctly

%%%
% Revision Date and Release Date definitions.
%
%       \RelDate - The last time this songbook was released.  Set this
%                  date each time a new release/update of the songbook
%                  is generated.
%       \RevDate - The last time a particular song was revised in any
%                  way.  This command will be renewed inside every
%                  song.
%%%
\newcommand{\RelDate}{31~August,~2003}
\newcommand{\RevDate}{\today}

%%%
% C.C.L.I. license number definition; for copyright licensing info.
% One of these macros will be manually inserted into the {SBMel}
% parameter of the {song} environment.
%
%       \CCLInumber - The actual copyright license number.  Don't
%               insert this command in the {SBMel} parameter, use one
%               of the others.
%       \CCLIed - Indicates a song falls under our CCLI license.
%       \NotCCLIed - Indicates a song doesn't fall under our CCLI
%               license.  Public Domain songs fall into this category.
%       \PGranted - We have received specific permission from the
%               copyright holder to use this song.
%       \PPending - We are in the process of obtaining permission to
%               use this song.
%%%
\newcommand{\CCLInumber}{Your CCLI Number}
\newcommand{\CCLIed}{{\SBMelInfoFont (CCLI \CCLInumber)}}
\newcommand{\NotCCLIed}{\relax}
\newcommand{\PGranted}{\relax}
\newcommand{\PPending}{{\SBMelInfoFont (Permission Pending)}}

%%%
% Title page information.
%%%
%\title{UNF Computer Science Camp 2019 Sangbog}
%\author{}
%\date{Revideret:  \RevDate}

%%%
% Redefine fonts from SongBook style that I don't like.
%%%
\font\myTinySF=cmss8 at 8pt
\renewcommand{\SBMelInfoFont}{\tiny\myTinySF}

%%%
% Define fonts to use in the headers and footers of the songbook.
%%%
\newcommand{\LHeadFont}{\normalsize}            % = cmr12  at 12pt
\newcommand{\CHeadFont}{\normalsize\rm}         % = cmr12  at 12pt
\newcommand{\RHeadFont}{\normalsize}            % = cmr12  at 12pt
\newcommand{\LFootFont}{\scriptsize}            % = cmr8   at  8pt
\newcommand{\CFootFont}{\tiny\myTinySF}         % = cmss8  at  8pt
\newcommand{\RFootFont}{\scriptsize}            % = cmr8   at  8pt

\def\repeat{%
  \stackanchor{.}{.}%
  \rule[-\dp\strutbox]{.3pt}{\normalbaselineskip}%
  \kern0.5pt%
  \rule[-\dp\strutbox]{1pt}{\normalbaselineskip}%
  \kern1pt%
}
\def\frepeat{%
  \kern1pt%
  \rule[-\dp\strutbox]{1pt}{\normalbaselineskip}%
  \kern0.5pt%
  \rule[-\dp\strutbox]{.3pt}{\normalbaselineskip}%
  \stackanchor{.}{.}%
}
% \newcommand{\SBRepeat}[1]{#1\\#1}
\newcommand{\SBRepeat}[1]{\frepeat #1\repeat}
\setcounter{SBSongCnt}{-1}
\renewcommand{\SBWAndMTag}{Forfatter:}
\renewcommand{\SBUnknownTag}{Ukendt}
\renewcommand{\SBChorusTag}{Ref.}
\renewcommand{\SBOrgMel}{Originalmelodi}
\renewcommand{\SpaceAfterChorus}   {\vspace{0ex plus1ex minus 0.5ex}}
\renewcommand{\SpaceAfterOpGroup}  {\vspace{0ex plus1ex minus 0.5ex}}
\renewcommand{\SpaceAfterSBBracket}{\vspace{0ex plus1ex minus 0.5ex}}
\renewcommand{\SpaceAfterSection}  {\vspace{0ex plus1ex minus 0.5ex}}
\renewcommand{\SpaceAfterSong}     {\vspace{0ex plus1ex minus 0.5ex}}
\renewcommand{\SpaceAfterVerse}    {\vspace{0ex plus1ex minus 0.5ex}}

% Tell LaTeX that \medskip is a good place to make a page break
\let\oldmedskip\medskip
\renewcommand{\medskip}{\oldmedskip\pagebreak[2]}

%%%
% Turn on/off index-file generation.  Uncomment the \makeindex line to
% turn index generation on;  comment it out to turn index generation
% off.
%%%
%\makeTitleIndex         %% Title and First Line Index.
%\makeTitleContents      %% Table of Contents.
%\makeKeyIndex           %% Index of song by key.
% \makeArtistIndex	%% Index of song by artist.
% \newcommand{\SBThechapter}[0]{}
% \newcommand{\SBChapter}[1]{
%     \startcontents
%     \chapter*{#1} 
%     % \input{unf-sangbog.toc}
%       \begin{minipage}{.8\textwidth}
%         \printcontents{}{1}{}
%       \end{minipage}%
%     \renewcommand{\SBThechapter}{#1}
%     \clearpage
% }

% \titleformat{\chapter}
% [display]
% {}
% {%\vspace*{\fill}
%  % \titlerule[1pt]%
%  % \vspace{1pt}%
%  % \titlerule
%  % \vspace{1pc}%
%  \chaptertitlename}
% {}
% {\Huge}



%%%%%%%%%%%%%%%%%%%%%%%%%%%%%%%%%%%%%%%%%%%%%%%%%%%%%%%%%%
%%%%%%%%%%%%%%%%%%%%%%%%%%%%%%%%%%%%%%%%%%%%%%%%%%%%%%%%%%
%%                                                      %%
%%           D O C U M E N T   B E G I N S              %%
%%                                                      %%
%%%%%%%%%%%%%%%%%%%%%%%%%%%%%%%%%%%%%%%%%%%%%%%%%%%%%%%%%%
%%%%%%%%%%%%%%%%%%%%%%%%%%%%%%%%%%%%%%%%%%%%%%%%%%%%%%%%%%
\begin{document}

%%%
% Uncomment "\maketitle" statement to make a title page.
%%%
%\maketitle
% \begin{titlepage}
%   \centering
%   \vspace{5cm}
% 	\includegraphics[width=1\textwidth]{unf_logo.jpeg}\par\vspace{1cm}
% 	{\scshape\LARGE Sangbog \par}
% 	\vspace{1cm}
% 	{\scshape\Large UNF Computer Science Camp 2019\par}
	
% 	\vfill

% % Bottom of the page
% 	{\large \today\par}
% \end{titlepage}
% \mainmatter
% \ifWordBk
%   \twocolumn
% \fi


%%% Kolofon
%\thispagestyle{empty}
%Sammensat til UNF Computer Science Camp 2019 - csc.unf.dk\\
%Redaktør: Andreas Mosbæk Jensen m.fl. efter tidligere sangbog af Steffen Strunge Mathiesen\\
%Indhold opsat i \LaTeX. 
%Digital version og kildekode: github.com/steffen555/UNF-sangbog\\
%Revision 1 med stave fejl korrektioner
%\par\vspace*{\fill}
%Hvis du har forslag til sange, rettelser, ris og ros, eller hvis du kender en ukendt forfatter, så skriv til sangbog@unf.dk.

%%%
% Turn on and define fancy page heading/footing definition.
%%%
% \pagestyle{fancy}

% \ifChordBk
%   % It's a words & chords songbook...
%   \addtolength{\headwidth}{\marginparsep}
%   \addtolength{\headwidth}{\marginparwidth}
%   \renewcommand{\headrulewidth}{0.4pt}
%   \renewcommand{\footrulewidth}{0.4pt}
%   \fancyhead[LE,RO]{\LHeadFont\emph{\leftmark\/}\SBContinueMark}
%   \fancyhead[CE,CO]{\CHeadFont\thepage}
%   \fancyhead[RE,LO]{\RHeadFont \chaptermark}
% \else\ifOverhead
%   % It's an overhead...
%   \renewcommand{\footrulewidth}{0pt}
%   \renewcommand{\headrulewidth}{0pt}
%   \fancyhead[LE,RO]{}
%   \fancyhead[CE,CO]{}
%   \fancyhead[RE,LO]{}
% \else\ifWordBk
%   % It's a words only songbook...
%   \addtolength{\headwidth}{\marginparsep}
%   \addtolength{\headwidth}{\marginparwidth}
%   \renewcommand{\headrulewidth}{0.4pt}
%   \renewcommand{\footrulewidth}{0.4pt}
%   \fancyhead[LE,RO]{\LHeadFont Naturvidenskab revy sange}
%   \fancyhead[CE,CO]{\CHeadFont\thepage}
%   \fancyhead[RE,LO]{\RHeadFont \SBThechapter}
% \fi\fi\fi

% \fancyfoot[LE,RO]{\LFootFont Computer Science Camp 2019}
% \ifSongEject
%   \fancyfoot[CE,CO]{\CFootFont Last Revised:  \RevDate}
% \else
%   \fancyfoot[CE,CO]{\CFootFont}
% \fi
% \fancyfoot[RE,LO]{\RFootFont Synges på eget ansvar}

%%%
% Table of contents
%%%

% \clearpage
% \twocolumn
% \font\myTinySF=cmss8    at  8pt
% \font\myHugeSF=cmssbx10 at 25pt
% \newcommand{\CpyRtInfoFont}{\tiny\myTinySF}
% \newcommand{\myTitleFont}{\Huge\myHugeSF}
% \newcommand{\mySubTitleFont}{\large\sf}
% \renewcommand{\indexspace}{\medskip}

% % {\parindent 8pt
% %   {\myTitleFont Indhold}}\par
% % \vskip 5pt
% \renewcommand{\SBThechapter}{Indhold}
% % {\parindent 20pt
% %   {\mySubTitleFont --- with first lines in italic ---}}
% % \vskip 20pt
% \let\olditem\item
% \let\oldsubitem\subitem
% \let\oldsubsubitem\subsubitem
% \renewcommand{\item}{\par\hangindent=40pt}
% \renewcommand{\subitem}{\par\hangindent=40pt \hspace*{20pt}}
% \renewcommand{\subsubitem}{\par\hangindent=40pt \hspace*{30pt}}

% %\input{unf-sangbog.tocx}

% \renewcommand{\item}{\olditem}
% \renewcommand{\subitem}{\oldsubitem}
% \renewcommand{\subsubitem}{\oldsubsubitem}

%%%
% Songbook begins.
%%%

\twocolumn
%It's just one page, don't print page numbers etc.
\pagestyle{empty}
%Songs included
\input{songs/matmatik.tex}
\input{songs/taal_daj.tex}
\input{songs/linieskriverdriver.tex}
\input{songs/steve_hawking.tex}
\input{songs/ode_til_kode.tex}
\input{songs/se_min_kode.tex}
\input{songs/vaabenfysik_kort.tex}
%Maybe include:
%\input{songs/kvanter_i_maaneskin.tex}
%\input{songs/mest_matematiske_dyr.tex}

% \input{songs/vi_kan_ikke_li.tex}
% \input{songs/selektionssangen.tex}
% \input{songs/alfabetsangen.tex}
% \input{songs/sciencecamps.tex}
% \input{songs/hvad_maa_man.tex}


% \input{songs/lambda_kalkylen.tex}
% \input{songs/puslespil.tex}
% \input{songs/null.tex}
% \input{songs/fasebal.tex}

% \input{songs/chifitter.tex}

% \input{songs/kun_fysik.tex}



% \input{songs/kanoniske.tex}
% \input{songs/jeg_er_en_matematiker_fra_hcoe.tex}


% \input{songs/rekursiv_skovsang.tex}
% \input{songs/laerkerede.tex}


% \clearpage
% \font\myTinySF=cmss8    at  8pt
% \font\myHugeSF=cmssbx10 at 25pt
% % \newcommand{\CpyRtInfoFont}{\tiny\myTinySF}
% % \newcommand{\myTitleFont}{\Huge\myHugeSF}
% % \newcommand{\mySubTitleFont}{\large\sf}
% \renewcommand{\indexspace}{\medskip}

% {\parindent 8pt
%   {\myTitleFont Index}}\par
% \vskip 5pt
% \renewcommand{\SBThechapter}{Index}
% % {\parindent 20pt
% %   {\mySubTitleFont --- with first lines in italic ---}}
% % \vskip 20pt
% \renewcommand{\item}{\par\hangindent=40pt}
% \renewcommand{\subitem}{\par\hangindent=40pt \hspace*{20pt}}
% \renewcommand{\subsubitem}{\par\hangindent=40pt \hspace*{30pt}}

%\input{unf-sangbog.tdx}

\end{document}
\bye
%
%%%
% Document ends.
%%%

%       \begin{minipage}{.8\textwidth}
%         \printcontents{}{1}{}
%       \end{minipage}%
%     \renewcommand{\SBThechapter}{#1}
%     \clearpage
% }

% \titleformat{\chapter}
% [display]
% {}
% {%\vspace*{\fill}
%  % \titlerule[1pt]%
%  % \vspace{1pt}%
%  % \titlerule
%  % \vspace{1pc}%
%  \chaptertitlename}
% {}
% {\Huge}



%%%%%%%%%%%%%%%%%%%%%%%%%%%%%%%%%%%%%%%%%%%%%%%%%%%%%%%%%%
%%%%%%%%%%%%%%%%%%%%%%%%%%%%%%%%%%%%%%%%%%%%%%%%%%%%%%%%%%
%%                                                      %%
%%           D O C U M E N T   B E G I N S              %%
%%                                                      %%
%%%%%%%%%%%%%%%%%%%%%%%%%%%%%%%%%%%%%%%%%%%%%%%%%%%%%%%%%%
%%%%%%%%%%%%%%%%%%%%%%%%%%%%%%%%%%%%%%%%%%%%%%%%%%%%%%%%%%
\begin{document}

%%%
% Uncomment "\maketitle" statement to make a title page.
%%%
%\maketitle
% \begin{titlepage}
%   \centering
%   \vspace{5cm}
% 	\includegraphics[width=1\textwidth]{unf_logo.jpeg}\par\vspace{1cm}
% 	{\scshape\LARGE Sangbog \par}
% 	\vspace{1cm}
% 	{\scshape\Large UNF Computer Science Camp 2019\par}
	
% 	\vfill

% % Bottom of the page
% 	{\large \today\par}
% \end{titlepage}
% \mainmatter
% \ifWordBk
%   \twocolumn
% \fi


%%% Kolofon
%\thispagestyle{empty}
%Sammensat til UNF Computer Science Camp 2019 - csc.unf.dk\\
%Redaktør: Andreas Mosbæk Jensen m.fl. efter tidligere sangbog af Steffen Strunge Mathiesen\\
%Indhold opsat i \LaTeX. 
%Digital version og kildekode: github.com/steffen555/UNF-sangbog\\
%Revision 1 med stave fejl korrektioner
%\par\vspace*{\fill}
%Hvis du har forslag til sange, rettelser, ris og ros, eller hvis du kender en ukendt forfatter, så skriv til sangbog@unf.dk.

%%%
% Turn on and define fancy page heading/footing definition.
%%%
% \pagestyle{fancy}

% \ifChordBk
%   % It's a words & chords songbook...
%   \addtolength{\headwidth}{\marginparsep}
%   \addtolength{\headwidth}{\marginparwidth}
%   \renewcommand{\headrulewidth}{0.4pt}
%   \renewcommand{\footrulewidth}{0.4pt}
%   \fancyhead[LE,RO]{\LHeadFont\emph{\leftmark\/}\SBContinueMark}
%   \fancyhead[CE,CO]{\CHeadFont\thepage}
%   \fancyhead[RE,LO]{\RHeadFont \chaptermark}
% \else\ifOverhead
%   % It's an overhead...
%   \renewcommand{\footrulewidth}{0pt}
%   \renewcommand{\headrulewidth}{0pt}
%   \fancyhead[LE,RO]{}
%   \fancyhead[CE,CO]{}
%   \fancyhead[RE,LO]{}
% \else\ifWordBk
%   % It's a words only songbook...
%   \addtolength{\headwidth}{\marginparsep}
%   \addtolength{\headwidth}{\marginparwidth}
%   \renewcommand{\headrulewidth}{0.4pt}
%   \renewcommand{\footrulewidth}{0.4pt}
%   \fancyhead[LE,RO]{\LHeadFont Naturvidenskab revy sange}
%   \fancyhead[CE,CO]{\CHeadFont\thepage}
%   \fancyhead[RE,LO]{\RHeadFont \SBThechapter}
% \fi\fi\fi

% \fancyfoot[LE,RO]{\LFootFont Computer Science Camp 2019}
% \ifSongEject
%   \fancyfoot[CE,CO]{\CFootFont Last Revised:  \RevDate}
% \else
%   \fancyfoot[CE,CO]{\CFootFont}
% \fi
% \fancyfoot[RE,LO]{\RFootFont Synges på eget ansvar}

%%%
% Table of contents
%%%

% \clearpage
% \twocolumn
% \font\myTinySF=cmss8    at  8pt
% \font\myHugeSF=cmssbx10 at 25pt
% \newcommand{\CpyRtInfoFont}{\tiny\myTinySF}
% \newcommand{\myTitleFont}{\Huge\myHugeSF}
% \newcommand{\mySubTitleFont}{\large\sf}
% \renewcommand{\indexspace}{\medskip}

% % {\parindent 8pt
% %   {\myTitleFont Indhold}}\par
% % \vskip 5pt
% \renewcommand{\SBThechapter}{Indhold}
% % {\parindent 20pt
% %   {\mySubTitleFont --- with first lines in italic ---}}
% % \vskip 20pt
% \let\olditem\item
% \let\oldsubitem\subitem
% \let\oldsubsubitem\subsubitem
% \renewcommand{\item}{\par\hangindent=40pt}
% \renewcommand{\subitem}{\par\hangindent=40pt \hspace*{20pt}}
% \renewcommand{\subsubitem}{\par\hangindent=40pt \hspace*{30pt}}

% %%%%%%% rcsid = @(#)$Id: sample-sb.tex,v 1.23 2010-04-12 18:04:11 rathc Exp $
%%%%%%
%%
%%      ===============================
%%      Sample Songbook (sample-sb.tex)
%%      ===============================
%%
%%      Version 4.5, 30 April, 2010
%%
%%      Copyright 1992--2010 Christopher Rath <christopher@rath.ca>
%%
%%      This package is free software; you can redistribute it and/or
%%      modify it under the terms of version 2.1 of the GNU Lesser
%%	General Public License as published by the Free Software 
%%	Foundation.
%%
%%      This package is distributed in the hope that it will be
%%      useful, but WITHOUT ANY WARRANTY; without even the implied
%%      warranty of MERCHANTABILITY or FITNESS FOR A PARTICULAR
%%      PURPOSE.  See the GNU Lesser General Public License for more
%%      details.
%%
%%      This file contains a subset of the songbook we distribute
%%      at our church.  To the best of my knowledge, all of the lyrics
%%      contained herein are freely distributable.  This file has been
%%      provided as a sample of what can be produced by the chordbk,
%%      wordbk, and overhead LaTeX styles.
%%
%%      NEEDED:  The fancyhdr LaTeX style is required to properly
%%              format this file.  If you don't have that then comment
%%              out the commands in the preamble which deal with the
%%              fancyhdr style.
%%
%%%%%%
%%%%%%
%%
%%      1. Chord notation.  Within this songbook the following
%%         conventions have been adopted:
%%
%%              "Minor" is entered as "m";
%%                      e.g. Cm7 for C minor 7th.
%%              "Major" is entered as "M";
%%                      e.g. CM7 for C major 7th.
%%
%%%%%%
%%%%%%
%%      ============
%%      Bibliography
%%      ============
%%
%%      Exalt Him!: Exalt Him!  Compiled by Tom Fettke.  (c)1989
%%                      Word Music.
%%
%%      Hosanna! Music Books: Hosanna! Music Books #1--#6.
%%                      (c)1987--92 Integrity Music, Inc.
%%
%%      Worship Him II: Worship Him II.  Compiled by Jesse Peterson
%%                      and Bruce Ballinger.  (c)1989 Tempo Music
%%                      Publications.
%%
%%      Worship Songs Of The Vineyard: Worship Songs Of The Vineyard
%%                      --- Volume 2.  (c)1989 Vineyard Ministries
%%                      International.
%%
%%%%%%
%%%%%%

%%%%%%%%%%%%%%%%%%%%%%%%%%%%%%%%%%%%%%%%%%%%%%%%%%%%%%%%%%
%%%%%%%%%%%%%%%%%%%%%%%%%%%%%%%%%%%%%%%%%%%%%%%%%%%%%%%%%%
%%                                                      %%
%%           P R E A M B L E   B E G I N S              %%
%%                                                      %%
%%%%%%%%%%%%%%%%%%%%%%%%%%%%%%%%%%%%%%%%%%%%%%%%%%%%%%%%%%
%%%%%%%%%%%%%%%%%%%%%%%%%%%%%%%%%%%%%%%%%%%%%%%%%%%%%%%%%%

\documentclass[a5paper]{book}
\usepackage{latexsym,
            fancyhdr,
            titlesec,
            amsmath,
            amssymb,
            multicol,
            amsthm,
            stmaryrd,
            amsthm,
            color,
            needspace,
            stackengine,
            wasysym}
\usepackage[utf8]{inputenc}
\usepackage[T1]{fontenc}
% \usepackage[chordbk]{songbook}                  %% Words & Chords edition.
%%\usepackage[compactallsongs,chordbk]{songbook}    %% Words & Chords edition.
\usepackage[wordbk]{songbook}                 %% Words Only edition.
%%\usepackage[overhead]{songbook}               %% Overhead Transparency edition.
\usepackage{titletoc}
\usepackage{tket}  % Draws "TÅGEKAMMERET" correctly

%%%
% Revision Date and Release Date definitions.
%
%       \RelDate - The last time this songbook was released.  Set this
%                  date each time a new release/update of the songbook
%                  is generated.
%       \RevDate - The last time a particular song was revised in any
%                  way.  This command will be renewed inside every
%                  song.
%%%
\newcommand{\RelDate}{31~August,~2003}
\newcommand{\RevDate}{\today}

%%%
% C.C.L.I. license number definition; for copyright licensing info.
% One of these macros will be manually inserted into the {SBMel}
% parameter of the {song} environment.
%
%       \CCLInumber - The actual copyright license number.  Don't
%               insert this command in the {SBMel} parameter, use one
%               of the others.
%       \CCLIed - Indicates a song falls under our CCLI license.
%       \NotCCLIed - Indicates a song doesn't fall under our CCLI
%               license.  Public Domain songs fall into this category.
%       \PGranted - We have received specific permission from the
%               copyright holder to use this song.
%       \PPending - We are in the process of obtaining permission to
%               use this song.
%%%
\newcommand{\CCLInumber}{Your CCLI Number}
\newcommand{\CCLIed}{{\SBMelInfoFont (CCLI \CCLInumber)}}
\newcommand{\NotCCLIed}{\relax}
\newcommand{\PGranted}{\relax}
\newcommand{\PPending}{{\SBMelInfoFont (Permission Pending)}}

%%%
% Title page information.
%%%
%\title{UNF Computer Science Camp 2019 Sangbog}
%\author{}
%\date{Revideret:  \RevDate}

%%%
% Redefine fonts from SongBook style that I don't like.
%%%
\font\myTinySF=cmss8 at 8pt
\renewcommand{\SBMelInfoFont}{\tiny\myTinySF}

%%%
% Define fonts to use in the headers and footers of the songbook.
%%%
\newcommand{\LHeadFont}{\normalsize}            % = cmr12  at 12pt
\newcommand{\CHeadFont}{\normalsize\rm}         % = cmr12  at 12pt
\newcommand{\RHeadFont}{\normalsize}            % = cmr12  at 12pt
\newcommand{\LFootFont}{\scriptsize}            % = cmr8   at  8pt
\newcommand{\CFootFont}{\tiny\myTinySF}         % = cmss8  at  8pt
\newcommand{\RFootFont}{\scriptsize}            % = cmr8   at  8pt

\def\repeat{%
  \stackanchor{.}{.}%
  \rule[-\dp\strutbox]{.3pt}{\normalbaselineskip}%
  \kern0.5pt%
  \rule[-\dp\strutbox]{1pt}{\normalbaselineskip}%
  \kern1pt%
}
\def\frepeat{%
  \kern1pt%
  \rule[-\dp\strutbox]{1pt}{\normalbaselineskip}%
  \kern0.5pt%
  \rule[-\dp\strutbox]{.3pt}{\normalbaselineskip}%
  \stackanchor{.}{.}%
}
% \newcommand{\SBRepeat}[1]{#1\\#1}
\newcommand{\SBRepeat}[1]{\frepeat #1\repeat}
\setcounter{SBSongCnt}{-1}
\renewcommand{\SBWAndMTag}{Forfatter:}
\renewcommand{\SBUnknownTag}{Ukendt}
\renewcommand{\SBChorusTag}{Ref.}
\renewcommand{\SBOrgMel}{Originalmelodi}
\renewcommand{\SpaceAfterChorus}   {\vspace{0ex plus1ex minus 0.5ex}}
\renewcommand{\SpaceAfterOpGroup}  {\vspace{0ex plus1ex minus 0.5ex}}
\renewcommand{\SpaceAfterSBBracket}{\vspace{0ex plus1ex minus 0.5ex}}
\renewcommand{\SpaceAfterSection}  {\vspace{0ex plus1ex minus 0.5ex}}
\renewcommand{\SpaceAfterSong}     {\vspace{0ex plus1ex minus 0.5ex}}
\renewcommand{\SpaceAfterVerse}    {\vspace{0ex plus1ex minus 0.5ex}}

% Tell LaTeX that \medskip is a good place to make a page break
\let\oldmedskip\medskip
\renewcommand{\medskip}{\oldmedskip\pagebreak[2]}

%%%
% Turn on/off index-file generation.  Uncomment the \makeindex line to
% turn index generation on;  comment it out to turn index generation
% off.
%%%
%\makeTitleIndex         %% Title and First Line Index.
%\makeTitleContents      %% Table of Contents.
%\makeKeyIndex           %% Index of song by key.
% \makeArtistIndex	%% Index of song by artist.
% \newcommand{\SBThechapter}[0]{}
% \newcommand{\SBChapter}[1]{
%     \startcontents
%     \chapter*{#1} 
%     % \input{unf-sangbog.toc}
%       \begin{minipage}{.8\textwidth}
%         \printcontents{}{1}{}
%       \end{minipage}%
%     \renewcommand{\SBThechapter}{#1}
%     \clearpage
% }

% \titleformat{\chapter}
% [display]
% {}
% {%\vspace*{\fill}
%  % \titlerule[1pt]%
%  % \vspace{1pt}%
%  % \titlerule
%  % \vspace{1pc}%
%  \chaptertitlename}
% {}
% {\Huge}



%%%%%%%%%%%%%%%%%%%%%%%%%%%%%%%%%%%%%%%%%%%%%%%%%%%%%%%%%%
%%%%%%%%%%%%%%%%%%%%%%%%%%%%%%%%%%%%%%%%%%%%%%%%%%%%%%%%%%
%%                                                      %%
%%           D O C U M E N T   B E G I N S              %%
%%                                                      %%
%%%%%%%%%%%%%%%%%%%%%%%%%%%%%%%%%%%%%%%%%%%%%%%%%%%%%%%%%%
%%%%%%%%%%%%%%%%%%%%%%%%%%%%%%%%%%%%%%%%%%%%%%%%%%%%%%%%%%
\begin{document}

%%%
% Uncomment "\maketitle" statement to make a title page.
%%%
%\maketitle
% \begin{titlepage}
%   \centering
%   \vspace{5cm}
% 	\includegraphics[width=1\textwidth]{unf_logo.jpeg}\par\vspace{1cm}
% 	{\scshape\LARGE Sangbog \par}
% 	\vspace{1cm}
% 	{\scshape\Large UNF Computer Science Camp 2019\par}
	
% 	\vfill

% % Bottom of the page
% 	{\large \today\par}
% \end{titlepage}
% \mainmatter
% \ifWordBk
%   \twocolumn
% \fi


%%% Kolofon
%\thispagestyle{empty}
%Sammensat til UNF Computer Science Camp 2019 - csc.unf.dk\\
%Redaktør: Andreas Mosbæk Jensen m.fl. efter tidligere sangbog af Steffen Strunge Mathiesen\\
%Indhold opsat i \LaTeX. 
%Digital version og kildekode: github.com/steffen555/UNF-sangbog\\
%Revision 1 med stave fejl korrektioner
%\par\vspace*{\fill}
%Hvis du har forslag til sange, rettelser, ris og ros, eller hvis du kender en ukendt forfatter, så skriv til sangbog@unf.dk.

%%%
% Turn on and define fancy page heading/footing definition.
%%%
% \pagestyle{fancy}

% \ifChordBk
%   % It's a words & chords songbook...
%   \addtolength{\headwidth}{\marginparsep}
%   \addtolength{\headwidth}{\marginparwidth}
%   \renewcommand{\headrulewidth}{0.4pt}
%   \renewcommand{\footrulewidth}{0.4pt}
%   \fancyhead[LE,RO]{\LHeadFont\emph{\leftmark\/}\SBContinueMark}
%   \fancyhead[CE,CO]{\CHeadFont\thepage}
%   \fancyhead[RE,LO]{\RHeadFont \chaptermark}
% \else\ifOverhead
%   % It's an overhead...
%   \renewcommand{\footrulewidth}{0pt}
%   \renewcommand{\headrulewidth}{0pt}
%   \fancyhead[LE,RO]{}
%   \fancyhead[CE,CO]{}
%   \fancyhead[RE,LO]{}
% \else\ifWordBk
%   % It's a words only songbook...
%   \addtolength{\headwidth}{\marginparsep}
%   \addtolength{\headwidth}{\marginparwidth}
%   \renewcommand{\headrulewidth}{0.4pt}
%   \renewcommand{\footrulewidth}{0.4pt}
%   \fancyhead[LE,RO]{\LHeadFont Naturvidenskab revy sange}
%   \fancyhead[CE,CO]{\CHeadFont\thepage}
%   \fancyhead[RE,LO]{\RHeadFont \SBThechapter}
% \fi\fi\fi

% \fancyfoot[LE,RO]{\LFootFont Computer Science Camp 2019}
% \ifSongEject
%   \fancyfoot[CE,CO]{\CFootFont Last Revised:  \RevDate}
% \else
%   \fancyfoot[CE,CO]{\CFootFont}
% \fi
% \fancyfoot[RE,LO]{\RFootFont Synges på eget ansvar}

%%%
% Table of contents
%%%

% \clearpage
% \twocolumn
% \font\myTinySF=cmss8    at  8pt
% \font\myHugeSF=cmssbx10 at 25pt
% \newcommand{\CpyRtInfoFont}{\tiny\myTinySF}
% \newcommand{\myTitleFont}{\Huge\myHugeSF}
% \newcommand{\mySubTitleFont}{\large\sf}
% \renewcommand{\indexspace}{\medskip}

% % {\parindent 8pt
% %   {\myTitleFont Indhold}}\par
% % \vskip 5pt
% \renewcommand{\SBThechapter}{Indhold}
% % {\parindent 20pt
% %   {\mySubTitleFont --- with first lines in italic ---}}
% % \vskip 20pt
% \let\olditem\item
% \let\oldsubitem\subitem
% \let\oldsubsubitem\subsubitem
% \renewcommand{\item}{\par\hangindent=40pt}
% \renewcommand{\subitem}{\par\hangindent=40pt \hspace*{20pt}}
% \renewcommand{\subsubitem}{\par\hangindent=40pt \hspace*{30pt}}

% %\input{unf-sangbog.tocx}

% \renewcommand{\item}{\olditem}
% \renewcommand{\subitem}{\oldsubitem}
% \renewcommand{\subsubitem}{\oldsubsubitem}

%%%
% Songbook begins.
%%%

\twocolumn
%It's just one page, don't print page numbers etc.
\pagestyle{empty}
%Songs included
\input{songs/matmatik.tex}
\input{songs/taal_daj.tex}
\input{songs/linieskriverdriver.tex}
\input{songs/steve_hawking.tex}
\input{songs/ode_til_kode.tex}
\input{songs/se_min_kode.tex}
\input{songs/vaabenfysik_kort.tex}
%Maybe include:
%\input{songs/kvanter_i_maaneskin.tex}
%\input{songs/mest_matematiske_dyr.tex}

% \input{songs/vi_kan_ikke_li.tex}
% \input{songs/selektionssangen.tex}
% \input{songs/alfabetsangen.tex}
% \input{songs/sciencecamps.tex}
% \input{songs/hvad_maa_man.tex}


% \input{songs/lambda_kalkylen.tex}
% \input{songs/puslespil.tex}
% \input{songs/null.tex}
% \input{songs/fasebal.tex}

% \input{songs/chifitter.tex}

% \input{songs/kun_fysik.tex}



% \input{songs/kanoniske.tex}
% \input{songs/jeg_er_en_matematiker_fra_hcoe.tex}


% \input{songs/rekursiv_skovsang.tex}
% \input{songs/laerkerede.tex}


% \clearpage
% \font\myTinySF=cmss8    at  8pt
% \font\myHugeSF=cmssbx10 at 25pt
% % \newcommand{\CpyRtInfoFont}{\tiny\myTinySF}
% % \newcommand{\myTitleFont}{\Huge\myHugeSF}
% % \newcommand{\mySubTitleFont}{\large\sf}
% \renewcommand{\indexspace}{\medskip}

% {\parindent 8pt
%   {\myTitleFont Index}}\par
% \vskip 5pt
% \renewcommand{\SBThechapter}{Index}
% % {\parindent 20pt
% %   {\mySubTitleFont --- with first lines in italic ---}}
% % \vskip 20pt
% \renewcommand{\item}{\par\hangindent=40pt}
% \renewcommand{\subitem}{\par\hangindent=40pt \hspace*{20pt}}
% \renewcommand{\subsubitem}{\par\hangindent=40pt \hspace*{30pt}}

%\input{unf-sangbog.tdx}

\end{document}
\bye
%
%%%
% Document ends.
%%%


% \renewcommand{\item}{\olditem}
% \renewcommand{\subitem}{\oldsubitem}
% \renewcommand{\subsubitem}{\oldsubsubitem}

%%%
% Songbook begins.
%%%

\twocolumn
%It's just one page, don't print page numbers etc.
\pagestyle{empty}
%Songs included
\input{songs/matmatik.tex}
\input{songs/taal_daj.tex}
\input{songs/linieskriverdriver.tex}
\input{songs/steve_hawking.tex}
\input{songs/ode_til_kode.tex}
\input{songs/se_min_kode.tex}
\input{songs/vaabenfysik_kort.tex}
%Maybe include:
%\input{songs/kvanter_i_maaneskin.tex}
%\input{songs/mest_matematiske_dyr.tex}

% \input{songs/vi_kan_ikke_li.tex}
% \input{songs/selektionssangen.tex}
% \input{songs/alfabetsangen.tex}
% \input{songs/sciencecamps.tex}
% \input{songs/hvad_maa_man.tex}


% \input{songs/lambda_kalkylen.tex}
% \input{songs/puslespil.tex}
% \input{songs/null.tex}
% \input{songs/fasebal.tex}

% \input{songs/chifitter.tex}

% \input{songs/kun_fysik.tex}



% \input{songs/kanoniske.tex}
% \input{songs/jeg_er_en_matematiker_fra_hcoe.tex}


% \input{songs/rekursiv_skovsang.tex}
% \input{songs/laerkerede.tex}


% \clearpage
% \font\myTinySF=cmss8    at  8pt
% \font\myHugeSF=cmssbx10 at 25pt
% % \newcommand{\CpyRtInfoFont}{\tiny\myTinySF}
% % \newcommand{\myTitleFont}{\Huge\myHugeSF}
% % \newcommand{\mySubTitleFont}{\large\sf}
% \renewcommand{\indexspace}{\medskip}

% {\parindent 8pt
%   {\myTitleFont Index}}\par
% \vskip 5pt
% \renewcommand{\SBThechapter}{Index}
% % {\parindent 20pt
% %   {\mySubTitleFont --- with first lines in italic ---}}
% % \vskip 20pt
% \renewcommand{\item}{\par\hangindent=40pt}
% \renewcommand{\subitem}{\par\hangindent=40pt \hspace*{20pt}}
% \renewcommand{\subsubitem}{\par\hangindent=40pt \hspace*{30pt}}

%%%%%%% rcsid = @(#)$Id: sample-sb.tex,v 1.23 2010-04-12 18:04:11 rathc Exp $
%%%%%%
%%
%%      ===============================
%%      Sample Songbook (sample-sb.tex)
%%      ===============================
%%
%%      Version 4.5, 30 April, 2010
%%
%%      Copyright 1992--2010 Christopher Rath <christopher@rath.ca>
%%
%%      This package is free software; you can redistribute it and/or
%%      modify it under the terms of version 2.1 of the GNU Lesser
%%	General Public License as published by the Free Software 
%%	Foundation.
%%
%%      This package is distributed in the hope that it will be
%%      useful, but WITHOUT ANY WARRANTY; without even the implied
%%      warranty of MERCHANTABILITY or FITNESS FOR A PARTICULAR
%%      PURPOSE.  See the GNU Lesser General Public License for more
%%      details.
%%
%%      This file contains a subset of the songbook we distribute
%%      at our church.  To the best of my knowledge, all of the lyrics
%%      contained herein are freely distributable.  This file has been
%%      provided as a sample of what can be produced by the chordbk,
%%      wordbk, and overhead LaTeX styles.
%%
%%      NEEDED:  The fancyhdr LaTeX style is required to properly
%%              format this file.  If you don't have that then comment
%%              out the commands in the preamble which deal with the
%%              fancyhdr style.
%%
%%%%%%
%%%%%%
%%
%%      1. Chord notation.  Within this songbook the following
%%         conventions have been adopted:
%%
%%              "Minor" is entered as "m";
%%                      e.g. Cm7 for C minor 7th.
%%              "Major" is entered as "M";
%%                      e.g. CM7 for C major 7th.
%%
%%%%%%
%%%%%%
%%      ============
%%      Bibliography
%%      ============
%%
%%      Exalt Him!: Exalt Him!  Compiled by Tom Fettke.  (c)1989
%%                      Word Music.
%%
%%      Hosanna! Music Books: Hosanna! Music Books #1--#6.
%%                      (c)1987--92 Integrity Music, Inc.
%%
%%      Worship Him II: Worship Him II.  Compiled by Jesse Peterson
%%                      and Bruce Ballinger.  (c)1989 Tempo Music
%%                      Publications.
%%
%%      Worship Songs Of The Vineyard: Worship Songs Of The Vineyard
%%                      --- Volume 2.  (c)1989 Vineyard Ministries
%%                      International.
%%
%%%%%%
%%%%%%

%%%%%%%%%%%%%%%%%%%%%%%%%%%%%%%%%%%%%%%%%%%%%%%%%%%%%%%%%%
%%%%%%%%%%%%%%%%%%%%%%%%%%%%%%%%%%%%%%%%%%%%%%%%%%%%%%%%%%
%%                                                      %%
%%           P R E A M B L E   B E G I N S              %%
%%                                                      %%
%%%%%%%%%%%%%%%%%%%%%%%%%%%%%%%%%%%%%%%%%%%%%%%%%%%%%%%%%%
%%%%%%%%%%%%%%%%%%%%%%%%%%%%%%%%%%%%%%%%%%%%%%%%%%%%%%%%%%

\documentclass[a5paper]{book}
\usepackage{latexsym,
            fancyhdr,
            titlesec,
            amsmath,
            amssymb,
            multicol,
            amsthm,
            stmaryrd,
            amsthm,
            color,
            needspace,
            stackengine,
            wasysym}
\usepackage[utf8]{inputenc}
\usepackage[T1]{fontenc}
% \usepackage[chordbk]{songbook}                  %% Words & Chords edition.
%%\usepackage[compactallsongs,chordbk]{songbook}    %% Words & Chords edition.
\usepackage[wordbk]{songbook}                 %% Words Only edition.
%%\usepackage[overhead]{songbook}               %% Overhead Transparency edition.
\usepackage{titletoc}
\usepackage{tket}  % Draws "TÅGEKAMMERET" correctly

%%%
% Revision Date and Release Date definitions.
%
%       \RelDate - The last time this songbook was released.  Set this
%                  date each time a new release/update of the songbook
%                  is generated.
%       \RevDate - The last time a particular song was revised in any
%                  way.  This command will be renewed inside every
%                  song.
%%%
\newcommand{\RelDate}{31~August,~2003}
\newcommand{\RevDate}{\today}

%%%
% C.C.L.I. license number definition; for copyright licensing info.
% One of these macros will be manually inserted into the {SBMel}
% parameter of the {song} environment.
%
%       \CCLInumber - The actual copyright license number.  Don't
%               insert this command in the {SBMel} parameter, use one
%               of the others.
%       \CCLIed - Indicates a song falls under our CCLI license.
%       \NotCCLIed - Indicates a song doesn't fall under our CCLI
%               license.  Public Domain songs fall into this category.
%       \PGranted - We have received specific permission from the
%               copyright holder to use this song.
%       \PPending - We are in the process of obtaining permission to
%               use this song.
%%%
\newcommand{\CCLInumber}{Your CCLI Number}
\newcommand{\CCLIed}{{\SBMelInfoFont (CCLI \CCLInumber)}}
\newcommand{\NotCCLIed}{\relax}
\newcommand{\PGranted}{\relax}
\newcommand{\PPending}{{\SBMelInfoFont (Permission Pending)}}

%%%
% Title page information.
%%%
%\title{UNF Computer Science Camp 2019 Sangbog}
%\author{}
%\date{Revideret:  \RevDate}

%%%
% Redefine fonts from SongBook style that I don't like.
%%%
\font\myTinySF=cmss8 at 8pt
\renewcommand{\SBMelInfoFont}{\tiny\myTinySF}

%%%
% Define fonts to use in the headers and footers of the songbook.
%%%
\newcommand{\LHeadFont}{\normalsize}            % = cmr12  at 12pt
\newcommand{\CHeadFont}{\normalsize\rm}         % = cmr12  at 12pt
\newcommand{\RHeadFont}{\normalsize}            % = cmr12  at 12pt
\newcommand{\LFootFont}{\scriptsize}            % = cmr8   at  8pt
\newcommand{\CFootFont}{\tiny\myTinySF}         % = cmss8  at  8pt
\newcommand{\RFootFont}{\scriptsize}            % = cmr8   at  8pt

\def\repeat{%
  \stackanchor{.}{.}%
  \rule[-\dp\strutbox]{.3pt}{\normalbaselineskip}%
  \kern0.5pt%
  \rule[-\dp\strutbox]{1pt}{\normalbaselineskip}%
  \kern1pt%
}
\def\frepeat{%
  \kern1pt%
  \rule[-\dp\strutbox]{1pt}{\normalbaselineskip}%
  \kern0.5pt%
  \rule[-\dp\strutbox]{.3pt}{\normalbaselineskip}%
  \stackanchor{.}{.}%
}
% \newcommand{\SBRepeat}[1]{#1\\#1}
\newcommand{\SBRepeat}[1]{\frepeat #1\repeat}
\setcounter{SBSongCnt}{-1}
\renewcommand{\SBWAndMTag}{Forfatter:}
\renewcommand{\SBUnknownTag}{Ukendt}
\renewcommand{\SBChorusTag}{Ref.}
\renewcommand{\SBOrgMel}{Originalmelodi}
\renewcommand{\SpaceAfterChorus}   {\vspace{0ex plus1ex minus 0.5ex}}
\renewcommand{\SpaceAfterOpGroup}  {\vspace{0ex plus1ex minus 0.5ex}}
\renewcommand{\SpaceAfterSBBracket}{\vspace{0ex plus1ex minus 0.5ex}}
\renewcommand{\SpaceAfterSection}  {\vspace{0ex plus1ex minus 0.5ex}}
\renewcommand{\SpaceAfterSong}     {\vspace{0ex plus1ex minus 0.5ex}}
\renewcommand{\SpaceAfterVerse}    {\vspace{0ex plus1ex minus 0.5ex}}

% Tell LaTeX that \medskip is a good place to make a page break
\let\oldmedskip\medskip
\renewcommand{\medskip}{\oldmedskip\pagebreak[2]}

%%%
% Turn on/off index-file generation.  Uncomment the \makeindex line to
% turn index generation on;  comment it out to turn index generation
% off.
%%%
%\makeTitleIndex         %% Title and First Line Index.
%\makeTitleContents      %% Table of Contents.
%\makeKeyIndex           %% Index of song by key.
% \makeArtistIndex	%% Index of song by artist.
% \newcommand{\SBThechapter}[0]{}
% \newcommand{\SBChapter}[1]{
%     \startcontents
%     \chapter*{#1} 
%     % \input{unf-sangbog.toc}
%       \begin{minipage}{.8\textwidth}
%         \printcontents{}{1}{}
%       \end{minipage}%
%     \renewcommand{\SBThechapter}{#1}
%     \clearpage
% }

% \titleformat{\chapter}
% [display]
% {}
% {%\vspace*{\fill}
%  % \titlerule[1pt]%
%  % \vspace{1pt}%
%  % \titlerule
%  % \vspace{1pc}%
%  \chaptertitlename}
% {}
% {\Huge}



%%%%%%%%%%%%%%%%%%%%%%%%%%%%%%%%%%%%%%%%%%%%%%%%%%%%%%%%%%
%%%%%%%%%%%%%%%%%%%%%%%%%%%%%%%%%%%%%%%%%%%%%%%%%%%%%%%%%%
%%                                                      %%
%%           D O C U M E N T   B E G I N S              %%
%%                                                      %%
%%%%%%%%%%%%%%%%%%%%%%%%%%%%%%%%%%%%%%%%%%%%%%%%%%%%%%%%%%
%%%%%%%%%%%%%%%%%%%%%%%%%%%%%%%%%%%%%%%%%%%%%%%%%%%%%%%%%%
\begin{document}

%%%
% Uncomment "\maketitle" statement to make a title page.
%%%
%\maketitle
% \begin{titlepage}
%   \centering
%   \vspace{5cm}
% 	\includegraphics[width=1\textwidth]{unf_logo.jpeg}\par\vspace{1cm}
% 	{\scshape\LARGE Sangbog \par}
% 	\vspace{1cm}
% 	{\scshape\Large UNF Computer Science Camp 2019\par}
	
% 	\vfill

% % Bottom of the page
% 	{\large \today\par}
% \end{titlepage}
% \mainmatter
% \ifWordBk
%   \twocolumn
% \fi


%%% Kolofon
%\thispagestyle{empty}
%Sammensat til UNF Computer Science Camp 2019 - csc.unf.dk\\
%Redaktør: Andreas Mosbæk Jensen m.fl. efter tidligere sangbog af Steffen Strunge Mathiesen\\
%Indhold opsat i \LaTeX. 
%Digital version og kildekode: github.com/steffen555/UNF-sangbog\\
%Revision 1 med stave fejl korrektioner
%\par\vspace*{\fill}
%Hvis du har forslag til sange, rettelser, ris og ros, eller hvis du kender en ukendt forfatter, så skriv til sangbog@unf.dk.

%%%
% Turn on and define fancy page heading/footing definition.
%%%
% \pagestyle{fancy}

% \ifChordBk
%   % It's a words & chords songbook...
%   \addtolength{\headwidth}{\marginparsep}
%   \addtolength{\headwidth}{\marginparwidth}
%   \renewcommand{\headrulewidth}{0.4pt}
%   \renewcommand{\footrulewidth}{0.4pt}
%   \fancyhead[LE,RO]{\LHeadFont\emph{\leftmark\/}\SBContinueMark}
%   \fancyhead[CE,CO]{\CHeadFont\thepage}
%   \fancyhead[RE,LO]{\RHeadFont \chaptermark}
% \else\ifOverhead
%   % It's an overhead...
%   \renewcommand{\footrulewidth}{0pt}
%   \renewcommand{\headrulewidth}{0pt}
%   \fancyhead[LE,RO]{}
%   \fancyhead[CE,CO]{}
%   \fancyhead[RE,LO]{}
% \else\ifWordBk
%   % It's a words only songbook...
%   \addtolength{\headwidth}{\marginparsep}
%   \addtolength{\headwidth}{\marginparwidth}
%   \renewcommand{\headrulewidth}{0.4pt}
%   \renewcommand{\footrulewidth}{0.4pt}
%   \fancyhead[LE,RO]{\LHeadFont Naturvidenskab revy sange}
%   \fancyhead[CE,CO]{\CHeadFont\thepage}
%   \fancyhead[RE,LO]{\RHeadFont \SBThechapter}
% \fi\fi\fi

% \fancyfoot[LE,RO]{\LFootFont Computer Science Camp 2019}
% \ifSongEject
%   \fancyfoot[CE,CO]{\CFootFont Last Revised:  \RevDate}
% \else
%   \fancyfoot[CE,CO]{\CFootFont}
% \fi
% \fancyfoot[RE,LO]{\RFootFont Synges på eget ansvar}

%%%
% Table of contents
%%%

% \clearpage
% \twocolumn
% \font\myTinySF=cmss8    at  8pt
% \font\myHugeSF=cmssbx10 at 25pt
% \newcommand{\CpyRtInfoFont}{\tiny\myTinySF}
% \newcommand{\myTitleFont}{\Huge\myHugeSF}
% \newcommand{\mySubTitleFont}{\large\sf}
% \renewcommand{\indexspace}{\medskip}

% % {\parindent 8pt
% %   {\myTitleFont Indhold}}\par
% % \vskip 5pt
% \renewcommand{\SBThechapter}{Indhold}
% % {\parindent 20pt
% %   {\mySubTitleFont --- with first lines in italic ---}}
% % \vskip 20pt
% \let\olditem\item
% \let\oldsubitem\subitem
% \let\oldsubsubitem\subsubitem
% \renewcommand{\item}{\par\hangindent=40pt}
% \renewcommand{\subitem}{\par\hangindent=40pt \hspace*{20pt}}
% \renewcommand{\subsubitem}{\par\hangindent=40pt \hspace*{30pt}}

% %\input{unf-sangbog.tocx}

% \renewcommand{\item}{\olditem}
% \renewcommand{\subitem}{\oldsubitem}
% \renewcommand{\subsubitem}{\oldsubsubitem}

%%%
% Songbook begins.
%%%

\twocolumn
%It's just one page, don't print page numbers etc.
\pagestyle{empty}
%Songs included
\input{songs/matmatik.tex}
\input{songs/taal_daj.tex}
\input{songs/linieskriverdriver.tex}
\input{songs/steve_hawking.tex}
\input{songs/ode_til_kode.tex}
\input{songs/se_min_kode.tex}
\input{songs/vaabenfysik_kort.tex}
%Maybe include:
%\input{songs/kvanter_i_maaneskin.tex}
%\input{songs/mest_matematiske_dyr.tex}

% \input{songs/vi_kan_ikke_li.tex}
% \input{songs/selektionssangen.tex}
% \input{songs/alfabetsangen.tex}
% \input{songs/sciencecamps.tex}
% \input{songs/hvad_maa_man.tex}


% \input{songs/lambda_kalkylen.tex}
% \input{songs/puslespil.tex}
% \input{songs/null.tex}
% \input{songs/fasebal.tex}

% \input{songs/chifitter.tex}

% \input{songs/kun_fysik.tex}



% \input{songs/kanoniske.tex}
% \input{songs/jeg_er_en_matematiker_fra_hcoe.tex}


% \input{songs/rekursiv_skovsang.tex}
% \input{songs/laerkerede.tex}


% \clearpage
% \font\myTinySF=cmss8    at  8pt
% \font\myHugeSF=cmssbx10 at 25pt
% % \newcommand{\CpyRtInfoFont}{\tiny\myTinySF}
% % \newcommand{\myTitleFont}{\Huge\myHugeSF}
% % \newcommand{\mySubTitleFont}{\large\sf}
% \renewcommand{\indexspace}{\medskip}

% {\parindent 8pt
%   {\myTitleFont Index}}\par
% \vskip 5pt
% \renewcommand{\SBThechapter}{Index}
% % {\parindent 20pt
% %   {\mySubTitleFont --- with first lines in italic ---}}
% % \vskip 20pt
% \renewcommand{\item}{\par\hangindent=40pt}
% \renewcommand{\subitem}{\par\hangindent=40pt \hspace*{20pt}}
% \renewcommand{\subsubitem}{\par\hangindent=40pt \hspace*{30pt}}

%\input{unf-sangbog.tdx}

\end{document}
\bye
%
%%%
% Document ends.
%%%


\end{document}
\bye
%
%%%
% Document ends.
%%%

%       \begin{minipage}{.8\textwidth}
%         \printcontents{}{1}{}
%       \end{minipage}%
%     \renewcommand{\SBThechapter}{#1}
%     \clearpage
% }

% \titleformat{\chapter}
% [display]
% {}
% {%\vspace*{\fill}
%  % \titlerule[1pt]%
%  % \vspace{1pt}%
%  % \titlerule
%  % \vspace{1pc}%
%  \chaptertitlename}
% {}
% {\Huge}



%%%%%%%%%%%%%%%%%%%%%%%%%%%%%%%%%%%%%%%%%%%%%%%%%%%%%%%%%%
%%%%%%%%%%%%%%%%%%%%%%%%%%%%%%%%%%%%%%%%%%%%%%%%%%%%%%%%%%
%%                                                      %%
%%           D O C U M E N T   B E G I N S              %%
%%                                                      %%
%%%%%%%%%%%%%%%%%%%%%%%%%%%%%%%%%%%%%%%%%%%%%%%%%%%%%%%%%%
%%%%%%%%%%%%%%%%%%%%%%%%%%%%%%%%%%%%%%%%%%%%%%%%%%%%%%%%%%
\begin{document}

%%%
% Uncomment "\maketitle" statement to make a title page.
%%%
%\maketitle
% \begin{titlepage}
%   \centering
%   \vspace{5cm}
% 	\includegraphics[width=1\textwidth]{unf_logo.jpeg}\par\vspace{1cm}
% 	{\scshape\LARGE Sangbog \par}
% 	\vspace{1cm}
% 	{\scshape\Large UNF Computer Science Camp 2019\par}
	
% 	\vfill

% % Bottom of the page
% 	{\large \today\par}
% \end{titlepage}
% \mainmatter
% \ifWordBk
%   \twocolumn
% \fi


%%% Kolofon
%\thispagestyle{empty}
%Sammensat til UNF Computer Science Camp 2019 - csc.unf.dk\\
%Redaktør: Andreas Mosbæk Jensen m.fl. efter tidligere sangbog af Steffen Strunge Mathiesen\\
%Indhold opsat i \LaTeX. 
%Digital version og kildekode: github.com/steffen555/UNF-sangbog\\
%Revision 1 med stave fejl korrektioner
%\par\vspace*{\fill}
%Hvis du har forslag til sange, rettelser, ris og ros, eller hvis du kender en ukendt forfatter, så skriv til sangbog@unf.dk.

%%%
% Turn on and define fancy page heading/footing definition.
%%%
% \pagestyle{fancy}

% \ifChordBk
%   % It's a words & chords songbook...
%   \addtolength{\headwidth}{\marginparsep}
%   \addtolength{\headwidth}{\marginparwidth}
%   \renewcommand{\headrulewidth}{0.4pt}
%   \renewcommand{\footrulewidth}{0.4pt}
%   \fancyhead[LE,RO]{\LHeadFont\emph{\leftmark\/}\SBContinueMark}
%   \fancyhead[CE,CO]{\CHeadFont\thepage}
%   \fancyhead[RE,LO]{\RHeadFont \chaptermark}
% \else\ifOverhead
%   % It's an overhead...
%   \renewcommand{\footrulewidth}{0pt}
%   \renewcommand{\headrulewidth}{0pt}
%   \fancyhead[LE,RO]{}
%   \fancyhead[CE,CO]{}
%   \fancyhead[RE,LO]{}
% \else\ifWordBk
%   % It's a words only songbook...
%   \addtolength{\headwidth}{\marginparsep}
%   \addtolength{\headwidth}{\marginparwidth}
%   \renewcommand{\headrulewidth}{0.4pt}
%   \renewcommand{\footrulewidth}{0.4pt}
%   \fancyhead[LE,RO]{\LHeadFont Naturvidenskab revy sange}
%   \fancyhead[CE,CO]{\CHeadFont\thepage}
%   \fancyhead[RE,LO]{\RHeadFont \SBThechapter}
% \fi\fi\fi

% \fancyfoot[LE,RO]{\LFootFont Computer Science Camp 2019}
% \ifSongEject
%   \fancyfoot[CE,CO]{\CFootFont Last Revised:  \RevDate}
% \else
%   \fancyfoot[CE,CO]{\CFootFont}
% \fi
% \fancyfoot[RE,LO]{\RFootFont Synges på eget ansvar}

%%%
% Table of contents
%%%

% \clearpage
% \twocolumn
% \font\myTinySF=cmss8    at  8pt
% \font\myHugeSF=cmssbx10 at 25pt
% \newcommand{\CpyRtInfoFont}{\tiny\myTinySF}
% \newcommand{\myTitleFont}{\Huge\myHugeSF}
% \newcommand{\mySubTitleFont}{\large\sf}
% \renewcommand{\indexspace}{\medskip}

% % {\parindent 8pt
% %   {\myTitleFont Indhold}}\par
% % \vskip 5pt
% \renewcommand{\SBThechapter}{Indhold}
% % {\parindent 20pt
% %   {\mySubTitleFont --- with first lines in italic ---}}
% % \vskip 20pt
% \let\olditem\item
% \let\oldsubitem\subitem
% \let\oldsubsubitem\subsubitem
% \renewcommand{\item}{\par\hangindent=40pt}
% \renewcommand{\subitem}{\par\hangindent=40pt \hspace*{20pt}}
% \renewcommand{\subsubitem}{\par\hangindent=40pt \hspace*{30pt}}

% %%%%%%% rcsid = @(#)$Id: sample-sb.tex,v 1.23 2010-04-12 18:04:11 rathc Exp $
%%%%%%
%%
%%      ===============================
%%      Sample Songbook (sample-sb.tex)
%%      ===============================
%%
%%      Version 4.5, 30 April, 2010
%%
%%      Copyright 1992--2010 Christopher Rath <christopher@rath.ca>
%%
%%      This package is free software; you can redistribute it and/or
%%      modify it under the terms of version 2.1 of the GNU Lesser
%%	General Public License as published by the Free Software 
%%	Foundation.
%%
%%      This package is distributed in the hope that it will be
%%      useful, but WITHOUT ANY WARRANTY; without even the implied
%%      warranty of MERCHANTABILITY or FITNESS FOR A PARTICULAR
%%      PURPOSE.  See the GNU Lesser General Public License for more
%%      details.
%%
%%      This file contains a subset of the songbook we distribute
%%      at our church.  To the best of my knowledge, all of the lyrics
%%      contained herein are freely distributable.  This file has been
%%      provided as a sample of what can be produced by the chordbk,
%%      wordbk, and overhead LaTeX styles.
%%
%%      NEEDED:  The fancyhdr LaTeX style is required to properly
%%              format this file.  If you don't have that then comment
%%              out the commands in the preamble which deal with the
%%              fancyhdr style.
%%
%%%%%%
%%%%%%
%%
%%      1. Chord notation.  Within this songbook the following
%%         conventions have been adopted:
%%
%%              "Minor" is entered as "m";
%%                      e.g. Cm7 for C minor 7th.
%%              "Major" is entered as "M";
%%                      e.g. CM7 for C major 7th.
%%
%%%%%%
%%%%%%
%%      ============
%%      Bibliography
%%      ============
%%
%%      Exalt Him!: Exalt Him!  Compiled by Tom Fettke.  (c)1989
%%                      Word Music.
%%
%%      Hosanna! Music Books: Hosanna! Music Books #1--#6.
%%                      (c)1987--92 Integrity Music, Inc.
%%
%%      Worship Him II: Worship Him II.  Compiled by Jesse Peterson
%%                      and Bruce Ballinger.  (c)1989 Tempo Music
%%                      Publications.
%%
%%      Worship Songs Of The Vineyard: Worship Songs Of The Vineyard
%%                      --- Volume 2.  (c)1989 Vineyard Ministries
%%                      International.
%%
%%%%%%
%%%%%%

%%%%%%%%%%%%%%%%%%%%%%%%%%%%%%%%%%%%%%%%%%%%%%%%%%%%%%%%%%
%%%%%%%%%%%%%%%%%%%%%%%%%%%%%%%%%%%%%%%%%%%%%%%%%%%%%%%%%%
%%                                                      %%
%%           P R E A M B L E   B E G I N S              %%
%%                                                      %%
%%%%%%%%%%%%%%%%%%%%%%%%%%%%%%%%%%%%%%%%%%%%%%%%%%%%%%%%%%
%%%%%%%%%%%%%%%%%%%%%%%%%%%%%%%%%%%%%%%%%%%%%%%%%%%%%%%%%%

\documentclass[a5paper]{book}
\usepackage{latexsym,
            fancyhdr,
            titlesec,
            amsmath,
            amssymb,
            multicol,
            amsthm,
            stmaryrd,
            amsthm,
            color,
            needspace,
            stackengine,
            wasysym}
\usepackage[utf8]{inputenc}
\usepackage[T1]{fontenc}
% \usepackage[chordbk]{songbook}                  %% Words & Chords edition.
%%\usepackage[compactallsongs,chordbk]{songbook}    %% Words & Chords edition.
\usepackage[wordbk]{songbook}                 %% Words Only edition.
%%\usepackage[overhead]{songbook}               %% Overhead Transparency edition.
\usepackage{titletoc}
\usepackage{tket}  % Draws "TÅGEKAMMERET" correctly

%%%
% Revision Date and Release Date definitions.
%
%       \RelDate - The last time this songbook was released.  Set this
%                  date each time a new release/update of the songbook
%                  is generated.
%       \RevDate - The last time a particular song was revised in any
%                  way.  This command will be renewed inside every
%                  song.
%%%
\newcommand{\RelDate}{31~August,~2003}
\newcommand{\RevDate}{\today}

%%%
% C.C.L.I. license number definition; for copyright licensing info.
% One of these macros will be manually inserted into the {SBMel}
% parameter of the {song} environment.
%
%       \CCLInumber - The actual copyright license number.  Don't
%               insert this command in the {SBMel} parameter, use one
%               of the others.
%       \CCLIed - Indicates a song falls under our CCLI license.
%       \NotCCLIed - Indicates a song doesn't fall under our CCLI
%               license.  Public Domain songs fall into this category.
%       \PGranted - We have received specific permission from the
%               copyright holder to use this song.
%       \PPending - We are in the process of obtaining permission to
%               use this song.
%%%
\newcommand{\CCLInumber}{Your CCLI Number}
\newcommand{\CCLIed}{{\SBMelInfoFont (CCLI \CCLInumber)}}
\newcommand{\NotCCLIed}{\relax}
\newcommand{\PGranted}{\relax}
\newcommand{\PPending}{{\SBMelInfoFont (Permission Pending)}}

%%%
% Title page information.
%%%
%\title{UNF Computer Science Camp 2019 Sangbog}
%\author{}
%\date{Revideret:  \RevDate}

%%%
% Redefine fonts from SongBook style that I don't like.
%%%
\font\myTinySF=cmss8 at 8pt
\renewcommand{\SBMelInfoFont}{\tiny\myTinySF}

%%%
% Define fonts to use in the headers and footers of the songbook.
%%%
\newcommand{\LHeadFont}{\normalsize}            % = cmr12  at 12pt
\newcommand{\CHeadFont}{\normalsize\rm}         % = cmr12  at 12pt
\newcommand{\RHeadFont}{\normalsize}            % = cmr12  at 12pt
\newcommand{\LFootFont}{\scriptsize}            % = cmr8   at  8pt
\newcommand{\CFootFont}{\tiny\myTinySF}         % = cmss8  at  8pt
\newcommand{\RFootFont}{\scriptsize}            % = cmr8   at  8pt

\def\repeat{%
  \stackanchor{.}{.}%
  \rule[-\dp\strutbox]{.3pt}{\normalbaselineskip}%
  \kern0.5pt%
  \rule[-\dp\strutbox]{1pt}{\normalbaselineskip}%
  \kern1pt%
}
\def\frepeat{%
  \kern1pt%
  \rule[-\dp\strutbox]{1pt}{\normalbaselineskip}%
  \kern0.5pt%
  \rule[-\dp\strutbox]{.3pt}{\normalbaselineskip}%
  \stackanchor{.}{.}%
}
% \newcommand{\SBRepeat}[1]{#1\\#1}
\newcommand{\SBRepeat}[1]{\frepeat #1\repeat}
\setcounter{SBSongCnt}{-1}
\renewcommand{\SBWAndMTag}{Forfatter:}
\renewcommand{\SBUnknownTag}{Ukendt}
\renewcommand{\SBChorusTag}{Ref.}
\renewcommand{\SBOrgMel}{Originalmelodi}
\renewcommand{\SpaceAfterChorus}   {\vspace{0ex plus1ex minus 0.5ex}}
\renewcommand{\SpaceAfterOpGroup}  {\vspace{0ex plus1ex minus 0.5ex}}
\renewcommand{\SpaceAfterSBBracket}{\vspace{0ex plus1ex minus 0.5ex}}
\renewcommand{\SpaceAfterSection}  {\vspace{0ex plus1ex minus 0.5ex}}
\renewcommand{\SpaceAfterSong}     {\vspace{0ex plus1ex minus 0.5ex}}
\renewcommand{\SpaceAfterVerse}    {\vspace{0ex plus1ex minus 0.5ex}}

% Tell LaTeX that \medskip is a good place to make a page break
\let\oldmedskip\medskip
\renewcommand{\medskip}{\oldmedskip\pagebreak[2]}

%%%
% Turn on/off index-file generation.  Uncomment the \makeindex line to
% turn index generation on;  comment it out to turn index generation
% off.
%%%
%\makeTitleIndex         %% Title and First Line Index.
%\makeTitleContents      %% Table of Contents.
%\makeKeyIndex           %% Index of song by key.
% \makeArtistIndex	%% Index of song by artist.
% \newcommand{\SBThechapter}[0]{}
% \newcommand{\SBChapter}[1]{
%     \startcontents
%     \chapter*{#1} 
%     % %%%%%% rcsid = @(#)$Id: sample-sb.tex,v 1.23 2010-04-12 18:04:11 rathc Exp $
%%%%%%
%%
%%      ===============================
%%      Sample Songbook (sample-sb.tex)
%%      ===============================
%%
%%      Version 4.5, 30 April, 2010
%%
%%      Copyright 1992--2010 Christopher Rath <christopher@rath.ca>
%%
%%      This package is free software; you can redistribute it and/or
%%      modify it under the terms of version 2.1 of the GNU Lesser
%%	General Public License as published by the Free Software 
%%	Foundation.
%%
%%      This package is distributed in the hope that it will be
%%      useful, but WITHOUT ANY WARRANTY; without even the implied
%%      warranty of MERCHANTABILITY or FITNESS FOR A PARTICULAR
%%      PURPOSE.  See the GNU Lesser General Public License for more
%%      details.
%%
%%      This file contains a subset of the songbook we distribute
%%      at our church.  To the best of my knowledge, all of the lyrics
%%      contained herein are freely distributable.  This file has been
%%      provided as a sample of what can be produced by the chordbk,
%%      wordbk, and overhead LaTeX styles.
%%
%%      NEEDED:  The fancyhdr LaTeX style is required to properly
%%              format this file.  If you don't have that then comment
%%              out the commands in the preamble which deal with the
%%              fancyhdr style.
%%
%%%%%%
%%%%%%
%%
%%      1. Chord notation.  Within this songbook the following
%%         conventions have been adopted:
%%
%%              "Minor" is entered as "m";
%%                      e.g. Cm7 for C minor 7th.
%%              "Major" is entered as "M";
%%                      e.g. CM7 for C major 7th.
%%
%%%%%%
%%%%%%
%%      ============
%%      Bibliography
%%      ============
%%
%%      Exalt Him!: Exalt Him!  Compiled by Tom Fettke.  (c)1989
%%                      Word Music.
%%
%%      Hosanna! Music Books: Hosanna! Music Books #1--#6.
%%                      (c)1987--92 Integrity Music, Inc.
%%
%%      Worship Him II: Worship Him II.  Compiled by Jesse Peterson
%%                      and Bruce Ballinger.  (c)1989 Tempo Music
%%                      Publications.
%%
%%      Worship Songs Of The Vineyard: Worship Songs Of The Vineyard
%%                      --- Volume 2.  (c)1989 Vineyard Ministries
%%                      International.
%%
%%%%%%
%%%%%%

%%%%%%%%%%%%%%%%%%%%%%%%%%%%%%%%%%%%%%%%%%%%%%%%%%%%%%%%%%
%%%%%%%%%%%%%%%%%%%%%%%%%%%%%%%%%%%%%%%%%%%%%%%%%%%%%%%%%%
%%                                                      %%
%%           P R E A M B L E   B E G I N S              %%
%%                                                      %%
%%%%%%%%%%%%%%%%%%%%%%%%%%%%%%%%%%%%%%%%%%%%%%%%%%%%%%%%%%
%%%%%%%%%%%%%%%%%%%%%%%%%%%%%%%%%%%%%%%%%%%%%%%%%%%%%%%%%%

\documentclass[a5paper]{book}
\usepackage{latexsym,
            fancyhdr,
            titlesec,
            amsmath,
            amssymb,
            multicol,
            amsthm,
            stmaryrd,
            amsthm,
            color,
            needspace,
            stackengine,
            wasysym}
\usepackage[utf8]{inputenc}
\usepackage[T1]{fontenc}
% \usepackage[chordbk]{songbook}                  %% Words & Chords edition.
%%\usepackage[compactallsongs,chordbk]{songbook}    %% Words & Chords edition.
\usepackage[wordbk]{songbook}                 %% Words Only edition.
%%\usepackage[overhead]{songbook}               %% Overhead Transparency edition.
\usepackage{titletoc}
\usepackage{tket}  % Draws "TÅGEKAMMERET" correctly

%%%
% Revision Date and Release Date definitions.
%
%       \RelDate - The last time this songbook was released.  Set this
%                  date each time a new release/update of the songbook
%                  is generated.
%       \RevDate - The last time a particular song was revised in any
%                  way.  This command will be renewed inside every
%                  song.
%%%
\newcommand{\RelDate}{31~August,~2003}
\newcommand{\RevDate}{\today}

%%%
% C.C.L.I. license number definition; for copyright licensing info.
% One of these macros will be manually inserted into the {SBMel}
% parameter of the {song} environment.
%
%       \CCLInumber - The actual copyright license number.  Don't
%               insert this command in the {SBMel} parameter, use one
%               of the others.
%       \CCLIed - Indicates a song falls under our CCLI license.
%       \NotCCLIed - Indicates a song doesn't fall under our CCLI
%               license.  Public Domain songs fall into this category.
%       \PGranted - We have received specific permission from the
%               copyright holder to use this song.
%       \PPending - We are in the process of obtaining permission to
%               use this song.
%%%
\newcommand{\CCLInumber}{Your CCLI Number}
\newcommand{\CCLIed}{{\SBMelInfoFont (CCLI \CCLInumber)}}
\newcommand{\NotCCLIed}{\relax}
\newcommand{\PGranted}{\relax}
\newcommand{\PPending}{{\SBMelInfoFont (Permission Pending)}}

%%%
% Title page information.
%%%
%\title{UNF Computer Science Camp 2019 Sangbog}
%\author{}
%\date{Revideret:  \RevDate}

%%%
% Redefine fonts from SongBook style that I don't like.
%%%
\font\myTinySF=cmss8 at 8pt
\renewcommand{\SBMelInfoFont}{\tiny\myTinySF}

%%%
% Define fonts to use in the headers and footers of the songbook.
%%%
\newcommand{\LHeadFont}{\normalsize}            % = cmr12  at 12pt
\newcommand{\CHeadFont}{\normalsize\rm}         % = cmr12  at 12pt
\newcommand{\RHeadFont}{\normalsize}            % = cmr12  at 12pt
\newcommand{\LFootFont}{\scriptsize}            % = cmr8   at  8pt
\newcommand{\CFootFont}{\tiny\myTinySF}         % = cmss8  at  8pt
\newcommand{\RFootFont}{\scriptsize}            % = cmr8   at  8pt

\def\repeat{%
  \stackanchor{.}{.}%
  \rule[-\dp\strutbox]{.3pt}{\normalbaselineskip}%
  \kern0.5pt%
  \rule[-\dp\strutbox]{1pt}{\normalbaselineskip}%
  \kern1pt%
}
\def\frepeat{%
  \kern1pt%
  \rule[-\dp\strutbox]{1pt}{\normalbaselineskip}%
  \kern0.5pt%
  \rule[-\dp\strutbox]{.3pt}{\normalbaselineskip}%
  \stackanchor{.}{.}%
}
% \newcommand{\SBRepeat}[1]{#1\\#1}
\newcommand{\SBRepeat}[1]{\frepeat #1\repeat}
\setcounter{SBSongCnt}{-1}
\renewcommand{\SBWAndMTag}{Forfatter:}
\renewcommand{\SBUnknownTag}{Ukendt}
\renewcommand{\SBChorusTag}{Ref.}
\renewcommand{\SBOrgMel}{Originalmelodi}
\renewcommand{\SpaceAfterChorus}   {\vspace{0ex plus1ex minus 0.5ex}}
\renewcommand{\SpaceAfterOpGroup}  {\vspace{0ex plus1ex minus 0.5ex}}
\renewcommand{\SpaceAfterSBBracket}{\vspace{0ex plus1ex minus 0.5ex}}
\renewcommand{\SpaceAfterSection}  {\vspace{0ex plus1ex minus 0.5ex}}
\renewcommand{\SpaceAfterSong}     {\vspace{0ex plus1ex minus 0.5ex}}
\renewcommand{\SpaceAfterVerse}    {\vspace{0ex plus1ex minus 0.5ex}}

% Tell LaTeX that \medskip is a good place to make a page break
\let\oldmedskip\medskip
\renewcommand{\medskip}{\oldmedskip\pagebreak[2]}

%%%
% Turn on/off index-file generation.  Uncomment the \makeindex line to
% turn index generation on;  comment it out to turn index generation
% off.
%%%
%\makeTitleIndex         %% Title and First Line Index.
%\makeTitleContents      %% Table of Contents.
%\makeKeyIndex           %% Index of song by key.
% \makeArtistIndex	%% Index of song by artist.
% \newcommand{\SBThechapter}[0]{}
% \newcommand{\SBChapter}[1]{
%     \startcontents
%     \chapter*{#1} 
%     % \input{unf-sangbog.toc}
%       \begin{minipage}{.8\textwidth}
%         \printcontents{}{1}{}
%       \end{minipage}%
%     \renewcommand{\SBThechapter}{#1}
%     \clearpage
% }

% \titleformat{\chapter}
% [display]
% {}
% {%\vspace*{\fill}
%  % \titlerule[1pt]%
%  % \vspace{1pt}%
%  % \titlerule
%  % \vspace{1pc}%
%  \chaptertitlename}
% {}
% {\Huge}



%%%%%%%%%%%%%%%%%%%%%%%%%%%%%%%%%%%%%%%%%%%%%%%%%%%%%%%%%%
%%%%%%%%%%%%%%%%%%%%%%%%%%%%%%%%%%%%%%%%%%%%%%%%%%%%%%%%%%
%%                                                      %%
%%           D O C U M E N T   B E G I N S              %%
%%                                                      %%
%%%%%%%%%%%%%%%%%%%%%%%%%%%%%%%%%%%%%%%%%%%%%%%%%%%%%%%%%%
%%%%%%%%%%%%%%%%%%%%%%%%%%%%%%%%%%%%%%%%%%%%%%%%%%%%%%%%%%
\begin{document}

%%%
% Uncomment "\maketitle" statement to make a title page.
%%%
%\maketitle
% \begin{titlepage}
%   \centering
%   \vspace{5cm}
% 	\includegraphics[width=1\textwidth]{unf_logo.jpeg}\par\vspace{1cm}
% 	{\scshape\LARGE Sangbog \par}
% 	\vspace{1cm}
% 	{\scshape\Large UNF Computer Science Camp 2019\par}
	
% 	\vfill

% % Bottom of the page
% 	{\large \today\par}
% \end{titlepage}
% \mainmatter
% \ifWordBk
%   \twocolumn
% \fi


%%% Kolofon
%\thispagestyle{empty}
%Sammensat til UNF Computer Science Camp 2019 - csc.unf.dk\\
%Redaktør: Andreas Mosbæk Jensen m.fl. efter tidligere sangbog af Steffen Strunge Mathiesen\\
%Indhold opsat i \LaTeX. 
%Digital version og kildekode: github.com/steffen555/UNF-sangbog\\
%Revision 1 med stave fejl korrektioner
%\par\vspace*{\fill}
%Hvis du har forslag til sange, rettelser, ris og ros, eller hvis du kender en ukendt forfatter, så skriv til sangbog@unf.dk.

%%%
% Turn on and define fancy page heading/footing definition.
%%%
% \pagestyle{fancy}

% \ifChordBk
%   % It's a words & chords songbook...
%   \addtolength{\headwidth}{\marginparsep}
%   \addtolength{\headwidth}{\marginparwidth}
%   \renewcommand{\headrulewidth}{0.4pt}
%   \renewcommand{\footrulewidth}{0.4pt}
%   \fancyhead[LE,RO]{\LHeadFont\emph{\leftmark\/}\SBContinueMark}
%   \fancyhead[CE,CO]{\CHeadFont\thepage}
%   \fancyhead[RE,LO]{\RHeadFont \chaptermark}
% \else\ifOverhead
%   % It's an overhead...
%   \renewcommand{\footrulewidth}{0pt}
%   \renewcommand{\headrulewidth}{0pt}
%   \fancyhead[LE,RO]{}
%   \fancyhead[CE,CO]{}
%   \fancyhead[RE,LO]{}
% \else\ifWordBk
%   % It's a words only songbook...
%   \addtolength{\headwidth}{\marginparsep}
%   \addtolength{\headwidth}{\marginparwidth}
%   \renewcommand{\headrulewidth}{0.4pt}
%   \renewcommand{\footrulewidth}{0.4pt}
%   \fancyhead[LE,RO]{\LHeadFont Naturvidenskab revy sange}
%   \fancyhead[CE,CO]{\CHeadFont\thepage}
%   \fancyhead[RE,LO]{\RHeadFont \SBThechapter}
% \fi\fi\fi

% \fancyfoot[LE,RO]{\LFootFont Computer Science Camp 2019}
% \ifSongEject
%   \fancyfoot[CE,CO]{\CFootFont Last Revised:  \RevDate}
% \else
%   \fancyfoot[CE,CO]{\CFootFont}
% \fi
% \fancyfoot[RE,LO]{\RFootFont Synges på eget ansvar}

%%%
% Table of contents
%%%

% \clearpage
% \twocolumn
% \font\myTinySF=cmss8    at  8pt
% \font\myHugeSF=cmssbx10 at 25pt
% \newcommand{\CpyRtInfoFont}{\tiny\myTinySF}
% \newcommand{\myTitleFont}{\Huge\myHugeSF}
% \newcommand{\mySubTitleFont}{\large\sf}
% \renewcommand{\indexspace}{\medskip}

% % {\parindent 8pt
% %   {\myTitleFont Indhold}}\par
% % \vskip 5pt
% \renewcommand{\SBThechapter}{Indhold}
% % {\parindent 20pt
% %   {\mySubTitleFont --- with first lines in italic ---}}
% % \vskip 20pt
% \let\olditem\item
% \let\oldsubitem\subitem
% \let\oldsubsubitem\subsubitem
% \renewcommand{\item}{\par\hangindent=40pt}
% \renewcommand{\subitem}{\par\hangindent=40pt \hspace*{20pt}}
% \renewcommand{\subsubitem}{\par\hangindent=40pt \hspace*{30pt}}

% %\input{unf-sangbog.tocx}

% \renewcommand{\item}{\olditem}
% \renewcommand{\subitem}{\oldsubitem}
% \renewcommand{\subsubitem}{\oldsubsubitem}

%%%
% Songbook begins.
%%%

\twocolumn
%It's just one page, don't print page numbers etc.
\pagestyle{empty}
%Songs included
\input{songs/matmatik.tex}
\input{songs/taal_daj.tex}
\input{songs/linieskriverdriver.tex}
\input{songs/steve_hawking.tex}
\input{songs/ode_til_kode.tex}
\input{songs/se_min_kode.tex}
\input{songs/vaabenfysik_kort.tex}
%Maybe include:
%\input{songs/kvanter_i_maaneskin.tex}
%\input{songs/mest_matematiske_dyr.tex}

% \input{songs/vi_kan_ikke_li.tex}
% \input{songs/selektionssangen.tex}
% \input{songs/alfabetsangen.tex}
% \input{songs/sciencecamps.tex}
% \input{songs/hvad_maa_man.tex}


% \input{songs/lambda_kalkylen.tex}
% \input{songs/puslespil.tex}
% \input{songs/null.tex}
% \input{songs/fasebal.tex}

% \input{songs/chifitter.tex}

% \input{songs/kun_fysik.tex}



% \input{songs/kanoniske.tex}
% \input{songs/jeg_er_en_matematiker_fra_hcoe.tex}


% \input{songs/rekursiv_skovsang.tex}
% \input{songs/laerkerede.tex}


% \clearpage
% \font\myTinySF=cmss8    at  8pt
% \font\myHugeSF=cmssbx10 at 25pt
% % \newcommand{\CpyRtInfoFont}{\tiny\myTinySF}
% % \newcommand{\myTitleFont}{\Huge\myHugeSF}
% % \newcommand{\mySubTitleFont}{\large\sf}
% \renewcommand{\indexspace}{\medskip}

% {\parindent 8pt
%   {\myTitleFont Index}}\par
% \vskip 5pt
% \renewcommand{\SBThechapter}{Index}
% % {\parindent 20pt
% %   {\mySubTitleFont --- with first lines in italic ---}}
% % \vskip 20pt
% \renewcommand{\item}{\par\hangindent=40pt}
% \renewcommand{\subitem}{\par\hangindent=40pt \hspace*{20pt}}
% \renewcommand{\subsubitem}{\par\hangindent=40pt \hspace*{30pt}}

%\input{unf-sangbog.tdx}

\end{document}
\bye
%
%%%
% Document ends.
%%%

%       \begin{minipage}{.8\textwidth}
%         \printcontents{}{1}{}
%       \end{minipage}%
%     \renewcommand{\SBThechapter}{#1}
%     \clearpage
% }

% \titleformat{\chapter}
% [display]
% {}
% {%\vspace*{\fill}
%  % \titlerule[1pt]%
%  % \vspace{1pt}%
%  % \titlerule
%  % \vspace{1pc}%
%  \chaptertitlename}
% {}
% {\Huge}



%%%%%%%%%%%%%%%%%%%%%%%%%%%%%%%%%%%%%%%%%%%%%%%%%%%%%%%%%%
%%%%%%%%%%%%%%%%%%%%%%%%%%%%%%%%%%%%%%%%%%%%%%%%%%%%%%%%%%
%%                                                      %%
%%           D O C U M E N T   B E G I N S              %%
%%                                                      %%
%%%%%%%%%%%%%%%%%%%%%%%%%%%%%%%%%%%%%%%%%%%%%%%%%%%%%%%%%%
%%%%%%%%%%%%%%%%%%%%%%%%%%%%%%%%%%%%%%%%%%%%%%%%%%%%%%%%%%
\begin{document}

%%%
% Uncomment "\maketitle" statement to make a title page.
%%%
%\maketitle
% \begin{titlepage}
%   \centering
%   \vspace{5cm}
% 	\includegraphics[width=1\textwidth]{unf_logo.jpeg}\par\vspace{1cm}
% 	{\scshape\LARGE Sangbog \par}
% 	\vspace{1cm}
% 	{\scshape\Large UNF Computer Science Camp 2019\par}
	
% 	\vfill

% % Bottom of the page
% 	{\large \today\par}
% \end{titlepage}
% \mainmatter
% \ifWordBk
%   \twocolumn
% \fi


%%% Kolofon
%\thispagestyle{empty}
%Sammensat til UNF Computer Science Camp 2019 - csc.unf.dk\\
%Redaktør: Andreas Mosbæk Jensen m.fl. efter tidligere sangbog af Steffen Strunge Mathiesen\\
%Indhold opsat i \LaTeX. 
%Digital version og kildekode: github.com/steffen555/UNF-sangbog\\
%Revision 1 med stave fejl korrektioner
%\par\vspace*{\fill}
%Hvis du har forslag til sange, rettelser, ris og ros, eller hvis du kender en ukendt forfatter, så skriv til sangbog@unf.dk.

%%%
% Turn on and define fancy page heading/footing definition.
%%%
% \pagestyle{fancy}

% \ifChordBk
%   % It's a words & chords songbook...
%   \addtolength{\headwidth}{\marginparsep}
%   \addtolength{\headwidth}{\marginparwidth}
%   \renewcommand{\headrulewidth}{0.4pt}
%   \renewcommand{\footrulewidth}{0.4pt}
%   \fancyhead[LE,RO]{\LHeadFont\emph{\leftmark\/}\SBContinueMark}
%   \fancyhead[CE,CO]{\CHeadFont\thepage}
%   \fancyhead[RE,LO]{\RHeadFont \chaptermark}
% \else\ifOverhead
%   % It's an overhead...
%   \renewcommand{\footrulewidth}{0pt}
%   \renewcommand{\headrulewidth}{0pt}
%   \fancyhead[LE,RO]{}
%   \fancyhead[CE,CO]{}
%   \fancyhead[RE,LO]{}
% \else\ifWordBk
%   % It's a words only songbook...
%   \addtolength{\headwidth}{\marginparsep}
%   \addtolength{\headwidth}{\marginparwidth}
%   \renewcommand{\headrulewidth}{0.4pt}
%   \renewcommand{\footrulewidth}{0.4pt}
%   \fancyhead[LE,RO]{\LHeadFont Naturvidenskab revy sange}
%   \fancyhead[CE,CO]{\CHeadFont\thepage}
%   \fancyhead[RE,LO]{\RHeadFont \SBThechapter}
% \fi\fi\fi

% \fancyfoot[LE,RO]{\LFootFont Computer Science Camp 2019}
% \ifSongEject
%   \fancyfoot[CE,CO]{\CFootFont Last Revised:  \RevDate}
% \else
%   \fancyfoot[CE,CO]{\CFootFont}
% \fi
% \fancyfoot[RE,LO]{\RFootFont Synges på eget ansvar}

%%%
% Table of contents
%%%

% \clearpage
% \twocolumn
% \font\myTinySF=cmss8    at  8pt
% \font\myHugeSF=cmssbx10 at 25pt
% \newcommand{\CpyRtInfoFont}{\tiny\myTinySF}
% \newcommand{\myTitleFont}{\Huge\myHugeSF}
% \newcommand{\mySubTitleFont}{\large\sf}
% \renewcommand{\indexspace}{\medskip}

% % {\parindent 8pt
% %   {\myTitleFont Indhold}}\par
% % \vskip 5pt
% \renewcommand{\SBThechapter}{Indhold}
% % {\parindent 20pt
% %   {\mySubTitleFont --- with first lines in italic ---}}
% % \vskip 20pt
% \let\olditem\item
% \let\oldsubitem\subitem
% \let\oldsubsubitem\subsubitem
% \renewcommand{\item}{\par\hangindent=40pt}
% \renewcommand{\subitem}{\par\hangindent=40pt \hspace*{20pt}}
% \renewcommand{\subsubitem}{\par\hangindent=40pt \hspace*{30pt}}

% %%%%%%% rcsid = @(#)$Id: sample-sb.tex,v 1.23 2010-04-12 18:04:11 rathc Exp $
%%%%%%
%%
%%      ===============================
%%      Sample Songbook (sample-sb.tex)
%%      ===============================
%%
%%      Version 4.5, 30 April, 2010
%%
%%      Copyright 1992--2010 Christopher Rath <christopher@rath.ca>
%%
%%      This package is free software; you can redistribute it and/or
%%      modify it under the terms of version 2.1 of the GNU Lesser
%%	General Public License as published by the Free Software 
%%	Foundation.
%%
%%      This package is distributed in the hope that it will be
%%      useful, but WITHOUT ANY WARRANTY; without even the implied
%%      warranty of MERCHANTABILITY or FITNESS FOR A PARTICULAR
%%      PURPOSE.  See the GNU Lesser General Public License for more
%%      details.
%%
%%      This file contains a subset of the songbook we distribute
%%      at our church.  To the best of my knowledge, all of the lyrics
%%      contained herein are freely distributable.  This file has been
%%      provided as a sample of what can be produced by the chordbk,
%%      wordbk, and overhead LaTeX styles.
%%
%%      NEEDED:  The fancyhdr LaTeX style is required to properly
%%              format this file.  If you don't have that then comment
%%              out the commands in the preamble which deal with the
%%              fancyhdr style.
%%
%%%%%%
%%%%%%
%%
%%      1. Chord notation.  Within this songbook the following
%%         conventions have been adopted:
%%
%%              "Minor" is entered as "m";
%%                      e.g. Cm7 for C minor 7th.
%%              "Major" is entered as "M";
%%                      e.g. CM7 for C major 7th.
%%
%%%%%%
%%%%%%
%%      ============
%%      Bibliography
%%      ============
%%
%%      Exalt Him!: Exalt Him!  Compiled by Tom Fettke.  (c)1989
%%                      Word Music.
%%
%%      Hosanna! Music Books: Hosanna! Music Books #1--#6.
%%                      (c)1987--92 Integrity Music, Inc.
%%
%%      Worship Him II: Worship Him II.  Compiled by Jesse Peterson
%%                      and Bruce Ballinger.  (c)1989 Tempo Music
%%                      Publications.
%%
%%      Worship Songs Of The Vineyard: Worship Songs Of The Vineyard
%%                      --- Volume 2.  (c)1989 Vineyard Ministries
%%                      International.
%%
%%%%%%
%%%%%%

%%%%%%%%%%%%%%%%%%%%%%%%%%%%%%%%%%%%%%%%%%%%%%%%%%%%%%%%%%
%%%%%%%%%%%%%%%%%%%%%%%%%%%%%%%%%%%%%%%%%%%%%%%%%%%%%%%%%%
%%                                                      %%
%%           P R E A M B L E   B E G I N S              %%
%%                                                      %%
%%%%%%%%%%%%%%%%%%%%%%%%%%%%%%%%%%%%%%%%%%%%%%%%%%%%%%%%%%
%%%%%%%%%%%%%%%%%%%%%%%%%%%%%%%%%%%%%%%%%%%%%%%%%%%%%%%%%%

\documentclass[a5paper]{book}
\usepackage{latexsym,
            fancyhdr,
            titlesec,
            amsmath,
            amssymb,
            multicol,
            amsthm,
            stmaryrd,
            amsthm,
            color,
            needspace,
            stackengine,
            wasysym}
\usepackage[utf8]{inputenc}
\usepackage[T1]{fontenc}
% \usepackage[chordbk]{songbook}                  %% Words & Chords edition.
%%\usepackage[compactallsongs,chordbk]{songbook}    %% Words & Chords edition.
\usepackage[wordbk]{songbook}                 %% Words Only edition.
%%\usepackage[overhead]{songbook}               %% Overhead Transparency edition.
\usepackage{titletoc}
\usepackage{tket}  % Draws "TÅGEKAMMERET" correctly

%%%
% Revision Date and Release Date definitions.
%
%       \RelDate - The last time this songbook was released.  Set this
%                  date each time a new release/update of the songbook
%                  is generated.
%       \RevDate - The last time a particular song was revised in any
%                  way.  This command will be renewed inside every
%                  song.
%%%
\newcommand{\RelDate}{31~August,~2003}
\newcommand{\RevDate}{\today}

%%%
% C.C.L.I. license number definition; for copyright licensing info.
% One of these macros will be manually inserted into the {SBMel}
% parameter of the {song} environment.
%
%       \CCLInumber - The actual copyright license number.  Don't
%               insert this command in the {SBMel} parameter, use one
%               of the others.
%       \CCLIed - Indicates a song falls under our CCLI license.
%       \NotCCLIed - Indicates a song doesn't fall under our CCLI
%               license.  Public Domain songs fall into this category.
%       \PGranted - We have received specific permission from the
%               copyright holder to use this song.
%       \PPending - We are in the process of obtaining permission to
%               use this song.
%%%
\newcommand{\CCLInumber}{Your CCLI Number}
\newcommand{\CCLIed}{{\SBMelInfoFont (CCLI \CCLInumber)}}
\newcommand{\NotCCLIed}{\relax}
\newcommand{\PGranted}{\relax}
\newcommand{\PPending}{{\SBMelInfoFont (Permission Pending)}}

%%%
% Title page information.
%%%
%\title{UNF Computer Science Camp 2019 Sangbog}
%\author{}
%\date{Revideret:  \RevDate}

%%%
% Redefine fonts from SongBook style that I don't like.
%%%
\font\myTinySF=cmss8 at 8pt
\renewcommand{\SBMelInfoFont}{\tiny\myTinySF}

%%%
% Define fonts to use in the headers and footers of the songbook.
%%%
\newcommand{\LHeadFont}{\normalsize}            % = cmr12  at 12pt
\newcommand{\CHeadFont}{\normalsize\rm}         % = cmr12  at 12pt
\newcommand{\RHeadFont}{\normalsize}            % = cmr12  at 12pt
\newcommand{\LFootFont}{\scriptsize}            % = cmr8   at  8pt
\newcommand{\CFootFont}{\tiny\myTinySF}         % = cmss8  at  8pt
\newcommand{\RFootFont}{\scriptsize}            % = cmr8   at  8pt

\def\repeat{%
  \stackanchor{.}{.}%
  \rule[-\dp\strutbox]{.3pt}{\normalbaselineskip}%
  \kern0.5pt%
  \rule[-\dp\strutbox]{1pt}{\normalbaselineskip}%
  \kern1pt%
}
\def\frepeat{%
  \kern1pt%
  \rule[-\dp\strutbox]{1pt}{\normalbaselineskip}%
  \kern0.5pt%
  \rule[-\dp\strutbox]{.3pt}{\normalbaselineskip}%
  \stackanchor{.}{.}%
}
% \newcommand{\SBRepeat}[1]{#1\\#1}
\newcommand{\SBRepeat}[1]{\frepeat #1\repeat}
\setcounter{SBSongCnt}{-1}
\renewcommand{\SBWAndMTag}{Forfatter:}
\renewcommand{\SBUnknownTag}{Ukendt}
\renewcommand{\SBChorusTag}{Ref.}
\renewcommand{\SBOrgMel}{Originalmelodi}
\renewcommand{\SpaceAfterChorus}   {\vspace{0ex plus1ex minus 0.5ex}}
\renewcommand{\SpaceAfterOpGroup}  {\vspace{0ex plus1ex minus 0.5ex}}
\renewcommand{\SpaceAfterSBBracket}{\vspace{0ex plus1ex minus 0.5ex}}
\renewcommand{\SpaceAfterSection}  {\vspace{0ex plus1ex minus 0.5ex}}
\renewcommand{\SpaceAfterSong}     {\vspace{0ex plus1ex minus 0.5ex}}
\renewcommand{\SpaceAfterVerse}    {\vspace{0ex plus1ex minus 0.5ex}}

% Tell LaTeX that \medskip is a good place to make a page break
\let\oldmedskip\medskip
\renewcommand{\medskip}{\oldmedskip\pagebreak[2]}

%%%
% Turn on/off index-file generation.  Uncomment the \makeindex line to
% turn index generation on;  comment it out to turn index generation
% off.
%%%
%\makeTitleIndex         %% Title and First Line Index.
%\makeTitleContents      %% Table of Contents.
%\makeKeyIndex           %% Index of song by key.
% \makeArtistIndex	%% Index of song by artist.
% \newcommand{\SBThechapter}[0]{}
% \newcommand{\SBChapter}[1]{
%     \startcontents
%     \chapter*{#1} 
%     % \input{unf-sangbog.toc}
%       \begin{minipage}{.8\textwidth}
%         \printcontents{}{1}{}
%       \end{minipage}%
%     \renewcommand{\SBThechapter}{#1}
%     \clearpage
% }

% \titleformat{\chapter}
% [display]
% {}
% {%\vspace*{\fill}
%  % \titlerule[1pt]%
%  % \vspace{1pt}%
%  % \titlerule
%  % \vspace{1pc}%
%  \chaptertitlename}
% {}
% {\Huge}



%%%%%%%%%%%%%%%%%%%%%%%%%%%%%%%%%%%%%%%%%%%%%%%%%%%%%%%%%%
%%%%%%%%%%%%%%%%%%%%%%%%%%%%%%%%%%%%%%%%%%%%%%%%%%%%%%%%%%
%%                                                      %%
%%           D O C U M E N T   B E G I N S              %%
%%                                                      %%
%%%%%%%%%%%%%%%%%%%%%%%%%%%%%%%%%%%%%%%%%%%%%%%%%%%%%%%%%%
%%%%%%%%%%%%%%%%%%%%%%%%%%%%%%%%%%%%%%%%%%%%%%%%%%%%%%%%%%
\begin{document}

%%%
% Uncomment "\maketitle" statement to make a title page.
%%%
%\maketitle
% \begin{titlepage}
%   \centering
%   \vspace{5cm}
% 	\includegraphics[width=1\textwidth]{unf_logo.jpeg}\par\vspace{1cm}
% 	{\scshape\LARGE Sangbog \par}
% 	\vspace{1cm}
% 	{\scshape\Large UNF Computer Science Camp 2019\par}
	
% 	\vfill

% % Bottom of the page
% 	{\large \today\par}
% \end{titlepage}
% \mainmatter
% \ifWordBk
%   \twocolumn
% \fi


%%% Kolofon
%\thispagestyle{empty}
%Sammensat til UNF Computer Science Camp 2019 - csc.unf.dk\\
%Redaktør: Andreas Mosbæk Jensen m.fl. efter tidligere sangbog af Steffen Strunge Mathiesen\\
%Indhold opsat i \LaTeX. 
%Digital version og kildekode: github.com/steffen555/UNF-sangbog\\
%Revision 1 med stave fejl korrektioner
%\par\vspace*{\fill}
%Hvis du har forslag til sange, rettelser, ris og ros, eller hvis du kender en ukendt forfatter, så skriv til sangbog@unf.dk.

%%%
% Turn on and define fancy page heading/footing definition.
%%%
% \pagestyle{fancy}

% \ifChordBk
%   % It's a words & chords songbook...
%   \addtolength{\headwidth}{\marginparsep}
%   \addtolength{\headwidth}{\marginparwidth}
%   \renewcommand{\headrulewidth}{0.4pt}
%   \renewcommand{\footrulewidth}{0.4pt}
%   \fancyhead[LE,RO]{\LHeadFont\emph{\leftmark\/}\SBContinueMark}
%   \fancyhead[CE,CO]{\CHeadFont\thepage}
%   \fancyhead[RE,LO]{\RHeadFont \chaptermark}
% \else\ifOverhead
%   % It's an overhead...
%   \renewcommand{\footrulewidth}{0pt}
%   \renewcommand{\headrulewidth}{0pt}
%   \fancyhead[LE,RO]{}
%   \fancyhead[CE,CO]{}
%   \fancyhead[RE,LO]{}
% \else\ifWordBk
%   % It's a words only songbook...
%   \addtolength{\headwidth}{\marginparsep}
%   \addtolength{\headwidth}{\marginparwidth}
%   \renewcommand{\headrulewidth}{0.4pt}
%   \renewcommand{\footrulewidth}{0.4pt}
%   \fancyhead[LE,RO]{\LHeadFont Naturvidenskab revy sange}
%   \fancyhead[CE,CO]{\CHeadFont\thepage}
%   \fancyhead[RE,LO]{\RHeadFont \SBThechapter}
% \fi\fi\fi

% \fancyfoot[LE,RO]{\LFootFont Computer Science Camp 2019}
% \ifSongEject
%   \fancyfoot[CE,CO]{\CFootFont Last Revised:  \RevDate}
% \else
%   \fancyfoot[CE,CO]{\CFootFont}
% \fi
% \fancyfoot[RE,LO]{\RFootFont Synges på eget ansvar}

%%%
% Table of contents
%%%

% \clearpage
% \twocolumn
% \font\myTinySF=cmss8    at  8pt
% \font\myHugeSF=cmssbx10 at 25pt
% \newcommand{\CpyRtInfoFont}{\tiny\myTinySF}
% \newcommand{\myTitleFont}{\Huge\myHugeSF}
% \newcommand{\mySubTitleFont}{\large\sf}
% \renewcommand{\indexspace}{\medskip}

% % {\parindent 8pt
% %   {\myTitleFont Indhold}}\par
% % \vskip 5pt
% \renewcommand{\SBThechapter}{Indhold}
% % {\parindent 20pt
% %   {\mySubTitleFont --- with first lines in italic ---}}
% % \vskip 20pt
% \let\olditem\item
% \let\oldsubitem\subitem
% \let\oldsubsubitem\subsubitem
% \renewcommand{\item}{\par\hangindent=40pt}
% \renewcommand{\subitem}{\par\hangindent=40pt \hspace*{20pt}}
% \renewcommand{\subsubitem}{\par\hangindent=40pt \hspace*{30pt}}

% %\input{unf-sangbog.tocx}

% \renewcommand{\item}{\olditem}
% \renewcommand{\subitem}{\oldsubitem}
% \renewcommand{\subsubitem}{\oldsubsubitem}

%%%
% Songbook begins.
%%%

\twocolumn
%It's just one page, don't print page numbers etc.
\pagestyle{empty}
%Songs included
\input{songs/matmatik.tex}
\input{songs/taal_daj.tex}
\input{songs/linieskriverdriver.tex}
\input{songs/steve_hawking.tex}
\input{songs/ode_til_kode.tex}
\input{songs/se_min_kode.tex}
\input{songs/vaabenfysik_kort.tex}
%Maybe include:
%\input{songs/kvanter_i_maaneskin.tex}
%\input{songs/mest_matematiske_dyr.tex}

% \input{songs/vi_kan_ikke_li.tex}
% \input{songs/selektionssangen.tex}
% \input{songs/alfabetsangen.tex}
% \input{songs/sciencecamps.tex}
% \input{songs/hvad_maa_man.tex}


% \input{songs/lambda_kalkylen.tex}
% \input{songs/puslespil.tex}
% \input{songs/null.tex}
% \input{songs/fasebal.tex}

% \input{songs/chifitter.tex}

% \input{songs/kun_fysik.tex}



% \input{songs/kanoniske.tex}
% \input{songs/jeg_er_en_matematiker_fra_hcoe.tex}


% \input{songs/rekursiv_skovsang.tex}
% \input{songs/laerkerede.tex}


% \clearpage
% \font\myTinySF=cmss8    at  8pt
% \font\myHugeSF=cmssbx10 at 25pt
% % \newcommand{\CpyRtInfoFont}{\tiny\myTinySF}
% % \newcommand{\myTitleFont}{\Huge\myHugeSF}
% % \newcommand{\mySubTitleFont}{\large\sf}
% \renewcommand{\indexspace}{\medskip}

% {\parindent 8pt
%   {\myTitleFont Index}}\par
% \vskip 5pt
% \renewcommand{\SBThechapter}{Index}
% % {\parindent 20pt
% %   {\mySubTitleFont --- with first lines in italic ---}}
% % \vskip 20pt
% \renewcommand{\item}{\par\hangindent=40pt}
% \renewcommand{\subitem}{\par\hangindent=40pt \hspace*{20pt}}
% \renewcommand{\subsubitem}{\par\hangindent=40pt \hspace*{30pt}}

%\input{unf-sangbog.tdx}

\end{document}
\bye
%
%%%
% Document ends.
%%%


% \renewcommand{\item}{\olditem}
% \renewcommand{\subitem}{\oldsubitem}
% \renewcommand{\subsubitem}{\oldsubsubitem}

%%%
% Songbook begins.
%%%

\twocolumn
%It's just one page, don't print page numbers etc.
\pagestyle{empty}
%Songs included
\input{songs/matmatik.tex}
\input{songs/taal_daj.tex}
\input{songs/linieskriverdriver.tex}
\input{songs/steve_hawking.tex}
\input{songs/ode_til_kode.tex}
\input{songs/se_min_kode.tex}
\input{songs/vaabenfysik_kort.tex}
%Maybe include:
%\input{songs/kvanter_i_maaneskin.tex}
%\input{songs/mest_matematiske_dyr.tex}

% \input{songs/vi_kan_ikke_li.tex}
% \input{songs/selektionssangen.tex}
% \input{songs/alfabetsangen.tex}
% \input{songs/sciencecamps.tex}
% \input{songs/hvad_maa_man.tex}


% \input{songs/lambda_kalkylen.tex}
% \input{songs/puslespil.tex}
% \input{songs/null.tex}
% \input{songs/fasebal.tex}

% \input{songs/chifitter.tex}

% \input{songs/kun_fysik.tex}



% \input{songs/kanoniske.tex}
% \input{songs/jeg_er_en_matematiker_fra_hcoe.tex}


% \input{songs/rekursiv_skovsang.tex}
% \input{songs/laerkerede.tex}


% \clearpage
% \font\myTinySF=cmss8    at  8pt
% \font\myHugeSF=cmssbx10 at 25pt
% % \newcommand{\CpyRtInfoFont}{\tiny\myTinySF}
% % \newcommand{\myTitleFont}{\Huge\myHugeSF}
% % \newcommand{\mySubTitleFont}{\large\sf}
% \renewcommand{\indexspace}{\medskip}

% {\parindent 8pt
%   {\myTitleFont Index}}\par
% \vskip 5pt
% \renewcommand{\SBThechapter}{Index}
% % {\parindent 20pt
% %   {\mySubTitleFont --- with first lines in italic ---}}
% % \vskip 20pt
% \renewcommand{\item}{\par\hangindent=40pt}
% \renewcommand{\subitem}{\par\hangindent=40pt \hspace*{20pt}}
% \renewcommand{\subsubitem}{\par\hangindent=40pt \hspace*{30pt}}

%%%%%%% rcsid = @(#)$Id: sample-sb.tex,v 1.23 2010-04-12 18:04:11 rathc Exp $
%%%%%%
%%
%%      ===============================
%%      Sample Songbook (sample-sb.tex)
%%      ===============================
%%
%%      Version 4.5, 30 April, 2010
%%
%%      Copyright 1992--2010 Christopher Rath <christopher@rath.ca>
%%
%%      This package is free software; you can redistribute it and/or
%%      modify it under the terms of version 2.1 of the GNU Lesser
%%	General Public License as published by the Free Software 
%%	Foundation.
%%
%%      This package is distributed in the hope that it will be
%%      useful, but WITHOUT ANY WARRANTY; without even the implied
%%      warranty of MERCHANTABILITY or FITNESS FOR A PARTICULAR
%%      PURPOSE.  See the GNU Lesser General Public License for more
%%      details.
%%
%%      This file contains a subset of the songbook we distribute
%%      at our church.  To the best of my knowledge, all of the lyrics
%%      contained herein are freely distributable.  This file has been
%%      provided as a sample of what can be produced by the chordbk,
%%      wordbk, and overhead LaTeX styles.
%%
%%      NEEDED:  The fancyhdr LaTeX style is required to properly
%%              format this file.  If you don't have that then comment
%%              out the commands in the preamble which deal with the
%%              fancyhdr style.
%%
%%%%%%
%%%%%%
%%
%%      1. Chord notation.  Within this songbook the following
%%         conventions have been adopted:
%%
%%              "Minor" is entered as "m";
%%                      e.g. Cm7 for C minor 7th.
%%              "Major" is entered as "M";
%%                      e.g. CM7 for C major 7th.
%%
%%%%%%
%%%%%%
%%      ============
%%      Bibliography
%%      ============
%%
%%      Exalt Him!: Exalt Him!  Compiled by Tom Fettke.  (c)1989
%%                      Word Music.
%%
%%      Hosanna! Music Books: Hosanna! Music Books #1--#6.
%%                      (c)1987--92 Integrity Music, Inc.
%%
%%      Worship Him II: Worship Him II.  Compiled by Jesse Peterson
%%                      and Bruce Ballinger.  (c)1989 Tempo Music
%%                      Publications.
%%
%%      Worship Songs Of The Vineyard: Worship Songs Of The Vineyard
%%                      --- Volume 2.  (c)1989 Vineyard Ministries
%%                      International.
%%
%%%%%%
%%%%%%

%%%%%%%%%%%%%%%%%%%%%%%%%%%%%%%%%%%%%%%%%%%%%%%%%%%%%%%%%%
%%%%%%%%%%%%%%%%%%%%%%%%%%%%%%%%%%%%%%%%%%%%%%%%%%%%%%%%%%
%%                                                      %%
%%           P R E A M B L E   B E G I N S              %%
%%                                                      %%
%%%%%%%%%%%%%%%%%%%%%%%%%%%%%%%%%%%%%%%%%%%%%%%%%%%%%%%%%%
%%%%%%%%%%%%%%%%%%%%%%%%%%%%%%%%%%%%%%%%%%%%%%%%%%%%%%%%%%

\documentclass[a5paper]{book}
\usepackage{latexsym,
            fancyhdr,
            titlesec,
            amsmath,
            amssymb,
            multicol,
            amsthm,
            stmaryrd,
            amsthm,
            color,
            needspace,
            stackengine,
            wasysym}
\usepackage[utf8]{inputenc}
\usepackage[T1]{fontenc}
% \usepackage[chordbk]{songbook}                  %% Words & Chords edition.
%%\usepackage[compactallsongs,chordbk]{songbook}    %% Words & Chords edition.
\usepackage[wordbk]{songbook}                 %% Words Only edition.
%%\usepackage[overhead]{songbook}               %% Overhead Transparency edition.
\usepackage{titletoc}
\usepackage{tket}  % Draws "TÅGEKAMMERET" correctly

%%%
% Revision Date and Release Date definitions.
%
%       \RelDate - The last time this songbook was released.  Set this
%                  date each time a new release/update of the songbook
%                  is generated.
%       \RevDate - The last time a particular song was revised in any
%                  way.  This command will be renewed inside every
%                  song.
%%%
\newcommand{\RelDate}{31~August,~2003}
\newcommand{\RevDate}{\today}

%%%
% C.C.L.I. license number definition; for copyright licensing info.
% One of these macros will be manually inserted into the {SBMel}
% parameter of the {song} environment.
%
%       \CCLInumber - The actual copyright license number.  Don't
%               insert this command in the {SBMel} parameter, use one
%               of the others.
%       \CCLIed - Indicates a song falls under our CCLI license.
%       \NotCCLIed - Indicates a song doesn't fall under our CCLI
%               license.  Public Domain songs fall into this category.
%       \PGranted - We have received specific permission from the
%               copyright holder to use this song.
%       \PPending - We are in the process of obtaining permission to
%               use this song.
%%%
\newcommand{\CCLInumber}{Your CCLI Number}
\newcommand{\CCLIed}{{\SBMelInfoFont (CCLI \CCLInumber)}}
\newcommand{\NotCCLIed}{\relax}
\newcommand{\PGranted}{\relax}
\newcommand{\PPending}{{\SBMelInfoFont (Permission Pending)}}

%%%
% Title page information.
%%%
%\title{UNF Computer Science Camp 2019 Sangbog}
%\author{}
%\date{Revideret:  \RevDate}

%%%
% Redefine fonts from SongBook style that I don't like.
%%%
\font\myTinySF=cmss8 at 8pt
\renewcommand{\SBMelInfoFont}{\tiny\myTinySF}

%%%
% Define fonts to use in the headers and footers of the songbook.
%%%
\newcommand{\LHeadFont}{\normalsize}            % = cmr12  at 12pt
\newcommand{\CHeadFont}{\normalsize\rm}         % = cmr12  at 12pt
\newcommand{\RHeadFont}{\normalsize}            % = cmr12  at 12pt
\newcommand{\LFootFont}{\scriptsize}            % = cmr8   at  8pt
\newcommand{\CFootFont}{\tiny\myTinySF}         % = cmss8  at  8pt
\newcommand{\RFootFont}{\scriptsize}            % = cmr8   at  8pt

\def\repeat{%
  \stackanchor{.}{.}%
  \rule[-\dp\strutbox]{.3pt}{\normalbaselineskip}%
  \kern0.5pt%
  \rule[-\dp\strutbox]{1pt}{\normalbaselineskip}%
  \kern1pt%
}
\def\frepeat{%
  \kern1pt%
  \rule[-\dp\strutbox]{1pt}{\normalbaselineskip}%
  \kern0.5pt%
  \rule[-\dp\strutbox]{.3pt}{\normalbaselineskip}%
  \stackanchor{.}{.}%
}
% \newcommand{\SBRepeat}[1]{#1\\#1}
\newcommand{\SBRepeat}[1]{\frepeat #1\repeat}
\setcounter{SBSongCnt}{-1}
\renewcommand{\SBWAndMTag}{Forfatter:}
\renewcommand{\SBUnknownTag}{Ukendt}
\renewcommand{\SBChorusTag}{Ref.}
\renewcommand{\SBOrgMel}{Originalmelodi}
\renewcommand{\SpaceAfterChorus}   {\vspace{0ex plus1ex minus 0.5ex}}
\renewcommand{\SpaceAfterOpGroup}  {\vspace{0ex plus1ex minus 0.5ex}}
\renewcommand{\SpaceAfterSBBracket}{\vspace{0ex plus1ex minus 0.5ex}}
\renewcommand{\SpaceAfterSection}  {\vspace{0ex plus1ex minus 0.5ex}}
\renewcommand{\SpaceAfterSong}     {\vspace{0ex plus1ex minus 0.5ex}}
\renewcommand{\SpaceAfterVerse}    {\vspace{0ex plus1ex minus 0.5ex}}

% Tell LaTeX that \medskip is a good place to make a page break
\let\oldmedskip\medskip
\renewcommand{\medskip}{\oldmedskip\pagebreak[2]}

%%%
% Turn on/off index-file generation.  Uncomment the \makeindex line to
% turn index generation on;  comment it out to turn index generation
% off.
%%%
%\makeTitleIndex         %% Title and First Line Index.
%\makeTitleContents      %% Table of Contents.
%\makeKeyIndex           %% Index of song by key.
% \makeArtistIndex	%% Index of song by artist.
% \newcommand{\SBThechapter}[0]{}
% \newcommand{\SBChapter}[1]{
%     \startcontents
%     \chapter*{#1} 
%     % \input{unf-sangbog.toc}
%       \begin{minipage}{.8\textwidth}
%         \printcontents{}{1}{}
%       \end{minipage}%
%     \renewcommand{\SBThechapter}{#1}
%     \clearpage
% }

% \titleformat{\chapter}
% [display]
% {}
% {%\vspace*{\fill}
%  % \titlerule[1pt]%
%  % \vspace{1pt}%
%  % \titlerule
%  % \vspace{1pc}%
%  \chaptertitlename}
% {}
% {\Huge}



%%%%%%%%%%%%%%%%%%%%%%%%%%%%%%%%%%%%%%%%%%%%%%%%%%%%%%%%%%
%%%%%%%%%%%%%%%%%%%%%%%%%%%%%%%%%%%%%%%%%%%%%%%%%%%%%%%%%%
%%                                                      %%
%%           D O C U M E N T   B E G I N S              %%
%%                                                      %%
%%%%%%%%%%%%%%%%%%%%%%%%%%%%%%%%%%%%%%%%%%%%%%%%%%%%%%%%%%
%%%%%%%%%%%%%%%%%%%%%%%%%%%%%%%%%%%%%%%%%%%%%%%%%%%%%%%%%%
\begin{document}

%%%
% Uncomment "\maketitle" statement to make a title page.
%%%
%\maketitle
% \begin{titlepage}
%   \centering
%   \vspace{5cm}
% 	\includegraphics[width=1\textwidth]{unf_logo.jpeg}\par\vspace{1cm}
% 	{\scshape\LARGE Sangbog \par}
% 	\vspace{1cm}
% 	{\scshape\Large UNF Computer Science Camp 2019\par}
	
% 	\vfill

% % Bottom of the page
% 	{\large \today\par}
% \end{titlepage}
% \mainmatter
% \ifWordBk
%   \twocolumn
% \fi


%%% Kolofon
%\thispagestyle{empty}
%Sammensat til UNF Computer Science Camp 2019 - csc.unf.dk\\
%Redaktør: Andreas Mosbæk Jensen m.fl. efter tidligere sangbog af Steffen Strunge Mathiesen\\
%Indhold opsat i \LaTeX. 
%Digital version og kildekode: github.com/steffen555/UNF-sangbog\\
%Revision 1 med stave fejl korrektioner
%\par\vspace*{\fill}
%Hvis du har forslag til sange, rettelser, ris og ros, eller hvis du kender en ukendt forfatter, så skriv til sangbog@unf.dk.

%%%
% Turn on and define fancy page heading/footing definition.
%%%
% \pagestyle{fancy}

% \ifChordBk
%   % It's a words & chords songbook...
%   \addtolength{\headwidth}{\marginparsep}
%   \addtolength{\headwidth}{\marginparwidth}
%   \renewcommand{\headrulewidth}{0.4pt}
%   \renewcommand{\footrulewidth}{0.4pt}
%   \fancyhead[LE,RO]{\LHeadFont\emph{\leftmark\/}\SBContinueMark}
%   \fancyhead[CE,CO]{\CHeadFont\thepage}
%   \fancyhead[RE,LO]{\RHeadFont \chaptermark}
% \else\ifOverhead
%   % It's an overhead...
%   \renewcommand{\footrulewidth}{0pt}
%   \renewcommand{\headrulewidth}{0pt}
%   \fancyhead[LE,RO]{}
%   \fancyhead[CE,CO]{}
%   \fancyhead[RE,LO]{}
% \else\ifWordBk
%   % It's a words only songbook...
%   \addtolength{\headwidth}{\marginparsep}
%   \addtolength{\headwidth}{\marginparwidth}
%   \renewcommand{\headrulewidth}{0.4pt}
%   \renewcommand{\footrulewidth}{0.4pt}
%   \fancyhead[LE,RO]{\LHeadFont Naturvidenskab revy sange}
%   \fancyhead[CE,CO]{\CHeadFont\thepage}
%   \fancyhead[RE,LO]{\RHeadFont \SBThechapter}
% \fi\fi\fi

% \fancyfoot[LE,RO]{\LFootFont Computer Science Camp 2019}
% \ifSongEject
%   \fancyfoot[CE,CO]{\CFootFont Last Revised:  \RevDate}
% \else
%   \fancyfoot[CE,CO]{\CFootFont}
% \fi
% \fancyfoot[RE,LO]{\RFootFont Synges på eget ansvar}

%%%
% Table of contents
%%%

% \clearpage
% \twocolumn
% \font\myTinySF=cmss8    at  8pt
% \font\myHugeSF=cmssbx10 at 25pt
% \newcommand{\CpyRtInfoFont}{\tiny\myTinySF}
% \newcommand{\myTitleFont}{\Huge\myHugeSF}
% \newcommand{\mySubTitleFont}{\large\sf}
% \renewcommand{\indexspace}{\medskip}

% % {\parindent 8pt
% %   {\myTitleFont Indhold}}\par
% % \vskip 5pt
% \renewcommand{\SBThechapter}{Indhold}
% % {\parindent 20pt
% %   {\mySubTitleFont --- with first lines in italic ---}}
% % \vskip 20pt
% \let\olditem\item
% \let\oldsubitem\subitem
% \let\oldsubsubitem\subsubitem
% \renewcommand{\item}{\par\hangindent=40pt}
% \renewcommand{\subitem}{\par\hangindent=40pt \hspace*{20pt}}
% \renewcommand{\subsubitem}{\par\hangindent=40pt \hspace*{30pt}}

% %\input{unf-sangbog.tocx}

% \renewcommand{\item}{\olditem}
% \renewcommand{\subitem}{\oldsubitem}
% \renewcommand{\subsubitem}{\oldsubsubitem}

%%%
% Songbook begins.
%%%

\twocolumn
%It's just one page, don't print page numbers etc.
\pagestyle{empty}
%Songs included
\input{songs/matmatik.tex}
\input{songs/taal_daj.tex}
\input{songs/linieskriverdriver.tex}
\input{songs/steve_hawking.tex}
\input{songs/ode_til_kode.tex}
\input{songs/se_min_kode.tex}
\input{songs/vaabenfysik_kort.tex}
%Maybe include:
%\input{songs/kvanter_i_maaneskin.tex}
%\input{songs/mest_matematiske_dyr.tex}

% \input{songs/vi_kan_ikke_li.tex}
% \input{songs/selektionssangen.tex}
% \input{songs/alfabetsangen.tex}
% \input{songs/sciencecamps.tex}
% \input{songs/hvad_maa_man.tex}


% \input{songs/lambda_kalkylen.tex}
% \input{songs/puslespil.tex}
% \input{songs/null.tex}
% \input{songs/fasebal.tex}

% \input{songs/chifitter.tex}

% \input{songs/kun_fysik.tex}



% \input{songs/kanoniske.tex}
% \input{songs/jeg_er_en_matematiker_fra_hcoe.tex}


% \input{songs/rekursiv_skovsang.tex}
% \input{songs/laerkerede.tex}


% \clearpage
% \font\myTinySF=cmss8    at  8pt
% \font\myHugeSF=cmssbx10 at 25pt
% % \newcommand{\CpyRtInfoFont}{\tiny\myTinySF}
% % \newcommand{\myTitleFont}{\Huge\myHugeSF}
% % \newcommand{\mySubTitleFont}{\large\sf}
% \renewcommand{\indexspace}{\medskip}

% {\parindent 8pt
%   {\myTitleFont Index}}\par
% \vskip 5pt
% \renewcommand{\SBThechapter}{Index}
% % {\parindent 20pt
% %   {\mySubTitleFont --- with first lines in italic ---}}
% % \vskip 20pt
% \renewcommand{\item}{\par\hangindent=40pt}
% \renewcommand{\subitem}{\par\hangindent=40pt \hspace*{20pt}}
% \renewcommand{\subsubitem}{\par\hangindent=40pt \hspace*{30pt}}

%\input{unf-sangbog.tdx}

\end{document}
\bye
%
%%%
% Document ends.
%%%


\end{document}
\bye
%
%%%
% Document ends.
%%%


% \renewcommand{\item}{\olditem}
% \renewcommand{\subitem}{\oldsubitem}
% \renewcommand{\subsubitem}{\oldsubsubitem}

%%%
% Songbook begins.
%%%

\twocolumn
%It's just one page, don't print page numbers etc.
\pagestyle{empty}
%Songs included
\input{songs/matmatik.tex}
\input{songs/taal_daj.tex}
\input{songs/linieskriverdriver.tex}
\input{songs/steve_hawking.tex}
\input{songs/ode_til_kode.tex}
\input{songs/se_min_kode.tex}
\input{songs/vaabenfysik_kort.tex}
%Maybe include:
%\input{songs/kvanter_i_maaneskin.tex}
%\input{songs/mest_matematiske_dyr.tex}

% \input{songs/vi_kan_ikke_li.tex}
% \input{songs/selektionssangen.tex}
% \input{songs/alfabetsangen.tex}
% \input{songs/sciencecamps.tex}
% \input{songs/hvad_maa_man.tex}


% \input{songs/lambda_kalkylen.tex}
% \input{songs/puslespil.tex}
% \input{songs/null.tex}
% \input{songs/fasebal.tex}

% \input{songs/chifitter.tex}

% \input{songs/kun_fysik.tex}



% \input{songs/kanoniske.tex}
% \input{songs/jeg_er_en_matematiker_fra_hcoe.tex}


% \input{songs/rekursiv_skovsang.tex}
% \input{songs/laerkerede.tex}


% \clearpage
% \font\myTinySF=cmss8    at  8pt
% \font\myHugeSF=cmssbx10 at 25pt
% % \newcommand{\CpyRtInfoFont}{\tiny\myTinySF}
% % \newcommand{\myTitleFont}{\Huge\myHugeSF}
% % \newcommand{\mySubTitleFont}{\large\sf}
% \renewcommand{\indexspace}{\medskip}

% {\parindent 8pt
%   {\myTitleFont Index}}\par
% \vskip 5pt
% \renewcommand{\SBThechapter}{Index}
% % {\parindent 20pt
% %   {\mySubTitleFont --- with first lines in italic ---}}
% % \vskip 20pt
% \renewcommand{\item}{\par\hangindent=40pt}
% \renewcommand{\subitem}{\par\hangindent=40pt \hspace*{20pt}}
% \renewcommand{\subsubitem}{\par\hangindent=40pt \hspace*{30pt}}

%%%%%%% rcsid = @(#)$Id: sample-sb.tex,v 1.23 2010-04-12 18:04:11 rathc Exp $
%%%%%%
%%
%%      ===============================
%%      Sample Songbook (sample-sb.tex)
%%      ===============================
%%
%%      Version 4.5, 30 April, 2010
%%
%%      Copyright 1992--2010 Christopher Rath <christopher@rath.ca>
%%
%%      This package is free software; you can redistribute it and/or
%%      modify it under the terms of version 2.1 of the GNU Lesser
%%	General Public License as published by the Free Software 
%%	Foundation.
%%
%%      This package is distributed in the hope that it will be
%%      useful, but WITHOUT ANY WARRANTY; without even the implied
%%      warranty of MERCHANTABILITY or FITNESS FOR A PARTICULAR
%%      PURPOSE.  See the GNU Lesser General Public License for more
%%      details.
%%
%%      This file contains a subset of the songbook we distribute
%%      at our church.  To the best of my knowledge, all of the lyrics
%%      contained herein are freely distributable.  This file has been
%%      provided as a sample of what can be produced by the chordbk,
%%      wordbk, and overhead LaTeX styles.
%%
%%      NEEDED:  The fancyhdr LaTeX style is required to properly
%%              format this file.  If you don't have that then comment
%%              out the commands in the preamble which deal with the
%%              fancyhdr style.
%%
%%%%%%
%%%%%%
%%
%%      1. Chord notation.  Within this songbook the following
%%         conventions have been adopted:
%%
%%              "Minor" is entered as "m";
%%                      e.g. Cm7 for C minor 7th.
%%              "Major" is entered as "M";
%%                      e.g. CM7 for C major 7th.
%%
%%%%%%
%%%%%%
%%      ============
%%      Bibliography
%%      ============
%%
%%      Exalt Him!: Exalt Him!  Compiled by Tom Fettke.  (c)1989
%%                      Word Music.
%%
%%      Hosanna! Music Books: Hosanna! Music Books #1--#6.
%%                      (c)1987--92 Integrity Music, Inc.
%%
%%      Worship Him II: Worship Him II.  Compiled by Jesse Peterson
%%                      and Bruce Ballinger.  (c)1989 Tempo Music
%%                      Publications.
%%
%%      Worship Songs Of The Vineyard: Worship Songs Of The Vineyard
%%                      --- Volume 2.  (c)1989 Vineyard Ministries
%%                      International.
%%
%%%%%%
%%%%%%

%%%%%%%%%%%%%%%%%%%%%%%%%%%%%%%%%%%%%%%%%%%%%%%%%%%%%%%%%%
%%%%%%%%%%%%%%%%%%%%%%%%%%%%%%%%%%%%%%%%%%%%%%%%%%%%%%%%%%
%%                                                      %%
%%           P R E A M B L E   B E G I N S              %%
%%                                                      %%
%%%%%%%%%%%%%%%%%%%%%%%%%%%%%%%%%%%%%%%%%%%%%%%%%%%%%%%%%%
%%%%%%%%%%%%%%%%%%%%%%%%%%%%%%%%%%%%%%%%%%%%%%%%%%%%%%%%%%

\documentclass[a5paper]{book}
\usepackage{latexsym,
            fancyhdr,
            titlesec,
            amsmath,
            amssymb,
            multicol,
            amsthm,
            stmaryrd,
            amsthm,
            color,
            needspace,
            stackengine,
            wasysym}
\usepackage[utf8]{inputenc}
\usepackage[T1]{fontenc}
% \usepackage[chordbk]{songbook}                  %% Words & Chords edition.
%%\usepackage[compactallsongs,chordbk]{songbook}    %% Words & Chords edition.
\usepackage[wordbk]{songbook}                 %% Words Only edition.
%%\usepackage[overhead]{songbook}               %% Overhead Transparency edition.
\usepackage{titletoc}
\usepackage{tket}  % Draws "TÅGEKAMMERET" correctly

%%%
% Revision Date and Release Date definitions.
%
%       \RelDate - The last time this songbook was released.  Set this
%                  date each time a new release/update of the songbook
%                  is generated.
%       \RevDate - The last time a particular song was revised in any
%                  way.  This command will be renewed inside every
%                  song.
%%%
\newcommand{\RelDate}{31~August,~2003}
\newcommand{\RevDate}{\today}

%%%
% C.C.L.I. license number definition; for copyright licensing info.
% One of these macros will be manually inserted into the {SBMel}
% parameter of the {song} environment.
%
%       \CCLInumber - The actual copyright license number.  Don't
%               insert this command in the {SBMel} parameter, use one
%               of the others.
%       \CCLIed - Indicates a song falls under our CCLI license.
%       \NotCCLIed - Indicates a song doesn't fall under our CCLI
%               license.  Public Domain songs fall into this category.
%       \PGranted - We have received specific permission from the
%               copyright holder to use this song.
%       \PPending - We are in the process of obtaining permission to
%               use this song.
%%%
\newcommand{\CCLInumber}{Your CCLI Number}
\newcommand{\CCLIed}{{\SBMelInfoFont (CCLI \CCLInumber)}}
\newcommand{\NotCCLIed}{\relax}
\newcommand{\PGranted}{\relax}
\newcommand{\PPending}{{\SBMelInfoFont (Permission Pending)}}

%%%
% Title page information.
%%%
%\title{UNF Computer Science Camp 2019 Sangbog}
%\author{}
%\date{Revideret:  \RevDate}

%%%
% Redefine fonts from SongBook style that I don't like.
%%%
\font\myTinySF=cmss8 at 8pt
\renewcommand{\SBMelInfoFont}{\tiny\myTinySF}

%%%
% Define fonts to use in the headers and footers of the songbook.
%%%
\newcommand{\LHeadFont}{\normalsize}            % = cmr12  at 12pt
\newcommand{\CHeadFont}{\normalsize\rm}         % = cmr12  at 12pt
\newcommand{\RHeadFont}{\normalsize}            % = cmr12  at 12pt
\newcommand{\LFootFont}{\scriptsize}            % = cmr8   at  8pt
\newcommand{\CFootFont}{\tiny\myTinySF}         % = cmss8  at  8pt
\newcommand{\RFootFont}{\scriptsize}            % = cmr8   at  8pt

\def\repeat{%
  \stackanchor{.}{.}%
  \rule[-\dp\strutbox]{.3pt}{\normalbaselineskip}%
  \kern0.5pt%
  \rule[-\dp\strutbox]{1pt}{\normalbaselineskip}%
  \kern1pt%
}
\def\frepeat{%
  \kern1pt%
  \rule[-\dp\strutbox]{1pt}{\normalbaselineskip}%
  \kern0.5pt%
  \rule[-\dp\strutbox]{.3pt}{\normalbaselineskip}%
  \stackanchor{.}{.}%
}
% \newcommand{\SBRepeat}[1]{#1\\#1}
\newcommand{\SBRepeat}[1]{\frepeat #1\repeat}
\setcounter{SBSongCnt}{-1}
\renewcommand{\SBWAndMTag}{Forfatter:}
\renewcommand{\SBUnknownTag}{Ukendt}
\renewcommand{\SBChorusTag}{Ref.}
\renewcommand{\SBOrgMel}{Originalmelodi}
\renewcommand{\SpaceAfterChorus}   {\vspace{0ex plus1ex minus 0.5ex}}
\renewcommand{\SpaceAfterOpGroup}  {\vspace{0ex plus1ex minus 0.5ex}}
\renewcommand{\SpaceAfterSBBracket}{\vspace{0ex plus1ex minus 0.5ex}}
\renewcommand{\SpaceAfterSection}  {\vspace{0ex plus1ex minus 0.5ex}}
\renewcommand{\SpaceAfterSong}     {\vspace{0ex plus1ex minus 0.5ex}}
\renewcommand{\SpaceAfterVerse}    {\vspace{0ex plus1ex minus 0.5ex}}

% Tell LaTeX that \medskip is a good place to make a page break
\let\oldmedskip\medskip
\renewcommand{\medskip}{\oldmedskip\pagebreak[2]}

%%%
% Turn on/off index-file generation.  Uncomment the \makeindex line to
% turn index generation on;  comment it out to turn index generation
% off.
%%%
%\makeTitleIndex         %% Title and First Line Index.
%\makeTitleContents      %% Table of Contents.
%\makeKeyIndex           %% Index of song by key.
% \makeArtistIndex	%% Index of song by artist.
% \newcommand{\SBThechapter}[0]{}
% \newcommand{\SBChapter}[1]{
%     \startcontents
%     \chapter*{#1} 
%     % %%%%%% rcsid = @(#)$Id: sample-sb.tex,v 1.23 2010-04-12 18:04:11 rathc Exp $
%%%%%%
%%
%%      ===============================
%%      Sample Songbook (sample-sb.tex)
%%      ===============================
%%
%%      Version 4.5, 30 April, 2010
%%
%%      Copyright 1992--2010 Christopher Rath <christopher@rath.ca>
%%
%%      This package is free software; you can redistribute it and/or
%%      modify it under the terms of version 2.1 of the GNU Lesser
%%	General Public License as published by the Free Software 
%%	Foundation.
%%
%%      This package is distributed in the hope that it will be
%%      useful, but WITHOUT ANY WARRANTY; without even the implied
%%      warranty of MERCHANTABILITY or FITNESS FOR A PARTICULAR
%%      PURPOSE.  See the GNU Lesser General Public License for more
%%      details.
%%
%%      This file contains a subset of the songbook we distribute
%%      at our church.  To the best of my knowledge, all of the lyrics
%%      contained herein are freely distributable.  This file has been
%%      provided as a sample of what can be produced by the chordbk,
%%      wordbk, and overhead LaTeX styles.
%%
%%      NEEDED:  The fancyhdr LaTeX style is required to properly
%%              format this file.  If you don't have that then comment
%%              out the commands in the preamble which deal with the
%%              fancyhdr style.
%%
%%%%%%
%%%%%%
%%
%%      1. Chord notation.  Within this songbook the following
%%         conventions have been adopted:
%%
%%              "Minor" is entered as "m";
%%                      e.g. Cm7 for C minor 7th.
%%              "Major" is entered as "M";
%%                      e.g. CM7 for C major 7th.
%%
%%%%%%
%%%%%%
%%      ============
%%      Bibliography
%%      ============
%%
%%      Exalt Him!: Exalt Him!  Compiled by Tom Fettke.  (c)1989
%%                      Word Music.
%%
%%      Hosanna! Music Books: Hosanna! Music Books #1--#6.
%%                      (c)1987--92 Integrity Music, Inc.
%%
%%      Worship Him II: Worship Him II.  Compiled by Jesse Peterson
%%                      and Bruce Ballinger.  (c)1989 Tempo Music
%%                      Publications.
%%
%%      Worship Songs Of The Vineyard: Worship Songs Of The Vineyard
%%                      --- Volume 2.  (c)1989 Vineyard Ministries
%%                      International.
%%
%%%%%%
%%%%%%

%%%%%%%%%%%%%%%%%%%%%%%%%%%%%%%%%%%%%%%%%%%%%%%%%%%%%%%%%%
%%%%%%%%%%%%%%%%%%%%%%%%%%%%%%%%%%%%%%%%%%%%%%%%%%%%%%%%%%
%%                                                      %%
%%           P R E A M B L E   B E G I N S              %%
%%                                                      %%
%%%%%%%%%%%%%%%%%%%%%%%%%%%%%%%%%%%%%%%%%%%%%%%%%%%%%%%%%%
%%%%%%%%%%%%%%%%%%%%%%%%%%%%%%%%%%%%%%%%%%%%%%%%%%%%%%%%%%

\documentclass[a5paper]{book}
\usepackage{latexsym,
            fancyhdr,
            titlesec,
            amsmath,
            amssymb,
            multicol,
            amsthm,
            stmaryrd,
            amsthm,
            color,
            needspace,
            stackengine,
            wasysym}
\usepackage[utf8]{inputenc}
\usepackage[T1]{fontenc}
% \usepackage[chordbk]{songbook}                  %% Words & Chords edition.
%%\usepackage[compactallsongs,chordbk]{songbook}    %% Words & Chords edition.
\usepackage[wordbk]{songbook}                 %% Words Only edition.
%%\usepackage[overhead]{songbook}               %% Overhead Transparency edition.
\usepackage{titletoc}
\usepackage{tket}  % Draws "TÅGEKAMMERET" correctly

%%%
% Revision Date and Release Date definitions.
%
%       \RelDate - The last time this songbook was released.  Set this
%                  date each time a new release/update of the songbook
%                  is generated.
%       \RevDate - The last time a particular song was revised in any
%                  way.  This command will be renewed inside every
%                  song.
%%%
\newcommand{\RelDate}{31~August,~2003}
\newcommand{\RevDate}{\today}

%%%
% C.C.L.I. license number definition; for copyright licensing info.
% One of these macros will be manually inserted into the {SBMel}
% parameter of the {song} environment.
%
%       \CCLInumber - The actual copyright license number.  Don't
%               insert this command in the {SBMel} parameter, use one
%               of the others.
%       \CCLIed - Indicates a song falls under our CCLI license.
%       \NotCCLIed - Indicates a song doesn't fall under our CCLI
%               license.  Public Domain songs fall into this category.
%       \PGranted - We have received specific permission from the
%               copyright holder to use this song.
%       \PPending - We are in the process of obtaining permission to
%               use this song.
%%%
\newcommand{\CCLInumber}{Your CCLI Number}
\newcommand{\CCLIed}{{\SBMelInfoFont (CCLI \CCLInumber)}}
\newcommand{\NotCCLIed}{\relax}
\newcommand{\PGranted}{\relax}
\newcommand{\PPending}{{\SBMelInfoFont (Permission Pending)}}

%%%
% Title page information.
%%%
%\title{UNF Computer Science Camp 2019 Sangbog}
%\author{}
%\date{Revideret:  \RevDate}

%%%
% Redefine fonts from SongBook style that I don't like.
%%%
\font\myTinySF=cmss8 at 8pt
\renewcommand{\SBMelInfoFont}{\tiny\myTinySF}

%%%
% Define fonts to use in the headers and footers of the songbook.
%%%
\newcommand{\LHeadFont}{\normalsize}            % = cmr12  at 12pt
\newcommand{\CHeadFont}{\normalsize\rm}         % = cmr12  at 12pt
\newcommand{\RHeadFont}{\normalsize}            % = cmr12  at 12pt
\newcommand{\LFootFont}{\scriptsize}            % = cmr8   at  8pt
\newcommand{\CFootFont}{\tiny\myTinySF}         % = cmss8  at  8pt
\newcommand{\RFootFont}{\scriptsize}            % = cmr8   at  8pt

\def\repeat{%
  \stackanchor{.}{.}%
  \rule[-\dp\strutbox]{.3pt}{\normalbaselineskip}%
  \kern0.5pt%
  \rule[-\dp\strutbox]{1pt}{\normalbaselineskip}%
  \kern1pt%
}
\def\frepeat{%
  \kern1pt%
  \rule[-\dp\strutbox]{1pt}{\normalbaselineskip}%
  \kern0.5pt%
  \rule[-\dp\strutbox]{.3pt}{\normalbaselineskip}%
  \stackanchor{.}{.}%
}
% \newcommand{\SBRepeat}[1]{#1\\#1}
\newcommand{\SBRepeat}[1]{\frepeat #1\repeat}
\setcounter{SBSongCnt}{-1}
\renewcommand{\SBWAndMTag}{Forfatter:}
\renewcommand{\SBUnknownTag}{Ukendt}
\renewcommand{\SBChorusTag}{Ref.}
\renewcommand{\SBOrgMel}{Originalmelodi}
\renewcommand{\SpaceAfterChorus}   {\vspace{0ex plus1ex minus 0.5ex}}
\renewcommand{\SpaceAfterOpGroup}  {\vspace{0ex plus1ex minus 0.5ex}}
\renewcommand{\SpaceAfterSBBracket}{\vspace{0ex plus1ex minus 0.5ex}}
\renewcommand{\SpaceAfterSection}  {\vspace{0ex plus1ex minus 0.5ex}}
\renewcommand{\SpaceAfterSong}     {\vspace{0ex plus1ex minus 0.5ex}}
\renewcommand{\SpaceAfterVerse}    {\vspace{0ex plus1ex minus 0.5ex}}

% Tell LaTeX that \medskip is a good place to make a page break
\let\oldmedskip\medskip
\renewcommand{\medskip}{\oldmedskip\pagebreak[2]}

%%%
% Turn on/off index-file generation.  Uncomment the \makeindex line to
% turn index generation on;  comment it out to turn index generation
% off.
%%%
%\makeTitleIndex         %% Title and First Line Index.
%\makeTitleContents      %% Table of Contents.
%\makeKeyIndex           %% Index of song by key.
% \makeArtistIndex	%% Index of song by artist.
% \newcommand{\SBThechapter}[0]{}
% \newcommand{\SBChapter}[1]{
%     \startcontents
%     \chapter*{#1} 
%     % \input{unf-sangbog.toc}
%       \begin{minipage}{.8\textwidth}
%         \printcontents{}{1}{}
%       \end{minipage}%
%     \renewcommand{\SBThechapter}{#1}
%     \clearpage
% }

% \titleformat{\chapter}
% [display]
% {}
% {%\vspace*{\fill}
%  % \titlerule[1pt]%
%  % \vspace{1pt}%
%  % \titlerule
%  % \vspace{1pc}%
%  \chaptertitlename}
% {}
% {\Huge}



%%%%%%%%%%%%%%%%%%%%%%%%%%%%%%%%%%%%%%%%%%%%%%%%%%%%%%%%%%
%%%%%%%%%%%%%%%%%%%%%%%%%%%%%%%%%%%%%%%%%%%%%%%%%%%%%%%%%%
%%                                                      %%
%%           D O C U M E N T   B E G I N S              %%
%%                                                      %%
%%%%%%%%%%%%%%%%%%%%%%%%%%%%%%%%%%%%%%%%%%%%%%%%%%%%%%%%%%
%%%%%%%%%%%%%%%%%%%%%%%%%%%%%%%%%%%%%%%%%%%%%%%%%%%%%%%%%%
\begin{document}

%%%
% Uncomment "\maketitle" statement to make a title page.
%%%
%\maketitle
% \begin{titlepage}
%   \centering
%   \vspace{5cm}
% 	\includegraphics[width=1\textwidth]{unf_logo.jpeg}\par\vspace{1cm}
% 	{\scshape\LARGE Sangbog \par}
% 	\vspace{1cm}
% 	{\scshape\Large UNF Computer Science Camp 2019\par}
	
% 	\vfill

% % Bottom of the page
% 	{\large \today\par}
% \end{titlepage}
% \mainmatter
% \ifWordBk
%   \twocolumn
% \fi


%%% Kolofon
%\thispagestyle{empty}
%Sammensat til UNF Computer Science Camp 2019 - csc.unf.dk\\
%Redaktør: Andreas Mosbæk Jensen m.fl. efter tidligere sangbog af Steffen Strunge Mathiesen\\
%Indhold opsat i \LaTeX. 
%Digital version og kildekode: github.com/steffen555/UNF-sangbog\\
%Revision 1 med stave fejl korrektioner
%\par\vspace*{\fill}
%Hvis du har forslag til sange, rettelser, ris og ros, eller hvis du kender en ukendt forfatter, så skriv til sangbog@unf.dk.

%%%
% Turn on and define fancy page heading/footing definition.
%%%
% \pagestyle{fancy}

% \ifChordBk
%   % It's a words & chords songbook...
%   \addtolength{\headwidth}{\marginparsep}
%   \addtolength{\headwidth}{\marginparwidth}
%   \renewcommand{\headrulewidth}{0.4pt}
%   \renewcommand{\footrulewidth}{0.4pt}
%   \fancyhead[LE,RO]{\LHeadFont\emph{\leftmark\/}\SBContinueMark}
%   \fancyhead[CE,CO]{\CHeadFont\thepage}
%   \fancyhead[RE,LO]{\RHeadFont \chaptermark}
% \else\ifOverhead
%   % It's an overhead...
%   \renewcommand{\footrulewidth}{0pt}
%   \renewcommand{\headrulewidth}{0pt}
%   \fancyhead[LE,RO]{}
%   \fancyhead[CE,CO]{}
%   \fancyhead[RE,LO]{}
% \else\ifWordBk
%   % It's a words only songbook...
%   \addtolength{\headwidth}{\marginparsep}
%   \addtolength{\headwidth}{\marginparwidth}
%   \renewcommand{\headrulewidth}{0.4pt}
%   \renewcommand{\footrulewidth}{0.4pt}
%   \fancyhead[LE,RO]{\LHeadFont Naturvidenskab revy sange}
%   \fancyhead[CE,CO]{\CHeadFont\thepage}
%   \fancyhead[RE,LO]{\RHeadFont \SBThechapter}
% \fi\fi\fi

% \fancyfoot[LE,RO]{\LFootFont Computer Science Camp 2019}
% \ifSongEject
%   \fancyfoot[CE,CO]{\CFootFont Last Revised:  \RevDate}
% \else
%   \fancyfoot[CE,CO]{\CFootFont}
% \fi
% \fancyfoot[RE,LO]{\RFootFont Synges på eget ansvar}

%%%
% Table of contents
%%%

% \clearpage
% \twocolumn
% \font\myTinySF=cmss8    at  8pt
% \font\myHugeSF=cmssbx10 at 25pt
% \newcommand{\CpyRtInfoFont}{\tiny\myTinySF}
% \newcommand{\myTitleFont}{\Huge\myHugeSF}
% \newcommand{\mySubTitleFont}{\large\sf}
% \renewcommand{\indexspace}{\medskip}

% % {\parindent 8pt
% %   {\myTitleFont Indhold}}\par
% % \vskip 5pt
% \renewcommand{\SBThechapter}{Indhold}
% % {\parindent 20pt
% %   {\mySubTitleFont --- with first lines in italic ---}}
% % \vskip 20pt
% \let\olditem\item
% \let\oldsubitem\subitem
% \let\oldsubsubitem\subsubitem
% \renewcommand{\item}{\par\hangindent=40pt}
% \renewcommand{\subitem}{\par\hangindent=40pt \hspace*{20pt}}
% \renewcommand{\subsubitem}{\par\hangindent=40pt \hspace*{30pt}}

% %\input{unf-sangbog.tocx}

% \renewcommand{\item}{\olditem}
% \renewcommand{\subitem}{\oldsubitem}
% \renewcommand{\subsubitem}{\oldsubsubitem}

%%%
% Songbook begins.
%%%

\twocolumn
%It's just one page, don't print page numbers etc.
\pagestyle{empty}
%Songs included
\input{songs/matmatik.tex}
\input{songs/taal_daj.tex}
\input{songs/linieskriverdriver.tex}
\input{songs/steve_hawking.tex}
\input{songs/ode_til_kode.tex}
\input{songs/se_min_kode.tex}
\input{songs/vaabenfysik_kort.tex}
%Maybe include:
%\input{songs/kvanter_i_maaneskin.tex}
%\input{songs/mest_matematiske_dyr.tex}

% \input{songs/vi_kan_ikke_li.tex}
% \input{songs/selektionssangen.tex}
% \input{songs/alfabetsangen.tex}
% \input{songs/sciencecamps.tex}
% \input{songs/hvad_maa_man.tex}


% \input{songs/lambda_kalkylen.tex}
% \input{songs/puslespil.tex}
% \input{songs/null.tex}
% \input{songs/fasebal.tex}

% \input{songs/chifitter.tex}

% \input{songs/kun_fysik.tex}



% \input{songs/kanoniske.tex}
% \input{songs/jeg_er_en_matematiker_fra_hcoe.tex}


% \input{songs/rekursiv_skovsang.tex}
% \input{songs/laerkerede.tex}


% \clearpage
% \font\myTinySF=cmss8    at  8pt
% \font\myHugeSF=cmssbx10 at 25pt
% % \newcommand{\CpyRtInfoFont}{\tiny\myTinySF}
% % \newcommand{\myTitleFont}{\Huge\myHugeSF}
% % \newcommand{\mySubTitleFont}{\large\sf}
% \renewcommand{\indexspace}{\medskip}

% {\parindent 8pt
%   {\myTitleFont Index}}\par
% \vskip 5pt
% \renewcommand{\SBThechapter}{Index}
% % {\parindent 20pt
% %   {\mySubTitleFont --- with first lines in italic ---}}
% % \vskip 20pt
% \renewcommand{\item}{\par\hangindent=40pt}
% \renewcommand{\subitem}{\par\hangindent=40pt \hspace*{20pt}}
% \renewcommand{\subsubitem}{\par\hangindent=40pt \hspace*{30pt}}

%\input{unf-sangbog.tdx}

\end{document}
\bye
%
%%%
% Document ends.
%%%

%       \begin{minipage}{.8\textwidth}
%         \printcontents{}{1}{}
%       \end{minipage}%
%     \renewcommand{\SBThechapter}{#1}
%     \clearpage
% }

% \titleformat{\chapter}
% [display]
% {}
% {%\vspace*{\fill}
%  % \titlerule[1pt]%
%  % \vspace{1pt}%
%  % \titlerule
%  % \vspace{1pc}%
%  \chaptertitlename}
% {}
% {\Huge}



%%%%%%%%%%%%%%%%%%%%%%%%%%%%%%%%%%%%%%%%%%%%%%%%%%%%%%%%%%
%%%%%%%%%%%%%%%%%%%%%%%%%%%%%%%%%%%%%%%%%%%%%%%%%%%%%%%%%%
%%                                                      %%
%%           D O C U M E N T   B E G I N S              %%
%%                                                      %%
%%%%%%%%%%%%%%%%%%%%%%%%%%%%%%%%%%%%%%%%%%%%%%%%%%%%%%%%%%
%%%%%%%%%%%%%%%%%%%%%%%%%%%%%%%%%%%%%%%%%%%%%%%%%%%%%%%%%%
\begin{document}

%%%
% Uncomment "\maketitle" statement to make a title page.
%%%
%\maketitle
% \begin{titlepage}
%   \centering
%   \vspace{5cm}
% 	\includegraphics[width=1\textwidth]{unf_logo.jpeg}\par\vspace{1cm}
% 	{\scshape\LARGE Sangbog \par}
% 	\vspace{1cm}
% 	{\scshape\Large UNF Computer Science Camp 2019\par}
	
% 	\vfill

% % Bottom of the page
% 	{\large \today\par}
% \end{titlepage}
% \mainmatter
% \ifWordBk
%   \twocolumn
% \fi


%%% Kolofon
%\thispagestyle{empty}
%Sammensat til UNF Computer Science Camp 2019 - csc.unf.dk\\
%Redaktør: Andreas Mosbæk Jensen m.fl. efter tidligere sangbog af Steffen Strunge Mathiesen\\
%Indhold opsat i \LaTeX. 
%Digital version og kildekode: github.com/steffen555/UNF-sangbog\\
%Revision 1 med stave fejl korrektioner
%\par\vspace*{\fill}
%Hvis du har forslag til sange, rettelser, ris og ros, eller hvis du kender en ukendt forfatter, så skriv til sangbog@unf.dk.

%%%
% Turn on and define fancy page heading/footing definition.
%%%
% \pagestyle{fancy}

% \ifChordBk
%   % It's a words & chords songbook...
%   \addtolength{\headwidth}{\marginparsep}
%   \addtolength{\headwidth}{\marginparwidth}
%   \renewcommand{\headrulewidth}{0.4pt}
%   \renewcommand{\footrulewidth}{0.4pt}
%   \fancyhead[LE,RO]{\LHeadFont\emph{\leftmark\/}\SBContinueMark}
%   \fancyhead[CE,CO]{\CHeadFont\thepage}
%   \fancyhead[RE,LO]{\RHeadFont \chaptermark}
% \else\ifOverhead
%   % It's an overhead...
%   \renewcommand{\footrulewidth}{0pt}
%   \renewcommand{\headrulewidth}{0pt}
%   \fancyhead[LE,RO]{}
%   \fancyhead[CE,CO]{}
%   \fancyhead[RE,LO]{}
% \else\ifWordBk
%   % It's a words only songbook...
%   \addtolength{\headwidth}{\marginparsep}
%   \addtolength{\headwidth}{\marginparwidth}
%   \renewcommand{\headrulewidth}{0.4pt}
%   \renewcommand{\footrulewidth}{0.4pt}
%   \fancyhead[LE,RO]{\LHeadFont Naturvidenskab revy sange}
%   \fancyhead[CE,CO]{\CHeadFont\thepage}
%   \fancyhead[RE,LO]{\RHeadFont \SBThechapter}
% \fi\fi\fi

% \fancyfoot[LE,RO]{\LFootFont Computer Science Camp 2019}
% \ifSongEject
%   \fancyfoot[CE,CO]{\CFootFont Last Revised:  \RevDate}
% \else
%   \fancyfoot[CE,CO]{\CFootFont}
% \fi
% \fancyfoot[RE,LO]{\RFootFont Synges på eget ansvar}

%%%
% Table of contents
%%%

% \clearpage
% \twocolumn
% \font\myTinySF=cmss8    at  8pt
% \font\myHugeSF=cmssbx10 at 25pt
% \newcommand{\CpyRtInfoFont}{\tiny\myTinySF}
% \newcommand{\myTitleFont}{\Huge\myHugeSF}
% \newcommand{\mySubTitleFont}{\large\sf}
% \renewcommand{\indexspace}{\medskip}

% % {\parindent 8pt
% %   {\myTitleFont Indhold}}\par
% % \vskip 5pt
% \renewcommand{\SBThechapter}{Indhold}
% % {\parindent 20pt
% %   {\mySubTitleFont --- with first lines in italic ---}}
% % \vskip 20pt
% \let\olditem\item
% \let\oldsubitem\subitem
% \let\oldsubsubitem\subsubitem
% \renewcommand{\item}{\par\hangindent=40pt}
% \renewcommand{\subitem}{\par\hangindent=40pt \hspace*{20pt}}
% \renewcommand{\subsubitem}{\par\hangindent=40pt \hspace*{30pt}}

% %%%%%%% rcsid = @(#)$Id: sample-sb.tex,v 1.23 2010-04-12 18:04:11 rathc Exp $
%%%%%%
%%
%%      ===============================
%%      Sample Songbook (sample-sb.tex)
%%      ===============================
%%
%%      Version 4.5, 30 April, 2010
%%
%%      Copyright 1992--2010 Christopher Rath <christopher@rath.ca>
%%
%%      This package is free software; you can redistribute it and/or
%%      modify it under the terms of version 2.1 of the GNU Lesser
%%	General Public License as published by the Free Software 
%%	Foundation.
%%
%%      This package is distributed in the hope that it will be
%%      useful, but WITHOUT ANY WARRANTY; without even the implied
%%      warranty of MERCHANTABILITY or FITNESS FOR A PARTICULAR
%%      PURPOSE.  See the GNU Lesser General Public License for more
%%      details.
%%
%%      This file contains a subset of the songbook we distribute
%%      at our church.  To the best of my knowledge, all of the lyrics
%%      contained herein are freely distributable.  This file has been
%%      provided as a sample of what can be produced by the chordbk,
%%      wordbk, and overhead LaTeX styles.
%%
%%      NEEDED:  The fancyhdr LaTeX style is required to properly
%%              format this file.  If you don't have that then comment
%%              out the commands in the preamble which deal with the
%%              fancyhdr style.
%%
%%%%%%
%%%%%%
%%
%%      1. Chord notation.  Within this songbook the following
%%         conventions have been adopted:
%%
%%              "Minor" is entered as "m";
%%                      e.g. Cm7 for C minor 7th.
%%              "Major" is entered as "M";
%%                      e.g. CM7 for C major 7th.
%%
%%%%%%
%%%%%%
%%      ============
%%      Bibliography
%%      ============
%%
%%      Exalt Him!: Exalt Him!  Compiled by Tom Fettke.  (c)1989
%%                      Word Music.
%%
%%      Hosanna! Music Books: Hosanna! Music Books #1--#6.
%%                      (c)1987--92 Integrity Music, Inc.
%%
%%      Worship Him II: Worship Him II.  Compiled by Jesse Peterson
%%                      and Bruce Ballinger.  (c)1989 Tempo Music
%%                      Publications.
%%
%%      Worship Songs Of The Vineyard: Worship Songs Of The Vineyard
%%                      --- Volume 2.  (c)1989 Vineyard Ministries
%%                      International.
%%
%%%%%%
%%%%%%

%%%%%%%%%%%%%%%%%%%%%%%%%%%%%%%%%%%%%%%%%%%%%%%%%%%%%%%%%%
%%%%%%%%%%%%%%%%%%%%%%%%%%%%%%%%%%%%%%%%%%%%%%%%%%%%%%%%%%
%%                                                      %%
%%           P R E A M B L E   B E G I N S              %%
%%                                                      %%
%%%%%%%%%%%%%%%%%%%%%%%%%%%%%%%%%%%%%%%%%%%%%%%%%%%%%%%%%%
%%%%%%%%%%%%%%%%%%%%%%%%%%%%%%%%%%%%%%%%%%%%%%%%%%%%%%%%%%

\documentclass[a5paper]{book}
\usepackage{latexsym,
            fancyhdr,
            titlesec,
            amsmath,
            amssymb,
            multicol,
            amsthm,
            stmaryrd,
            amsthm,
            color,
            needspace,
            stackengine,
            wasysym}
\usepackage[utf8]{inputenc}
\usepackage[T1]{fontenc}
% \usepackage[chordbk]{songbook}                  %% Words & Chords edition.
%%\usepackage[compactallsongs,chordbk]{songbook}    %% Words & Chords edition.
\usepackage[wordbk]{songbook}                 %% Words Only edition.
%%\usepackage[overhead]{songbook}               %% Overhead Transparency edition.
\usepackage{titletoc}
\usepackage{tket}  % Draws "TÅGEKAMMERET" correctly

%%%
% Revision Date and Release Date definitions.
%
%       \RelDate - The last time this songbook was released.  Set this
%                  date each time a new release/update of the songbook
%                  is generated.
%       \RevDate - The last time a particular song was revised in any
%                  way.  This command will be renewed inside every
%                  song.
%%%
\newcommand{\RelDate}{31~August,~2003}
\newcommand{\RevDate}{\today}

%%%
% C.C.L.I. license number definition; for copyright licensing info.
% One of these macros will be manually inserted into the {SBMel}
% parameter of the {song} environment.
%
%       \CCLInumber - The actual copyright license number.  Don't
%               insert this command in the {SBMel} parameter, use one
%               of the others.
%       \CCLIed - Indicates a song falls under our CCLI license.
%       \NotCCLIed - Indicates a song doesn't fall under our CCLI
%               license.  Public Domain songs fall into this category.
%       \PGranted - We have received specific permission from the
%               copyright holder to use this song.
%       \PPending - We are in the process of obtaining permission to
%               use this song.
%%%
\newcommand{\CCLInumber}{Your CCLI Number}
\newcommand{\CCLIed}{{\SBMelInfoFont (CCLI \CCLInumber)}}
\newcommand{\NotCCLIed}{\relax}
\newcommand{\PGranted}{\relax}
\newcommand{\PPending}{{\SBMelInfoFont (Permission Pending)}}

%%%
% Title page information.
%%%
%\title{UNF Computer Science Camp 2019 Sangbog}
%\author{}
%\date{Revideret:  \RevDate}

%%%
% Redefine fonts from SongBook style that I don't like.
%%%
\font\myTinySF=cmss8 at 8pt
\renewcommand{\SBMelInfoFont}{\tiny\myTinySF}

%%%
% Define fonts to use in the headers and footers of the songbook.
%%%
\newcommand{\LHeadFont}{\normalsize}            % = cmr12  at 12pt
\newcommand{\CHeadFont}{\normalsize\rm}         % = cmr12  at 12pt
\newcommand{\RHeadFont}{\normalsize}            % = cmr12  at 12pt
\newcommand{\LFootFont}{\scriptsize}            % = cmr8   at  8pt
\newcommand{\CFootFont}{\tiny\myTinySF}         % = cmss8  at  8pt
\newcommand{\RFootFont}{\scriptsize}            % = cmr8   at  8pt

\def\repeat{%
  \stackanchor{.}{.}%
  \rule[-\dp\strutbox]{.3pt}{\normalbaselineskip}%
  \kern0.5pt%
  \rule[-\dp\strutbox]{1pt}{\normalbaselineskip}%
  \kern1pt%
}
\def\frepeat{%
  \kern1pt%
  \rule[-\dp\strutbox]{1pt}{\normalbaselineskip}%
  \kern0.5pt%
  \rule[-\dp\strutbox]{.3pt}{\normalbaselineskip}%
  \stackanchor{.}{.}%
}
% \newcommand{\SBRepeat}[1]{#1\\#1}
\newcommand{\SBRepeat}[1]{\frepeat #1\repeat}
\setcounter{SBSongCnt}{-1}
\renewcommand{\SBWAndMTag}{Forfatter:}
\renewcommand{\SBUnknownTag}{Ukendt}
\renewcommand{\SBChorusTag}{Ref.}
\renewcommand{\SBOrgMel}{Originalmelodi}
\renewcommand{\SpaceAfterChorus}   {\vspace{0ex plus1ex minus 0.5ex}}
\renewcommand{\SpaceAfterOpGroup}  {\vspace{0ex plus1ex minus 0.5ex}}
\renewcommand{\SpaceAfterSBBracket}{\vspace{0ex plus1ex minus 0.5ex}}
\renewcommand{\SpaceAfterSection}  {\vspace{0ex plus1ex minus 0.5ex}}
\renewcommand{\SpaceAfterSong}     {\vspace{0ex plus1ex minus 0.5ex}}
\renewcommand{\SpaceAfterVerse}    {\vspace{0ex plus1ex minus 0.5ex}}

% Tell LaTeX that \medskip is a good place to make a page break
\let\oldmedskip\medskip
\renewcommand{\medskip}{\oldmedskip\pagebreak[2]}

%%%
% Turn on/off index-file generation.  Uncomment the \makeindex line to
% turn index generation on;  comment it out to turn index generation
% off.
%%%
%\makeTitleIndex         %% Title and First Line Index.
%\makeTitleContents      %% Table of Contents.
%\makeKeyIndex           %% Index of song by key.
% \makeArtistIndex	%% Index of song by artist.
% \newcommand{\SBThechapter}[0]{}
% \newcommand{\SBChapter}[1]{
%     \startcontents
%     \chapter*{#1} 
%     % \input{unf-sangbog.toc}
%       \begin{minipage}{.8\textwidth}
%         \printcontents{}{1}{}
%       \end{minipage}%
%     \renewcommand{\SBThechapter}{#1}
%     \clearpage
% }

% \titleformat{\chapter}
% [display]
% {}
% {%\vspace*{\fill}
%  % \titlerule[1pt]%
%  % \vspace{1pt}%
%  % \titlerule
%  % \vspace{1pc}%
%  \chaptertitlename}
% {}
% {\Huge}



%%%%%%%%%%%%%%%%%%%%%%%%%%%%%%%%%%%%%%%%%%%%%%%%%%%%%%%%%%
%%%%%%%%%%%%%%%%%%%%%%%%%%%%%%%%%%%%%%%%%%%%%%%%%%%%%%%%%%
%%                                                      %%
%%           D O C U M E N T   B E G I N S              %%
%%                                                      %%
%%%%%%%%%%%%%%%%%%%%%%%%%%%%%%%%%%%%%%%%%%%%%%%%%%%%%%%%%%
%%%%%%%%%%%%%%%%%%%%%%%%%%%%%%%%%%%%%%%%%%%%%%%%%%%%%%%%%%
\begin{document}

%%%
% Uncomment "\maketitle" statement to make a title page.
%%%
%\maketitle
% \begin{titlepage}
%   \centering
%   \vspace{5cm}
% 	\includegraphics[width=1\textwidth]{unf_logo.jpeg}\par\vspace{1cm}
% 	{\scshape\LARGE Sangbog \par}
% 	\vspace{1cm}
% 	{\scshape\Large UNF Computer Science Camp 2019\par}
	
% 	\vfill

% % Bottom of the page
% 	{\large \today\par}
% \end{titlepage}
% \mainmatter
% \ifWordBk
%   \twocolumn
% \fi


%%% Kolofon
%\thispagestyle{empty}
%Sammensat til UNF Computer Science Camp 2019 - csc.unf.dk\\
%Redaktør: Andreas Mosbæk Jensen m.fl. efter tidligere sangbog af Steffen Strunge Mathiesen\\
%Indhold opsat i \LaTeX. 
%Digital version og kildekode: github.com/steffen555/UNF-sangbog\\
%Revision 1 med stave fejl korrektioner
%\par\vspace*{\fill}
%Hvis du har forslag til sange, rettelser, ris og ros, eller hvis du kender en ukendt forfatter, så skriv til sangbog@unf.dk.

%%%
% Turn on and define fancy page heading/footing definition.
%%%
% \pagestyle{fancy}

% \ifChordBk
%   % It's a words & chords songbook...
%   \addtolength{\headwidth}{\marginparsep}
%   \addtolength{\headwidth}{\marginparwidth}
%   \renewcommand{\headrulewidth}{0.4pt}
%   \renewcommand{\footrulewidth}{0.4pt}
%   \fancyhead[LE,RO]{\LHeadFont\emph{\leftmark\/}\SBContinueMark}
%   \fancyhead[CE,CO]{\CHeadFont\thepage}
%   \fancyhead[RE,LO]{\RHeadFont \chaptermark}
% \else\ifOverhead
%   % It's an overhead...
%   \renewcommand{\footrulewidth}{0pt}
%   \renewcommand{\headrulewidth}{0pt}
%   \fancyhead[LE,RO]{}
%   \fancyhead[CE,CO]{}
%   \fancyhead[RE,LO]{}
% \else\ifWordBk
%   % It's a words only songbook...
%   \addtolength{\headwidth}{\marginparsep}
%   \addtolength{\headwidth}{\marginparwidth}
%   \renewcommand{\headrulewidth}{0.4pt}
%   \renewcommand{\footrulewidth}{0.4pt}
%   \fancyhead[LE,RO]{\LHeadFont Naturvidenskab revy sange}
%   \fancyhead[CE,CO]{\CHeadFont\thepage}
%   \fancyhead[RE,LO]{\RHeadFont \SBThechapter}
% \fi\fi\fi

% \fancyfoot[LE,RO]{\LFootFont Computer Science Camp 2019}
% \ifSongEject
%   \fancyfoot[CE,CO]{\CFootFont Last Revised:  \RevDate}
% \else
%   \fancyfoot[CE,CO]{\CFootFont}
% \fi
% \fancyfoot[RE,LO]{\RFootFont Synges på eget ansvar}

%%%
% Table of contents
%%%

% \clearpage
% \twocolumn
% \font\myTinySF=cmss8    at  8pt
% \font\myHugeSF=cmssbx10 at 25pt
% \newcommand{\CpyRtInfoFont}{\tiny\myTinySF}
% \newcommand{\myTitleFont}{\Huge\myHugeSF}
% \newcommand{\mySubTitleFont}{\large\sf}
% \renewcommand{\indexspace}{\medskip}

% % {\parindent 8pt
% %   {\myTitleFont Indhold}}\par
% % \vskip 5pt
% \renewcommand{\SBThechapter}{Indhold}
% % {\parindent 20pt
% %   {\mySubTitleFont --- with first lines in italic ---}}
% % \vskip 20pt
% \let\olditem\item
% \let\oldsubitem\subitem
% \let\oldsubsubitem\subsubitem
% \renewcommand{\item}{\par\hangindent=40pt}
% \renewcommand{\subitem}{\par\hangindent=40pt \hspace*{20pt}}
% \renewcommand{\subsubitem}{\par\hangindent=40pt \hspace*{30pt}}

% %\input{unf-sangbog.tocx}

% \renewcommand{\item}{\olditem}
% \renewcommand{\subitem}{\oldsubitem}
% \renewcommand{\subsubitem}{\oldsubsubitem}

%%%
% Songbook begins.
%%%

\twocolumn
%It's just one page, don't print page numbers etc.
\pagestyle{empty}
%Songs included
\input{songs/matmatik.tex}
\input{songs/taal_daj.tex}
\input{songs/linieskriverdriver.tex}
\input{songs/steve_hawking.tex}
\input{songs/ode_til_kode.tex}
\input{songs/se_min_kode.tex}
\input{songs/vaabenfysik_kort.tex}
%Maybe include:
%\input{songs/kvanter_i_maaneskin.tex}
%\input{songs/mest_matematiske_dyr.tex}

% \input{songs/vi_kan_ikke_li.tex}
% \input{songs/selektionssangen.tex}
% \input{songs/alfabetsangen.tex}
% \input{songs/sciencecamps.tex}
% \input{songs/hvad_maa_man.tex}


% \input{songs/lambda_kalkylen.tex}
% \input{songs/puslespil.tex}
% \input{songs/null.tex}
% \input{songs/fasebal.tex}

% \input{songs/chifitter.tex}

% \input{songs/kun_fysik.tex}



% \input{songs/kanoniske.tex}
% \input{songs/jeg_er_en_matematiker_fra_hcoe.tex}


% \input{songs/rekursiv_skovsang.tex}
% \input{songs/laerkerede.tex}


% \clearpage
% \font\myTinySF=cmss8    at  8pt
% \font\myHugeSF=cmssbx10 at 25pt
% % \newcommand{\CpyRtInfoFont}{\tiny\myTinySF}
% % \newcommand{\myTitleFont}{\Huge\myHugeSF}
% % \newcommand{\mySubTitleFont}{\large\sf}
% \renewcommand{\indexspace}{\medskip}

% {\parindent 8pt
%   {\myTitleFont Index}}\par
% \vskip 5pt
% \renewcommand{\SBThechapter}{Index}
% % {\parindent 20pt
% %   {\mySubTitleFont --- with first lines in italic ---}}
% % \vskip 20pt
% \renewcommand{\item}{\par\hangindent=40pt}
% \renewcommand{\subitem}{\par\hangindent=40pt \hspace*{20pt}}
% \renewcommand{\subsubitem}{\par\hangindent=40pt \hspace*{30pt}}

%\input{unf-sangbog.tdx}

\end{document}
\bye
%
%%%
% Document ends.
%%%


% \renewcommand{\item}{\olditem}
% \renewcommand{\subitem}{\oldsubitem}
% \renewcommand{\subsubitem}{\oldsubsubitem}

%%%
% Songbook begins.
%%%

\twocolumn
%It's just one page, don't print page numbers etc.
\pagestyle{empty}
%Songs included
\input{songs/matmatik.tex}
\input{songs/taal_daj.tex}
\input{songs/linieskriverdriver.tex}
\input{songs/steve_hawking.tex}
\input{songs/ode_til_kode.tex}
\input{songs/se_min_kode.tex}
\input{songs/vaabenfysik_kort.tex}
%Maybe include:
%\input{songs/kvanter_i_maaneskin.tex}
%\input{songs/mest_matematiske_dyr.tex}

% \input{songs/vi_kan_ikke_li.tex}
% \input{songs/selektionssangen.tex}
% \input{songs/alfabetsangen.tex}
% \input{songs/sciencecamps.tex}
% \input{songs/hvad_maa_man.tex}


% \input{songs/lambda_kalkylen.tex}
% \input{songs/puslespil.tex}
% \input{songs/null.tex}
% \input{songs/fasebal.tex}

% \input{songs/chifitter.tex}

% \input{songs/kun_fysik.tex}



% \input{songs/kanoniske.tex}
% \input{songs/jeg_er_en_matematiker_fra_hcoe.tex}


% \input{songs/rekursiv_skovsang.tex}
% \input{songs/laerkerede.tex}


% \clearpage
% \font\myTinySF=cmss8    at  8pt
% \font\myHugeSF=cmssbx10 at 25pt
% % \newcommand{\CpyRtInfoFont}{\tiny\myTinySF}
% % \newcommand{\myTitleFont}{\Huge\myHugeSF}
% % \newcommand{\mySubTitleFont}{\large\sf}
% \renewcommand{\indexspace}{\medskip}

% {\parindent 8pt
%   {\myTitleFont Index}}\par
% \vskip 5pt
% \renewcommand{\SBThechapter}{Index}
% % {\parindent 20pt
% %   {\mySubTitleFont --- with first lines in italic ---}}
% % \vskip 20pt
% \renewcommand{\item}{\par\hangindent=40pt}
% \renewcommand{\subitem}{\par\hangindent=40pt \hspace*{20pt}}
% \renewcommand{\subsubitem}{\par\hangindent=40pt \hspace*{30pt}}

%%%%%%% rcsid = @(#)$Id: sample-sb.tex,v 1.23 2010-04-12 18:04:11 rathc Exp $
%%%%%%
%%
%%      ===============================
%%      Sample Songbook (sample-sb.tex)
%%      ===============================
%%
%%      Version 4.5, 30 April, 2010
%%
%%      Copyright 1992--2010 Christopher Rath <christopher@rath.ca>
%%
%%      This package is free software; you can redistribute it and/or
%%      modify it under the terms of version 2.1 of the GNU Lesser
%%	General Public License as published by the Free Software 
%%	Foundation.
%%
%%      This package is distributed in the hope that it will be
%%      useful, but WITHOUT ANY WARRANTY; without even the implied
%%      warranty of MERCHANTABILITY or FITNESS FOR A PARTICULAR
%%      PURPOSE.  See the GNU Lesser General Public License for more
%%      details.
%%
%%      This file contains a subset of the songbook we distribute
%%      at our church.  To the best of my knowledge, all of the lyrics
%%      contained herein are freely distributable.  This file has been
%%      provided as a sample of what can be produced by the chordbk,
%%      wordbk, and overhead LaTeX styles.
%%
%%      NEEDED:  The fancyhdr LaTeX style is required to properly
%%              format this file.  If you don't have that then comment
%%              out the commands in the preamble which deal with the
%%              fancyhdr style.
%%
%%%%%%
%%%%%%
%%
%%      1. Chord notation.  Within this songbook the following
%%         conventions have been adopted:
%%
%%              "Minor" is entered as "m";
%%                      e.g. Cm7 for C minor 7th.
%%              "Major" is entered as "M";
%%                      e.g. CM7 for C major 7th.
%%
%%%%%%
%%%%%%
%%      ============
%%      Bibliography
%%      ============
%%
%%      Exalt Him!: Exalt Him!  Compiled by Tom Fettke.  (c)1989
%%                      Word Music.
%%
%%      Hosanna! Music Books: Hosanna! Music Books #1--#6.
%%                      (c)1987--92 Integrity Music, Inc.
%%
%%      Worship Him II: Worship Him II.  Compiled by Jesse Peterson
%%                      and Bruce Ballinger.  (c)1989 Tempo Music
%%                      Publications.
%%
%%      Worship Songs Of The Vineyard: Worship Songs Of The Vineyard
%%                      --- Volume 2.  (c)1989 Vineyard Ministries
%%                      International.
%%
%%%%%%
%%%%%%

%%%%%%%%%%%%%%%%%%%%%%%%%%%%%%%%%%%%%%%%%%%%%%%%%%%%%%%%%%
%%%%%%%%%%%%%%%%%%%%%%%%%%%%%%%%%%%%%%%%%%%%%%%%%%%%%%%%%%
%%                                                      %%
%%           P R E A M B L E   B E G I N S              %%
%%                                                      %%
%%%%%%%%%%%%%%%%%%%%%%%%%%%%%%%%%%%%%%%%%%%%%%%%%%%%%%%%%%
%%%%%%%%%%%%%%%%%%%%%%%%%%%%%%%%%%%%%%%%%%%%%%%%%%%%%%%%%%

\documentclass[a5paper]{book}
\usepackage{latexsym,
            fancyhdr,
            titlesec,
            amsmath,
            amssymb,
            multicol,
            amsthm,
            stmaryrd,
            amsthm,
            color,
            needspace,
            stackengine,
            wasysym}
\usepackage[utf8]{inputenc}
\usepackage[T1]{fontenc}
% \usepackage[chordbk]{songbook}                  %% Words & Chords edition.
%%\usepackage[compactallsongs,chordbk]{songbook}    %% Words & Chords edition.
\usepackage[wordbk]{songbook}                 %% Words Only edition.
%%\usepackage[overhead]{songbook}               %% Overhead Transparency edition.
\usepackage{titletoc}
\usepackage{tket}  % Draws "TÅGEKAMMERET" correctly

%%%
% Revision Date and Release Date definitions.
%
%       \RelDate - The last time this songbook was released.  Set this
%                  date each time a new release/update of the songbook
%                  is generated.
%       \RevDate - The last time a particular song was revised in any
%                  way.  This command will be renewed inside every
%                  song.
%%%
\newcommand{\RelDate}{31~August,~2003}
\newcommand{\RevDate}{\today}

%%%
% C.C.L.I. license number definition; for copyright licensing info.
% One of these macros will be manually inserted into the {SBMel}
% parameter of the {song} environment.
%
%       \CCLInumber - The actual copyright license number.  Don't
%               insert this command in the {SBMel} parameter, use one
%               of the others.
%       \CCLIed - Indicates a song falls under our CCLI license.
%       \NotCCLIed - Indicates a song doesn't fall under our CCLI
%               license.  Public Domain songs fall into this category.
%       \PGranted - We have received specific permission from the
%               copyright holder to use this song.
%       \PPending - We are in the process of obtaining permission to
%               use this song.
%%%
\newcommand{\CCLInumber}{Your CCLI Number}
\newcommand{\CCLIed}{{\SBMelInfoFont (CCLI \CCLInumber)}}
\newcommand{\NotCCLIed}{\relax}
\newcommand{\PGranted}{\relax}
\newcommand{\PPending}{{\SBMelInfoFont (Permission Pending)}}

%%%
% Title page information.
%%%
%\title{UNF Computer Science Camp 2019 Sangbog}
%\author{}
%\date{Revideret:  \RevDate}

%%%
% Redefine fonts from SongBook style that I don't like.
%%%
\font\myTinySF=cmss8 at 8pt
\renewcommand{\SBMelInfoFont}{\tiny\myTinySF}

%%%
% Define fonts to use in the headers and footers of the songbook.
%%%
\newcommand{\LHeadFont}{\normalsize}            % = cmr12  at 12pt
\newcommand{\CHeadFont}{\normalsize\rm}         % = cmr12  at 12pt
\newcommand{\RHeadFont}{\normalsize}            % = cmr12  at 12pt
\newcommand{\LFootFont}{\scriptsize}            % = cmr8   at  8pt
\newcommand{\CFootFont}{\tiny\myTinySF}         % = cmss8  at  8pt
\newcommand{\RFootFont}{\scriptsize}            % = cmr8   at  8pt

\def\repeat{%
  \stackanchor{.}{.}%
  \rule[-\dp\strutbox]{.3pt}{\normalbaselineskip}%
  \kern0.5pt%
  \rule[-\dp\strutbox]{1pt}{\normalbaselineskip}%
  \kern1pt%
}
\def\frepeat{%
  \kern1pt%
  \rule[-\dp\strutbox]{1pt}{\normalbaselineskip}%
  \kern0.5pt%
  \rule[-\dp\strutbox]{.3pt}{\normalbaselineskip}%
  \stackanchor{.}{.}%
}
% \newcommand{\SBRepeat}[1]{#1\\#1}
\newcommand{\SBRepeat}[1]{\frepeat #1\repeat}
\setcounter{SBSongCnt}{-1}
\renewcommand{\SBWAndMTag}{Forfatter:}
\renewcommand{\SBUnknownTag}{Ukendt}
\renewcommand{\SBChorusTag}{Ref.}
\renewcommand{\SBOrgMel}{Originalmelodi}
\renewcommand{\SpaceAfterChorus}   {\vspace{0ex plus1ex minus 0.5ex}}
\renewcommand{\SpaceAfterOpGroup}  {\vspace{0ex plus1ex minus 0.5ex}}
\renewcommand{\SpaceAfterSBBracket}{\vspace{0ex plus1ex minus 0.5ex}}
\renewcommand{\SpaceAfterSection}  {\vspace{0ex plus1ex minus 0.5ex}}
\renewcommand{\SpaceAfterSong}     {\vspace{0ex plus1ex minus 0.5ex}}
\renewcommand{\SpaceAfterVerse}    {\vspace{0ex plus1ex minus 0.5ex}}

% Tell LaTeX that \medskip is a good place to make a page break
\let\oldmedskip\medskip
\renewcommand{\medskip}{\oldmedskip\pagebreak[2]}

%%%
% Turn on/off index-file generation.  Uncomment the \makeindex line to
% turn index generation on;  comment it out to turn index generation
% off.
%%%
%\makeTitleIndex         %% Title and First Line Index.
%\makeTitleContents      %% Table of Contents.
%\makeKeyIndex           %% Index of song by key.
% \makeArtistIndex	%% Index of song by artist.
% \newcommand{\SBThechapter}[0]{}
% \newcommand{\SBChapter}[1]{
%     \startcontents
%     \chapter*{#1} 
%     % \input{unf-sangbog.toc}
%       \begin{minipage}{.8\textwidth}
%         \printcontents{}{1}{}
%       \end{minipage}%
%     \renewcommand{\SBThechapter}{#1}
%     \clearpage
% }

% \titleformat{\chapter}
% [display]
% {}
% {%\vspace*{\fill}
%  % \titlerule[1pt]%
%  % \vspace{1pt}%
%  % \titlerule
%  % \vspace{1pc}%
%  \chaptertitlename}
% {}
% {\Huge}



%%%%%%%%%%%%%%%%%%%%%%%%%%%%%%%%%%%%%%%%%%%%%%%%%%%%%%%%%%
%%%%%%%%%%%%%%%%%%%%%%%%%%%%%%%%%%%%%%%%%%%%%%%%%%%%%%%%%%
%%                                                      %%
%%           D O C U M E N T   B E G I N S              %%
%%                                                      %%
%%%%%%%%%%%%%%%%%%%%%%%%%%%%%%%%%%%%%%%%%%%%%%%%%%%%%%%%%%
%%%%%%%%%%%%%%%%%%%%%%%%%%%%%%%%%%%%%%%%%%%%%%%%%%%%%%%%%%
\begin{document}

%%%
% Uncomment "\maketitle" statement to make a title page.
%%%
%\maketitle
% \begin{titlepage}
%   \centering
%   \vspace{5cm}
% 	\includegraphics[width=1\textwidth]{unf_logo.jpeg}\par\vspace{1cm}
% 	{\scshape\LARGE Sangbog \par}
% 	\vspace{1cm}
% 	{\scshape\Large UNF Computer Science Camp 2019\par}
	
% 	\vfill

% % Bottom of the page
% 	{\large \today\par}
% \end{titlepage}
% \mainmatter
% \ifWordBk
%   \twocolumn
% \fi


%%% Kolofon
%\thispagestyle{empty}
%Sammensat til UNF Computer Science Camp 2019 - csc.unf.dk\\
%Redaktør: Andreas Mosbæk Jensen m.fl. efter tidligere sangbog af Steffen Strunge Mathiesen\\
%Indhold opsat i \LaTeX. 
%Digital version og kildekode: github.com/steffen555/UNF-sangbog\\
%Revision 1 med stave fejl korrektioner
%\par\vspace*{\fill}
%Hvis du har forslag til sange, rettelser, ris og ros, eller hvis du kender en ukendt forfatter, så skriv til sangbog@unf.dk.

%%%
% Turn on and define fancy page heading/footing definition.
%%%
% \pagestyle{fancy}

% \ifChordBk
%   % It's a words & chords songbook...
%   \addtolength{\headwidth}{\marginparsep}
%   \addtolength{\headwidth}{\marginparwidth}
%   \renewcommand{\headrulewidth}{0.4pt}
%   \renewcommand{\footrulewidth}{0.4pt}
%   \fancyhead[LE,RO]{\LHeadFont\emph{\leftmark\/}\SBContinueMark}
%   \fancyhead[CE,CO]{\CHeadFont\thepage}
%   \fancyhead[RE,LO]{\RHeadFont \chaptermark}
% \else\ifOverhead
%   % It's an overhead...
%   \renewcommand{\footrulewidth}{0pt}
%   \renewcommand{\headrulewidth}{0pt}
%   \fancyhead[LE,RO]{}
%   \fancyhead[CE,CO]{}
%   \fancyhead[RE,LO]{}
% \else\ifWordBk
%   % It's a words only songbook...
%   \addtolength{\headwidth}{\marginparsep}
%   \addtolength{\headwidth}{\marginparwidth}
%   \renewcommand{\headrulewidth}{0.4pt}
%   \renewcommand{\footrulewidth}{0.4pt}
%   \fancyhead[LE,RO]{\LHeadFont Naturvidenskab revy sange}
%   \fancyhead[CE,CO]{\CHeadFont\thepage}
%   \fancyhead[RE,LO]{\RHeadFont \SBThechapter}
% \fi\fi\fi

% \fancyfoot[LE,RO]{\LFootFont Computer Science Camp 2019}
% \ifSongEject
%   \fancyfoot[CE,CO]{\CFootFont Last Revised:  \RevDate}
% \else
%   \fancyfoot[CE,CO]{\CFootFont}
% \fi
% \fancyfoot[RE,LO]{\RFootFont Synges på eget ansvar}

%%%
% Table of contents
%%%

% \clearpage
% \twocolumn
% \font\myTinySF=cmss8    at  8pt
% \font\myHugeSF=cmssbx10 at 25pt
% \newcommand{\CpyRtInfoFont}{\tiny\myTinySF}
% \newcommand{\myTitleFont}{\Huge\myHugeSF}
% \newcommand{\mySubTitleFont}{\large\sf}
% \renewcommand{\indexspace}{\medskip}

% % {\parindent 8pt
% %   {\myTitleFont Indhold}}\par
% % \vskip 5pt
% \renewcommand{\SBThechapter}{Indhold}
% % {\parindent 20pt
% %   {\mySubTitleFont --- with first lines in italic ---}}
% % \vskip 20pt
% \let\olditem\item
% \let\oldsubitem\subitem
% \let\oldsubsubitem\subsubitem
% \renewcommand{\item}{\par\hangindent=40pt}
% \renewcommand{\subitem}{\par\hangindent=40pt \hspace*{20pt}}
% \renewcommand{\subsubitem}{\par\hangindent=40pt \hspace*{30pt}}

% %\input{unf-sangbog.tocx}

% \renewcommand{\item}{\olditem}
% \renewcommand{\subitem}{\oldsubitem}
% \renewcommand{\subsubitem}{\oldsubsubitem}

%%%
% Songbook begins.
%%%

\twocolumn
%It's just one page, don't print page numbers etc.
\pagestyle{empty}
%Songs included
\input{songs/matmatik.tex}
\input{songs/taal_daj.tex}
\input{songs/linieskriverdriver.tex}
\input{songs/steve_hawking.tex}
\input{songs/ode_til_kode.tex}
\input{songs/se_min_kode.tex}
\input{songs/vaabenfysik_kort.tex}
%Maybe include:
%\input{songs/kvanter_i_maaneskin.tex}
%\input{songs/mest_matematiske_dyr.tex}

% \input{songs/vi_kan_ikke_li.tex}
% \input{songs/selektionssangen.tex}
% \input{songs/alfabetsangen.tex}
% \input{songs/sciencecamps.tex}
% \input{songs/hvad_maa_man.tex}


% \input{songs/lambda_kalkylen.tex}
% \input{songs/puslespil.tex}
% \input{songs/null.tex}
% \input{songs/fasebal.tex}

% \input{songs/chifitter.tex}

% \input{songs/kun_fysik.tex}



% \input{songs/kanoniske.tex}
% \input{songs/jeg_er_en_matematiker_fra_hcoe.tex}


% \input{songs/rekursiv_skovsang.tex}
% \input{songs/laerkerede.tex}


% \clearpage
% \font\myTinySF=cmss8    at  8pt
% \font\myHugeSF=cmssbx10 at 25pt
% % \newcommand{\CpyRtInfoFont}{\tiny\myTinySF}
% % \newcommand{\myTitleFont}{\Huge\myHugeSF}
% % \newcommand{\mySubTitleFont}{\large\sf}
% \renewcommand{\indexspace}{\medskip}

% {\parindent 8pt
%   {\myTitleFont Index}}\par
% \vskip 5pt
% \renewcommand{\SBThechapter}{Index}
% % {\parindent 20pt
% %   {\mySubTitleFont --- with first lines in italic ---}}
% % \vskip 20pt
% \renewcommand{\item}{\par\hangindent=40pt}
% \renewcommand{\subitem}{\par\hangindent=40pt \hspace*{20pt}}
% \renewcommand{\subsubitem}{\par\hangindent=40pt \hspace*{30pt}}

%\input{unf-sangbog.tdx}

\end{document}
\bye
%
%%%
% Document ends.
%%%


\end{document}
\bye
%
%%%
% Document ends.
%%%


\end{document}
\bye
%
%%%
% Document ends.
%%%

%       \begin{minipage}{.8\textwidth}
%         \printcontents{}{1}{}
%       \end{minipage}%
%     \renewcommand{\SBThechapter}{#1}
%     \clearpage
% }

% \titleformat{\chapter}
% [display]
% {}
% {%\vspace*{\fill}
%  % \titlerule[1pt]%
%  % \vspace{1pt}%
%  % \titlerule
%  % \vspace{1pc}%
%  \chaptertitlename}
% {}
% {\Huge}



%%%%%%%%%%%%%%%%%%%%%%%%%%%%%%%%%%%%%%%%%%%%%%%%%%%%%%%%%%
%%%%%%%%%%%%%%%%%%%%%%%%%%%%%%%%%%%%%%%%%%%%%%%%%%%%%%%%%%
%%                                                      %%
%%           D O C U M E N T   B E G I N S              %%
%%                                                      %%
%%%%%%%%%%%%%%%%%%%%%%%%%%%%%%%%%%%%%%%%%%%%%%%%%%%%%%%%%%
%%%%%%%%%%%%%%%%%%%%%%%%%%%%%%%%%%%%%%%%%%%%%%%%%%%%%%%%%%
\begin{document}

%%%
% Uncomment "\maketitle" statement to make a title page.
%%%
%\maketitle
% \begin{titlepage}
%   \centering
%   \vspace{5cm}
% 	\includegraphics[width=1\textwidth]{unf_logo.jpeg}\par\vspace{1cm}
% 	{\scshape\LARGE Sangbog \par}
% 	\vspace{1cm}
% 	{\scshape\Large UNF Computer Science Camp 2019\par}
	
% 	\vfill

% % Bottom of the page
% 	{\large \today\par}
% \end{titlepage}
% \mainmatter
% \ifWordBk
%   \twocolumn
% \fi


%%% Kolofon
%\thispagestyle{empty}
%Sammensat til UNF Computer Science Camp 2019 - csc.unf.dk\\
%Redaktør: Andreas Mosbæk Jensen m.fl. efter tidligere sangbog af Steffen Strunge Mathiesen\\
%Indhold opsat i \LaTeX. 
%Digital version og kildekode: github.com/steffen555/UNF-sangbog\\
%Revision 1 med stave fejl korrektioner
%\par\vspace*{\fill}
%Hvis du har forslag til sange, rettelser, ris og ros, eller hvis du kender en ukendt forfatter, så skriv til sangbog@unf.dk.

%%%
% Turn on and define fancy page heading/footing definition.
%%%
% \pagestyle{fancy}

% \ifChordBk
%   % It's a words & chords songbook...
%   \addtolength{\headwidth}{\marginparsep}
%   \addtolength{\headwidth}{\marginparwidth}
%   \renewcommand{\headrulewidth}{0.4pt}
%   \renewcommand{\footrulewidth}{0.4pt}
%   \fancyhead[LE,RO]{\LHeadFont\emph{\leftmark\/}\SBContinueMark}
%   \fancyhead[CE,CO]{\CHeadFont\thepage}
%   \fancyhead[RE,LO]{\RHeadFont \chaptermark}
% \else\ifOverhead
%   % It's an overhead...
%   \renewcommand{\footrulewidth}{0pt}
%   \renewcommand{\headrulewidth}{0pt}
%   \fancyhead[LE,RO]{}
%   \fancyhead[CE,CO]{}
%   \fancyhead[RE,LO]{}
% \else\ifWordBk
%   % It's a words only songbook...
%   \addtolength{\headwidth}{\marginparsep}
%   \addtolength{\headwidth}{\marginparwidth}
%   \renewcommand{\headrulewidth}{0.4pt}
%   \renewcommand{\footrulewidth}{0.4pt}
%   \fancyhead[LE,RO]{\LHeadFont Naturvidenskab revy sange}
%   \fancyhead[CE,CO]{\CHeadFont\thepage}
%   \fancyhead[RE,LO]{\RHeadFont \SBThechapter}
% \fi\fi\fi

% \fancyfoot[LE,RO]{\LFootFont Computer Science Camp 2019}
% \ifSongEject
%   \fancyfoot[CE,CO]{\CFootFont Last Revised:  \RevDate}
% \else
%   \fancyfoot[CE,CO]{\CFootFont}
% \fi
% \fancyfoot[RE,LO]{\RFootFont Synges på eget ansvar}

%%%
% Table of contents
%%%

% \clearpage
% \twocolumn
% \font\myTinySF=cmss8    at  8pt
% \font\myHugeSF=cmssbx10 at 25pt
% \newcommand{\CpyRtInfoFont}{\tiny\myTinySF}
% \newcommand{\myTitleFont}{\Huge\myHugeSF}
% \newcommand{\mySubTitleFont}{\large\sf}
% \renewcommand{\indexspace}{\medskip}

% % {\parindent 8pt
% %   {\myTitleFont Indhold}}\par
% % \vskip 5pt
% \renewcommand{\SBThechapter}{Indhold}
% % {\parindent 20pt
% %   {\mySubTitleFont --- with first lines in italic ---}}
% % \vskip 20pt
% \let\olditem\item
% \let\oldsubitem\subitem
% \let\oldsubsubitem\subsubitem
% \renewcommand{\item}{\par\hangindent=40pt}
% \renewcommand{\subitem}{\par\hangindent=40pt \hspace*{20pt}}
% \renewcommand{\subsubitem}{\par\hangindent=40pt \hspace*{30pt}}

% %%%%%%% rcsid = @(#)$Id: sample-sb.tex,v 1.23 2010-04-12 18:04:11 rathc Exp $
%%%%%%
%%
%%      ===============================
%%      Sample Songbook (sample-sb.tex)
%%      ===============================
%%
%%      Version 4.5, 30 April, 2010
%%
%%      Copyright 1992--2010 Christopher Rath <christopher@rath.ca>
%%
%%      This package is free software; you can redistribute it and/or
%%      modify it under the terms of version 2.1 of the GNU Lesser
%%	General Public License as published by the Free Software 
%%	Foundation.
%%
%%      This package is distributed in the hope that it will be
%%      useful, but WITHOUT ANY WARRANTY; without even the implied
%%      warranty of MERCHANTABILITY or FITNESS FOR A PARTICULAR
%%      PURPOSE.  See the GNU Lesser General Public License for more
%%      details.
%%
%%      This file contains a subset of the songbook we distribute
%%      at our church.  To the best of my knowledge, all of the lyrics
%%      contained herein are freely distributable.  This file has been
%%      provided as a sample of what can be produced by the chordbk,
%%      wordbk, and overhead LaTeX styles.
%%
%%      NEEDED:  The fancyhdr LaTeX style is required to properly
%%              format this file.  If you don't have that then comment
%%              out the commands in the preamble which deal with the
%%              fancyhdr style.
%%
%%%%%%
%%%%%%
%%
%%      1. Chord notation.  Within this songbook the following
%%         conventions have been adopted:
%%
%%              "Minor" is entered as "m";
%%                      e.g. Cm7 for C minor 7th.
%%              "Major" is entered as "M";
%%                      e.g. CM7 for C major 7th.
%%
%%%%%%
%%%%%%
%%      ============
%%      Bibliography
%%      ============
%%
%%      Exalt Him!: Exalt Him!  Compiled by Tom Fettke.  (c)1989
%%                      Word Music.
%%
%%      Hosanna! Music Books: Hosanna! Music Books #1--#6.
%%                      (c)1987--92 Integrity Music, Inc.
%%
%%      Worship Him II: Worship Him II.  Compiled by Jesse Peterson
%%                      and Bruce Ballinger.  (c)1989 Tempo Music
%%                      Publications.
%%
%%      Worship Songs Of The Vineyard: Worship Songs Of The Vineyard
%%                      --- Volume 2.  (c)1989 Vineyard Ministries
%%                      International.
%%
%%%%%%
%%%%%%

%%%%%%%%%%%%%%%%%%%%%%%%%%%%%%%%%%%%%%%%%%%%%%%%%%%%%%%%%%
%%%%%%%%%%%%%%%%%%%%%%%%%%%%%%%%%%%%%%%%%%%%%%%%%%%%%%%%%%
%%                                                      %%
%%           P R E A M B L E   B E G I N S              %%
%%                                                      %%
%%%%%%%%%%%%%%%%%%%%%%%%%%%%%%%%%%%%%%%%%%%%%%%%%%%%%%%%%%
%%%%%%%%%%%%%%%%%%%%%%%%%%%%%%%%%%%%%%%%%%%%%%%%%%%%%%%%%%

\documentclass[a5paper]{book}
\usepackage{latexsym,
            fancyhdr,
            titlesec,
            amsmath,
            amssymb,
            multicol,
            amsthm,
            stmaryrd,
            amsthm,
            color,
            needspace,
            stackengine,
            wasysym}
\usepackage[utf8]{inputenc}
\usepackage[T1]{fontenc}
% \usepackage[chordbk]{songbook}                  %% Words & Chords edition.
%%\usepackage[compactallsongs,chordbk]{songbook}    %% Words & Chords edition.
\usepackage[wordbk]{songbook}                 %% Words Only edition.
%%\usepackage[overhead]{songbook}               %% Overhead Transparency edition.
\usepackage{titletoc}
\usepackage{tket}  % Draws "TÅGEKAMMERET" correctly

%%%
% Revision Date and Release Date definitions.
%
%       \RelDate - The last time this songbook was released.  Set this
%                  date each time a new release/update of the songbook
%                  is generated.
%       \RevDate - The last time a particular song was revised in any
%                  way.  This command will be renewed inside every
%                  song.
%%%
\newcommand{\RelDate}{31~August,~2003}
\newcommand{\RevDate}{\today}

%%%
% C.C.L.I. license number definition; for copyright licensing info.
% One of these macros will be manually inserted into the {SBMel}
% parameter of the {song} environment.
%
%       \CCLInumber - The actual copyright license number.  Don't
%               insert this command in the {SBMel} parameter, use one
%               of the others.
%       \CCLIed - Indicates a song falls under our CCLI license.
%       \NotCCLIed - Indicates a song doesn't fall under our CCLI
%               license.  Public Domain songs fall into this category.
%       \PGranted - We have received specific permission from the
%               copyright holder to use this song.
%       \PPending - We are in the process of obtaining permission to
%               use this song.
%%%
\newcommand{\CCLInumber}{Your CCLI Number}
\newcommand{\CCLIed}{{\SBMelInfoFont (CCLI \CCLInumber)}}
\newcommand{\NotCCLIed}{\relax}
\newcommand{\PGranted}{\relax}
\newcommand{\PPending}{{\SBMelInfoFont (Permission Pending)}}

%%%
% Title page information.
%%%
%\title{UNF Computer Science Camp 2019 Sangbog}
%\author{}
%\date{Revideret:  \RevDate}

%%%
% Redefine fonts from SongBook style that I don't like.
%%%
\font\myTinySF=cmss8 at 8pt
\renewcommand{\SBMelInfoFont}{\tiny\myTinySF}

%%%
% Define fonts to use in the headers and footers of the songbook.
%%%
\newcommand{\LHeadFont}{\normalsize}            % = cmr12  at 12pt
\newcommand{\CHeadFont}{\normalsize\rm}         % = cmr12  at 12pt
\newcommand{\RHeadFont}{\normalsize}            % = cmr12  at 12pt
\newcommand{\LFootFont}{\scriptsize}            % = cmr8   at  8pt
\newcommand{\CFootFont}{\tiny\myTinySF}         % = cmss8  at  8pt
\newcommand{\RFootFont}{\scriptsize}            % = cmr8   at  8pt

\def\repeat{%
  \stackanchor{.}{.}%
  \rule[-\dp\strutbox]{.3pt}{\normalbaselineskip}%
  \kern0.5pt%
  \rule[-\dp\strutbox]{1pt}{\normalbaselineskip}%
  \kern1pt%
}
\def\frepeat{%
  \kern1pt%
  \rule[-\dp\strutbox]{1pt}{\normalbaselineskip}%
  \kern0.5pt%
  \rule[-\dp\strutbox]{.3pt}{\normalbaselineskip}%
  \stackanchor{.}{.}%
}
% \newcommand{\SBRepeat}[1]{#1\\#1}
\newcommand{\SBRepeat}[1]{\frepeat #1\repeat}
\setcounter{SBSongCnt}{-1}
\renewcommand{\SBWAndMTag}{Forfatter:}
\renewcommand{\SBUnknownTag}{Ukendt}
\renewcommand{\SBChorusTag}{Ref.}
\renewcommand{\SBOrgMel}{Originalmelodi}
\renewcommand{\SpaceAfterChorus}   {\vspace{0ex plus1ex minus 0.5ex}}
\renewcommand{\SpaceAfterOpGroup}  {\vspace{0ex plus1ex minus 0.5ex}}
\renewcommand{\SpaceAfterSBBracket}{\vspace{0ex plus1ex minus 0.5ex}}
\renewcommand{\SpaceAfterSection}  {\vspace{0ex plus1ex minus 0.5ex}}
\renewcommand{\SpaceAfterSong}     {\vspace{0ex plus1ex minus 0.5ex}}
\renewcommand{\SpaceAfterVerse}    {\vspace{0ex plus1ex minus 0.5ex}}

% Tell LaTeX that \medskip is a good place to make a page break
\let\oldmedskip\medskip
\renewcommand{\medskip}{\oldmedskip\pagebreak[2]}

%%%
% Turn on/off index-file generation.  Uncomment the \makeindex line to
% turn index generation on;  comment it out to turn index generation
% off.
%%%
%\makeTitleIndex         %% Title and First Line Index.
%\makeTitleContents      %% Table of Contents.
%\makeKeyIndex           %% Index of song by key.
% \makeArtistIndex	%% Index of song by artist.
% \newcommand{\SBThechapter}[0]{}
% \newcommand{\SBChapter}[1]{
%     \startcontents
%     \chapter*{#1} 
%     % %%%%%% rcsid = @(#)$Id: sample-sb.tex,v 1.23 2010-04-12 18:04:11 rathc Exp $
%%%%%%
%%
%%      ===============================
%%      Sample Songbook (sample-sb.tex)
%%      ===============================
%%
%%      Version 4.5, 30 April, 2010
%%
%%      Copyright 1992--2010 Christopher Rath <christopher@rath.ca>
%%
%%      This package is free software; you can redistribute it and/or
%%      modify it under the terms of version 2.1 of the GNU Lesser
%%	General Public License as published by the Free Software 
%%	Foundation.
%%
%%      This package is distributed in the hope that it will be
%%      useful, but WITHOUT ANY WARRANTY; without even the implied
%%      warranty of MERCHANTABILITY or FITNESS FOR A PARTICULAR
%%      PURPOSE.  See the GNU Lesser General Public License for more
%%      details.
%%
%%      This file contains a subset of the songbook we distribute
%%      at our church.  To the best of my knowledge, all of the lyrics
%%      contained herein are freely distributable.  This file has been
%%      provided as a sample of what can be produced by the chordbk,
%%      wordbk, and overhead LaTeX styles.
%%
%%      NEEDED:  The fancyhdr LaTeX style is required to properly
%%              format this file.  If you don't have that then comment
%%              out the commands in the preamble which deal with the
%%              fancyhdr style.
%%
%%%%%%
%%%%%%
%%
%%      1. Chord notation.  Within this songbook the following
%%         conventions have been adopted:
%%
%%              "Minor" is entered as "m";
%%                      e.g. Cm7 for C minor 7th.
%%              "Major" is entered as "M";
%%                      e.g. CM7 for C major 7th.
%%
%%%%%%
%%%%%%
%%      ============
%%      Bibliography
%%      ============
%%
%%      Exalt Him!: Exalt Him!  Compiled by Tom Fettke.  (c)1989
%%                      Word Music.
%%
%%      Hosanna! Music Books: Hosanna! Music Books #1--#6.
%%                      (c)1987--92 Integrity Music, Inc.
%%
%%      Worship Him II: Worship Him II.  Compiled by Jesse Peterson
%%                      and Bruce Ballinger.  (c)1989 Tempo Music
%%                      Publications.
%%
%%      Worship Songs Of The Vineyard: Worship Songs Of The Vineyard
%%                      --- Volume 2.  (c)1989 Vineyard Ministries
%%                      International.
%%
%%%%%%
%%%%%%

%%%%%%%%%%%%%%%%%%%%%%%%%%%%%%%%%%%%%%%%%%%%%%%%%%%%%%%%%%
%%%%%%%%%%%%%%%%%%%%%%%%%%%%%%%%%%%%%%%%%%%%%%%%%%%%%%%%%%
%%                                                      %%
%%           P R E A M B L E   B E G I N S              %%
%%                                                      %%
%%%%%%%%%%%%%%%%%%%%%%%%%%%%%%%%%%%%%%%%%%%%%%%%%%%%%%%%%%
%%%%%%%%%%%%%%%%%%%%%%%%%%%%%%%%%%%%%%%%%%%%%%%%%%%%%%%%%%

\documentclass[a5paper]{book}
\usepackage{latexsym,
            fancyhdr,
            titlesec,
            amsmath,
            amssymb,
            multicol,
            amsthm,
            stmaryrd,
            amsthm,
            color,
            needspace,
            stackengine,
            wasysym}
\usepackage[utf8]{inputenc}
\usepackage[T1]{fontenc}
% \usepackage[chordbk]{songbook}                  %% Words & Chords edition.
%%\usepackage[compactallsongs,chordbk]{songbook}    %% Words & Chords edition.
\usepackage[wordbk]{songbook}                 %% Words Only edition.
%%\usepackage[overhead]{songbook}               %% Overhead Transparency edition.
\usepackage{titletoc}
\usepackage{tket}  % Draws "TÅGEKAMMERET" correctly

%%%
% Revision Date and Release Date definitions.
%
%       \RelDate - The last time this songbook was released.  Set this
%                  date each time a new release/update of the songbook
%                  is generated.
%       \RevDate - The last time a particular song was revised in any
%                  way.  This command will be renewed inside every
%                  song.
%%%
\newcommand{\RelDate}{31~August,~2003}
\newcommand{\RevDate}{\today}

%%%
% C.C.L.I. license number definition; for copyright licensing info.
% One of these macros will be manually inserted into the {SBMel}
% parameter of the {song} environment.
%
%       \CCLInumber - The actual copyright license number.  Don't
%               insert this command in the {SBMel} parameter, use one
%               of the others.
%       \CCLIed - Indicates a song falls under our CCLI license.
%       \NotCCLIed - Indicates a song doesn't fall under our CCLI
%               license.  Public Domain songs fall into this category.
%       \PGranted - We have received specific permission from the
%               copyright holder to use this song.
%       \PPending - We are in the process of obtaining permission to
%               use this song.
%%%
\newcommand{\CCLInumber}{Your CCLI Number}
\newcommand{\CCLIed}{{\SBMelInfoFont (CCLI \CCLInumber)}}
\newcommand{\NotCCLIed}{\relax}
\newcommand{\PGranted}{\relax}
\newcommand{\PPending}{{\SBMelInfoFont (Permission Pending)}}

%%%
% Title page information.
%%%
%\title{UNF Computer Science Camp 2019 Sangbog}
%\author{}
%\date{Revideret:  \RevDate}

%%%
% Redefine fonts from SongBook style that I don't like.
%%%
\font\myTinySF=cmss8 at 8pt
\renewcommand{\SBMelInfoFont}{\tiny\myTinySF}

%%%
% Define fonts to use in the headers and footers of the songbook.
%%%
\newcommand{\LHeadFont}{\normalsize}            % = cmr12  at 12pt
\newcommand{\CHeadFont}{\normalsize\rm}         % = cmr12  at 12pt
\newcommand{\RHeadFont}{\normalsize}            % = cmr12  at 12pt
\newcommand{\LFootFont}{\scriptsize}            % = cmr8   at  8pt
\newcommand{\CFootFont}{\tiny\myTinySF}         % = cmss8  at  8pt
\newcommand{\RFootFont}{\scriptsize}            % = cmr8   at  8pt

\def\repeat{%
  \stackanchor{.}{.}%
  \rule[-\dp\strutbox]{.3pt}{\normalbaselineskip}%
  \kern0.5pt%
  \rule[-\dp\strutbox]{1pt}{\normalbaselineskip}%
  \kern1pt%
}
\def\frepeat{%
  \kern1pt%
  \rule[-\dp\strutbox]{1pt}{\normalbaselineskip}%
  \kern0.5pt%
  \rule[-\dp\strutbox]{.3pt}{\normalbaselineskip}%
  \stackanchor{.}{.}%
}
% \newcommand{\SBRepeat}[1]{#1\\#1}
\newcommand{\SBRepeat}[1]{\frepeat #1\repeat}
\setcounter{SBSongCnt}{-1}
\renewcommand{\SBWAndMTag}{Forfatter:}
\renewcommand{\SBUnknownTag}{Ukendt}
\renewcommand{\SBChorusTag}{Ref.}
\renewcommand{\SBOrgMel}{Originalmelodi}
\renewcommand{\SpaceAfterChorus}   {\vspace{0ex plus1ex minus 0.5ex}}
\renewcommand{\SpaceAfterOpGroup}  {\vspace{0ex plus1ex minus 0.5ex}}
\renewcommand{\SpaceAfterSBBracket}{\vspace{0ex plus1ex minus 0.5ex}}
\renewcommand{\SpaceAfterSection}  {\vspace{0ex plus1ex minus 0.5ex}}
\renewcommand{\SpaceAfterSong}     {\vspace{0ex plus1ex minus 0.5ex}}
\renewcommand{\SpaceAfterVerse}    {\vspace{0ex plus1ex minus 0.5ex}}

% Tell LaTeX that \medskip is a good place to make a page break
\let\oldmedskip\medskip
\renewcommand{\medskip}{\oldmedskip\pagebreak[2]}

%%%
% Turn on/off index-file generation.  Uncomment the \makeindex line to
% turn index generation on;  comment it out to turn index generation
% off.
%%%
%\makeTitleIndex         %% Title and First Line Index.
%\makeTitleContents      %% Table of Contents.
%\makeKeyIndex           %% Index of song by key.
% \makeArtistIndex	%% Index of song by artist.
% \newcommand{\SBThechapter}[0]{}
% \newcommand{\SBChapter}[1]{
%     \startcontents
%     \chapter*{#1} 
%     % %%%%%% rcsid = @(#)$Id: sample-sb.tex,v 1.23 2010-04-12 18:04:11 rathc Exp $
%%%%%%
%%
%%      ===============================
%%      Sample Songbook (sample-sb.tex)
%%      ===============================
%%
%%      Version 4.5, 30 April, 2010
%%
%%      Copyright 1992--2010 Christopher Rath <christopher@rath.ca>
%%
%%      This package is free software; you can redistribute it and/or
%%      modify it under the terms of version 2.1 of the GNU Lesser
%%	General Public License as published by the Free Software 
%%	Foundation.
%%
%%      This package is distributed in the hope that it will be
%%      useful, but WITHOUT ANY WARRANTY; without even the implied
%%      warranty of MERCHANTABILITY or FITNESS FOR A PARTICULAR
%%      PURPOSE.  See the GNU Lesser General Public License for more
%%      details.
%%
%%      This file contains a subset of the songbook we distribute
%%      at our church.  To the best of my knowledge, all of the lyrics
%%      contained herein are freely distributable.  This file has been
%%      provided as a sample of what can be produced by the chordbk,
%%      wordbk, and overhead LaTeX styles.
%%
%%      NEEDED:  The fancyhdr LaTeX style is required to properly
%%              format this file.  If you don't have that then comment
%%              out the commands in the preamble which deal with the
%%              fancyhdr style.
%%
%%%%%%
%%%%%%
%%
%%      1. Chord notation.  Within this songbook the following
%%         conventions have been adopted:
%%
%%              "Minor" is entered as "m";
%%                      e.g. Cm7 for C minor 7th.
%%              "Major" is entered as "M";
%%                      e.g. CM7 for C major 7th.
%%
%%%%%%
%%%%%%
%%      ============
%%      Bibliography
%%      ============
%%
%%      Exalt Him!: Exalt Him!  Compiled by Tom Fettke.  (c)1989
%%                      Word Music.
%%
%%      Hosanna! Music Books: Hosanna! Music Books #1--#6.
%%                      (c)1987--92 Integrity Music, Inc.
%%
%%      Worship Him II: Worship Him II.  Compiled by Jesse Peterson
%%                      and Bruce Ballinger.  (c)1989 Tempo Music
%%                      Publications.
%%
%%      Worship Songs Of The Vineyard: Worship Songs Of The Vineyard
%%                      --- Volume 2.  (c)1989 Vineyard Ministries
%%                      International.
%%
%%%%%%
%%%%%%

%%%%%%%%%%%%%%%%%%%%%%%%%%%%%%%%%%%%%%%%%%%%%%%%%%%%%%%%%%
%%%%%%%%%%%%%%%%%%%%%%%%%%%%%%%%%%%%%%%%%%%%%%%%%%%%%%%%%%
%%                                                      %%
%%           P R E A M B L E   B E G I N S              %%
%%                                                      %%
%%%%%%%%%%%%%%%%%%%%%%%%%%%%%%%%%%%%%%%%%%%%%%%%%%%%%%%%%%
%%%%%%%%%%%%%%%%%%%%%%%%%%%%%%%%%%%%%%%%%%%%%%%%%%%%%%%%%%

\documentclass[a5paper]{book}
\usepackage{latexsym,
            fancyhdr,
            titlesec,
            amsmath,
            amssymb,
            multicol,
            amsthm,
            stmaryrd,
            amsthm,
            color,
            needspace,
            stackengine,
            wasysym}
\usepackage[utf8]{inputenc}
\usepackage[T1]{fontenc}
% \usepackage[chordbk]{songbook}                  %% Words & Chords edition.
%%\usepackage[compactallsongs,chordbk]{songbook}    %% Words & Chords edition.
\usepackage[wordbk]{songbook}                 %% Words Only edition.
%%\usepackage[overhead]{songbook}               %% Overhead Transparency edition.
\usepackage{titletoc}
\usepackage{tket}  % Draws "TÅGEKAMMERET" correctly

%%%
% Revision Date and Release Date definitions.
%
%       \RelDate - The last time this songbook was released.  Set this
%                  date each time a new release/update of the songbook
%                  is generated.
%       \RevDate - The last time a particular song was revised in any
%                  way.  This command will be renewed inside every
%                  song.
%%%
\newcommand{\RelDate}{31~August,~2003}
\newcommand{\RevDate}{\today}

%%%
% C.C.L.I. license number definition; for copyright licensing info.
% One of these macros will be manually inserted into the {SBMel}
% parameter of the {song} environment.
%
%       \CCLInumber - The actual copyright license number.  Don't
%               insert this command in the {SBMel} parameter, use one
%               of the others.
%       \CCLIed - Indicates a song falls under our CCLI license.
%       \NotCCLIed - Indicates a song doesn't fall under our CCLI
%               license.  Public Domain songs fall into this category.
%       \PGranted - We have received specific permission from the
%               copyright holder to use this song.
%       \PPending - We are in the process of obtaining permission to
%               use this song.
%%%
\newcommand{\CCLInumber}{Your CCLI Number}
\newcommand{\CCLIed}{{\SBMelInfoFont (CCLI \CCLInumber)}}
\newcommand{\NotCCLIed}{\relax}
\newcommand{\PGranted}{\relax}
\newcommand{\PPending}{{\SBMelInfoFont (Permission Pending)}}

%%%
% Title page information.
%%%
%\title{UNF Computer Science Camp 2019 Sangbog}
%\author{}
%\date{Revideret:  \RevDate}

%%%
% Redefine fonts from SongBook style that I don't like.
%%%
\font\myTinySF=cmss8 at 8pt
\renewcommand{\SBMelInfoFont}{\tiny\myTinySF}

%%%
% Define fonts to use in the headers and footers of the songbook.
%%%
\newcommand{\LHeadFont}{\normalsize}            % = cmr12  at 12pt
\newcommand{\CHeadFont}{\normalsize\rm}         % = cmr12  at 12pt
\newcommand{\RHeadFont}{\normalsize}            % = cmr12  at 12pt
\newcommand{\LFootFont}{\scriptsize}            % = cmr8   at  8pt
\newcommand{\CFootFont}{\tiny\myTinySF}         % = cmss8  at  8pt
\newcommand{\RFootFont}{\scriptsize}            % = cmr8   at  8pt

\def\repeat{%
  \stackanchor{.}{.}%
  \rule[-\dp\strutbox]{.3pt}{\normalbaselineskip}%
  \kern0.5pt%
  \rule[-\dp\strutbox]{1pt}{\normalbaselineskip}%
  \kern1pt%
}
\def\frepeat{%
  \kern1pt%
  \rule[-\dp\strutbox]{1pt}{\normalbaselineskip}%
  \kern0.5pt%
  \rule[-\dp\strutbox]{.3pt}{\normalbaselineskip}%
  \stackanchor{.}{.}%
}
% \newcommand{\SBRepeat}[1]{#1\\#1}
\newcommand{\SBRepeat}[1]{\frepeat #1\repeat}
\setcounter{SBSongCnt}{-1}
\renewcommand{\SBWAndMTag}{Forfatter:}
\renewcommand{\SBUnknownTag}{Ukendt}
\renewcommand{\SBChorusTag}{Ref.}
\renewcommand{\SBOrgMel}{Originalmelodi}
\renewcommand{\SpaceAfterChorus}   {\vspace{0ex plus1ex minus 0.5ex}}
\renewcommand{\SpaceAfterOpGroup}  {\vspace{0ex plus1ex minus 0.5ex}}
\renewcommand{\SpaceAfterSBBracket}{\vspace{0ex plus1ex minus 0.5ex}}
\renewcommand{\SpaceAfterSection}  {\vspace{0ex plus1ex minus 0.5ex}}
\renewcommand{\SpaceAfterSong}     {\vspace{0ex plus1ex minus 0.5ex}}
\renewcommand{\SpaceAfterVerse}    {\vspace{0ex plus1ex minus 0.5ex}}

% Tell LaTeX that \medskip is a good place to make a page break
\let\oldmedskip\medskip
\renewcommand{\medskip}{\oldmedskip\pagebreak[2]}

%%%
% Turn on/off index-file generation.  Uncomment the \makeindex line to
% turn index generation on;  comment it out to turn index generation
% off.
%%%
%\makeTitleIndex         %% Title and First Line Index.
%\makeTitleContents      %% Table of Contents.
%\makeKeyIndex           %% Index of song by key.
% \makeArtistIndex	%% Index of song by artist.
% \newcommand{\SBThechapter}[0]{}
% \newcommand{\SBChapter}[1]{
%     \startcontents
%     \chapter*{#1} 
%     % \input{unf-sangbog.toc}
%       \begin{minipage}{.8\textwidth}
%         \printcontents{}{1}{}
%       \end{minipage}%
%     \renewcommand{\SBThechapter}{#1}
%     \clearpage
% }

% \titleformat{\chapter}
% [display]
% {}
% {%\vspace*{\fill}
%  % \titlerule[1pt]%
%  % \vspace{1pt}%
%  % \titlerule
%  % \vspace{1pc}%
%  \chaptertitlename}
% {}
% {\Huge}



%%%%%%%%%%%%%%%%%%%%%%%%%%%%%%%%%%%%%%%%%%%%%%%%%%%%%%%%%%
%%%%%%%%%%%%%%%%%%%%%%%%%%%%%%%%%%%%%%%%%%%%%%%%%%%%%%%%%%
%%                                                      %%
%%           D O C U M E N T   B E G I N S              %%
%%                                                      %%
%%%%%%%%%%%%%%%%%%%%%%%%%%%%%%%%%%%%%%%%%%%%%%%%%%%%%%%%%%
%%%%%%%%%%%%%%%%%%%%%%%%%%%%%%%%%%%%%%%%%%%%%%%%%%%%%%%%%%
\begin{document}

%%%
% Uncomment "\maketitle" statement to make a title page.
%%%
%\maketitle
% \begin{titlepage}
%   \centering
%   \vspace{5cm}
% 	\includegraphics[width=1\textwidth]{unf_logo.jpeg}\par\vspace{1cm}
% 	{\scshape\LARGE Sangbog \par}
% 	\vspace{1cm}
% 	{\scshape\Large UNF Computer Science Camp 2019\par}
	
% 	\vfill

% % Bottom of the page
% 	{\large \today\par}
% \end{titlepage}
% \mainmatter
% \ifWordBk
%   \twocolumn
% \fi


%%% Kolofon
%\thispagestyle{empty}
%Sammensat til UNF Computer Science Camp 2019 - csc.unf.dk\\
%Redaktør: Andreas Mosbæk Jensen m.fl. efter tidligere sangbog af Steffen Strunge Mathiesen\\
%Indhold opsat i \LaTeX. 
%Digital version og kildekode: github.com/steffen555/UNF-sangbog\\
%Revision 1 med stave fejl korrektioner
%\par\vspace*{\fill}
%Hvis du har forslag til sange, rettelser, ris og ros, eller hvis du kender en ukendt forfatter, så skriv til sangbog@unf.dk.

%%%
% Turn on and define fancy page heading/footing definition.
%%%
% \pagestyle{fancy}

% \ifChordBk
%   % It's a words & chords songbook...
%   \addtolength{\headwidth}{\marginparsep}
%   \addtolength{\headwidth}{\marginparwidth}
%   \renewcommand{\headrulewidth}{0.4pt}
%   \renewcommand{\footrulewidth}{0.4pt}
%   \fancyhead[LE,RO]{\LHeadFont\emph{\leftmark\/}\SBContinueMark}
%   \fancyhead[CE,CO]{\CHeadFont\thepage}
%   \fancyhead[RE,LO]{\RHeadFont \chaptermark}
% \else\ifOverhead
%   % It's an overhead...
%   \renewcommand{\footrulewidth}{0pt}
%   \renewcommand{\headrulewidth}{0pt}
%   \fancyhead[LE,RO]{}
%   \fancyhead[CE,CO]{}
%   \fancyhead[RE,LO]{}
% \else\ifWordBk
%   % It's a words only songbook...
%   \addtolength{\headwidth}{\marginparsep}
%   \addtolength{\headwidth}{\marginparwidth}
%   \renewcommand{\headrulewidth}{0.4pt}
%   \renewcommand{\footrulewidth}{0.4pt}
%   \fancyhead[LE,RO]{\LHeadFont Naturvidenskab revy sange}
%   \fancyhead[CE,CO]{\CHeadFont\thepage}
%   \fancyhead[RE,LO]{\RHeadFont \SBThechapter}
% \fi\fi\fi

% \fancyfoot[LE,RO]{\LFootFont Computer Science Camp 2019}
% \ifSongEject
%   \fancyfoot[CE,CO]{\CFootFont Last Revised:  \RevDate}
% \else
%   \fancyfoot[CE,CO]{\CFootFont}
% \fi
% \fancyfoot[RE,LO]{\RFootFont Synges på eget ansvar}

%%%
% Table of contents
%%%

% \clearpage
% \twocolumn
% \font\myTinySF=cmss8    at  8pt
% \font\myHugeSF=cmssbx10 at 25pt
% \newcommand{\CpyRtInfoFont}{\tiny\myTinySF}
% \newcommand{\myTitleFont}{\Huge\myHugeSF}
% \newcommand{\mySubTitleFont}{\large\sf}
% \renewcommand{\indexspace}{\medskip}

% % {\parindent 8pt
% %   {\myTitleFont Indhold}}\par
% % \vskip 5pt
% \renewcommand{\SBThechapter}{Indhold}
% % {\parindent 20pt
% %   {\mySubTitleFont --- with first lines in italic ---}}
% % \vskip 20pt
% \let\olditem\item
% \let\oldsubitem\subitem
% \let\oldsubsubitem\subsubitem
% \renewcommand{\item}{\par\hangindent=40pt}
% \renewcommand{\subitem}{\par\hangindent=40pt \hspace*{20pt}}
% \renewcommand{\subsubitem}{\par\hangindent=40pt \hspace*{30pt}}

% %\input{unf-sangbog.tocx}

% \renewcommand{\item}{\olditem}
% \renewcommand{\subitem}{\oldsubitem}
% \renewcommand{\subsubitem}{\oldsubsubitem}

%%%
% Songbook begins.
%%%

\twocolumn
%It's just one page, don't print page numbers etc.
\pagestyle{empty}
%Songs included
\input{songs/matmatik.tex}
\input{songs/taal_daj.tex}
\input{songs/linieskriverdriver.tex}
\input{songs/steve_hawking.tex}
\input{songs/ode_til_kode.tex}
\input{songs/se_min_kode.tex}
\input{songs/vaabenfysik_kort.tex}
%Maybe include:
%\input{songs/kvanter_i_maaneskin.tex}
%\input{songs/mest_matematiske_dyr.tex}

% \input{songs/vi_kan_ikke_li.tex}
% \input{songs/selektionssangen.tex}
% \input{songs/alfabetsangen.tex}
% \input{songs/sciencecamps.tex}
% \input{songs/hvad_maa_man.tex}


% \input{songs/lambda_kalkylen.tex}
% \input{songs/puslespil.tex}
% \input{songs/null.tex}
% \input{songs/fasebal.tex}

% \input{songs/chifitter.tex}

% \input{songs/kun_fysik.tex}



% \input{songs/kanoniske.tex}
% \input{songs/jeg_er_en_matematiker_fra_hcoe.tex}


% \input{songs/rekursiv_skovsang.tex}
% \input{songs/laerkerede.tex}


% \clearpage
% \font\myTinySF=cmss8    at  8pt
% \font\myHugeSF=cmssbx10 at 25pt
% % \newcommand{\CpyRtInfoFont}{\tiny\myTinySF}
% % \newcommand{\myTitleFont}{\Huge\myHugeSF}
% % \newcommand{\mySubTitleFont}{\large\sf}
% \renewcommand{\indexspace}{\medskip}

% {\parindent 8pt
%   {\myTitleFont Index}}\par
% \vskip 5pt
% \renewcommand{\SBThechapter}{Index}
% % {\parindent 20pt
% %   {\mySubTitleFont --- with first lines in italic ---}}
% % \vskip 20pt
% \renewcommand{\item}{\par\hangindent=40pt}
% \renewcommand{\subitem}{\par\hangindent=40pt \hspace*{20pt}}
% \renewcommand{\subsubitem}{\par\hangindent=40pt \hspace*{30pt}}

%\input{unf-sangbog.tdx}

\end{document}
\bye
%
%%%
% Document ends.
%%%

%       \begin{minipage}{.8\textwidth}
%         \printcontents{}{1}{}
%       \end{minipage}%
%     \renewcommand{\SBThechapter}{#1}
%     \clearpage
% }

% \titleformat{\chapter}
% [display]
% {}
% {%\vspace*{\fill}
%  % \titlerule[1pt]%
%  % \vspace{1pt}%
%  % \titlerule
%  % \vspace{1pc}%
%  \chaptertitlename}
% {}
% {\Huge}



%%%%%%%%%%%%%%%%%%%%%%%%%%%%%%%%%%%%%%%%%%%%%%%%%%%%%%%%%%
%%%%%%%%%%%%%%%%%%%%%%%%%%%%%%%%%%%%%%%%%%%%%%%%%%%%%%%%%%
%%                                                      %%
%%           D O C U M E N T   B E G I N S              %%
%%                                                      %%
%%%%%%%%%%%%%%%%%%%%%%%%%%%%%%%%%%%%%%%%%%%%%%%%%%%%%%%%%%
%%%%%%%%%%%%%%%%%%%%%%%%%%%%%%%%%%%%%%%%%%%%%%%%%%%%%%%%%%
\begin{document}

%%%
% Uncomment "\maketitle" statement to make a title page.
%%%
%\maketitle
% \begin{titlepage}
%   \centering
%   \vspace{5cm}
% 	\includegraphics[width=1\textwidth]{unf_logo.jpeg}\par\vspace{1cm}
% 	{\scshape\LARGE Sangbog \par}
% 	\vspace{1cm}
% 	{\scshape\Large UNF Computer Science Camp 2019\par}
	
% 	\vfill

% % Bottom of the page
% 	{\large \today\par}
% \end{titlepage}
% \mainmatter
% \ifWordBk
%   \twocolumn
% \fi


%%% Kolofon
%\thispagestyle{empty}
%Sammensat til UNF Computer Science Camp 2019 - csc.unf.dk\\
%Redaktør: Andreas Mosbæk Jensen m.fl. efter tidligere sangbog af Steffen Strunge Mathiesen\\
%Indhold opsat i \LaTeX. 
%Digital version og kildekode: github.com/steffen555/UNF-sangbog\\
%Revision 1 med stave fejl korrektioner
%\par\vspace*{\fill}
%Hvis du har forslag til sange, rettelser, ris og ros, eller hvis du kender en ukendt forfatter, så skriv til sangbog@unf.dk.

%%%
% Turn on and define fancy page heading/footing definition.
%%%
% \pagestyle{fancy}

% \ifChordBk
%   % It's a words & chords songbook...
%   \addtolength{\headwidth}{\marginparsep}
%   \addtolength{\headwidth}{\marginparwidth}
%   \renewcommand{\headrulewidth}{0.4pt}
%   \renewcommand{\footrulewidth}{0.4pt}
%   \fancyhead[LE,RO]{\LHeadFont\emph{\leftmark\/}\SBContinueMark}
%   \fancyhead[CE,CO]{\CHeadFont\thepage}
%   \fancyhead[RE,LO]{\RHeadFont \chaptermark}
% \else\ifOverhead
%   % It's an overhead...
%   \renewcommand{\footrulewidth}{0pt}
%   \renewcommand{\headrulewidth}{0pt}
%   \fancyhead[LE,RO]{}
%   \fancyhead[CE,CO]{}
%   \fancyhead[RE,LO]{}
% \else\ifWordBk
%   % It's a words only songbook...
%   \addtolength{\headwidth}{\marginparsep}
%   \addtolength{\headwidth}{\marginparwidth}
%   \renewcommand{\headrulewidth}{0.4pt}
%   \renewcommand{\footrulewidth}{0.4pt}
%   \fancyhead[LE,RO]{\LHeadFont Naturvidenskab revy sange}
%   \fancyhead[CE,CO]{\CHeadFont\thepage}
%   \fancyhead[RE,LO]{\RHeadFont \SBThechapter}
% \fi\fi\fi

% \fancyfoot[LE,RO]{\LFootFont Computer Science Camp 2019}
% \ifSongEject
%   \fancyfoot[CE,CO]{\CFootFont Last Revised:  \RevDate}
% \else
%   \fancyfoot[CE,CO]{\CFootFont}
% \fi
% \fancyfoot[RE,LO]{\RFootFont Synges på eget ansvar}

%%%
% Table of contents
%%%

% \clearpage
% \twocolumn
% \font\myTinySF=cmss8    at  8pt
% \font\myHugeSF=cmssbx10 at 25pt
% \newcommand{\CpyRtInfoFont}{\tiny\myTinySF}
% \newcommand{\myTitleFont}{\Huge\myHugeSF}
% \newcommand{\mySubTitleFont}{\large\sf}
% \renewcommand{\indexspace}{\medskip}

% % {\parindent 8pt
% %   {\myTitleFont Indhold}}\par
% % \vskip 5pt
% \renewcommand{\SBThechapter}{Indhold}
% % {\parindent 20pt
% %   {\mySubTitleFont --- with first lines in italic ---}}
% % \vskip 20pt
% \let\olditem\item
% \let\oldsubitem\subitem
% \let\oldsubsubitem\subsubitem
% \renewcommand{\item}{\par\hangindent=40pt}
% \renewcommand{\subitem}{\par\hangindent=40pt \hspace*{20pt}}
% \renewcommand{\subsubitem}{\par\hangindent=40pt \hspace*{30pt}}

% %%%%%%% rcsid = @(#)$Id: sample-sb.tex,v 1.23 2010-04-12 18:04:11 rathc Exp $
%%%%%%
%%
%%      ===============================
%%      Sample Songbook (sample-sb.tex)
%%      ===============================
%%
%%      Version 4.5, 30 April, 2010
%%
%%      Copyright 1992--2010 Christopher Rath <christopher@rath.ca>
%%
%%      This package is free software; you can redistribute it and/or
%%      modify it under the terms of version 2.1 of the GNU Lesser
%%	General Public License as published by the Free Software 
%%	Foundation.
%%
%%      This package is distributed in the hope that it will be
%%      useful, but WITHOUT ANY WARRANTY; without even the implied
%%      warranty of MERCHANTABILITY or FITNESS FOR A PARTICULAR
%%      PURPOSE.  See the GNU Lesser General Public License for more
%%      details.
%%
%%      This file contains a subset of the songbook we distribute
%%      at our church.  To the best of my knowledge, all of the lyrics
%%      contained herein are freely distributable.  This file has been
%%      provided as a sample of what can be produced by the chordbk,
%%      wordbk, and overhead LaTeX styles.
%%
%%      NEEDED:  The fancyhdr LaTeX style is required to properly
%%              format this file.  If you don't have that then comment
%%              out the commands in the preamble which deal with the
%%              fancyhdr style.
%%
%%%%%%
%%%%%%
%%
%%      1. Chord notation.  Within this songbook the following
%%         conventions have been adopted:
%%
%%              "Minor" is entered as "m";
%%                      e.g. Cm7 for C minor 7th.
%%              "Major" is entered as "M";
%%                      e.g. CM7 for C major 7th.
%%
%%%%%%
%%%%%%
%%      ============
%%      Bibliography
%%      ============
%%
%%      Exalt Him!: Exalt Him!  Compiled by Tom Fettke.  (c)1989
%%                      Word Music.
%%
%%      Hosanna! Music Books: Hosanna! Music Books #1--#6.
%%                      (c)1987--92 Integrity Music, Inc.
%%
%%      Worship Him II: Worship Him II.  Compiled by Jesse Peterson
%%                      and Bruce Ballinger.  (c)1989 Tempo Music
%%                      Publications.
%%
%%      Worship Songs Of The Vineyard: Worship Songs Of The Vineyard
%%                      --- Volume 2.  (c)1989 Vineyard Ministries
%%                      International.
%%
%%%%%%
%%%%%%

%%%%%%%%%%%%%%%%%%%%%%%%%%%%%%%%%%%%%%%%%%%%%%%%%%%%%%%%%%
%%%%%%%%%%%%%%%%%%%%%%%%%%%%%%%%%%%%%%%%%%%%%%%%%%%%%%%%%%
%%                                                      %%
%%           P R E A M B L E   B E G I N S              %%
%%                                                      %%
%%%%%%%%%%%%%%%%%%%%%%%%%%%%%%%%%%%%%%%%%%%%%%%%%%%%%%%%%%
%%%%%%%%%%%%%%%%%%%%%%%%%%%%%%%%%%%%%%%%%%%%%%%%%%%%%%%%%%

\documentclass[a5paper]{book}
\usepackage{latexsym,
            fancyhdr,
            titlesec,
            amsmath,
            amssymb,
            multicol,
            amsthm,
            stmaryrd,
            amsthm,
            color,
            needspace,
            stackengine,
            wasysym}
\usepackage[utf8]{inputenc}
\usepackage[T1]{fontenc}
% \usepackage[chordbk]{songbook}                  %% Words & Chords edition.
%%\usepackage[compactallsongs,chordbk]{songbook}    %% Words & Chords edition.
\usepackage[wordbk]{songbook}                 %% Words Only edition.
%%\usepackage[overhead]{songbook}               %% Overhead Transparency edition.
\usepackage{titletoc}
\usepackage{tket}  % Draws "TÅGEKAMMERET" correctly

%%%
% Revision Date and Release Date definitions.
%
%       \RelDate - The last time this songbook was released.  Set this
%                  date each time a new release/update of the songbook
%                  is generated.
%       \RevDate - The last time a particular song was revised in any
%                  way.  This command will be renewed inside every
%                  song.
%%%
\newcommand{\RelDate}{31~August,~2003}
\newcommand{\RevDate}{\today}

%%%
% C.C.L.I. license number definition; for copyright licensing info.
% One of these macros will be manually inserted into the {SBMel}
% parameter of the {song} environment.
%
%       \CCLInumber - The actual copyright license number.  Don't
%               insert this command in the {SBMel} parameter, use one
%               of the others.
%       \CCLIed - Indicates a song falls under our CCLI license.
%       \NotCCLIed - Indicates a song doesn't fall under our CCLI
%               license.  Public Domain songs fall into this category.
%       \PGranted - We have received specific permission from the
%               copyright holder to use this song.
%       \PPending - We are in the process of obtaining permission to
%               use this song.
%%%
\newcommand{\CCLInumber}{Your CCLI Number}
\newcommand{\CCLIed}{{\SBMelInfoFont (CCLI \CCLInumber)}}
\newcommand{\NotCCLIed}{\relax}
\newcommand{\PGranted}{\relax}
\newcommand{\PPending}{{\SBMelInfoFont (Permission Pending)}}

%%%
% Title page information.
%%%
%\title{UNF Computer Science Camp 2019 Sangbog}
%\author{}
%\date{Revideret:  \RevDate}

%%%
% Redefine fonts from SongBook style that I don't like.
%%%
\font\myTinySF=cmss8 at 8pt
\renewcommand{\SBMelInfoFont}{\tiny\myTinySF}

%%%
% Define fonts to use in the headers and footers of the songbook.
%%%
\newcommand{\LHeadFont}{\normalsize}            % = cmr12  at 12pt
\newcommand{\CHeadFont}{\normalsize\rm}         % = cmr12  at 12pt
\newcommand{\RHeadFont}{\normalsize}            % = cmr12  at 12pt
\newcommand{\LFootFont}{\scriptsize}            % = cmr8   at  8pt
\newcommand{\CFootFont}{\tiny\myTinySF}         % = cmss8  at  8pt
\newcommand{\RFootFont}{\scriptsize}            % = cmr8   at  8pt

\def\repeat{%
  \stackanchor{.}{.}%
  \rule[-\dp\strutbox]{.3pt}{\normalbaselineskip}%
  \kern0.5pt%
  \rule[-\dp\strutbox]{1pt}{\normalbaselineskip}%
  \kern1pt%
}
\def\frepeat{%
  \kern1pt%
  \rule[-\dp\strutbox]{1pt}{\normalbaselineskip}%
  \kern0.5pt%
  \rule[-\dp\strutbox]{.3pt}{\normalbaselineskip}%
  \stackanchor{.}{.}%
}
% \newcommand{\SBRepeat}[1]{#1\\#1}
\newcommand{\SBRepeat}[1]{\frepeat #1\repeat}
\setcounter{SBSongCnt}{-1}
\renewcommand{\SBWAndMTag}{Forfatter:}
\renewcommand{\SBUnknownTag}{Ukendt}
\renewcommand{\SBChorusTag}{Ref.}
\renewcommand{\SBOrgMel}{Originalmelodi}
\renewcommand{\SpaceAfterChorus}   {\vspace{0ex plus1ex minus 0.5ex}}
\renewcommand{\SpaceAfterOpGroup}  {\vspace{0ex plus1ex minus 0.5ex}}
\renewcommand{\SpaceAfterSBBracket}{\vspace{0ex plus1ex minus 0.5ex}}
\renewcommand{\SpaceAfterSection}  {\vspace{0ex plus1ex minus 0.5ex}}
\renewcommand{\SpaceAfterSong}     {\vspace{0ex plus1ex minus 0.5ex}}
\renewcommand{\SpaceAfterVerse}    {\vspace{0ex plus1ex minus 0.5ex}}

% Tell LaTeX that \medskip is a good place to make a page break
\let\oldmedskip\medskip
\renewcommand{\medskip}{\oldmedskip\pagebreak[2]}

%%%
% Turn on/off index-file generation.  Uncomment the \makeindex line to
% turn index generation on;  comment it out to turn index generation
% off.
%%%
%\makeTitleIndex         %% Title and First Line Index.
%\makeTitleContents      %% Table of Contents.
%\makeKeyIndex           %% Index of song by key.
% \makeArtistIndex	%% Index of song by artist.
% \newcommand{\SBThechapter}[0]{}
% \newcommand{\SBChapter}[1]{
%     \startcontents
%     \chapter*{#1} 
%     % \input{unf-sangbog.toc}
%       \begin{minipage}{.8\textwidth}
%         \printcontents{}{1}{}
%       \end{minipage}%
%     \renewcommand{\SBThechapter}{#1}
%     \clearpage
% }

% \titleformat{\chapter}
% [display]
% {}
% {%\vspace*{\fill}
%  % \titlerule[1pt]%
%  % \vspace{1pt}%
%  % \titlerule
%  % \vspace{1pc}%
%  \chaptertitlename}
% {}
% {\Huge}



%%%%%%%%%%%%%%%%%%%%%%%%%%%%%%%%%%%%%%%%%%%%%%%%%%%%%%%%%%
%%%%%%%%%%%%%%%%%%%%%%%%%%%%%%%%%%%%%%%%%%%%%%%%%%%%%%%%%%
%%                                                      %%
%%           D O C U M E N T   B E G I N S              %%
%%                                                      %%
%%%%%%%%%%%%%%%%%%%%%%%%%%%%%%%%%%%%%%%%%%%%%%%%%%%%%%%%%%
%%%%%%%%%%%%%%%%%%%%%%%%%%%%%%%%%%%%%%%%%%%%%%%%%%%%%%%%%%
\begin{document}

%%%
% Uncomment "\maketitle" statement to make a title page.
%%%
%\maketitle
% \begin{titlepage}
%   \centering
%   \vspace{5cm}
% 	\includegraphics[width=1\textwidth]{unf_logo.jpeg}\par\vspace{1cm}
% 	{\scshape\LARGE Sangbog \par}
% 	\vspace{1cm}
% 	{\scshape\Large UNF Computer Science Camp 2019\par}
	
% 	\vfill

% % Bottom of the page
% 	{\large \today\par}
% \end{titlepage}
% \mainmatter
% \ifWordBk
%   \twocolumn
% \fi


%%% Kolofon
%\thispagestyle{empty}
%Sammensat til UNF Computer Science Camp 2019 - csc.unf.dk\\
%Redaktør: Andreas Mosbæk Jensen m.fl. efter tidligere sangbog af Steffen Strunge Mathiesen\\
%Indhold opsat i \LaTeX. 
%Digital version og kildekode: github.com/steffen555/UNF-sangbog\\
%Revision 1 med stave fejl korrektioner
%\par\vspace*{\fill}
%Hvis du har forslag til sange, rettelser, ris og ros, eller hvis du kender en ukendt forfatter, så skriv til sangbog@unf.dk.

%%%
% Turn on and define fancy page heading/footing definition.
%%%
% \pagestyle{fancy}

% \ifChordBk
%   % It's a words & chords songbook...
%   \addtolength{\headwidth}{\marginparsep}
%   \addtolength{\headwidth}{\marginparwidth}
%   \renewcommand{\headrulewidth}{0.4pt}
%   \renewcommand{\footrulewidth}{0.4pt}
%   \fancyhead[LE,RO]{\LHeadFont\emph{\leftmark\/}\SBContinueMark}
%   \fancyhead[CE,CO]{\CHeadFont\thepage}
%   \fancyhead[RE,LO]{\RHeadFont \chaptermark}
% \else\ifOverhead
%   % It's an overhead...
%   \renewcommand{\footrulewidth}{0pt}
%   \renewcommand{\headrulewidth}{0pt}
%   \fancyhead[LE,RO]{}
%   \fancyhead[CE,CO]{}
%   \fancyhead[RE,LO]{}
% \else\ifWordBk
%   % It's a words only songbook...
%   \addtolength{\headwidth}{\marginparsep}
%   \addtolength{\headwidth}{\marginparwidth}
%   \renewcommand{\headrulewidth}{0.4pt}
%   \renewcommand{\footrulewidth}{0.4pt}
%   \fancyhead[LE,RO]{\LHeadFont Naturvidenskab revy sange}
%   \fancyhead[CE,CO]{\CHeadFont\thepage}
%   \fancyhead[RE,LO]{\RHeadFont \SBThechapter}
% \fi\fi\fi

% \fancyfoot[LE,RO]{\LFootFont Computer Science Camp 2019}
% \ifSongEject
%   \fancyfoot[CE,CO]{\CFootFont Last Revised:  \RevDate}
% \else
%   \fancyfoot[CE,CO]{\CFootFont}
% \fi
% \fancyfoot[RE,LO]{\RFootFont Synges på eget ansvar}

%%%
% Table of contents
%%%

% \clearpage
% \twocolumn
% \font\myTinySF=cmss8    at  8pt
% \font\myHugeSF=cmssbx10 at 25pt
% \newcommand{\CpyRtInfoFont}{\tiny\myTinySF}
% \newcommand{\myTitleFont}{\Huge\myHugeSF}
% \newcommand{\mySubTitleFont}{\large\sf}
% \renewcommand{\indexspace}{\medskip}

% % {\parindent 8pt
% %   {\myTitleFont Indhold}}\par
% % \vskip 5pt
% \renewcommand{\SBThechapter}{Indhold}
% % {\parindent 20pt
% %   {\mySubTitleFont --- with first lines in italic ---}}
% % \vskip 20pt
% \let\olditem\item
% \let\oldsubitem\subitem
% \let\oldsubsubitem\subsubitem
% \renewcommand{\item}{\par\hangindent=40pt}
% \renewcommand{\subitem}{\par\hangindent=40pt \hspace*{20pt}}
% \renewcommand{\subsubitem}{\par\hangindent=40pt \hspace*{30pt}}

% %\input{unf-sangbog.tocx}

% \renewcommand{\item}{\olditem}
% \renewcommand{\subitem}{\oldsubitem}
% \renewcommand{\subsubitem}{\oldsubsubitem}

%%%
% Songbook begins.
%%%

\twocolumn
%It's just one page, don't print page numbers etc.
\pagestyle{empty}
%Songs included
\input{songs/matmatik.tex}
\input{songs/taal_daj.tex}
\input{songs/linieskriverdriver.tex}
\input{songs/steve_hawking.tex}
\input{songs/ode_til_kode.tex}
\input{songs/se_min_kode.tex}
\input{songs/vaabenfysik_kort.tex}
%Maybe include:
%\input{songs/kvanter_i_maaneskin.tex}
%\input{songs/mest_matematiske_dyr.tex}

% \input{songs/vi_kan_ikke_li.tex}
% \input{songs/selektionssangen.tex}
% \input{songs/alfabetsangen.tex}
% \input{songs/sciencecamps.tex}
% \input{songs/hvad_maa_man.tex}


% \input{songs/lambda_kalkylen.tex}
% \input{songs/puslespil.tex}
% \input{songs/null.tex}
% \input{songs/fasebal.tex}

% \input{songs/chifitter.tex}

% \input{songs/kun_fysik.tex}



% \input{songs/kanoniske.tex}
% \input{songs/jeg_er_en_matematiker_fra_hcoe.tex}


% \input{songs/rekursiv_skovsang.tex}
% \input{songs/laerkerede.tex}


% \clearpage
% \font\myTinySF=cmss8    at  8pt
% \font\myHugeSF=cmssbx10 at 25pt
% % \newcommand{\CpyRtInfoFont}{\tiny\myTinySF}
% % \newcommand{\myTitleFont}{\Huge\myHugeSF}
% % \newcommand{\mySubTitleFont}{\large\sf}
% \renewcommand{\indexspace}{\medskip}

% {\parindent 8pt
%   {\myTitleFont Index}}\par
% \vskip 5pt
% \renewcommand{\SBThechapter}{Index}
% % {\parindent 20pt
% %   {\mySubTitleFont --- with first lines in italic ---}}
% % \vskip 20pt
% \renewcommand{\item}{\par\hangindent=40pt}
% \renewcommand{\subitem}{\par\hangindent=40pt \hspace*{20pt}}
% \renewcommand{\subsubitem}{\par\hangindent=40pt \hspace*{30pt}}

%\input{unf-sangbog.tdx}

\end{document}
\bye
%
%%%
% Document ends.
%%%


% \renewcommand{\item}{\olditem}
% \renewcommand{\subitem}{\oldsubitem}
% \renewcommand{\subsubitem}{\oldsubsubitem}

%%%
% Songbook begins.
%%%

\twocolumn
%It's just one page, don't print page numbers etc.
\pagestyle{empty}
%Songs included
\input{songs/matmatik.tex}
\input{songs/taal_daj.tex}
\input{songs/linieskriverdriver.tex}
\input{songs/steve_hawking.tex}
\input{songs/ode_til_kode.tex}
\input{songs/se_min_kode.tex}
\input{songs/vaabenfysik_kort.tex}
%Maybe include:
%\input{songs/kvanter_i_maaneskin.tex}
%\input{songs/mest_matematiske_dyr.tex}

% \input{songs/vi_kan_ikke_li.tex}
% \input{songs/selektionssangen.tex}
% \input{songs/alfabetsangen.tex}
% \input{songs/sciencecamps.tex}
% \input{songs/hvad_maa_man.tex}


% \input{songs/lambda_kalkylen.tex}
% \input{songs/puslespil.tex}
% \input{songs/null.tex}
% \input{songs/fasebal.tex}

% \input{songs/chifitter.tex}

% \input{songs/kun_fysik.tex}



% \input{songs/kanoniske.tex}
% \input{songs/jeg_er_en_matematiker_fra_hcoe.tex}


% \input{songs/rekursiv_skovsang.tex}
% \input{songs/laerkerede.tex}


% \clearpage
% \font\myTinySF=cmss8    at  8pt
% \font\myHugeSF=cmssbx10 at 25pt
% % \newcommand{\CpyRtInfoFont}{\tiny\myTinySF}
% % \newcommand{\myTitleFont}{\Huge\myHugeSF}
% % \newcommand{\mySubTitleFont}{\large\sf}
% \renewcommand{\indexspace}{\medskip}

% {\parindent 8pt
%   {\myTitleFont Index}}\par
% \vskip 5pt
% \renewcommand{\SBThechapter}{Index}
% % {\parindent 20pt
% %   {\mySubTitleFont --- with first lines in italic ---}}
% % \vskip 20pt
% \renewcommand{\item}{\par\hangindent=40pt}
% \renewcommand{\subitem}{\par\hangindent=40pt \hspace*{20pt}}
% \renewcommand{\subsubitem}{\par\hangindent=40pt \hspace*{30pt}}

%%%%%%% rcsid = @(#)$Id: sample-sb.tex,v 1.23 2010-04-12 18:04:11 rathc Exp $
%%%%%%
%%
%%      ===============================
%%      Sample Songbook (sample-sb.tex)
%%      ===============================
%%
%%      Version 4.5, 30 April, 2010
%%
%%      Copyright 1992--2010 Christopher Rath <christopher@rath.ca>
%%
%%      This package is free software; you can redistribute it and/or
%%      modify it under the terms of version 2.1 of the GNU Lesser
%%	General Public License as published by the Free Software 
%%	Foundation.
%%
%%      This package is distributed in the hope that it will be
%%      useful, but WITHOUT ANY WARRANTY; without even the implied
%%      warranty of MERCHANTABILITY or FITNESS FOR A PARTICULAR
%%      PURPOSE.  See the GNU Lesser General Public License for more
%%      details.
%%
%%      This file contains a subset of the songbook we distribute
%%      at our church.  To the best of my knowledge, all of the lyrics
%%      contained herein are freely distributable.  This file has been
%%      provided as a sample of what can be produced by the chordbk,
%%      wordbk, and overhead LaTeX styles.
%%
%%      NEEDED:  The fancyhdr LaTeX style is required to properly
%%              format this file.  If you don't have that then comment
%%              out the commands in the preamble which deal with the
%%              fancyhdr style.
%%
%%%%%%
%%%%%%
%%
%%      1. Chord notation.  Within this songbook the following
%%         conventions have been adopted:
%%
%%              "Minor" is entered as "m";
%%                      e.g. Cm7 for C minor 7th.
%%              "Major" is entered as "M";
%%                      e.g. CM7 for C major 7th.
%%
%%%%%%
%%%%%%
%%      ============
%%      Bibliography
%%      ============
%%
%%      Exalt Him!: Exalt Him!  Compiled by Tom Fettke.  (c)1989
%%                      Word Music.
%%
%%      Hosanna! Music Books: Hosanna! Music Books #1--#6.
%%                      (c)1987--92 Integrity Music, Inc.
%%
%%      Worship Him II: Worship Him II.  Compiled by Jesse Peterson
%%                      and Bruce Ballinger.  (c)1989 Tempo Music
%%                      Publications.
%%
%%      Worship Songs Of The Vineyard: Worship Songs Of The Vineyard
%%                      --- Volume 2.  (c)1989 Vineyard Ministries
%%                      International.
%%
%%%%%%
%%%%%%

%%%%%%%%%%%%%%%%%%%%%%%%%%%%%%%%%%%%%%%%%%%%%%%%%%%%%%%%%%
%%%%%%%%%%%%%%%%%%%%%%%%%%%%%%%%%%%%%%%%%%%%%%%%%%%%%%%%%%
%%                                                      %%
%%           P R E A M B L E   B E G I N S              %%
%%                                                      %%
%%%%%%%%%%%%%%%%%%%%%%%%%%%%%%%%%%%%%%%%%%%%%%%%%%%%%%%%%%
%%%%%%%%%%%%%%%%%%%%%%%%%%%%%%%%%%%%%%%%%%%%%%%%%%%%%%%%%%

\documentclass[a5paper]{book}
\usepackage{latexsym,
            fancyhdr,
            titlesec,
            amsmath,
            amssymb,
            multicol,
            amsthm,
            stmaryrd,
            amsthm,
            color,
            needspace,
            stackengine,
            wasysym}
\usepackage[utf8]{inputenc}
\usepackage[T1]{fontenc}
% \usepackage[chordbk]{songbook}                  %% Words & Chords edition.
%%\usepackage[compactallsongs,chordbk]{songbook}    %% Words & Chords edition.
\usepackage[wordbk]{songbook}                 %% Words Only edition.
%%\usepackage[overhead]{songbook}               %% Overhead Transparency edition.
\usepackage{titletoc}
\usepackage{tket}  % Draws "TÅGEKAMMERET" correctly

%%%
% Revision Date and Release Date definitions.
%
%       \RelDate - The last time this songbook was released.  Set this
%                  date each time a new release/update of the songbook
%                  is generated.
%       \RevDate - The last time a particular song was revised in any
%                  way.  This command will be renewed inside every
%                  song.
%%%
\newcommand{\RelDate}{31~August,~2003}
\newcommand{\RevDate}{\today}

%%%
% C.C.L.I. license number definition; for copyright licensing info.
% One of these macros will be manually inserted into the {SBMel}
% parameter of the {song} environment.
%
%       \CCLInumber - The actual copyright license number.  Don't
%               insert this command in the {SBMel} parameter, use one
%               of the others.
%       \CCLIed - Indicates a song falls under our CCLI license.
%       \NotCCLIed - Indicates a song doesn't fall under our CCLI
%               license.  Public Domain songs fall into this category.
%       \PGranted - We have received specific permission from the
%               copyright holder to use this song.
%       \PPending - We are in the process of obtaining permission to
%               use this song.
%%%
\newcommand{\CCLInumber}{Your CCLI Number}
\newcommand{\CCLIed}{{\SBMelInfoFont (CCLI \CCLInumber)}}
\newcommand{\NotCCLIed}{\relax}
\newcommand{\PGranted}{\relax}
\newcommand{\PPending}{{\SBMelInfoFont (Permission Pending)}}

%%%
% Title page information.
%%%
%\title{UNF Computer Science Camp 2019 Sangbog}
%\author{}
%\date{Revideret:  \RevDate}

%%%
% Redefine fonts from SongBook style that I don't like.
%%%
\font\myTinySF=cmss8 at 8pt
\renewcommand{\SBMelInfoFont}{\tiny\myTinySF}

%%%
% Define fonts to use in the headers and footers of the songbook.
%%%
\newcommand{\LHeadFont}{\normalsize}            % = cmr12  at 12pt
\newcommand{\CHeadFont}{\normalsize\rm}         % = cmr12  at 12pt
\newcommand{\RHeadFont}{\normalsize}            % = cmr12  at 12pt
\newcommand{\LFootFont}{\scriptsize}            % = cmr8   at  8pt
\newcommand{\CFootFont}{\tiny\myTinySF}         % = cmss8  at  8pt
\newcommand{\RFootFont}{\scriptsize}            % = cmr8   at  8pt

\def\repeat{%
  \stackanchor{.}{.}%
  \rule[-\dp\strutbox]{.3pt}{\normalbaselineskip}%
  \kern0.5pt%
  \rule[-\dp\strutbox]{1pt}{\normalbaselineskip}%
  \kern1pt%
}
\def\frepeat{%
  \kern1pt%
  \rule[-\dp\strutbox]{1pt}{\normalbaselineskip}%
  \kern0.5pt%
  \rule[-\dp\strutbox]{.3pt}{\normalbaselineskip}%
  \stackanchor{.}{.}%
}
% \newcommand{\SBRepeat}[1]{#1\\#1}
\newcommand{\SBRepeat}[1]{\frepeat #1\repeat}
\setcounter{SBSongCnt}{-1}
\renewcommand{\SBWAndMTag}{Forfatter:}
\renewcommand{\SBUnknownTag}{Ukendt}
\renewcommand{\SBChorusTag}{Ref.}
\renewcommand{\SBOrgMel}{Originalmelodi}
\renewcommand{\SpaceAfterChorus}   {\vspace{0ex plus1ex minus 0.5ex}}
\renewcommand{\SpaceAfterOpGroup}  {\vspace{0ex plus1ex minus 0.5ex}}
\renewcommand{\SpaceAfterSBBracket}{\vspace{0ex plus1ex minus 0.5ex}}
\renewcommand{\SpaceAfterSection}  {\vspace{0ex plus1ex minus 0.5ex}}
\renewcommand{\SpaceAfterSong}     {\vspace{0ex plus1ex minus 0.5ex}}
\renewcommand{\SpaceAfterVerse}    {\vspace{0ex plus1ex minus 0.5ex}}

% Tell LaTeX that \medskip is a good place to make a page break
\let\oldmedskip\medskip
\renewcommand{\medskip}{\oldmedskip\pagebreak[2]}

%%%
% Turn on/off index-file generation.  Uncomment the \makeindex line to
% turn index generation on;  comment it out to turn index generation
% off.
%%%
%\makeTitleIndex         %% Title and First Line Index.
%\makeTitleContents      %% Table of Contents.
%\makeKeyIndex           %% Index of song by key.
% \makeArtistIndex	%% Index of song by artist.
% \newcommand{\SBThechapter}[0]{}
% \newcommand{\SBChapter}[1]{
%     \startcontents
%     \chapter*{#1} 
%     % \input{unf-sangbog.toc}
%       \begin{minipage}{.8\textwidth}
%         \printcontents{}{1}{}
%       \end{minipage}%
%     \renewcommand{\SBThechapter}{#1}
%     \clearpage
% }

% \titleformat{\chapter}
% [display]
% {}
% {%\vspace*{\fill}
%  % \titlerule[1pt]%
%  % \vspace{1pt}%
%  % \titlerule
%  % \vspace{1pc}%
%  \chaptertitlename}
% {}
% {\Huge}



%%%%%%%%%%%%%%%%%%%%%%%%%%%%%%%%%%%%%%%%%%%%%%%%%%%%%%%%%%
%%%%%%%%%%%%%%%%%%%%%%%%%%%%%%%%%%%%%%%%%%%%%%%%%%%%%%%%%%
%%                                                      %%
%%           D O C U M E N T   B E G I N S              %%
%%                                                      %%
%%%%%%%%%%%%%%%%%%%%%%%%%%%%%%%%%%%%%%%%%%%%%%%%%%%%%%%%%%
%%%%%%%%%%%%%%%%%%%%%%%%%%%%%%%%%%%%%%%%%%%%%%%%%%%%%%%%%%
\begin{document}

%%%
% Uncomment "\maketitle" statement to make a title page.
%%%
%\maketitle
% \begin{titlepage}
%   \centering
%   \vspace{5cm}
% 	\includegraphics[width=1\textwidth]{unf_logo.jpeg}\par\vspace{1cm}
% 	{\scshape\LARGE Sangbog \par}
% 	\vspace{1cm}
% 	{\scshape\Large UNF Computer Science Camp 2019\par}
	
% 	\vfill

% % Bottom of the page
% 	{\large \today\par}
% \end{titlepage}
% \mainmatter
% \ifWordBk
%   \twocolumn
% \fi


%%% Kolofon
%\thispagestyle{empty}
%Sammensat til UNF Computer Science Camp 2019 - csc.unf.dk\\
%Redaktør: Andreas Mosbæk Jensen m.fl. efter tidligere sangbog af Steffen Strunge Mathiesen\\
%Indhold opsat i \LaTeX. 
%Digital version og kildekode: github.com/steffen555/UNF-sangbog\\
%Revision 1 med stave fejl korrektioner
%\par\vspace*{\fill}
%Hvis du har forslag til sange, rettelser, ris og ros, eller hvis du kender en ukendt forfatter, så skriv til sangbog@unf.dk.

%%%
% Turn on and define fancy page heading/footing definition.
%%%
% \pagestyle{fancy}

% \ifChordBk
%   % It's a words & chords songbook...
%   \addtolength{\headwidth}{\marginparsep}
%   \addtolength{\headwidth}{\marginparwidth}
%   \renewcommand{\headrulewidth}{0.4pt}
%   \renewcommand{\footrulewidth}{0.4pt}
%   \fancyhead[LE,RO]{\LHeadFont\emph{\leftmark\/}\SBContinueMark}
%   \fancyhead[CE,CO]{\CHeadFont\thepage}
%   \fancyhead[RE,LO]{\RHeadFont \chaptermark}
% \else\ifOverhead
%   % It's an overhead...
%   \renewcommand{\footrulewidth}{0pt}
%   \renewcommand{\headrulewidth}{0pt}
%   \fancyhead[LE,RO]{}
%   \fancyhead[CE,CO]{}
%   \fancyhead[RE,LO]{}
% \else\ifWordBk
%   % It's a words only songbook...
%   \addtolength{\headwidth}{\marginparsep}
%   \addtolength{\headwidth}{\marginparwidth}
%   \renewcommand{\headrulewidth}{0.4pt}
%   \renewcommand{\footrulewidth}{0.4pt}
%   \fancyhead[LE,RO]{\LHeadFont Naturvidenskab revy sange}
%   \fancyhead[CE,CO]{\CHeadFont\thepage}
%   \fancyhead[RE,LO]{\RHeadFont \SBThechapter}
% \fi\fi\fi

% \fancyfoot[LE,RO]{\LFootFont Computer Science Camp 2019}
% \ifSongEject
%   \fancyfoot[CE,CO]{\CFootFont Last Revised:  \RevDate}
% \else
%   \fancyfoot[CE,CO]{\CFootFont}
% \fi
% \fancyfoot[RE,LO]{\RFootFont Synges på eget ansvar}

%%%
% Table of contents
%%%

% \clearpage
% \twocolumn
% \font\myTinySF=cmss8    at  8pt
% \font\myHugeSF=cmssbx10 at 25pt
% \newcommand{\CpyRtInfoFont}{\tiny\myTinySF}
% \newcommand{\myTitleFont}{\Huge\myHugeSF}
% \newcommand{\mySubTitleFont}{\large\sf}
% \renewcommand{\indexspace}{\medskip}

% % {\parindent 8pt
% %   {\myTitleFont Indhold}}\par
% % \vskip 5pt
% \renewcommand{\SBThechapter}{Indhold}
% % {\parindent 20pt
% %   {\mySubTitleFont --- with first lines in italic ---}}
% % \vskip 20pt
% \let\olditem\item
% \let\oldsubitem\subitem
% \let\oldsubsubitem\subsubitem
% \renewcommand{\item}{\par\hangindent=40pt}
% \renewcommand{\subitem}{\par\hangindent=40pt \hspace*{20pt}}
% \renewcommand{\subsubitem}{\par\hangindent=40pt \hspace*{30pt}}

% %\input{unf-sangbog.tocx}

% \renewcommand{\item}{\olditem}
% \renewcommand{\subitem}{\oldsubitem}
% \renewcommand{\subsubitem}{\oldsubsubitem}

%%%
% Songbook begins.
%%%

\twocolumn
%It's just one page, don't print page numbers etc.
\pagestyle{empty}
%Songs included
\input{songs/matmatik.tex}
\input{songs/taal_daj.tex}
\input{songs/linieskriverdriver.tex}
\input{songs/steve_hawking.tex}
\input{songs/ode_til_kode.tex}
\input{songs/se_min_kode.tex}
\input{songs/vaabenfysik_kort.tex}
%Maybe include:
%\input{songs/kvanter_i_maaneskin.tex}
%\input{songs/mest_matematiske_dyr.tex}

% \input{songs/vi_kan_ikke_li.tex}
% \input{songs/selektionssangen.tex}
% \input{songs/alfabetsangen.tex}
% \input{songs/sciencecamps.tex}
% \input{songs/hvad_maa_man.tex}


% \input{songs/lambda_kalkylen.tex}
% \input{songs/puslespil.tex}
% \input{songs/null.tex}
% \input{songs/fasebal.tex}

% \input{songs/chifitter.tex}

% \input{songs/kun_fysik.tex}



% \input{songs/kanoniske.tex}
% \input{songs/jeg_er_en_matematiker_fra_hcoe.tex}


% \input{songs/rekursiv_skovsang.tex}
% \input{songs/laerkerede.tex}


% \clearpage
% \font\myTinySF=cmss8    at  8pt
% \font\myHugeSF=cmssbx10 at 25pt
% % \newcommand{\CpyRtInfoFont}{\tiny\myTinySF}
% % \newcommand{\myTitleFont}{\Huge\myHugeSF}
% % \newcommand{\mySubTitleFont}{\large\sf}
% \renewcommand{\indexspace}{\medskip}

% {\parindent 8pt
%   {\myTitleFont Index}}\par
% \vskip 5pt
% \renewcommand{\SBThechapter}{Index}
% % {\parindent 20pt
% %   {\mySubTitleFont --- with first lines in italic ---}}
% % \vskip 20pt
% \renewcommand{\item}{\par\hangindent=40pt}
% \renewcommand{\subitem}{\par\hangindent=40pt \hspace*{20pt}}
% \renewcommand{\subsubitem}{\par\hangindent=40pt \hspace*{30pt}}

%\input{unf-sangbog.tdx}

\end{document}
\bye
%
%%%
% Document ends.
%%%


\end{document}
\bye
%
%%%
% Document ends.
%%%

%       \begin{minipage}{.8\textwidth}
%         \printcontents{}{1}{}
%       \end{minipage}%
%     \renewcommand{\SBThechapter}{#1}
%     \clearpage
% }

% \titleformat{\chapter}
% [display]
% {}
% {%\vspace*{\fill}
%  % \titlerule[1pt]%
%  % \vspace{1pt}%
%  % \titlerule
%  % \vspace{1pc}%
%  \chaptertitlename}
% {}
% {\Huge}



%%%%%%%%%%%%%%%%%%%%%%%%%%%%%%%%%%%%%%%%%%%%%%%%%%%%%%%%%%
%%%%%%%%%%%%%%%%%%%%%%%%%%%%%%%%%%%%%%%%%%%%%%%%%%%%%%%%%%
%%                                                      %%
%%           D O C U M E N T   B E G I N S              %%
%%                                                      %%
%%%%%%%%%%%%%%%%%%%%%%%%%%%%%%%%%%%%%%%%%%%%%%%%%%%%%%%%%%
%%%%%%%%%%%%%%%%%%%%%%%%%%%%%%%%%%%%%%%%%%%%%%%%%%%%%%%%%%
\begin{document}

%%%
% Uncomment "\maketitle" statement to make a title page.
%%%
%\maketitle
% \begin{titlepage}
%   \centering
%   \vspace{5cm}
% 	\includegraphics[width=1\textwidth]{unf_logo.jpeg}\par\vspace{1cm}
% 	{\scshape\LARGE Sangbog \par}
% 	\vspace{1cm}
% 	{\scshape\Large UNF Computer Science Camp 2019\par}
	
% 	\vfill

% % Bottom of the page
% 	{\large \today\par}
% \end{titlepage}
% \mainmatter
% \ifWordBk
%   \twocolumn
% \fi


%%% Kolofon
%\thispagestyle{empty}
%Sammensat til UNF Computer Science Camp 2019 - csc.unf.dk\\
%Redaktør: Andreas Mosbæk Jensen m.fl. efter tidligere sangbog af Steffen Strunge Mathiesen\\
%Indhold opsat i \LaTeX. 
%Digital version og kildekode: github.com/steffen555/UNF-sangbog\\
%Revision 1 med stave fejl korrektioner
%\par\vspace*{\fill}
%Hvis du har forslag til sange, rettelser, ris og ros, eller hvis du kender en ukendt forfatter, så skriv til sangbog@unf.dk.

%%%
% Turn on and define fancy page heading/footing definition.
%%%
% \pagestyle{fancy}

% \ifChordBk
%   % It's a words & chords songbook...
%   \addtolength{\headwidth}{\marginparsep}
%   \addtolength{\headwidth}{\marginparwidth}
%   \renewcommand{\headrulewidth}{0.4pt}
%   \renewcommand{\footrulewidth}{0.4pt}
%   \fancyhead[LE,RO]{\LHeadFont\emph{\leftmark\/}\SBContinueMark}
%   \fancyhead[CE,CO]{\CHeadFont\thepage}
%   \fancyhead[RE,LO]{\RHeadFont \chaptermark}
% \else\ifOverhead
%   % It's an overhead...
%   \renewcommand{\footrulewidth}{0pt}
%   \renewcommand{\headrulewidth}{0pt}
%   \fancyhead[LE,RO]{}
%   \fancyhead[CE,CO]{}
%   \fancyhead[RE,LO]{}
% \else\ifWordBk
%   % It's a words only songbook...
%   \addtolength{\headwidth}{\marginparsep}
%   \addtolength{\headwidth}{\marginparwidth}
%   \renewcommand{\headrulewidth}{0.4pt}
%   \renewcommand{\footrulewidth}{0.4pt}
%   \fancyhead[LE,RO]{\LHeadFont Naturvidenskab revy sange}
%   \fancyhead[CE,CO]{\CHeadFont\thepage}
%   \fancyhead[RE,LO]{\RHeadFont \SBThechapter}
% \fi\fi\fi

% \fancyfoot[LE,RO]{\LFootFont Computer Science Camp 2019}
% \ifSongEject
%   \fancyfoot[CE,CO]{\CFootFont Last Revised:  \RevDate}
% \else
%   \fancyfoot[CE,CO]{\CFootFont}
% \fi
% \fancyfoot[RE,LO]{\RFootFont Synges på eget ansvar}

%%%
% Table of contents
%%%

% \clearpage
% \twocolumn
% \font\myTinySF=cmss8    at  8pt
% \font\myHugeSF=cmssbx10 at 25pt
% \newcommand{\CpyRtInfoFont}{\tiny\myTinySF}
% \newcommand{\myTitleFont}{\Huge\myHugeSF}
% \newcommand{\mySubTitleFont}{\large\sf}
% \renewcommand{\indexspace}{\medskip}

% % {\parindent 8pt
% %   {\myTitleFont Indhold}}\par
% % \vskip 5pt
% \renewcommand{\SBThechapter}{Indhold}
% % {\parindent 20pt
% %   {\mySubTitleFont --- with first lines in italic ---}}
% % \vskip 20pt
% \let\olditem\item
% \let\oldsubitem\subitem
% \let\oldsubsubitem\subsubitem
% \renewcommand{\item}{\par\hangindent=40pt}
% \renewcommand{\subitem}{\par\hangindent=40pt \hspace*{20pt}}
% \renewcommand{\subsubitem}{\par\hangindent=40pt \hspace*{30pt}}

% %%%%%%% rcsid = @(#)$Id: sample-sb.tex,v 1.23 2010-04-12 18:04:11 rathc Exp $
%%%%%%
%%
%%      ===============================
%%      Sample Songbook (sample-sb.tex)
%%      ===============================
%%
%%      Version 4.5, 30 April, 2010
%%
%%      Copyright 1992--2010 Christopher Rath <christopher@rath.ca>
%%
%%      This package is free software; you can redistribute it and/or
%%      modify it under the terms of version 2.1 of the GNU Lesser
%%	General Public License as published by the Free Software 
%%	Foundation.
%%
%%      This package is distributed in the hope that it will be
%%      useful, but WITHOUT ANY WARRANTY; without even the implied
%%      warranty of MERCHANTABILITY or FITNESS FOR A PARTICULAR
%%      PURPOSE.  See the GNU Lesser General Public License for more
%%      details.
%%
%%      This file contains a subset of the songbook we distribute
%%      at our church.  To the best of my knowledge, all of the lyrics
%%      contained herein are freely distributable.  This file has been
%%      provided as a sample of what can be produced by the chordbk,
%%      wordbk, and overhead LaTeX styles.
%%
%%      NEEDED:  The fancyhdr LaTeX style is required to properly
%%              format this file.  If you don't have that then comment
%%              out the commands in the preamble which deal with the
%%              fancyhdr style.
%%
%%%%%%
%%%%%%
%%
%%      1. Chord notation.  Within this songbook the following
%%         conventions have been adopted:
%%
%%              "Minor" is entered as "m";
%%                      e.g. Cm7 for C minor 7th.
%%              "Major" is entered as "M";
%%                      e.g. CM7 for C major 7th.
%%
%%%%%%
%%%%%%
%%      ============
%%      Bibliography
%%      ============
%%
%%      Exalt Him!: Exalt Him!  Compiled by Tom Fettke.  (c)1989
%%                      Word Music.
%%
%%      Hosanna! Music Books: Hosanna! Music Books #1--#6.
%%                      (c)1987--92 Integrity Music, Inc.
%%
%%      Worship Him II: Worship Him II.  Compiled by Jesse Peterson
%%                      and Bruce Ballinger.  (c)1989 Tempo Music
%%                      Publications.
%%
%%      Worship Songs Of The Vineyard: Worship Songs Of The Vineyard
%%                      --- Volume 2.  (c)1989 Vineyard Ministries
%%                      International.
%%
%%%%%%
%%%%%%

%%%%%%%%%%%%%%%%%%%%%%%%%%%%%%%%%%%%%%%%%%%%%%%%%%%%%%%%%%
%%%%%%%%%%%%%%%%%%%%%%%%%%%%%%%%%%%%%%%%%%%%%%%%%%%%%%%%%%
%%                                                      %%
%%           P R E A M B L E   B E G I N S              %%
%%                                                      %%
%%%%%%%%%%%%%%%%%%%%%%%%%%%%%%%%%%%%%%%%%%%%%%%%%%%%%%%%%%
%%%%%%%%%%%%%%%%%%%%%%%%%%%%%%%%%%%%%%%%%%%%%%%%%%%%%%%%%%

\documentclass[a5paper]{book}
\usepackage{latexsym,
            fancyhdr,
            titlesec,
            amsmath,
            amssymb,
            multicol,
            amsthm,
            stmaryrd,
            amsthm,
            color,
            needspace,
            stackengine,
            wasysym}
\usepackage[utf8]{inputenc}
\usepackage[T1]{fontenc}
% \usepackage[chordbk]{songbook}                  %% Words & Chords edition.
%%\usepackage[compactallsongs,chordbk]{songbook}    %% Words & Chords edition.
\usepackage[wordbk]{songbook}                 %% Words Only edition.
%%\usepackage[overhead]{songbook}               %% Overhead Transparency edition.
\usepackage{titletoc}
\usepackage{tket}  % Draws "TÅGEKAMMERET" correctly

%%%
% Revision Date and Release Date definitions.
%
%       \RelDate - The last time this songbook was released.  Set this
%                  date each time a new release/update of the songbook
%                  is generated.
%       \RevDate - The last time a particular song was revised in any
%                  way.  This command will be renewed inside every
%                  song.
%%%
\newcommand{\RelDate}{31~August,~2003}
\newcommand{\RevDate}{\today}

%%%
% C.C.L.I. license number definition; for copyright licensing info.
% One of these macros will be manually inserted into the {SBMel}
% parameter of the {song} environment.
%
%       \CCLInumber - The actual copyright license number.  Don't
%               insert this command in the {SBMel} parameter, use one
%               of the others.
%       \CCLIed - Indicates a song falls under our CCLI license.
%       \NotCCLIed - Indicates a song doesn't fall under our CCLI
%               license.  Public Domain songs fall into this category.
%       \PGranted - We have received specific permission from the
%               copyright holder to use this song.
%       \PPending - We are in the process of obtaining permission to
%               use this song.
%%%
\newcommand{\CCLInumber}{Your CCLI Number}
\newcommand{\CCLIed}{{\SBMelInfoFont (CCLI \CCLInumber)}}
\newcommand{\NotCCLIed}{\relax}
\newcommand{\PGranted}{\relax}
\newcommand{\PPending}{{\SBMelInfoFont (Permission Pending)}}

%%%
% Title page information.
%%%
%\title{UNF Computer Science Camp 2019 Sangbog}
%\author{}
%\date{Revideret:  \RevDate}

%%%
% Redefine fonts from SongBook style that I don't like.
%%%
\font\myTinySF=cmss8 at 8pt
\renewcommand{\SBMelInfoFont}{\tiny\myTinySF}

%%%
% Define fonts to use in the headers and footers of the songbook.
%%%
\newcommand{\LHeadFont}{\normalsize}            % = cmr12  at 12pt
\newcommand{\CHeadFont}{\normalsize\rm}         % = cmr12  at 12pt
\newcommand{\RHeadFont}{\normalsize}            % = cmr12  at 12pt
\newcommand{\LFootFont}{\scriptsize}            % = cmr8   at  8pt
\newcommand{\CFootFont}{\tiny\myTinySF}         % = cmss8  at  8pt
\newcommand{\RFootFont}{\scriptsize}            % = cmr8   at  8pt

\def\repeat{%
  \stackanchor{.}{.}%
  \rule[-\dp\strutbox]{.3pt}{\normalbaselineskip}%
  \kern0.5pt%
  \rule[-\dp\strutbox]{1pt}{\normalbaselineskip}%
  \kern1pt%
}
\def\frepeat{%
  \kern1pt%
  \rule[-\dp\strutbox]{1pt}{\normalbaselineskip}%
  \kern0.5pt%
  \rule[-\dp\strutbox]{.3pt}{\normalbaselineskip}%
  \stackanchor{.}{.}%
}
% \newcommand{\SBRepeat}[1]{#1\\#1}
\newcommand{\SBRepeat}[1]{\frepeat #1\repeat}
\setcounter{SBSongCnt}{-1}
\renewcommand{\SBWAndMTag}{Forfatter:}
\renewcommand{\SBUnknownTag}{Ukendt}
\renewcommand{\SBChorusTag}{Ref.}
\renewcommand{\SBOrgMel}{Originalmelodi}
\renewcommand{\SpaceAfterChorus}   {\vspace{0ex plus1ex minus 0.5ex}}
\renewcommand{\SpaceAfterOpGroup}  {\vspace{0ex plus1ex minus 0.5ex}}
\renewcommand{\SpaceAfterSBBracket}{\vspace{0ex plus1ex minus 0.5ex}}
\renewcommand{\SpaceAfterSection}  {\vspace{0ex plus1ex minus 0.5ex}}
\renewcommand{\SpaceAfterSong}     {\vspace{0ex plus1ex minus 0.5ex}}
\renewcommand{\SpaceAfterVerse}    {\vspace{0ex plus1ex minus 0.5ex}}

% Tell LaTeX that \medskip is a good place to make a page break
\let\oldmedskip\medskip
\renewcommand{\medskip}{\oldmedskip\pagebreak[2]}

%%%
% Turn on/off index-file generation.  Uncomment the \makeindex line to
% turn index generation on;  comment it out to turn index generation
% off.
%%%
%\makeTitleIndex         %% Title and First Line Index.
%\makeTitleContents      %% Table of Contents.
%\makeKeyIndex           %% Index of song by key.
% \makeArtistIndex	%% Index of song by artist.
% \newcommand{\SBThechapter}[0]{}
% \newcommand{\SBChapter}[1]{
%     \startcontents
%     \chapter*{#1} 
%     % %%%%%% rcsid = @(#)$Id: sample-sb.tex,v 1.23 2010-04-12 18:04:11 rathc Exp $
%%%%%%
%%
%%      ===============================
%%      Sample Songbook (sample-sb.tex)
%%      ===============================
%%
%%      Version 4.5, 30 April, 2010
%%
%%      Copyright 1992--2010 Christopher Rath <christopher@rath.ca>
%%
%%      This package is free software; you can redistribute it and/or
%%      modify it under the terms of version 2.1 of the GNU Lesser
%%	General Public License as published by the Free Software 
%%	Foundation.
%%
%%      This package is distributed in the hope that it will be
%%      useful, but WITHOUT ANY WARRANTY; without even the implied
%%      warranty of MERCHANTABILITY or FITNESS FOR A PARTICULAR
%%      PURPOSE.  See the GNU Lesser General Public License for more
%%      details.
%%
%%      This file contains a subset of the songbook we distribute
%%      at our church.  To the best of my knowledge, all of the lyrics
%%      contained herein are freely distributable.  This file has been
%%      provided as a sample of what can be produced by the chordbk,
%%      wordbk, and overhead LaTeX styles.
%%
%%      NEEDED:  The fancyhdr LaTeX style is required to properly
%%              format this file.  If you don't have that then comment
%%              out the commands in the preamble which deal with the
%%              fancyhdr style.
%%
%%%%%%
%%%%%%
%%
%%      1. Chord notation.  Within this songbook the following
%%         conventions have been adopted:
%%
%%              "Minor" is entered as "m";
%%                      e.g. Cm7 for C minor 7th.
%%              "Major" is entered as "M";
%%                      e.g. CM7 for C major 7th.
%%
%%%%%%
%%%%%%
%%      ============
%%      Bibliography
%%      ============
%%
%%      Exalt Him!: Exalt Him!  Compiled by Tom Fettke.  (c)1989
%%                      Word Music.
%%
%%      Hosanna! Music Books: Hosanna! Music Books #1--#6.
%%                      (c)1987--92 Integrity Music, Inc.
%%
%%      Worship Him II: Worship Him II.  Compiled by Jesse Peterson
%%                      and Bruce Ballinger.  (c)1989 Tempo Music
%%                      Publications.
%%
%%      Worship Songs Of The Vineyard: Worship Songs Of The Vineyard
%%                      --- Volume 2.  (c)1989 Vineyard Ministries
%%                      International.
%%
%%%%%%
%%%%%%

%%%%%%%%%%%%%%%%%%%%%%%%%%%%%%%%%%%%%%%%%%%%%%%%%%%%%%%%%%
%%%%%%%%%%%%%%%%%%%%%%%%%%%%%%%%%%%%%%%%%%%%%%%%%%%%%%%%%%
%%                                                      %%
%%           P R E A M B L E   B E G I N S              %%
%%                                                      %%
%%%%%%%%%%%%%%%%%%%%%%%%%%%%%%%%%%%%%%%%%%%%%%%%%%%%%%%%%%
%%%%%%%%%%%%%%%%%%%%%%%%%%%%%%%%%%%%%%%%%%%%%%%%%%%%%%%%%%

\documentclass[a5paper]{book}
\usepackage{latexsym,
            fancyhdr,
            titlesec,
            amsmath,
            amssymb,
            multicol,
            amsthm,
            stmaryrd,
            amsthm,
            color,
            needspace,
            stackengine,
            wasysym}
\usepackage[utf8]{inputenc}
\usepackage[T1]{fontenc}
% \usepackage[chordbk]{songbook}                  %% Words & Chords edition.
%%\usepackage[compactallsongs,chordbk]{songbook}    %% Words & Chords edition.
\usepackage[wordbk]{songbook}                 %% Words Only edition.
%%\usepackage[overhead]{songbook}               %% Overhead Transparency edition.
\usepackage{titletoc}
\usepackage{tket}  % Draws "TÅGEKAMMERET" correctly

%%%
% Revision Date and Release Date definitions.
%
%       \RelDate - The last time this songbook was released.  Set this
%                  date each time a new release/update of the songbook
%                  is generated.
%       \RevDate - The last time a particular song was revised in any
%                  way.  This command will be renewed inside every
%                  song.
%%%
\newcommand{\RelDate}{31~August,~2003}
\newcommand{\RevDate}{\today}

%%%
% C.C.L.I. license number definition; for copyright licensing info.
% One of these macros will be manually inserted into the {SBMel}
% parameter of the {song} environment.
%
%       \CCLInumber - The actual copyright license number.  Don't
%               insert this command in the {SBMel} parameter, use one
%               of the others.
%       \CCLIed - Indicates a song falls under our CCLI license.
%       \NotCCLIed - Indicates a song doesn't fall under our CCLI
%               license.  Public Domain songs fall into this category.
%       \PGranted - We have received specific permission from the
%               copyright holder to use this song.
%       \PPending - We are in the process of obtaining permission to
%               use this song.
%%%
\newcommand{\CCLInumber}{Your CCLI Number}
\newcommand{\CCLIed}{{\SBMelInfoFont (CCLI \CCLInumber)}}
\newcommand{\NotCCLIed}{\relax}
\newcommand{\PGranted}{\relax}
\newcommand{\PPending}{{\SBMelInfoFont (Permission Pending)}}

%%%
% Title page information.
%%%
%\title{UNF Computer Science Camp 2019 Sangbog}
%\author{}
%\date{Revideret:  \RevDate}

%%%
% Redefine fonts from SongBook style that I don't like.
%%%
\font\myTinySF=cmss8 at 8pt
\renewcommand{\SBMelInfoFont}{\tiny\myTinySF}

%%%
% Define fonts to use in the headers and footers of the songbook.
%%%
\newcommand{\LHeadFont}{\normalsize}            % = cmr12  at 12pt
\newcommand{\CHeadFont}{\normalsize\rm}         % = cmr12  at 12pt
\newcommand{\RHeadFont}{\normalsize}            % = cmr12  at 12pt
\newcommand{\LFootFont}{\scriptsize}            % = cmr8   at  8pt
\newcommand{\CFootFont}{\tiny\myTinySF}         % = cmss8  at  8pt
\newcommand{\RFootFont}{\scriptsize}            % = cmr8   at  8pt

\def\repeat{%
  \stackanchor{.}{.}%
  \rule[-\dp\strutbox]{.3pt}{\normalbaselineskip}%
  \kern0.5pt%
  \rule[-\dp\strutbox]{1pt}{\normalbaselineskip}%
  \kern1pt%
}
\def\frepeat{%
  \kern1pt%
  \rule[-\dp\strutbox]{1pt}{\normalbaselineskip}%
  \kern0.5pt%
  \rule[-\dp\strutbox]{.3pt}{\normalbaselineskip}%
  \stackanchor{.}{.}%
}
% \newcommand{\SBRepeat}[1]{#1\\#1}
\newcommand{\SBRepeat}[1]{\frepeat #1\repeat}
\setcounter{SBSongCnt}{-1}
\renewcommand{\SBWAndMTag}{Forfatter:}
\renewcommand{\SBUnknownTag}{Ukendt}
\renewcommand{\SBChorusTag}{Ref.}
\renewcommand{\SBOrgMel}{Originalmelodi}
\renewcommand{\SpaceAfterChorus}   {\vspace{0ex plus1ex minus 0.5ex}}
\renewcommand{\SpaceAfterOpGroup}  {\vspace{0ex plus1ex minus 0.5ex}}
\renewcommand{\SpaceAfterSBBracket}{\vspace{0ex plus1ex minus 0.5ex}}
\renewcommand{\SpaceAfterSection}  {\vspace{0ex plus1ex minus 0.5ex}}
\renewcommand{\SpaceAfterSong}     {\vspace{0ex plus1ex minus 0.5ex}}
\renewcommand{\SpaceAfterVerse}    {\vspace{0ex plus1ex minus 0.5ex}}

% Tell LaTeX that \medskip is a good place to make a page break
\let\oldmedskip\medskip
\renewcommand{\medskip}{\oldmedskip\pagebreak[2]}

%%%
% Turn on/off index-file generation.  Uncomment the \makeindex line to
% turn index generation on;  comment it out to turn index generation
% off.
%%%
%\makeTitleIndex         %% Title and First Line Index.
%\makeTitleContents      %% Table of Contents.
%\makeKeyIndex           %% Index of song by key.
% \makeArtistIndex	%% Index of song by artist.
% \newcommand{\SBThechapter}[0]{}
% \newcommand{\SBChapter}[1]{
%     \startcontents
%     \chapter*{#1} 
%     % \input{unf-sangbog.toc}
%       \begin{minipage}{.8\textwidth}
%         \printcontents{}{1}{}
%       \end{minipage}%
%     \renewcommand{\SBThechapter}{#1}
%     \clearpage
% }

% \titleformat{\chapter}
% [display]
% {}
% {%\vspace*{\fill}
%  % \titlerule[1pt]%
%  % \vspace{1pt}%
%  % \titlerule
%  % \vspace{1pc}%
%  \chaptertitlename}
% {}
% {\Huge}



%%%%%%%%%%%%%%%%%%%%%%%%%%%%%%%%%%%%%%%%%%%%%%%%%%%%%%%%%%
%%%%%%%%%%%%%%%%%%%%%%%%%%%%%%%%%%%%%%%%%%%%%%%%%%%%%%%%%%
%%                                                      %%
%%           D O C U M E N T   B E G I N S              %%
%%                                                      %%
%%%%%%%%%%%%%%%%%%%%%%%%%%%%%%%%%%%%%%%%%%%%%%%%%%%%%%%%%%
%%%%%%%%%%%%%%%%%%%%%%%%%%%%%%%%%%%%%%%%%%%%%%%%%%%%%%%%%%
\begin{document}

%%%
% Uncomment "\maketitle" statement to make a title page.
%%%
%\maketitle
% \begin{titlepage}
%   \centering
%   \vspace{5cm}
% 	\includegraphics[width=1\textwidth]{unf_logo.jpeg}\par\vspace{1cm}
% 	{\scshape\LARGE Sangbog \par}
% 	\vspace{1cm}
% 	{\scshape\Large UNF Computer Science Camp 2019\par}
	
% 	\vfill

% % Bottom of the page
% 	{\large \today\par}
% \end{titlepage}
% \mainmatter
% \ifWordBk
%   \twocolumn
% \fi


%%% Kolofon
%\thispagestyle{empty}
%Sammensat til UNF Computer Science Camp 2019 - csc.unf.dk\\
%Redaktør: Andreas Mosbæk Jensen m.fl. efter tidligere sangbog af Steffen Strunge Mathiesen\\
%Indhold opsat i \LaTeX. 
%Digital version og kildekode: github.com/steffen555/UNF-sangbog\\
%Revision 1 med stave fejl korrektioner
%\par\vspace*{\fill}
%Hvis du har forslag til sange, rettelser, ris og ros, eller hvis du kender en ukendt forfatter, så skriv til sangbog@unf.dk.

%%%
% Turn on and define fancy page heading/footing definition.
%%%
% \pagestyle{fancy}

% \ifChordBk
%   % It's a words & chords songbook...
%   \addtolength{\headwidth}{\marginparsep}
%   \addtolength{\headwidth}{\marginparwidth}
%   \renewcommand{\headrulewidth}{0.4pt}
%   \renewcommand{\footrulewidth}{0.4pt}
%   \fancyhead[LE,RO]{\LHeadFont\emph{\leftmark\/}\SBContinueMark}
%   \fancyhead[CE,CO]{\CHeadFont\thepage}
%   \fancyhead[RE,LO]{\RHeadFont \chaptermark}
% \else\ifOverhead
%   % It's an overhead...
%   \renewcommand{\footrulewidth}{0pt}
%   \renewcommand{\headrulewidth}{0pt}
%   \fancyhead[LE,RO]{}
%   \fancyhead[CE,CO]{}
%   \fancyhead[RE,LO]{}
% \else\ifWordBk
%   % It's a words only songbook...
%   \addtolength{\headwidth}{\marginparsep}
%   \addtolength{\headwidth}{\marginparwidth}
%   \renewcommand{\headrulewidth}{0.4pt}
%   \renewcommand{\footrulewidth}{0.4pt}
%   \fancyhead[LE,RO]{\LHeadFont Naturvidenskab revy sange}
%   \fancyhead[CE,CO]{\CHeadFont\thepage}
%   \fancyhead[RE,LO]{\RHeadFont \SBThechapter}
% \fi\fi\fi

% \fancyfoot[LE,RO]{\LFootFont Computer Science Camp 2019}
% \ifSongEject
%   \fancyfoot[CE,CO]{\CFootFont Last Revised:  \RevDate}
% \else
%   \fancyfoot[CE,CO]{\CFootFont}
% \fi
% \fancyfoot[RE,LO]{\RFootFont Synges på eget ansvar}

%%%
% Table of contents
%%%

% \clearpage
% \twocolumn
% \font\myTinySF=cmss8    at  8pt
% \font\myHugeSF=cmssbx10 at 25pt
% \newcommand{\CpyRtInfoFont}{\tiny\myTinySF}
% \newcommand{\myTitleFont}{\Huge\myHugeSF}
% \newcommand{\mySubTitleFont}{\large\sf}
% \renewcommand{\indexspace}{\medskip}

% % {\parindent 8pt
% %   {\myTitleFont Indhold}}\par
% % \vskip 5pt
% \renewcommand{\SBThechapter}{Indhold}
% % {\parindent 20pt
% %   {\mySubTitleFont --- with first lines in italic ---}}
% % \vskip 20pt
% \let\olditem\item
% \let\oldsubitem\subitem
% \let\oldsubsubitem\subsubitem
% \renewcommand{\item}{\par\hangindent=40pt}
% \renewcommand{\subitem}{\par\hangindent=40pt \hspace*{20pt}}
% \renewcommand{\subsubitem}{\par\hangindent=40pt \hspace*{30pt}}

% %\input{unf-sangbog.tocx}

% \renewcommand{\item}{\olditem}
% \renewcommand{\subitem}{\oldsubitem}
% \renewcommand{\subsubitem}{\oldsubsubitem}

%%%
% Songbook begins.
%%%

\twocolumn
%It's just one page, don't print page numbers etc.
\pagestyle{empty}
%Songs included
\input{songs/matmatik.tex}
\input{songs/taal_daj.tex}
\input{songs/linieskriverdriver.tex}
\input{songs/steve_hawking.tex}
\input{songs/ode_til_kode.tex}
\input{songs/se_min_kode.tex}
\input{songs/vaabenfysik_kort.tex}
%Maybe include:
%\input{songs/kvanter_i_maaneskin.tex}
%\input{songs/mest_matematiske_dyr.tex}

% \input{songs/vi_kan_ikke_li.tex}
% \input{songs/selektionssangen.tex}
% \input{songs/alfabetsangen.tex}
% \input{songs/sciencecamps.tex}
% \input{songs/hvad_maa_man.tex}


% \input{songs/lambda_kalkylen.tex}
% \input{songs/puslespil.tex}
% \input{songs/null.tex}
% \input{songs/fasebal.tex}

% \input{songs/chifitter.tex}

% \input{songs/kun_fysik.tex}



% \input{songs/kanoniske.tex}
% \input{songs/jeg_er_en_matematiker_fra_hcoe.tex}


% \input{songs/rekursiv_skovsang.tex}
% \input{songs/laerkerede.tex}


% \clearpage
% \font\myTinySF=cmss8    at  8pt
% \font\myHugeSF=cmssbx10 at 25pt
% % \newcommand{\CpyRtInfoFont}{\tiny\myTinySF}
% % \newcommand{\myTitleFont}{\Huge\myHugeSF}
% % \newcommand{\mySubTitleFont}{\large\sf}
% \renewcommand{\indexspace}{\medskip}

% {\parindent 8pt
%   {\myTitleFont Index}}\par
% \vskip 5pt
% \renewcommand{\SBThechapter}{Index}
% % {\parindent 20pt
% %   {\mySubTitleFont --- with first lines in italic ---}}
% % \vskip 20pt
% \renewcommand{\item}{\par\hangindent=40pt}
% \renewcommand{\subitem}{\par\hangindent=40pt \hspace*{20pt}}
% \renewcommand{\subsubitem}{\par\hangindent=40pt \hspace*{30pt}}

%\input{unf-sangbog.tdx}

\end{document}
\bye
%
%%%
% Document ends.
%%%

%       \begin{minipage}{.8\textwidth}
%         \printcontents{}{1}{}
%       \end{minipage}%
%     \renewcommand{\SBThechapter}{#1}
%     \clearpage
% }

% \titleformat{\chapter}
% [display]
% {}
% {%\vspace*{\fill}
%  % \titlerule[1pt]%
%  % \vspace{1pt}%
%  % \titlerule
%  % \vspace{1pc}%
%  \chaptertitlename}
% {}
% {\Huge}



%%%%%%%%%%%%%%%%%%%%%%%%%%%%%%%%%%%%%%%%%%%%%%%%%%%%%%%%%%
%%%%%%%%%%%%%%%%%%%%%%%%%%%%%%%%%%%%%%%%%%%%%%%%%%%%%%%%%%
%%                                                      %%
%%           D O C U M E N T   B E G I N S              %%
%%                                                      %%
%%%%%%%%%%%%%%%%%%%%%%%%%%%%%%%%%%%%%%%%%%%%%%%%%%%%%%%%%%
%%%%%%%%%%%%%%%%%%%%%%%%%%%%%%%%%%%%%%%%%%%%%%%%%%%%%%%%%%
\begin{document}

%%%
% Uncomment "\maketitle" statement to make a title page.
%%%
%\maketitle
% \begin{titlepage}
%   \centering
%   \vspace{5cm}
% 	\includegraphics[width=1\textwidth]{unf_logo.jpeg}\par\vspace{1cm}
% 	{\scshape\LARGE Sangbog \par}
% 	\vspace{1cm}
% 	{\scshape\Large UNF Computer Science Camp 2019\par}
	
% 	\vfill

% % Bottom of the page
% 	{\large \today\par}
% \end{titlepage}
% \mainmatter
% \ifWordBk
%   \twocolumn
% \fi


%%% Kolofon
%\thispagestyle{empty}
%Sammensat til UNF Computer Science Camp 2019 - csc.unf.dk\\
%Redaktør: Andreas Mosbæk Jensen m.fl. efter tidligere sangbog af Steffen Strunge Mathiesen\\
%Indhold opsat i \LaTeX. 
%Digital version og kildekode: github.com/steffen555/UNF-sangbog\\
%Revision 1 med stave fejl korrektioner
%\par\vspace*{\fill}
%Hvis du har forslag til sange, rettelser, ris og ros, eller hvis du kender en ukendt forfatter, så skriv til sangbog@unf.dk.

%%%
% Turn on and define fancy page heading/footing definition.
%%%
% \pagestyle{fancy}

% \ifChordBk
%   % It's a words & chords songbook...
%   \addtolength{\headwidth}{\marginparsep}
%   \addtolength{\headwidth}{\marginparwidth}
%   \renewcommand{\headrulewidth}{0.4pt}
%   \renewcommand{\footrulewidth}{0.4pt}
%   \fancyhead[LE,RO]{\LHeadFont\emph{\leftmark\/}\SBContinueMark}
%   \fancyhead[CE,CO]{\CHeadFont\thepage}
%   \fancyhead[RE,LO]{\RHeadFont \chaptermark}
% \else\ifOverhead
%   % It's an overhead...
%   \renewcommand{\footrulewidth}{0pt}
%   \renewcommand{\headrulewidth}{0pt}
%   \fancyhead[LE,RO]{}
%   \fancyhead[CE,CO]{}
%   \fancyhead[RE,LO]{}
% \else\ifWordBk
%   % It's a words only songbook...
%   \addtolength{\headwidth}{\marginparsep}
%   \addtolength{\headwidth}{\marginparwidth}
%   \renewcommand{\headrulewidth}{0.4pt}
%   \renewcommand{\footrulewidth}{0.4pt}
%   \fancyhead[LE,RO]{\LHeadFont Naturvidenskab revy sange}
%   \fancyhead[CE,CO]{\CHeadFont\thepage}
%   \fancyhead[RE,LO]{\RHeadFont \SBThechapter}
% \fi\fi\fi

% \fancyfoot[LE,RO]{\LFootFont Computer Science Camp 2019}
% \ifSongEject
%   \fancyfoot[CE,CO]{\CFootFont Last Revised:  \RevDate}
% \else
%   \fancyfoot[CE,CO]{\CFootFont}
% \fi
% \fancyfoot[RE,LO]{\RFootFont Synges på eget ansvar}

%%%
% Table of contents
%%%

% \clearpage
% \twocolumn
% \font\myTinySF=cmss8    at  8pt
% \font\myHugeSF=cmssbx10 at 25pt
% \newcommand{\CpyRtInfoFont}{\tiny\myTinySF}
% \newcommand{\myTitleFont}{\Huge\myHugeSF}
% \newcommand{\mySubTitleFont}{\large\sf}
% \renewcommand{\indexspace}{\medskip}

% % {\parindent 8pt
% %   {\myTitleFont Indhold}}\par
% % \vskip 5pt
% \renewcommand{\SBThechapter}{Indhold}
% % {\parindent 20pt
% %   {\mySubTitleFont --- with first lines in italic ---}}
% % \vskip 20pt
% \let\olditem\item
% \let\oldsubitem\subitem
% \let\oldsubsubitem\subsubitem
% \renewcommand{\item}{\par\hangindent=40pt}
% \renewcommand{\subitem}{\par\hangindent=40pt \hspace*{20pt}}
% \renewcommand{\subsubitem}{\par\hangindent=40pt \hspace*{30pt}}

% %%%%%%% rcsid = @(#)$Id: sample-sb.tex,v 1.23 2010-04-12 18:04:11 rathc Exp $
%%%%%%
%%
%%      ===============================
%%      Sample Songbook (sample-sb.tex)
%%      ===============================
%%
%%      Version 4.5, 30 April, 2010
%%
%%      Copyright 1992--2010 Christopher Rath <christopher@rath.ca>
%%
%%      This package is free software; you can redistribute it and/or
%%      modify it under the terms of version 2.1 of the GNU Lesser
%%	General Public License as published by the Free Software 
%%	Foundation.
%%
%%      This package is distributed in the hope that it will be
%%      useful, but WITHOUT ANY WARRANTY; without even the implied
%%      warranty of MERCHANTABILITY or FITNESS FOR A PARTICULAR
%%      PURPOSE.  See the GNU Lesser General Public License for more
%%      details.
%%
%%      This file contains a subset of the songbook we distribute
%%      at our church.  To the best of my knowledge, all of the lyrics
%%      contained herein are freely distributable.  This file has been
%%      provided as a sample of what can be produced by the chordbk,
%%      wordbk, and overhead LaTeX styles.
%%
%%      NEEDED:  The fancyhdr LaTeX style is required to properly
%%              format this file.  If you don't have that then comment
%%              out the commands in the preamble which deal with the
%%              fancyhdr style.
%%
%%%%%%
%%%%%%
%%
%%      1. Chord notation.  Within this songbook the following
%%         conventions have been adopted:
%%
%%              "Minor" is entered as "m";
%%                      e.g. Cm7 for C minor 7th.
%%              "Major" is entered as "M";
%%                      e.g. CM7 for C major 7th.
%%
%%%%%%
%%%%%%
%%      ============
%%      Bibliography
%%      ============
%%
%%      Exalt Him!: Exalt Him!  Compiled by Tom Fettke.  (c)1989
%%                      Word Music.
%%
%%      Hosanna! Music Books: Hosanna! Music Books #1--#6.
%%                      (c)1987--92 Integrity Music, Inc.
%%
%%      Worship Him II: Worship Him II.  Compiled by Jesse Peterson
%%                      and Bruce Ballinger.  (c)1989 Tempo Music
%%                      Publications.
%%
%%      Worship Songs Of The Vineyard: Worship Songs Of The Vineyard
%%                      --- Volume 2.  (c)1989 Vineyard Ministries
%%                      International.
%%
%%%%%%
%%%%%%

%%%%%%%%%%%%%%%%%%%%%%%%%%%%%%%%%%%%%%%%%%%%%%%%%%%%%%%%%%
%%%%%%%%%%%%%%%%%%%%%%%%%%%%%%%%%%%%%%%%%%%%%%%%%%%%%%%%%%
%%                                                      %%
%%           P R E A M B L E   B E G I N S              %%
%%                                                      %%
%%%%%%%%%%%%%%%%%%%%%%%%%%%%%%%%%%%%%%%%%%%%%%%%%%%%%%%%%%
%%%%%%%%%%%%%%%%%%%%%%%%%%%%%%%%%%%%%%%%%%%%%%%%%%%%%%%%%%

\documentclass[a5paper]{book}
\usepackage{latexsym,
            fancyhdr,
            titlesec,
            amsmath,
            amssymb,
            multicol,
            amsthm,
            stmaryrd,
            amsthm,
            color,
            needspace,
            stackengine,
            wasysym}
\usepackage[utf8]{inputenc}
\usepackage[T1]{fontenc}
% \usepackage[chordbk]{songbook}                  %% Words & Chords edition.
%%\usepackage[compactallsongs,chordbk]{songbook}    %% Words & Chords edition.
\usepackage[wordbk]{songbook}                 %% Words Only edition.
%%\usepackage[overhead]{songbook}               %% Overhead Transparency edition.
\usepackage{titletoc}
\usepackage{tket}  % Draws "TÅGEKAMMERET" correctly

%%%
% Revision Date and Release Date definitions.
%
%       \RelDate - The last time this songbook was released.  Set this
%                  date each time a new release/update of the songbook
%                  is generated.
%       \RevDate - The last time a particular song was revised in any
%                  way.  This command will be renewed inside every
%                  song.
%%%
\newcommand{\RelDate}{31~August,~2003}
\newcommand{\RevDate}{\today}

%%%
% C.C.L.I. license number definition; for copyright licensing info.
% One of these macros will be manually inserted into the {SBMel}
% parameter of the {song} environment.
%
%       \CCLInumber - The actual copyright license number.  Don't
%               insert this command in the {SBMel} parameter, use one
%               of the others.
%       \CCLIed - Indicates a song falls under our CCLI license.
%       \NotCCLIed - Indicates a song doesn't fall under our CCLI
%               license.  Public Domain songs fall into this category.
%       \PGranted - We have received specific permission from the
%               copyright holder to use this song.
%       \PPending - We are in the process of obtaining permission to
%               use this song.
%%%
\newcommand{\CCLInumber}{Your CCLI Number}
\newcommand{\CCLIed}{{\SBMelInfoFont (CCLI \CCLInumber)}}
\newcommand{\NotCCLIed}{\relax}
\newcommand{\PGranted}{\relax}
\newcommand{\PPending}{{\SBMelInfoFont (Permission Pending)}}

%%%
% Title page information.
%%%
%\title{UNF Computer Science Camp 2019 Sangbog}
%\author{}
%\date{Revideret:  \RevDate}

%%%
% Redefine fonts from SongBook style that I don't like.
%%%
\font\myTinySF=cmss8 at 8pt
\renewcommand{\SBMelInfoFont}{\tiny\myTinySF}

%%%
% Define fonts to use in the headers and footers of the songbook.
%%%
\newcommand{\LHeadFont}{\normalsize}            % = cmr12  at 12pt
\newcommand{\CHeadFont}{\normalsize\rm}         % = cmr12  at 12pt
\newcommand{\RHeadFont}{\normalsize}            % = cmr12  at 12pt
\newcommand{\LFootFont}{\scriptsize}            % = cmr8   at  8pt
\newcommand{\CFootFont}{\tiny\myTinySF}         % = cmss8  at  8pt
\newcommand{\RFootFont}{\scriptsize}            % = cmr8   at  8pt

\def\repeat{%
  \stackanchor{.}{.}%
  \rule[-\dp\strutbox]{.3pt}{\normalbaselineskip}%
  \kern0.5pt%
  \rule[-\dp\strutbox]{1pt}{\normalbaselineskip}%
  \kern1pt%
}
\def\frepeat{%
  \kern1pt%
  \rule[-\dp\strutbox]{1pt}{\normalbaselineskip}%
  \kern0.5pt%
  \rule[-\dp\strutbox]{.3pt}{\normalbaselineskip}%
  \stackanchor{.}{.}%
}
% \newcommand{\SBRepeat}[1]{#1\\#1}
\newcommand{\SBRepeat}[1]{\frepeat #1\repeat}
\setcounter{SBSongCnt}{-1}
\renewcommand{\SBWAndMTag}{Forfatter:}
\renewcommand{\SBUnknownTag}{Ukendt}
\renewcommand{\SBChorusTag}{Ref.}
\renewcommand{\SBOrgMel}{Originalmelodi}
\renewcommand{\SpaceAfterChorus}   {\vspace{0ex plus1ex minus 0.5ex}}
\renewcommand{\SpaceAfterOpGroup}  {\vspace{0ex plus1ex minus 0.5ex}}
\renewcommand{\SpaceAfterSBBracket}{\vspace{0ex plus1ex minus 0.5ex}}
\renewcommand{\SpaceAfterSection}  {\vspace{0ex plus1ex minus 0.5ex}}
\renewcommand{\SpaceAfterSong}     {\vspace{0ex plus1ex minus 0.5ex}}
\renewcommand{\SpaceAfterVerse}    {\vspace{0ex plus1ex minus 0.5ex}}

% Tell LaTeX that \medskip is a good place to make a page break
\let\oldmedskip\medskip
\renewcommand{\medskip}{\oldmedskip\pagebreak[2]}

%%%
% Turn on/off index-file generation.  Uncomment the \makeindex line to
% turn index generation on;  comment it out to turn index generation
% off.
%%%
%\makeTitleIndex         %% Title and First Line Index.
%\makeTitleContents      %% Table of Contents.
%\makeKeyIndex           %% Index of song by key.
% \makeArtistIndex	%% Index of song by artist.
% \newcommand{\SBThechapter}[0]{}
% \newcommand{\SBChapter}[1]{
%     \startcontents
%     \chapter*{#1} 
%     % \input{unf-sangbog.toc}
%       \begin{minipage}{.8\textwidth}
%         \printcontents{}{1}{}
%       \end{minipage}%
%     \renewcommand{\SBThechapter}{#1}
%     \clearpage
% }

% \titleformat{\chapter}
% [display]
% {}
% {%\vspace*{\fill}
%  % \titlerule[1pt]%
%  % \vspace{1pt}%
%  % \titlerule
%  % \vspace{1pc}%
%  \chaptertitlename}
% {}
% {\Huge}



%%%%%%%%%%%%%%%%%%%%%%%%%%%%%%%%%%%%%%%%%%%%%%%%%%%%%%%%%%
%%%%%%%%%%%%%%%%%%%%%%%%%%%%%%%%%%%%%%%%%%%%%%%%%%%%%%%%%%
%%                                                      %%
%%           D O C U M E N T   B E G I N S              %%
%%                                                      %%
%%%%%%%%%%%%%%%%%%%%%%%%%%%%%%%%%%%%%%%%%%%%%%%%%%%%%%%%%%
%%%%%%%%%%%%%%%%%%%%%%%%%%%%%%%%%%%%%%%%%%%%%%%%%%%%%%%%%%
\begin{document}

%%%
% Uncomment "\maketitle" statement to make a title page.
%%%
%\maketitle
% \begin{titlepage}
%   \centering
%   \vspace{5cm}
% 	\includegraphics[width=1\textwidth]{unf_logo.jpeg}\par\vspace{1cm}
% 	{\scshape\LARGE Sangbog \par}
% 	\vspace{1cm}
% 	{\scshape\Large UNF Computer Science Camp 2019\par}
	
% 	\vfill

% % Bottom of the page
% 	{\large \today\par}
% \end{titlepage}
% \mainmatter
% \ifWordBk
%   \twocolumn
% \fi


%%% Kolofon
%\thispagestyle{empty}
%Sammensat til UNF Computer Science Camp 2019 - csc.unf.dk\\
%Redaktør: Andreas Mosbæk Jensen m.fl. efter tidligere sangbog af Steffen Strunge Mathiesen\\
%Indhold opsat i \LaTeX. 
%Digital version og kildekode: github.com/steffen555/UNF-sangbog\\
%Revision 1 med stave fejl korrektioner
%\par\vspace*{\fill}
%Hvis du har forslag til sange, rettelser, ris og ros, eller hvis du kender en ukendt forfatter, så skriv til sangbog@unf.dk.

%%%
% Turn on and define fancy page heading/footing definition.
%%%
% \pagestyle{fancy}

% \ifChordBk
%   % It's a words & chords songbook...
%   \addtolength{\headwidth}{\marginparsep}
%   \addtolength{\headwidth}{\marginparwidth}
%   \renewcommand{\headrulewidth}{0.4pt}
%   \renewcommand{\footrulewidth}{0.4pt}
%   \fancyhead[LE,RO]{\LHeadFont\emph{\leftmark\/}\SBContinueMark}
%   \fancyhead[CE,CO]{\CHeadFont\thepage}
%   \fancyhead[RE,LO]{\RHeadFont \chaptermark}
% \else\ifOverhead
%   % It's an overhead...
%   \renewcommand{\footrulewidth}{0pt}
%   \renewcommand{\headrulewidth}{0pt}
%   \fancyhead[LE,RO]{}
%   \fancyhead[CE,CO]{}
%   \fancyhead[RE,LO]{}
% \else\ifWordBk
%   % It's a words only songbook...
%   \addtolength{\headwidth}{\marginparsep}
%   \addtolength{\headwidth}{\marginparwidth}
%   \renewcommand{\headrulewidth}{0.4pt}
%   \renewcommand{\footrulewidth}{0.4pt}
%   \fancyhead[LE,RO]{\LHeadFont Naturvidenskab revy sange}
%   \fancyhead[CE,CO]{\CHeadFont\thepage}
%   \fancyhead[RE,LO]{\RHeadFont \SBThechapter}
% \fi\fi\fi

% \fancyfoot[LE,RO]{\LFootFont Computer Science Camp 2019}
% \ifSongEject
%   \fancyfoot[CE,CO]{\CFootFont Last Revised:  \RevDate}
% \else
%   \fancyfoot[CE,CO]{\CFootFont}
% \fi
% \fancyfoot[RE,LO]{\RFootFont Synges på eget ansvar}

%%%
% Table of contents
%%%

% \clearpage
% \twocolumn
% \font\myTinySF=cmss8    at  8pt
% \font\myHugeSF=cmssbx10 at 25pt
% \newcommand{\CpyRtInfoFont}{\tiny\myTinySF}
% \newcommand{\myTitleFont}{\Huge\myHugeSF}
% \newcommand{\mySubTitleFont}{\large\sf}
% \renewcommand{\indexspace}{\medskip}

% % {\parindent 8pt
% %   {\myTitleFont Indhold}}\par
% % \vskip 5pt
% \renewcommand{\SBThechapter}{Indhold}
% % {\parindent 20pt
% %   {\mySubTitleFont --- with first lines in italic ---}}
% % \vskip 20pt
% \let\olditem\item
% \let\oldsubitem\subitem
% \let\oldsubsubitem\subsubitem
% \renewcommand{\item}{\par\hangindent=40pt}
% \renewcommand{\subitem}{\par\hangindent=40pt \hspace*{20pt}}
% \renewcommand{\subsubitem}{\par\hangindent=40pt \hspace*{30pt}}

% %\input{unf-sangbog.tocx}

% \renewcommand{\item}{\olditem}
% \renewcommand{\subitem}{\oldsubitem}
% \renewcommand{\subsubitem}{\oldsubsubitem}

%%%
% Songbook begins.
%%%

\twocolumn
%It's just one page, don't print page numbers etc.
\pagestyle{empty}
%Songs included
\input{songs/matmatik.tex}
\input{songs/taal_daj.tex}
\input{songs/linieskriverdriver.tex}
\input{songs/steve_hawking.tex}
\input{songs/ode_til_kode.tex}
\input{songs/se_min_kode.tex}
\input{songs/vaabenfysik_kort.tex}
%Maybe include:
%\input{songs/kvanter_i_maaneskin.tex}
%\input{songs/mest_matematiske_dyr.tex}

% \input{songs/vi_kan_ikke_li.tex}
% \input{songs/selektionssangen.tex}
% \input{songs/alfabetsangen.tex}
% \input{songs/sciencecamps.tex}
% \input{songs/hvad_maa_man.tex}


% \input{songs/lambda_kalkylen.tex}
% \input{songs/puslespil.tex}
% \input{songs/null.tex}
% \input{songs/fasebal.tex}

% \input{songs/chifitter.tex}

% \input{songs/kun_fysik.tex}



% \input{songs/kanoniske.tex}
% \input{songs/jeg_er_en_matematiker_fra_hcoe.tex}


% \input{songs/rekursiv_skovsang.tex}
% \input{songs/laerkerede.tex}


% \clearpage
% \font\myTinySF=cmss8    at  8pt
% \font\myHugeSF=cmssbx10 at 25pt
% % \newcommand{\CpyRtInfoFont}{\tiny\myTinySF}
% % \newcommand{\myTitleFont}{\Huge\myHugeSF}
% % \newcommand{\mySubTitleFont}{\large\sf}
% \renewcommand{\indexspace}{\medskip}

% {\parindent 8pt
%   {\myTitleFont Index}}\par
% \vskip 5pt
% \renewcommand{\SBThechapter}{Index}
% % {\parindent 20pt
% %   {\mySubTitleFont --- with first lines in italic ---}}
% % \vskip 20pt
% \renewcommand{\item}{\par\hangindent=40pt}
% \renewcommand{\subitem}{\par\hangindent=40pt \hspace*{20pt}}
% \renewcommand{\subsubitem}{\par\hangindent=40pt \hspace*{30pt}}

%\input{unf-sangbog.tdx}

\end{document}
\bye
%
%%%
% Document ends.
%%%


% \renewcommand{\item}{\olditem}
% \renewcommand{\subitem}{\oldsubitem}
% \renewcommand{\subsubitem}{\oldsubsubitem}

%%%
% Songbook begins.
%%%

\twocolumn
%It's just one page, don't print page numbers etc.
\pagestyle{empty}
%Songs included
\input{songs/matmatik.tex}
\input{songs/taal_daj.tex}
\input{songs/linieskriverdriver.tex}
\input{songs/steve_hawking.tex}
\input{songs/ode_til_kode.tex}
\input{songs/se_min_kode.tex}
\input{songs/vaabenfysik_kort.tex}
%Maybe include:
%\input{songs/kvanter_i_maaneskin.tex}
%\input{songs/mest_matematiske_dyr.tex}

% \input{songs/vi_kan_ikke_li.tex}
% \input{songs/selektionssangen.tex}
% \input{songs/alfabetsangen.tex}
% \input{songs/sciencecamps.tex}
% \input{songs/hvad_maa_man.tex}


% \input{songs/lambda_kalkylen.tex}
% \input{songs/puslespil.tex}
% \input{songs/null.tex}
% \input{songs/fasebal.tex}

% \input{songs/chifitter.tex}

% \input{songs/kun_fysik.tex}



% \input{songs/kanoniske.tex}
% \input{songs/jeg_er_en_matematiker_fra_hcoe.tex}


% \input{songs/rekursiv_skovsang.tex}
% \input{songs/laerkerede.tex}


% \clearpage
% \font\myTinySF=cmss8    at  8pt
% \font\myHugeSF=cmssbx10 at 25pt
% % \newcommand{\CpyRtInfoFont}{\tiny\myTinySF}
% % \newcommand{\myTitleFont}{\Huge\myHugeSF}
% % \newcommand{\mySubTitleFont}{\large\sf}
% \renewcommand{\indexspace}{\medskip}

% {\parindent 8pt
%   {\myTitleFont Index}}\par
% \vskip 5pt
% \renewcommand{\SBThechapter}{Index}
% % {\parindent 20pt
% %   {\mySubTitleFont --- with first lines in italic ---}}
% % \vskip 20pt
% \renewcommand{\item}{\par\hangindent=40pt}
% \renewcommand{\subitem}{\par\hangindent=40pt \hspace*{20pt}}
% \renewcommand{\subsubitem}{\par\hangindent=40pt \hspace*{30pt}}

%%%%%%% rcsid = @(#)$Id: sample-sb.tex,v 1.23 2010-04-12 18:04:11 rathc Exp $
%%%%%%
%%
%%      ===============================
%%      Sample Songbook (sample-sb.tex)
%%      ===============================
%%
%%      Version 4.5, 30 April, 2010
%%
%%      Copyright 1992--2010 Christopher Rath <christopher@rath.ca>
%%
%%      This package is free software; you can redistribute it and/or
%%      modify it under the terms of version 2.1 of the GNU Lesser
%%	General Public License as published by the Free Software 
%%	Foundation.
%%
%%      This package is distributed in the hope that it will be
%%      useful, but WITHOUT ANY WARRANTY; without even the implied
%%      warranty of MERCHANTABILITY or FITNESS FOR A PARTICULAR
%%      PURPOSE.  See the GNU Lesser General Public License for more
%%      details.
%%
%%      This file contains a subset of the songbook we distribute
%%      at our church.  To the best of my knowledge, all of the lyrics
%%      contained herein are freely distributable.  This file has been
%%      provided as a sample of what can be produced by the chordbk,
%%      wordbk, and overhead LaTeX styles.
%%
%%      NEEDED:  The fancyhdr LaTeX style is required to properly
%%              format this file.  If you don't have that then comment
%%              out the commands in the preamble which deal with the
%%              fancyhdr style.
%%
%%%%%%
%%%%%%
%%
%%      1. Chord notation.  Within this songbook the following
%%         conventions have been adopted:
%%
%%              "Minor" is entered as "m";
%%                      e.g. Cm7 for C minor 7th.
%%              "Major" is entered as "M";
%%                      e.g. CM7 for C major 7th.
%%
%%%%%%
%%%%%%
%%      ============
%%      Bibliography
%%      ============
%%
%%      Exalt Him!: Exalt Him!  Compiled by Tom Fettke.  (c)1989
%%                      Word Music.
%%
%%      Hosanna! Music Books: Hosanna! Music Books #1--#6.
%%                      (c)1987--92 Integrity Music, Inc.
%%
%%      Worship Him II: Worship Him II.  Compiled by Jesse Peterson
%%                      and Bruce Ballinger.  (c)1989 Tempo Music
%%                      Publications.
%%
%%      Worship Songs Of The Vineyard: Worship Songs Of The Vineyard
%%                      --- Volume 2.  (c)1989 Vineyard Ministries
%%                      International.
%%
%%%%%%
%%%%%%

%%%%%%%%%%%%%%%%%%%%%%%%%%%%%%%%%%%%%%%%%%%%%%%%%%%%%%%%%%
%%%%%%%%%%%%%%%%%%%%%%%%%%%%%%%%%%%%%%%%%%%%%%%%%%%%%%%%%%
%%                                                      %%
%%           P R E A M B L E   B E G I N S              %%
%%                                                      %%
%%%%%%%%%%%%%%%%%%%%%%%%%%%%%%%%%%%%%%%%%%%%%%%%%%%%%%%%%%
%%%%%%%%%%%%%%%%%%%%%%%%%%%%%%%%%%%%%%%%%%%%%%%%%%%%%%%%%%

\documentclass[a5paper]{book}
\usepackage{latexsym,
            fancyhdr,
            titlesec,
            amsmath,
            amssymb,
            multicol,
            amsthm,
            stmaryrd,
            amsthm,
            color,
            needspace,
            stackengine,
            wasysym}
\usepackage[utf8]{inputenc}
\usepackage[T1]{fontenc}
% \usepackage[chordbk]{songbook}                  %% Words & Chords edition.
%%\usepackage[compactallsongs,chordbk]{songbook}    %% Words & Chords edition.
\usepackage[wordbk]{songbook}                 %% Words Only edition.
%%\usepackage[overhead]{songbook}               %% Overhead Transparency edition.
\usepackage{titletoc}
\usepackage{tket}  % Draws "TÅGEKAMMERET" correctly

%%%
% Revision Date and Release Date definitions.
%
%       \RelDate - The last time this songbook was released.  Set this
%                  date each time a new release/update of the songbook
%                  is generated.
%       \RevDate - The last time a particular song was revised in any
%                  way.  This command will be renewed inside every
%                  song.
%%%
\newcommand{\RelDate}{31~August,~2003}
\newcommand{\RevDate}{\today}

%%%
% C.C.L.I. license number definition; for copyright licensing info.
% One of these macros will be manually inserted into the {SBMel}
% parameter of the {song} environment.
%
%       \CCLInumber - The actual copyright license number.  Don't
%               insert this command in the {SBMel} parameter, use one
%               of the others.
%       \CCLIed - Indicates a song falls under our CCLI license.
%       \NotCCLIed - Indicates a song doesn't fall under our CCLI
%               license.  Public Domain songs fall into this category.
%       \PGranted - We have received specific permission from the
%               copyright holder to use this song.
%       \PPending - We are in the process of obtaining permission to
%               use this song.
%%%
\newcommand{\CCLInumber}{Your CCLI Number}
\newcommand{\CCLIed}{{\SBMelInfoFont (CCLI \CCLInumber)}}
\newcommand{\NotCCLIed}{\relax}
\newcommand{\PGranted}{\relax}
\newcommand{\PPending}{{\SBMelInfoFont (Permission Pending)}}

%%%
% Title page information.
%%%
%\title{UNF Computer Science Camp 2019 Sangbog}
%\author{}
%\date{Revideret:  \RevDate}

%%%
% Redefine fonts from SongBook style that I don't like.
%%%
\font\myTinySF=cmss8 at 8pt
\renewcommand{\SBMelInfoFont}{\tiny\myTinySF}

%%%
% Define fonts to use in the headers and footers of the songbook.
%%%
\newcommand{\LHeadFont}{\normalsize}            % = cmr12  at 12pt
\newcommand{\CHeadFont}{\normalsize\rm}         % = cmr12  at 12pt
\newcommand{\RHeadFont}{\normalsize}            % = cmr12  at 12pt
\newcommand{\LFootFont}{\scriptsize}            % = cmr8   at  8pt
\newcommand{\CFootFont}{\tiny\myTinySF}         % = cmss8  at  8pt
\newcommand{\RFootFont}{\scriptsize}            % = cmr8   at  8pt

\def\repeat{%
  \stackanchor{.}{.}%
  \rule[-\dp\strutbox]{.3pt}{\normalbaselineskip}%
  \kern0.5pt%
  \rule[-\dp\strutbox]{1pt}{\normalbaselineskip}%
  \kern1pt%
}
\def\frepeat{%
  \kern1pt%
  \rule[-\dp\strutbox]{1pt}{\normalbaselineskip}%
  \kern0.5pt%
  \rule[-\dp\strutbox]{.3pt}{\normalbaselineskip}%
  \stackanchor{.}{.}%
}
% \newcommand{\SBRepeat}[1]{#1\\#1}
\newcommand{\SBRepeat}[1]{\frepeat #1\repeat}
\setcounter{SBSongCnt}{-1}
\renewcommand{\SBWAndMTag}{Forfatter:}
\renewcommand{\SBUnknownTag}{Ukendt}
\renewcommand{\SBChorusTag}{Ref.}
\renewcommand{\SBOrgMel}{Originalmelodi}
\renewcommand{\SpaceAfterChorus}   {\vspace{0ex plus1ex minus 0.5ex}}
\renewcommand{\SpaceAfterOpGroup}  {\vspace{0ex plus1ex minus 0.5ex}}
\renewcommand{\SpaceAfterSBBracket}{\vspace{0ex plus1ex minus 0.5ex}}
\renewcommand{\SpaceAfterSection}  {\vspace{0ex plus1ex minus 0.5ex}}
\renewcommand{\SpaceAfterSong}     {\vspace{0ex plus1ex minus 0.5ex}}
\renewcommand{\SpaceAfterVerse}    {\vspace{0ex plus1ex minus 0.5ex}}

% Tell LaTeX that \medskip is a good place to make a page break
\let\oldmedskip\medskip
\renewcommand{\medskip}{\oldmedskip\pagebreak[2]}

%%%
% Turn on/off index-file generation.  Uncomment the \makeindex line to
% turn index generation on;  comment it out to turn index generation
% off.
%%%
%\makeTitleIndex         %% Title and First Line Index.
%\makeTitleContents      %% Table of Contents.
%\makeKeyIndex           %% Index of song by key.
% \makeArtistIndex	%% Index of song by artist.
% \newcommand{\SBThechapter}[0]{}
% \newcommand{\SBChapter}[1]{
%     \startcontents
%     \chapter*{#1} 
%     % \input{unf-sangbog.toc}
%       \begin{minipage}{.8\textwidth}
%         \printcontents{}{1}{}
%       \end{minipage}%
%     \renewcommand{\SBThechapter}{#1}
%     \clearpage
% }

% \titleformat{\chapter}
% [display]
% {}
% {%\vspace*{\fill}
%  % \titlerule[1pt]%
%  % \vspace{1pt}%
%  % \titlerule
%  % \vspace{1pc}%
%  \chaptertitlename}
% {}
% {\Huge}



%%%%%%%%%%%%%%%%%%%%%%%%%%%%%%%%%%%%%%%%%%%%%%%%%%%%%%%%%%
%%%%%%%%%%%%%%%%%%%%%%%%%%%%%%%%%%%%%%%%%%%%%%%%%%%%%%%%%%
%%                                                      %%
%%           D O C U M E N T   B E G I N S              %%
%%                                                      %%
%%%%%%%%%%%%%%%%%%%%%%%%%%%%%%%%%%%%%%%%%%%%%%%%%%%%%%%%%%
%%%%%%%%%%%%%%%%%%%%%%%%%%%%%%%%%%%%%%%%%%%%%%%%%%%%%%%%%%
\begin{document}

%%%
% Uncomment "\maketitle" statement to make a title page.
%%%
%\maketitle
% \begin{titlepage}
%   \centering
%   \vspace{5cm}
% 	\includegraphics[width=1\textwidth]{unf_logo.jpeg}\par\vspace{1cm}
% 	{\scshape\LARGE Sangbog \par}
% 	\vspace{1cm}
% 	{\scshape\Large UNF Computer Science Camp 2019\par}
	
% 	\vfill

% % Bottom of the page
% 	{\large \today\par}
% \end{titlepage}
% \mainmatter
% \ifWordBk
%   \twocolumn
% \fi


%%% Kolofon
%\thispagestyle{empty}
%Sammensat til UNF Computer Science Camp 2019 - csc.unf.dk\\
%Redaktør: Andreas Mosbæk Jensen m.fl. efter tidligere sangbog af Steffen Strunge Mathiesen\\
%Indhold opsat i \LaTeX. 
%Digital version og kildekode: github.com/steffen555/UNF-sangbog\\
%Revision 1 med stave fejl korrektioner
%\par\vspace*{\fill}
%Hvis du har forslag til sange, rettelser, ris og ros, eller hvis du kender en ukendt forfatter, så skriv til sangbog@unf.dk.

%%%
% Turn on and define fancy page heading/footing definition.
%%%
% \pagestyle{fancy}

% \ifChordBk
%   % It's a words & chords songbook...
%   \addtolength{\headwidth}{\marginparsep}
%   \addtolength{\headwidth}{\marginparwidth}
%   \renewcommand{\headrulewidth}{0.4pt}
%   \renewcommand{\footrulewidth}{0.4pt}
%   \fancyhead[LE,RO]{\LHeadFont\emph{\leftmark\/}\SBContinueMark}
%   \fancyhead[CE,CO]{\CHeadFont\thepage}
%   \fancyhead[RE,LO]{\RHeadFont \chaptermark}
% \else\ifOverhead
%   % It's an overhead...
%   \renewcommand{\footrulewidth}{0pt}
%   \renewcommand{\headrulewidth}{0pt}
%   \fancyhead[LE,RO]{}
%   \fancyhead[CE,CO]{}
%   \fancyhead[RE,LO]{}
% \else\ifWordBk
%   % It's a words only songbook...
%   \addtolength{\headwidth}{\marginparsep}
%   \addtolength{\headwidth}{\marginparwidth}
%   \renewcommand{\headrulewidth}{0.4pt}
%   \renewcommand{\footrulewidth}{0.4pt}
%   \fancyhead[LE,RO]{\LHeadFont Naturvidenskab revy sange}
%   \fancyhead[CE,CO]{\CHeadFont\thepage}
%   \fancyhead[RE,LO]{\RHeadFont \SBThechapter}
% \fi\fi\fi

% \fancyfoot[LE,RO]{\LFootFont Computer Science Camp 2019}
% \ifSongEject
%   \fancyfoot[CE,CO]{\CFootFont Last Revised:  \RevDate}
% \else
%   \fancyfoot[CE,CO]{\CFootFont}
% \fi
% \fancyfoot[RE,LO]{\RFootFont Synges på eget ansvar}

%%%
% Table of contents
%%%

% \clearpage
% \twocolumn
% \font\myTinySF=cmss8    at  8pt
% \font\myHugeSF=cmssbx10 at 25pt
% \newcommand{\CpyRtInfoFont}{\tiny\myTinySF}
% \newcommand{\myTitleFont}{\Huge\myHugeSF}
% \newcommand{\mySubTitleFont}{\large\sf}
% \renewcommand{\indexspace}{\medskip}

% % {\parindent 8pt
% %   {\myTitleFont Indhold}}\par
% % \vskip 5pt
% \renewcommand{\SBThechapter}{Indhold}
% % {\parindent 20pt
% %   {\mySubTitleFont --- with first lines in italic ---}}
% % \vskip 20pt
% \let\olditem\item
% \let\oldsubitem\subitem
% \let\oldsubsubitem\subsubitem
% \renewcommand{\item}{\par\hangindent=40pt}
% \renewcommand{\subitem}{\par\hangindent=40pt \hspace*{20pt}}
% \renewcommand{\subsubitem}{\par\hangindent=40pt \hspace*{30pt}}

% %\input{unf-sangbog.tocx}

% \renewcommand{\item}{\olditem}
% \renewcommand{\subitem}{\oldsubitem}
% \renewcommand{\subsubitem}{\oldsubsubitem}

%%%
% Songbook begins.
%%%

\twocolumn
%It's just one page, don't print page numbers etc.
\pagestyle{empty}
%Songs included
\input{songs/matmatik.tex}
\input{songs/taal_daj.tex}
\input{songs/linieskriverdriver.tex}
\input{songs/steve_hawking.tex}
\input{songs/ode_til_kode.tex}
\input{songs/se_min_kode.tex}
\input{songs/vaabenfysik_kort.tex}
%Maybe include:
%\input{songs/kvanter_i_maaneskin.tex}
%\input{songs/mest_matematiske_dyr.tex}

% \input{songs/vi_kan_ikke_li.tex}
% \input{songs/selektionssangen.tex}
% \input{songs/alfabetsangen.tex}
% \input{songs/sciencecamps.tex}
% \input{songs/hvad_maa_man.tex}


% \input{songs/lambda_kalkylen.tex}
% \input{songs/puslespil.tex}
% \input{songs/null.tex}
% \input{songs/fasebal.tex}

% \input{songs/chifitter.tex}

% \input{songs/kun_fysik.tex}



% \input{songs/kanoniske.tex}
% \input{songs/jeg_er_en_matematiker_fra_hcoe.tex}


% \input{songs/rekursiv_skovsang.tex}
% \input{songs/laerkerede.tex}


% \clearpage
% \font\myTinySF=cmss8    at  8pt
% \font\myHugeSF=cmssbx10 at 25pt
% % \newcommand{\CpyRtInfoFont}{\tiny\myTinySF}
% % \newcommand{\myTitleFont}{\Huge\myHugeSF}
% % \newcommand{\mySubTitleFont}{\large\sf}
% \renewcommand{\indexspace}{\medskip}

% {\parindent 8pt
%   {\myTitleFont Index}}\par
% \vskip 5pt
% \renewcommand{\SBThechapter}{Index}
% % {\parindent 20pt
% %   {\mySubTitleFont --- with first lines in italic ---}}
% % \vskip 20pt
% \renewcommand{\item}{\par\hangindent=40pt}
% \renewcommand{\subitem}{\par\hangindent=40pt \hspace*{20pt}}
% \renewcommand{\subsubitem}{\par\hangindent=40pt \hspace*{30pt}}

%\input{unf-sangbog.tdx}

\end{document}
\bye
%
%%%
% Document ends.
%%%


\end{document}
\bye
%
%%%
% Document ends.
%%%


% \renewcommand{\item}{\olditem}
% \renewcommand{\subitem}{\oldsubitem}
% \renewcommand{\subsubitem}{\oldsubsubitem}

%%%
% Songbook begins.
%%%

\twocolumn
%It's just one page, don't print page numbers etc.
\pagestyle{empty}
%Songs included
\input{songs/matmatik.tex}
\input{songs/taal_daj.tex}
\input{songs/linieskriverdriver.tex}
\input{songs/steve_hawking.tex}
\input{songs/ode_til_kode.tex}
\input{songs/se_min_kode.tex}
\input{songs/vaabenfysik_kort.tex}
%Maybe include:
%\input{songs/kvanter_i_maaneskin.tex}
%\input{songs/mest_matematiske_dyr.tex}

% \input{songs/vi_kan_ikke_li.tex}
% \input{songs/selektionssangen.tex}
% \input{songs/alfabetsangen.tex}
% \input{songs/sciencecamps.tex}
% \input{songs/hvad_maa_man.tex}


% \input{songs/lambda_kalkylen.tex}
% \input{songs/puslespil.tex}
% \input{songs/null.tex}
% \input{songs/fasebal.tex}

% \input{songs/chifitter.tex}

% \input{songs/kun_fysik.tex}



% \input{songs/kanoniske.tex}
% \input{songs/jeg_er_en_matematiker_fra_hcoe.tex}


% \input{songs/rekursiv_skovsang.tex}
% \input{songs/laerkerede.tex}


% \clearpage
% \font\myTinySF=cmss8    at  8pt
% \font\myHugeSF=cmssbx10 at 25pt
% % \newcommand{\CpyRtInfoFont}{\tiny\myTinySF}
% % \newcommand{\myTitleFont}{\Huge\myHugeSF}
% % \newcommand{\mySubTitleFont}{\large\sf}
% \renewcommand{\indexspace}{\medskip}

% {\parindent 8pt
%   {\myTitleFont Index}}\par
% \vskip 5pt
% \renewcommand{\SBThechapter}{Index}
% % {\parindent 20pt
% %   {\mySubTitleFont --- with first lines in italic ---}}
% % \vskip 20pt
% \renewcommand{\item}{\par\hangindent=40pt}
% \renewcommand{\subitem}{\par\hangindent=40pt \hspace*{20pt}}
% \renewcommand{\subsubitem}{\par\hangindent=40pt \hspace*{30pt}}

%%%%%%% rcsid = @(#)$Id: sample-sb.tex,v 1.23 2010-04-12 18:04:11 rathc Exp $
%%%%%%
%%
%%      ===============================
%%      Sample Songbook (sample-sb.tex)
%%      ===============================
%%
%%      Version 4.5, 30 April, 2010
%%
%%      Copyright 1992--2010 Christopher Rath <christopher@rath.ca>
%%
%%      This package is free software; you can redistribute it and/or
%%      modify it under the terms of version 2.1 of the GNU Lesser
%%	General Public License as published by the Free Software 
%%	Foundation.
%%
%%      This package is distributed in the hope that it will be
%%      useful, but WITHOUT ANY WARRANTY; without even the implied
%%      warranty of MERCHANTABILITY or FITNESS FOR A PARTICULAR
%%      PURPOSE.  See the GNU Lesser General Public License for more
%%      details.
%%
%%      This file contains a subset of the songbook we distribute
%%      at our church.  To the best of my knowledge, all of the lyrics
%%      contained herein are freely distributable.  This file has been
%%      provided as a sample of what can be produced by the chordbk,
%%      wordbk, and overhead LaTeX styles.
%%
%%      NEEDED:  The fancyhdr LaTeX style is required to properly
%%              format this file.  If you don't have that then comment
%%              out the commands in the preamble which deal with the
%%              fancyhdr style.
%%
%%%%%%
%%%%%%
%%
%%      1. Chord notation.  Within this songbook the following
%%         conventions have been adopted:
%%
%%              "Minor" is entered as "m";
%%                      e.g. Cm7 for C minor 7th.
%%              "Major" is entered as "M";
%%                      e.g. CM7 for C major 7th.
%%
%%%%%%
%%%%%%
%%      ============
%%      Bibliography
%%      ============
%%
%%      Exalt Him!: Exalt Him!  Compiled by Tom Fettke.  (c)1989
%%                      Word Music.
%%
%%      Hosanna! Music Books: Hosanna! Music Books #1--#6.
%%                      (c)1987--92 Integrity Music, Inc.
%%
%%      Worship Him II: Worship Him II.  Compiled by Jesse Peterson
%%                      and Bruce Ballinger.  (c)1989 Tempo Music
%%                      Publications.
%%
%%      Worship Songs Of The Vineyard: Worship Songs Of The Vineyard
%%                      --- Volume 2.  (c)1989 Vineyard Ministries
%%                      International.
%%
%%%%%%
%%%%%%

%%%%%%%%%%%%%%%%%%%%%%%%%%%%%%%%%%%%%%%%%%%%%%%%%%%%%%%%%%
%%%%%%%%%%%%%%%%%%%%%%%%%%%%%%%%%%%%%%%%%%%%%%%%%%%%%%%%%%
%%                                                      %%
%%           P R E A M B L E   B E G I N S              %%
%%                                                      %%
%%%%%%%%%%%%%%%%%%%%%%%%%%%%%%%%%%%%%%%%%%%%%%%%%%%%%%%%%%
%%%%%%%%%%%%%%%%%%%%%%%%%%%%%%%%%%%%%%%%%%%%%%%%%%%%%%%%%%

\documentclass[a5paper]{book}
\usepackage{latexsym,
            fancyhdr,
            titlesec,
            amsmath,
            amssymb,
            multicol,
            amsthm,
            stmaryrd,
            amsthm,
            color,
            needspace,
            stackengine,
            wasysym}
\usepackage[utf8]{inputenc}
\usepackage[T1]{fontenc}
% \usepackage[chordbk]{songbook}                  %% Words & Chords edition.
%%\usepackage[compactallsongs,chordbk]{songbook}    %% Words & Chords edition.
\usepackage[wordbk]{songbook}                 %% Words Only edition.
%%\usepackage[overhead]{songbook}               %% Overhead Transparency edition.
\usepackage{titletoc}
\usepackage{tket}  % Draws "TÅGEKAMMERET" correctly

%%%
% Revision Date and Release Date definitions.
%
%       \RelDate - The last time this songbook was released.  Set this
%                  date each time a new release/update of the songbook
%                  is generated.
%       \RevDate - The last time a particular song was revised in any
%                  way.  This command will be renewed inside every
%                  song.
%%%
\newcommand{\RelDate}{31~August,~2003}
\newcommand{\RevDate}{\today}

%%%
% C.C.L.I. license number definition; for copyright licensing info.
% One of these macros will be manually inserted into the {SBMel}
% parameter of the {song} environment.
%
%       \CCLInumber - The actual copyright license number.  Don't
%               insert this command in the {SBMel} parameter, use one
%               of the others.
%       \CCLIed - Indicates a song falls under our CCLI license.
%       \NotCCLIed - Indicates a song doesn't fall under our CCLI
%               license.  Public Domain songs fall into this category.
%       \PGranted - We have received specific permission from the
%               copyright holder to use this song.
%       \PPending - We are in the process of obtaining permission to
%               use this song.
%%%
\newcommand{\CCLInumber}{Your CCLI Number}
\newcommand{\CCLIed}{{\SBMelInfoFont (CCLI \CCLInumber)}}
\newcommand{\NotCCLIed}{\relax}
\newcommand{\PGranted}{\relax}
\newcommand{\PPending}{{\SBMelInfoFont (Permission Pending)}}

%%%
% Title page information.
%%%
%\title{UNF Computer Science Camp 2019 Sangbog}
%\author{}
%\date{Revideret:  \RevDate}

%%%
% Redefine fonts from SongBook style that I don't like.
%%%
\font\myTinySF=cmss8 at 8pt
\renewcommand{\SBMelInfoFont}{\tiny\myTinySF}

%%%
% Define fonts to use in the headers and footers of the songbook.
%%%
\newcommand{\LHeadFont}{\normalsize}            % = cmr12  at 12pt
\newcommand{\CHeadFont}{\normalsize\rm}         % = cmr12  at 12pt
\newcommand{\RHeadFont}{\normalsize}            % = cmr12  at 12pt
\newcommand{\LFootFont}{\scriptsize}            % = cmr8   at  8pt
\newcommand{\CFootFont}{\tiny\myTinySF}         % = cmss8  at  8pt
\newcommand{\RFootFont}{\scriptsize}            % = cmr8   at  8pt

\def\repeat{%
  \stackanchor{.}{.}%
  \rule[-\dp\strutbox]{.3pt}{\normalbaselineskip}%
  \kern0.5pt%
  \rule[-\dp\strutbox]{1pt}{\normalbaselineskip}%
  \kern1pt%
}
\def\frepeat{%
  \kern1pt%
  \rule[-\dp\strutbox]{1pt}{\normalbaselineskip}%
  \kern0.5pt%
  \rule[-\dp\strutbox]{.3pt}{\normalbaselineskip}%
  \stackanchor{.}{.}%
}
% \newcommand{\SBRepeat}[1]{#1\\#1}
\newcommand{\SBRepeat}[1]{\frepeat #1\repeat}
\setcounter{SBSongCnt}{-1}
\renewcommand{\SBWAndMTag}{Forfatter:}
\renewcommand{\SBUnknownTag}{Ukendt}
\renewcommand{\SBChorusTag}{Ref.}
\renewcommand{\SBOrgMel}{Originalmelodi}
\renewcommand{\SpaceAfterChorus}   {\vspace{0ex plus1ex minus 0.5ex}}
\renewcommand{\SpaceAfterOpGroup}  {\vspace{0ex plus1ex minus 0.5ex}}
\renewcommand{\SpaceAfterSBBracket}{\vspace{0ex plus1ex minus 0.5ex}}
\renewcommand{\SpaceAfterSection}  {\vspace{0ex plus1ex minus 0.5ex}}
\renewcommand{\SpaceAfterSong}     {\vspace{0ex plus1ex minus 0.5ex}}
\renewcommand{\SpaceAfterVerse}    {\vspace{0ex plus1ex minus 0.5ex}}

% Tell LaTeX that \medskip is a good place to make a page break
\let\oldmedskip\medskip
\renewcommand{\medskip}{\oldmedskip\pagebreak[2]}

%%%
% Turn on/off index-file generation.  Uncomment the \makeindex line to
% turn index generation on;  comment it out to turn index generation
% off.
%%%
%\makeTitleIndex         %% Title and First Line Index.
%\makeTitleContents      %% Table of Contents.
%\makeKeyIndex           %% Index of song by key.
% \makeArtistIndex	%% Index of song by artist.
% \newcommand{\SBThechapter}[0]{}
% \newcommand{\SBChapter}[1]{
%     \startcontents
%     \chapter*{#1} 
%     % %%%%%% rcsid = @(#)$Id: sample-sb.tex,v 1.23 2010-04-12 18:04:11 rathc Exp $
%%%%%%
%%
%%      ===============================
%%      Sample Songbook (sample-sb.tex)
%%      ===============================
%%
%%      Version 4.5, 30 April, 2010
%%
%%      Copyright 1992--2010 Christopher Rath <christopher@rath.ca>
%%
%%      This package is free software; you can redistribute it and/or
%%      modify it under the terms of version 2.1 of the GNU Lesser
%%	General Public License as published by the Free Software 
%%	Foundation.
%%
%%      This package is distributed in the hope that it will be
%%      useful, but WITHOUT ANY WARRANTY; without even the implied
%%      warranty of MERCHANTABILITY or FITNESS FOR A PARTICULAR
%%      PURPOSE.  See the GNU Lesser General Public License for more
%%      details.
%%
%%      This file contains a subset of the songbook we distribute
%%      at our church.  To the best of my knowledge, all of the lyrics
%%      contained herein are freely distributable.  This file has been
%%      provided as a sample of what can be produced by the chordbk,
%%      wordbk, and overhead LaTeX styles.
%%
%%      NEEDED:  The fancyhdr LaTeX style is required to properly
%%              format this file.  If you don't have that then comment
%%              out the commands in the preamble which deal with the
%%              fancyhdr style.
%%
%%%%%%
%%%%%%
%%
%%      1. Chord notation.  Within this songbook the following
%%         conventions have been adopted:
%%
%%              "Minor" is entered as "m";
%%                      e.g. Cm7 for C minor 7th.
%%              "Major" is entered as "M";
%%                      e.g. CM7 for C major 7th.
%%
%%%%%%
%%%%%%
%%      ============
%%      Bibliography
%%      ============
%%
%%      Exalt Him!: Exalt Him!  Compiled by Tom Fettke.  (c)1989
%%                      Word Music.
%%
%%      Hosanna! Music Books: Hosanna! Music Books #1--#6.
%%                      (c)1987--92 Integrity Music, Inc.
%%
%%      Worship Him II: Worship Him II.  Compiled by Jesse Peterson
%%                      and Bruce Ballinger.  (c)1989 Tempo Music
%%                      Publications.
%%
%%      Worship Songs Of The Vineyard: Worship Songs Of The Vineyard
%%                      --- Volume 2.  (c)1989 Vineyard Ministries
%%                      International.
%%
%%%%%%
%%%%%%

%%%%%%%%%%%%%%%%%%%%%%%%%%%%%%%%%%%%%%%%%%%%%%%%%%%%%%%%%%
%%%%%%%%%%%%%%%%%%%%%%%%%%%%%%%%%%%%%%%%%%%%%%%%%%%%%%%%%%
%%                                                      %%
%%           P R E A M B L E   B E G I N S              %%
%%                                                      %%
%%%%%%%%%%%%%%%%%%%%%%%%%%%%%%%%%%%%%%%%%%%%%%%%%%%%%%%%%%
%%%%%%%%%%%%%%%%%%%%%%%%%%%%%%%%%%%%%%%%%%%%%%%%%%%%%%%%%%

\documentclass[a5paper]{book}
\usepackage{latexsym,
            fancyhdr,
            titlesec,
            amsmath,
            amssymb,
            multicol,
            amsthm,
            stmaryrd,
            amsthm,
            color,
            needspace,
            stackengine,
            wasysym}
\usepackage[utf8]{inputenc}
\usepackage[T1]{fontenc}
% \usepackage[chordbk]{songbook}                  %% Words & Chords edition.
%%\usepackage[compactallsongs,chordbk]{songbook}    %% Words & Chords edition.
\usepackage[wordbk]{songbook}                 %% Words Only edition.
%%\usepackage[overhead]{songbook}               %% Overhead Transparency edition.
\usepackage{titletoc}
\usepackage{tket}  % Draws "TÅGEKAMMERET" correctly

%%%
% Revision Date and Release Date definitions.
%
%       \RelDate - The last time this songbook was released.  Set this
%                  date each time a new release/update of the songbook
%                  is generated.
%       \RevDate - The last time a particular song was revised in any
%                  way.  This command will be renewed inside every
%                  song.
%%%
\newcommand{\RelDate}{31~August,~2003}
\newcommand{\RevDate}{\today}

%%%
% C.C.L.I. license number definition; for copyright licensing info.
% One of these macros will be manually inserted into the {SBMel}
% parameter of the {song} environment.
%
%       \CCLInumber - The actual copyright license number.  Don't
%               insert this command in the {SBMel} parameter, use one
%               of the others.
%       \CCLIed - Indicates a song falls under our CCLI license.
%       \NotCCLIed - Indicates a song doesn't fall under our CCLI
%               license.  Public Domain songs fall into this category.
%       \PGranted - We have received specific permission from the
%               copyright holder to use this song.
%       \PPending - We are in the process of obtaining permission to
%               use this song.
%%%
\newcommand{\CCLInumber}{Your CCLI Number}
\newcommand{\CCLIed}{{\SBMelInfoFont (CCLI \CCLInumber)}}
\newcommand{\NotCCLIed}{\relax}
\newcommand{\PGranted}{\relax}
\newcommand{\PPending}{{\SBMelInfoFont (Permission Pending)}}

%%%
% Title page information.
%%%
%\title{UNF Computer Science Camp 2019 Sangbog}
%\author{}
%\date{Revideret:  \RevDate}

%%%
% Redefine fonts from SongBook style that I don't like.
%%%
\font\myTinySF=cmss8 at 8pt
\renewcommand{\SBMelInfoFont}{\tiny\myTinySF}

%%%
% Define fonts to use in the headers and footers of the songbook.
%%%
\newcommand{\LHeadFont}{\normalsize}            % = cmr12  at 12pt
\newcommand{\CHeadFont}{\normalsize\rm}         % = cmr12  at 12pt
\newcommand{\RHeadFont}{\normalsize}            % = cmr12  at 12pt
\newcommand{\LFootFont}{\scriptsize}            % = cmr8   at  8pt
\newcommand{\CFootFont}{\tiny\myTinySF}         % = cmss8  at  8pt
\newcommand{\RFootFont}{\scriptsize}            % = cmr8   at  8pt

\def\repeat{%
  \stackanchor{.}{.}%
  \rule[-\dp\strutbox]{.3pt}{\normalbaselineskip}%
  \kern0.5pt%
  \rule[-\dp\strutbox]{1pt}{\normalbaselineskip}%
  \kern1pt%
}
\def\frepeat{%
  \kern1pt%
  \rule[-\dp\strutbox]{1pt}{\normalbaselineskip}%
  \kern0.5pt%
  \rule[-\dp\strutbox]{.3pt}{\normalbaselineskip}%
  \stackanchor{.}{.}%
}
% \newcommand{\SBRepeat}[1]{#1\\#1}
\newcommand{\SBRepeat}[1]{\frepeat #1\repeat}
\setcounter{SBSongCnt}{-1}
\renewcommand{\SBWAndMTag}{Forfatter:}
\renewcommand{\SBUnknownTag}{Ukendt}
\renewcommand{\SBChorusTag}{Ref.}
\renewcommand{\SBOrgMel}{Originalmelodi}
\renewcommand{\SpaceAfterChorus}   {\vspace{0ex plus1ex minus 0.5ex}}
\renewcommand{\SpaceAfterOpGroup}  {\vspace{0ex plus1ex minus 0.5ex}}
\renewcommand{\SpaceAfterSBBracket}{\vspace{0ex plus1ex minus 0.5ex}}
\renewcommand{\SpaceAfterSection}  {\vspace{0ex plus1ex minus 0.5ex}}
\renewcommand{\SpaceAfterSong}     {\vspace{0ex plus1ex minus 0.5ex}}
\renewcommand{\SpaceAfterVerse}    {\vspace{0ex plus1ex minus 0.5ex}}

% Tell LaTeX that \medskip is a good place to make a page break
\let\oldmedskip\medskip
\renewcommand{\medskip}{\oldmedskip\pagebreak[2]}

%%%
% Turn on/off index-file generation.  Uncomment the \makeindex line to
% turn index generation on;  comment it out to turn index generation
% off.
%%%
%\makeTitleIndex         %% Title and First Line Index.
%\makeTitleContents      %% Table of Contents.
%\makeKeyIndex           %% Index of song by key.
% \makeArtistIndex	%% Index of song by artist.
% \newcommand{\SBThechapter}[0]{}
% \newcommand{\SBChapter}[1]{
%     \startcontents
%     \chapter*{#1} 
%     % \input{unf-sangbog.toc}
%       \begin{minipage}{.8\textwidth}
%         \printcontents{}{1}{}
%       \end{minipage}%
%     \renewcommand{\SBThechapter}{#1}
%     \clearpage
% }

% \titleformat{\chapter}
% [display]
% {}
% {%\vspace*{\fill}
%  % \titlerule[1pt]%
%  % \vspace{1pt}%
%  % \titlerule
%  % \vspace{1pc}%
%  \chaptertitlename}
% {}
% {\Huge}



%%%%%%%%%%%%%%%%%%%%%%%%%%%%%%%%%%%%%%%%%%%%%%%%%%%%%%%%%%
%%%%%%%%%%%%%%%%%%%%%%%%%%%%%%%%%%%%%%%%%%%%%%%%%%%%%%%%%%
%%                                                      %%
%%           D O C U M E N T   B E G I N S              %%
%%                                                      %%
%%%%%%%%%%%%%%%%%%%%%%%%%%%%%%%%%%%%%%%%%%%%%%%%%%%%%%%%%%
%%%%%%%%%%%%%%%%%%%%%%%%%%%%%%%%%%%%%%%%%%%%%%%%%%%%%%%%%%
\begin{document}

%%%
% Uncomment "\maketitle" statement to make a title page.
%%%
%\maketitle
% \begin{titlepage}
%   \centering
%   \vspace{5cm}
% 	\includegraphics[width=1\textwidth]{unf_logo.jpeg}\par\vspace{1cm}
% 	{\scshape\LARGE Sangbog \par}
% 	\vspace{1cm}
% 	{\scshape\Large UNF Computer Science Camp 2019\par}
	
% 	\vfill

% % Bottom of the page
% 	{\large \today\par}
% \end{titlepage}
% \mainmatter
% \ifWordBk
%   \twocolumn
% \fi


%%% Kolofon
%\thispagestyle{empty}
%Sammensat til UNF Computer Science Camp 2019 - csc.unf.dk\\
%Redaktør: Andreas Mosbæk Jensen m.fl. efter tidligere sangbog af Steffen Strunge Mathiesen\\
%Indhold opsat i \LaTeX. 
%Digital version og kildekode: github.com/steffen555/UNF-sangbog\\
%Revision 1 med stave fejl korrektioner
%\par\vspace*{\fill}
%Hvis du har forslag til sange, rettelser, ris og ros, eller hvis du kender en ukendt forfatter, så skriv til sangbog@unf.dk.

%%%
% Turn on and define fancy page heading/footing definition.
%%%
% \pagestyle{fancy}

% \ifChordBk
%   % It's a words & chords songbook...
%   \addtolength{\headwidth}{\marginparsep}
%   \addtolength{\headwidth}{\marginparwidth}
%   \renewcommand{\headrulewidth}{0.4pt}
%   \renewcommand{\footrulewidth}{0.4pt}
%   \fancyhead[LE,RO]{\LHeadFont\emph{\leftmark\/}\SBContinueMark}
%   \fancyhead[CE,CO]{\CHeadFont\thepage}
%   \fancyhead[RE,LO]{\RHeadFont \chaptermark}
% \else\ifOverhead
%   % It's an overhead...
%   \renewcommand{\footrulewidth}{0pt}
%   \renewcommand{\headrulewidth}{0pt}
%   \fancyhead[LE,RO]{}
%   \fancyhead[CE,CO]{}
%   \fancyhead[RE,LO]{}
% \else\ifWordBk
%   % It's a words only songbook...
%   \addtolength{\headwidth}{\marginparsep}
%   \addtolength{\headwidth}{\marginparwidth}
%   \renewcommand{\headrulewidth}{0.4pt}
%   \renewcommand{\footrulewidth}{0.4pt}
%   \fancyhead[LE,RO]{\LHeadFont Naturvidenskab revy sange}
%   \fancyhead[CE,CO]{\CHeadFont\thepage}
%   \fancyhead[RE,LO]{\RHeadFont \SBThechapter}
% \fi\fi\fi

% \fancyfoot[LE,RO]{\LFootFont Computer Science Camp 2019}
% \ifSongEject
%   \fancyfoot[CE,CO]{\CFootFont Last Revised:  \RevDate}
% \else
%   \fancyfoot[CE,CO]{\CFootFont}
% \fi
% \fancyfoot[RE,LO]{\RFootFont Synges på eget ansvar}

%%%
% Table of contents
%%%

% \clearpage
% \twocolumn
% \font\myTinySF=cmss8    at  8pt
% \font\myHugeSF=cmssbx10 at 25pt
% \newcommand{\CpyRtInfoFont}{\tiny\myTinySF}
% \newcommand{\myTitleFont}{\Huge\myHugeSF}
% \newcommand{\mySubTitleFont}{\large\sf}
% \renewcommand{\indexspace}{\medskip}

% % {\parindent 8pt
% %   {\myTitleFont Indhold}}\par
% % \vskip 5pt
% \renewcommand{\SBThechapter}{Indhold}
% % {\parindent 20pt
% %   {\mySubTitleFont --- with first lines in italic ---}}
% % \vskip 20pt
% \let\olditem\item
% \let\oldsubitem\subitem
% \let\oldsubsubitem\subsubitem
% \renewcommand{\item}{\par\hangindent=40pt}
% \renewcommand{\subitem}{\par\hangindent=40pt \hspace*{20pt}}
% \renewcommand{\subsubitem}{\par\hangindent=40pt \hspace*{30pt}}

% %\input{unf-sangbog.tocx}

% \renewcommand{\item}{\olditem}
% \renewcommand{\subitem}{\oldsubitem}
% \renewcommand{\subsubitem}{\oldsubsubitem}

%%%
% Songbook begins.
%%%

\twocolumn
%It's just one page, don't print page numbers etc.
\pagestyle{empty}
%Songs included
\input{songs/matmatik.tex}
\input{songs/taal_daj.tex}
\input{songs/linieskriverdriver.tex}
\input{songs/steve_hawking.tex}
\input{songs/ode_til_kode.tex}
\input{songs/se_min_kode.tex}
\input{songs/vaabenfysik_kort.tex}
%Maybe include:
%\input{songs/kvanter_i_maaneskin.tex}
%\input{songs/mest_matematiske_dyr.tex}

% \input{songs/vi_kan_ikke_li.tex}
% \input{songs/selektionssangen.tex}
% \input{songs/alfabetsangen.tex}
% \input{songs/sciencecamps.tex}
% \input{songs/hvad_maa_man.tex}


% \input{songs/lambda_kalkylen.tex}
% \input{songs/puslespil.tex}
% \input{songs/null.tex}
% \input{songs/fasebal.tex}

% \input{songs/chifitter.tex}

% \input{songs/kun_fysik.tex}



% \input{songs/kanoniske.tex}
% \input{songs/jeg_er_en_matematiker_fra_hcoe.tex}


% \input{songs/rekursiv_skovsang.tex}
% \input{songs/laerkerede.tex}


% \clearpage
% \font\myTinySF=cmss8    at  8pt
% \font\myHugeSF=cmssbx10 at 25pt
% % \newcommand{\CpyRtInfoFont}{\tiny\myTinySF}
% % \newcommand{\myTitleFont}{\Huge\myHugeSF}
% % \newcommand{\mySubTitleFont}{\large\sf}
% \renewcommand{\indexspace}{\medskip}

% {\parindent 8pt
%   {\myTitleFont Index}}\par
% \vskip 5pt
% \renewcommand{\SBThechapter}{Index}
% % {\parindent 20pt
% %   {\mySubTitleFont --- with first lines in italic ---}}
% % \vskip 20pt
% \renewcommand{\item}{\par\hangindent=40pt}
% \renewcommand{\subitem}{\par\hangindent=40pt \hspace*{20pt}}
% \renewcommand{\subsubitem}{\par\hangindent=40pt \hspace*{30pt}}

%\input{unf-sangbog.tdx}

\end{document}
\bye
%
%%%
% Document ends.
%%%

%       \begin{minipage}{.8\textwidth}
%         \printcontents{}{1}{}
%       \end{minipage}%
%     \renewcommand{\SBThechapter}{#1}
%     \clearpage
% }

% \titleformat{\chapter}
% [display]
% {}
% {%\vspace*{\fill}
%  % \titlerule[1pt]%
%  % \vspace{1pt}%
%  % \titlerule
%  % \vspace{1pc}%
%  \chaptertitlename}
% {}
% {\Huge}



%%%%%%%%%%%%%%%%%%%%%%%%%%%%%%%%%%%%%%%%%%%%%%%%%%%%%%%%%%
%%%%%%%%%%%%%%%%%%%%%%%%%%%%%%%%%%%%%%%%%%%%%%%%%%%%%%%%%%
%%                                                      %%
%%           D O C U M E N T   B E G I N S              %%
%%                                                      %%
%%%%%%%%%%%%%%%%%%%%%%%%%%%%%%%%%%%%%%%%%%%%%%%%%%%%%%%%%%
%%%%%%%%%%%%%%%%%%%%%%%%%%%%%%%%%%%%%%%%%%%%%%%%%%%%%%%%%%
\begin{document}

%%%
% Uncomment "\maketitle" statement to make a title page.
%%%
%\maketitle
% \begin{titlepage}
%   \centering
%   \vspace{5cm}
% 	\includegraphics[width=1\textwidth]{unf_logo.jpeg}\par\vspace{1cm}
% 	{\scshape\LARGE Sangbog \par}
% 	\vspace{1cm}
% 	{\scshape\Large UNF Computer Science Camp 2019\par}
	
% 	\vfill

% % Bottom of the page
% 	{\large \today\par}
% \end{titlepage}
% \mainmatter
% \ifWordBk
%   \twocolumn
% \fi


%%% Kolofon
%\thispagestyle{empty}
%Sammensat til UNF Computer Science Camp 2019 - csc.unf.dk\\
%Redaktør: Andreas Mosbæk Jensen m.fl. efter tidligere sangbog af Steffen Strunge Mathiesen\\
%Indhold opsat i \LaTeX. 
%Digital version og kildekode: github.com/steffen555/UNF-sangbog\\
%Revision 1 med stave fejl korrektioner
%\par\vspace*{\fill}
%Hvis du har forslag til sange, rettelser, ris og ros, eller hvis du kender en ukendt forfatter, så skriv til sangbog@unf.dk.

%%%
% Turn on and define fancy page heading/footing definition.
%%%
% \pagestyle{fancy}

% \ifChordBk
%   % It's a words & chords songbook...
%   \addtolength{\headwidth}{\marginparsep}
%   \addtolength{\headwidth}{\marginparwidth}
%   \renewcommand{\headrulewidth}{0.4pt}
%   \renewcommand{\footrulewidth}{0.4pt}
%   \fancyhead[LE,RO]{\LHeadFont\emph{\leftmark\/}\SBContinueMark}
%   \fancyhead[CE,CO]{\CHeadFont\thepage}
%   \fancyhead[RE,LO]{\RHeadFont \chaptermark}
% \else\ifOverhead
%   % It's an overhead...
%   \renewcommand{\footrulewidth}{0pt}
%   \renewcommand{\headrulewidth}{0pt}
%   \fancyhead[LE,RO]{}
%   \fancyhead[CE,CO]{}
%   \fancyhead[RE,LO]{}
% \else\ifWordBk
%   % It's a words only songbook...
%   \addtolength{\headwidth}{\marginparsep}
%   \addtolength{\headwidth}{\marginparwidth}
%   \renewcommand{\headrulewidth}{0.4pt}
%   \renewcommand{\footrulewidth}{0.4pt}
%   \fancyhead[LE,RO]{\LHeadFont Naturvidenskab revy sange}
%   \fancyhead[CE,CO]{\CHeadFont\thepage}
%   \fancyhead[RE,LO]{\RHeadFont \SBThechapter}
% \fi\fi\fi

% \fancyfoot[LE,RO]{\LFootFont Computer Science Camp 2019}
% \ifSongEject
%   \fancyfoot[CE,CO]{\CFootFont Last Revised:  \RevDate}
% \else
%   \fancyfoot[CE,CO]{\CFootFont}
% \fi
% \fancyfoot[RE,LO]{\RFootFont Synges på eget ansvar}

%%%
% Table of contents
%%%

% \clearpage
% \twocolumn
% \font\myTinySF=cmss8    at  8pt
% \font\myHugeSF=cmssbx10 at 25pt
% \newcommand{\CpyRtInfoFont}{\tiny\myTinySF}
% \newcommand{\myTitleFont}{\Huge\myHugeSF}
% \newcommand{\mySubTitleFont}{\large\sf}
% \renewcommand{\indexspace}{\medskip}

% % {\parindent 8pt
% %   {\myTitleFont Indhold}}\par
% % \vskip 5pt
% \renewcommand{\SBThechapter}{Indhold}
% % {\parindent 20pt
% %   {\mySubTitleFont --- with first lines in italic ---}}
% % \vskip 20pt
% \let\olditem\item
% \let\oldsubitem\subitem
% \let\oldsubsubitem\subsubitem
% \renewcommand{\item}{\par\hangindent=40pt}
% \renewcommand{\subitem}{\par\hangindent=40pt \hspace*{20pt}}
% \renewcommand{\subsubitem}{\par\hangindent=40pt \hspace*{30pt}}

% %%%%%%% rcsid = @(#)$Id: sample-sb.tex,v 1.23 2010-04-12 18:04:11 rathc Exp $
%%%%%%
%%
%%      ===============================
%%      Sample Songbook (sample-sb.tex)
%%      ===============================
%%
%%      Version 4.5, 30 April, 2010
%%
%%      Copyright 1992--2010 Christopher Rath <christopher@rath.ca>
%%
%%      This package is free software; you can redistribute it and/or
%%      modify it under the terms of version 2.1 of the GNU Lesser
%%	General Public License as published by the Free Software 
%%	Foundation.
%%
%%      This package is distributed in the hope that it will be
%%      useful, but WITHOUT ANY WARRANTY; without even the implied
%%      warranty of MERCHANTABILITY or FITNESS FOR A PARTICULAR
%%      PURPOSE.  See the GNU Lesser General Public License for more
%%      details.
%%
%%      This file contains a subset of the songbook we distribute
%%      at our church.  To the best of my knowledge, all of the lyrics
%%      contained herein are freely distributable.  This file has been
%%      provided as a sample of what can be produced by the chordbk,
%%      wordbk, and overhead LaTeX styles.
%%
%%      NEEDED:  The fancyhdr LaTeX style is required to properly
%%              format this file.  If you don't have that then comment
%%              out the commands in the preamble which deal with the
%%              fancyhdr style.
%%
%%%%%%
%%%%%%
%%
%%      1. Chord notation.  Within this songbook the following
%%         conventions have been adopted:
%%
%%              "Minor" is entered as "m";
%%                      e.g. Cm7 for C minor 7th.
%%              "Major" is entered as "M";
%%                      e.g. CM7 for C major 7th.
%%
%%%%%%
%%%%%%
%%      ============
%%      Bibliography
%%      ============
%%
%%      Exalt Him!: Exalt Him!  Compiled by Tom Fettke.  (c)1989
%%                      Word Music.
%%
%%      Hosanna! Music Books: Hosanna! Music Books #1--#6.
%%                      (c)1987--92 Integrity Music, Inc.
%%
%%      Worship Him II: Worship Him II.  Compiled by Jesse Peterson
%%                      and Bruce Ballinger.  (c)1989 Tempo Music
%%                      Publications.
%%
%%      Worship Songs Of The Vineyard: Worship Songs Of The Vineyard
%%                      --- Volume 2.  (c)1989 Vineyard Ministries
%%                      International.
%%
%%%%%%
%%%%%%

%%%%%%%%%%%%%%%%%%%%%%%%%%%%%%%%%%%%%%%%%%%%%%%%%%%%%%%%%%
%%%%%%%%%%%%%%%%%%%%%%%%%%%%%%%%%%%%%%%%%%%%%%%%%%%%%%%%%%
%%                                                      %%
%%           P R E A M B L E   B E G I N S              %%
%%                                                      %%
%%%%%%%%%%%%%%%%%%%%%%%%%%%%%%%%%%%%%%%%%%%%%%%%%%%%%%%%%%
%%%%%%%%%%%%%%%%%%%%%%%%%%%%%%%%%%%%%%%%%%%%%%%%%%%%%%%%%%

\documentclass[a5paper]{book}
\usepackage{latexsym,
            fancyhdr,
            titlesec,
            amsmath,
            amssymb,
            multicol,
            amsthm,
            stmaryrd,
            amsthm,
            color,
            needspace,
            stackengine,
            wasysym}
\usepackage[utf8]{inputenc}
\usepackage[T1]{fontenc}
% \usepackage[chordbk]{songbook}                  %% Words & Chords edition.
%%\usepackage[compactallsongs,chordbk]{songbook}    %% Words & Chords edition.
\usepackage[wordbk]{songbook}                 %% Words Only edition.
%%\usepackage[overhead]{songbook}               %% Overhead Transparency edition.
\usepackage{titletoc}
\usepackage{tket}  % Draws "TÅGEKAMMERET" correctly

%%%
% Revision Date and Release Date definitions.
%
%       \RelDate - The last time this songbook was released.  Set this
%                  date each time a new release/update of the songbook
%                  is generated.
%       \RevDate - The last time a particular song was revised in any
%                  way.  This command will be renewed inside every
%                  song.
%%%
\newcommand{\RelDate}{31~August,~2003}
\newcommand{\RevDate}{\today}

%%%
% C.C.L.I. license number definition; for copyright licensing info.
% One of these macros will be manually inserted into the {SBMel}
% parameter of the {song} environment.
%
%       \CCLInumber - The actual copyright license number.  Don't
%               insert this command in the {SBMel} parameter, use one
%               of the others.
%       \CCLIed - Indicates a song falls under our CCLI license.
%       \NotCCLIed - Indicates a song doesn't fall under our CCLI
%               license.  Public Domain songs fall into this category.
%       \PGranted - We have received specific permission from the
%               copyright holder to use this song.
%       \PPending - We are in the process of obtaining permission to
%               use this song.
%%%
\newcommand{\CCLInumber}{Your CCLI Number}
\newcommand{\CCLIed}{{\SBMelInfoFont (CCLI \CCLInumber)}}
\newcommand{\NotCCLIed}{\relax}
\newcommand{\PGranted}{\relax}
\newcommand{\PPending}{{\SBMelInfoFont (Permission Pending)}}

%%%
% Title page information.
%%%
%\title{UNF Computer Science Camp 2019 Sangbog}
%\author{}
%\date{Revideret:  \RevDate}

%%%
% Redefine fonts from SongBook style that I don't like.
%%%
\font\myTinySF=cmss8 at 8pt
\renewcommand{\SBMelInfoFont}{\tiny\myTinySF}

%%%
% Define fonts to use in the headers and footers of the songbook.
%%%
\newcommand{\LHeadFont}{\normalsize}            % = cmr12  at 12pt
\newcommand{\CHeadFont}{\normalsize\rm}         % = cmr12  at 12pt
\newcommand{\RHeadFont}{\normalsize}            % = cmr12  at 12pt
\newcommand{\LFootFont}{\scriptsize}            % = cmr8   at  8pt
\newcommand{\CFootFont}{\tiny\myTinySF}         % = cmss8  at  8pt
\newcommand{\RFootFont}{\scriptsize}            % = cmr8   at  8pt

\def\repeat{%
  \stackanchor{.}{.}%
  \rule[-\dp\strutbox]{.3pt}{\normalbaselineskip}%
  \kern0.5pt%
  \rule[-\dp\strutbox]{1pt}{\normalbaselineskip}%
  \kern1pt%
}
\def\frepeat{%
  \kern1pt%
  \rule[-\dp\strutbox]{1pt}{\normalbaselineskip}%
  \kern0.5pt%
  \rule[-\dp\strutbox]{.3pt}{\normalbaselineskip}%
  \stackanchor{.}{.}%
}
% \newcommand{\SBRepeat}[1]{#1\\#1}
\newcommand{\SBRepeat}[1]{\frepeat #1\repeat}
\setcounter{SBSongCnt}{-1}
\renewcommand{\SBWAndMTag}{Forfatter:}
\renewcommand{\SBUnknownTag}{Ukendt}
\renewcommand{\SBChorusTag}{Ref.}
\renewcommand{\SBOrgMel}{Originalmelodi}
\renewcommand{\SpaceAfterChorus}   {\vspace{0ex plus1ex minus 0.5ex}}
\renewcommand{\SpaceAfterOpGroup}  {\vspace{0ex plus1ex minus 0.5ex}}
\renewcommand{\SpaceAfterSBBracket}{\vspace{0ex plus1ex minus 0.5ex}}
\renewcommand{\SpaceAfterSection}  {\vspace{0ex plus1ex minus 0.5ex}}
\renewcommand{\SpaceAfterSong}     {\vspace{0ex plus1ex minus 0.5ex}}
\renewcommand{\SpaceAfterVerse}    {\vspace{0ex plus1ex minus 0.5ex}}

% Tell LaTeX that \medskip is a good place to make a page break
\let\oldmedskip\medskip
\renewcommand{\medskip}{\oldmedskip\pagebreak[2]}

%%%
% Turn on/off index-file generation.  Uncomment the \makeindex line to
% turn index generation on;  comment it out to turn index generation
% off.
%%%
%\makeTitleIndex         %% Title and First Line Index.
%\makeTitleContents      %% Table of Contents.
%\makeKeyIndex           %% Index of song by key.
% \makeArtistIndex	%% Index of song by artist.
% \newcommand{\SBThechapter}[0]{}
% \newcommand{\SBChapter}[1]{
%     \startcontents
%     \chapter*{#1} 
%     % \input{unf-sangbog.toc}
%       \begin{minipage}{.8\textwidth}
%         \printcontents{}{1}{}
%       \end{minipage}%
%     \renewcommand{\SBThechapter}{#1}
%     \clearpage
% }

% \titleformat{\chapter}
% [display]
% {}
% {%\vspace*{\fill}
%  % \titlerule[1pt]%
%  % \vspace{1pt}%
%  % \titlerule
%  % \vspace{1pc}%
%  \chaptertitlename}
% {}
% {\Huge}



%%%%%%%%%%%%%%%%%%%%%%%%%%%%%%%%%%%%%%%%%%%%%%%%%%%%%%%%%%
%%%%%%%%%%%%%%%%%%%%%%%%%%%%%%%%%%%%%%%%%%%%%%%%%%%%%%%%%%
%%                                                      %%
%%           D O C U M E N T   B E G I N S              %%
%%                                                      %%
%%%%%%%%%%%%%%%%%%%%%%%%%%%%%%%%%%%%%%%%%%%%%%%%%%%%%%%%%%
%%%%%%%%%%%%%%%%%%%%%%%%%%%%%%%%%%%%%%%%%%%%%%%%%%%%%%%%%%
\begin{document}

%%%
% Uncomment "\maketitle" statement to make a title page.
%%%
%\maketitle
% \begin{titlepage}
%   \centering
%   \vspace{5cm}
% 	\includegraphics[width=1\textwidth]{unf_logo.jpeg}\par\vspace{1cm}
% 	{\scshape\LARGE Sangbog \par}
% 	\vspace{1cm}
% 	{\scshape\Large UNF Computer Science Camp 2019\par}
	
% 	\vfill

% % Bottom of the page
% 	{\large \today\par}
% \end{titlepage}
% \mainmatter
% \ifWordBk
%   \twocolumn
% \fi


%%% Kolofon
%\thispagestyle{empty}
%Sammensat til UNF Computer Science Camp 2019 - csc.unf.dk\\
%Redaktør: Andreas Mosbæk Jensen m.fl. efter tidligere sangbog af Steffen Strunge Mathiesen\\
%Indhold opsat i \LaTeX. 
%Digital version og kildekode: github.com/steffen555/UNF-sangbog\\
%Revision 1 med stave fejl korrektioner
%\par\vspace*{\fill}
%Hvis du har forslag til sange, rettelser, ris og ros, eller hvis du kender en ukendt forfatter, så skriv til sangbog@unf.dk.

%%%
% Turn on and define fancy page heading/footing definition.
%%%
% \pagestyle{fancy}

% \ifChordBk
%   % It's a words & chords songbook...
%   \addtolength{\headwidth}{\marginparsep}
%   \addtolength{\headwidth}{\marginparwidth}
%   \renewcommand{\headrulewidth}{0.4pt}
%   \renewcommand{\footrulewidth}{0.4pt}
%   \fancyhead[LE,RO]{\LHeadFont\emph{\leftmark\/}\SBContinueMark}
%   \fancyhead[CE,CO]{\CHeadFont\thepage}
%   \fancyhead[RE,LO]{\RHeadFont \chaptermark}
% \else\ifOverhead
%   % It's an overhead...
%   \renewcommand{\footrulewidth}{0pt}
%   \renewcommand{\headrulewidth}{0pt}
%   \fancyhead[LE,RO]{}
%   \fancyhead[CE,CO]{}
%   \fancyhead[RE,LO]{}
% \else\ifWordBk
%   % It's a words only songbook...
%   \addtolength{\headwidth}{\marginparsep}
%   \addtolength{\headwidth}{\marginparwidth}
%   \renewcommand{\headrulewidth}{0.4pt}
%   \renewcommand{\footrulewidth}{0.4pt}
%   \fancyhead[LE,RO]{\LHeadFont Naturvidenskab revy sange}
%   \fancyhead[CE,CO]{\CHeadFont\thepage}
%   \fancyhead[RE,LO]{\RHeadFont \SBThechapter}
% \fi\fi\fi

% \fancyfoot[LE,RO]{\LFootFont Computer Science Camp 2019}
% \ifSongEject
%   \fancyfoot[CE,CO]{\CFootFont Last Revised:  \RevDate}
% \else
%   \fancyfoot[CE,CO]{\CFootFont}
% \fi
% \fancyfoot[RE,LO]{\RFootFont Synges på eget ansvar}

%%%
% Table of contents
%%%

% \clearpage
% \twocolumn
% \font\myTinySF=cmss8    at  8pt
% \font\myHugeSF=cmssbx10 at 25pt
% \newcommand{\CpyRtInfoFont}{\tiny\myTinySF}
% \newcommand{\myTitleFont}{\Huge\myHugeSF}
% \newcommand{\mySubTitleFont}{\large\sf}
% \renewcommand{\indexspace}{\medskip}

% % {\parindent 8pt
% %   {\myTitleFont Indhold}}\par
% % \vskip 5pt
% \renewcommand{\SBThechapter}{Indhold}
% % {\parindent 20pt
% %   {\mySubTitleFont --- with first lines in italic ---}}
% % \vskip 20pt
% \let\olditem\item
% \let\oldsubitem\subitem
% \let\oldsubsubitem\subsubitem
% \renewcommand{\item}{\par\hangindent=40pt}
% \renewcommand{\subitem}{\par\hangindent=40pt \hspace*{20pt}}
% \renewcommand{\subsubitem}{\par\hangindent=40pt \hspace*{30pt}}

% %\input{unf-sangbog.tocx}

% \renewcommand{\item}{\olditem}
% \renewcommand{\subitem}{\oldsubitem}
% \renewcommand{\subsubitem}{\oldsubsubitem}

%%%
% Songbook begins.
%%%

\twocolumn
%It's just one page, don't print page numbers etc.
\pagestyle{empty}
%Songs included
\input{songs/matmatik.tex}
\input{songs/taal_daj.tex}
\input{songs/linieskriverdriver.tex}
\input{songs/steve_hawking.tex}
\input{songs/ode_til_kode.tex}
\input{songs/se_min_kode.tex}
\input{songs/vaabenfysik_kort.tex}
%Maybe include:
%\input{songs/kvanter_i_maaneskin.tex}
%\input{songs/mest_matematiske_dyr.tex}

% \input{songs/vi_kan_ikke_li.tex}
% \input{songs/selektionssangen.tex}
% \input{songs/alfabetsangen.tex}
% \input{songs/sciencecamps.tex}
% \input{songs/hvad_maa_man.tex}


% \input{songs/lambda_kalkylen.tex}
% \input{songs/puslespil.tex}
% \input{songs/null.tex}
% \input{songs/fasebal.tex}

% \input{songs/chifitter.tex}

% \input{songs/kun_fysik.tex}



% \input{songs/kanoniske.tex}
% \input{songs/jeg_er_en_matematiker_fra_hcoe.tex}


% \input{songs/rekursiv_skovsang.tex}
% \input{songs/laerkerede.tex}


% \clearpage
% \font\myTinySF=cmss8    at  8pt
% \font\myHugeSF=cmssbx10 at 25pt
% % \newcommand{\CpyRtInfoFont}{\tiny\myTinySF}
% % \newcommand{\myTitleFont}{\Huge\myHugeSF}
% % \newcommand{\mySubTitleFont}{\large\sf}
% \renewcommand{\indexspace}{\medskip}

% {\parindent 8pt
%   {\myTitleFont Index}}\par
% \vskip 5pt
% \renewcommand{\SBThechapter}{Index}
% % {\parindent 20pt
% %   {\mySubTitleFont --- with first lines in italic ---}}
% % \vskip 20pt
% \renewcommand{\item}{\par\hangindent=40pt}
% \renewcommand{\subitem}{\par\hangindent=40pt \hspace*{20pt}}
% \renewcommand{\subsubitem}{\par\hangindent=40pt \hspace*{30pt}}

%\input{unf-sangbog.tdx}

\end{document}
\bye
%
%%%
% Document ends.
%%%


% \renewcommand{\item}{\olditem}
% \renewcommand{\subitem}{\oldsubitem}
% \renewcommand{\subsubitem}{\oldsubsubitem}

%%%
% Songbook begins.
%%%

\twocolumn
%It's just one page, don't print page numbers etc.
\pagestyle{empty}
%Songs included
\input{songs/matmatik.tex}
\input{songs/taal_daj.tex}
\input{songs/linieskriverdriver.tex}
\input{songs/steve_hawking.tex}
\input{songs/ode_til_kode.tex}
\input{songs/se_min_kode.tex}
\input{songs/vaabenfysik_kort.tex}
%Maybe include:
%\input{songs/kvanter_i_maaneskin.tex}
%\input{songs/mest_matematiske_dyr.tex}

% \input{songs/vi_kan_ikke_li.tex}
% \input{songs/selektionssangen.tex}
% \input{songs/alfabetsangen.tex}
% \input{songs/sciencecamps.tex}
% \input{songs/hvad_maa_man.tex}


% \input{songs/lambda_kalkylen.tex}
% \input{songs/puslespil.tex}
% \input{songs/null.tex}
% \input{songs/fasebal.tex}

% \input{songs/chifitter.tex}

% \input{songs/kun_fysik.tex}



% \input{songs/kanoniske.tex}
% \input{songs/jeg_er_en_matematiker_fra_hcoe.tex}


% \input{songs/rekursiv_skovsang.tex}
% \input{songs/laerkerede.tex}


% \clearpage
% \font\myTinySF=cmss8    at  8pt
% \font\myHugeSF=cmssbx10 at 25pt
% % \newcommand{\CpyRtInfoFont}{\tiny\myTinySF}
% % \newcommand{\myTitleFont}{\Huge\myHugeSF}
% % \newcommand{\mySubTitleFont}{\large\sf}
% \renewcommand{\indexspace}{\medskip}

% {\parindent 8pt
%   {\myTitleFont Index}}\par
% \vskip 5pt
% \renewcommand{\SBThechapter}{Index}
% % {\parindent 20pt
% %   {\mySubTitleFont --- with first lines in italic ---}}
% % \vskip 20pt
% \renewcommand{\item}{\par\hangindent=40pt}
% \renewcommand{\subitem}{\par\hangindent=40pt \hspace*{20pt}}
% \renewcommand{\subsubitem}{\par\hangindent=40pt \hspace*{30pt}}

%%%%%%% rcsid = @(#)$Id: sample-sb.tex,v 1.23 2010-04-12 18:04:11 rathc Exp $
%%%%%%
%%
%%      ===============================
%%      Sample Songbook (sample-sb.tex)
%%      ===============================
%%
%%      Version 4.5, 30 April, 2010
%%
%%      Copyright 1992--2010 Christopher Rath <christopher@rath.ca>
%%
%%      This package is free software; you can redistribute it and/or
%%      modify it under the terms of version 2.1 of the GNU Lesser
%%	General Public License as published by the Free Software 
%%	Foundation.
%%
%%      This package is distributed in the hope that it will be
%%      useful, but WITHOUT ANY WARRANTY; without even the implied
%%      warranty of MERCHANTABILITY or FITNESS FOR A PARTICULAR
%%      PURPOSE.  See the GNU Lesser General Public License for more
%%      details.
%%
%%      This file contains a subset of the songbook we distribute
%%      at our church.  To the best of my knowledge, all of the lyrics
%%      contained herein are freely distributable.  This file has been
%%      provided as a sample of what can be produced by the chordbk,
%%      wordbk, and overhead LaTeX styles.
%%
%%      NEEDED:  The fancyhdr LaTeX style is required to properly
%%              format this file.  If you don't have that then comment
%%              out the commands in the preamble which deal with the
%%              fancyhdr style.
%%
%%%%%%
%%%%%%
%%
%%      1. Chord notation.  Within this songbook the following
%%         conventions have been adopted:
%%
%%              "Minor" is entered as "m";
%%                      e.g. Cm7 for C minor 7th.
%%              "Major" is entered as "M";
%%                      e.g. CM7 for C major 7th.
%%
%%%%%%
%%%%%%
%%      ============
%%      Bibliography
%%      ============
%%
%%      Exalt Him!: Exalt Him!  Compiled by Tom Fettke.  (c)1989
%%                      Word Music.
%%
%%      Hosanna! Music Books: Hosanna! Music Books #1--#6.
%%                      (c)1987--92 Integrity Music, Inc.
%%
%%      Worship Him II: Worship Him II.  Compiled by Jesse Peterson
%%                      and Bruce Ballinger.  (c)1989 Tempo Music
%%                      Publications.
%%
%%      Worship Songs Of The Vineyard: Worship Songs Of The Vineyard
%%                      --- Volume 2.  (c)1989 Vineyard Ministries
%%                      International.
%%
%%%%%%
%%%%%%

%%%%%%%%%%%%%%%%%%%%%%%%%%%%%%%%%%%%%%%%%%%%%%%%%%%%%%%%%%
%%%%%%%%%%%%%%%%%%%%%%%%%%%%%%%%%%%%%%%%%%%%%%%%%%%%%%%%%%
%%                                                      %%
%%           P R E A M B L E   B E G I N S              %%
%%                                                      %%
%%%%%%%%%%%%%%%%%%%%%%%%%%%%%%%%%%%%%%%%%%%%%%%%%%%%%%%%%%
%%%%%%%%%%%%%%%%%%%%%%%%%%%%%%%%%%%%%%%%%%%%%%%%%%%%%%%%%%

\documentclass[a5paper]{book}
\usepackage{latexsym,
            fancyhdr,
            titlesec,
            amsmath,
            amssymb,
            multicol,
            amsthm,
            stmaryrd,
            amsthm,
            color,
            needspace,
            stackengine,
            wasysym}
\usepackage[utf8]{inputenc}
\usepackage[T1]{fontenc}
% \usepackage[chordbk]{songbook}                  %% Words & Chords edition.
%%\usepackage[compactallsongs,chordbk]{songbook}    %% Words & Chords edition.
\usepackage[wordbk]{songbook}                 %% Words Only edition.
%%\usepackage[overhead]{songbook}               %% Overhead Transparency edition.
\usepackage{titletoc}
\usepackage{tket}  % Draws "TÅGEKAMMERET" correctly

%%%
% Revision Date and Release Date definitions.
%
%       \RelDate - The last time this songbook was released.  Set this
%                  date each time a new release/update of the songbook
%                  is generated.
%       \RevDate - The last time a particular song was revised in any
%                  way.  This command will be renewed inside every
%                  song.
%%%
\newcommand{\RelDate}{31~August,~2003}
\newcommand{\RevDate}{\today}

%%%
% C.C.L.I. license number definition; for copyright licensing info.
% One of these macros will be manually inserted into the {SBMel}
% parameter of the {song} environment.
%
%       \CCLInumber - The actual copyright license number.  Don't
%               insert this command in the {SBMel} parameter, use one
%               of the others.
%       \CCLIed - Indicates a song falls under our CCLI license.
%       \NotCCLIed - Indicates a song doesn't fall under our CCLI
%               license.  Public Domain songs fall into this category.
%       \PGranted - We have received specific permission from the
%               copyright holder to use this song.
%       \PPending - We are in the process of obtaining permission to
%               use this song.
%%%
\newcommand{\CCLInumber}{Your CCLI Number}
\newcommand{\CCLIed}{{\SBMelInfoFont (CCLI \CCLInumber)}}
\newcommand{\NotCCLIed}{\relax}
\newcommand{\PGranted}{\relax}
\newcommand{\PPending}{{\SBMelInfoFont (Permission Pending)}}

%%%
% Title page information.
%%%
%\title{UNF Computer Science Camp 2019 Sangbog}
%\author{}
%\date{Revideret:  \RevDate}

%%%
% Redefine fonts from SongBook style that I don't like.
%%%
\font\myTinySF=cmss8 at 8pt
\renewcommand{\SBMelInfoFont}{\tiny\myTinySF}

%%%
% Define fonts to use in the headers and footers of the songbook.
%%%
\newcommand{\LHeadFont}{\normalsize}            % = cmr12  at 12pt
\newcommand{\CHeadFont}{\normalsize\rm}         % = cmr12  at 12pt
\newcommand{\RHeadFont}{\normalsize}            % = cmr12  at 12pt
\newcommand{\LFootFont}{\scriptsize}            % = cmr8   at  8pt
\newcommand{\CFootFont}{\tiny\myTinySF}         % = cmss8  at  8pt
\newcommand{\RFootFont}{\scriptsize}            % = cmr8   at  8pt

\def\repeat{%
  \stackanchor{.}{.}%
  \rule[-\dp\strutbox]{.3pt}{\normalbaselineskip}%
  \kern0.5pt%
  \rule[-\dp\strutbox]{1pt}{\normalbaselineskip}%
  \kern1pt%
}
\def\frepeat{%
  \kern1pt%
  \rule[-\dp\strutbox]{1pt}{\normalbaselineskip}%
  \kern0.5pt%
  \rule[-\dp\strutbox]{.3pt}{\normalbaselineskip}%
  \stackanchor{.}{.}%
}
% \newcommand{\SBRepeat}[1]{#1\\#1}
\newcommand{\SBRepeat}[1]{\frepeat #1\repeat}
\setcounter{SBSongCnt}{-1}
\renewcommand{\SBWAndMTag}{Forfatter:}
\renewcommand{\SBUnknownTag}{Ukendt}
\renewcommand{\SBChorusTag}{Ref.}
\renewcommand{\SBOrgMel}{Originalmelodi}
\renewcommand{\SpaceAfterChorus}   {\vspace{0ex plus1ex minus 0.5ex}}
\renewcommand{\SpaceAfterOpGroup}  {\vspace{0ex plus1ex minus 0.5ex}}
\renewcommand{\SpaceAfterSBBracket}{\vspace{0ex plus1ex minus 0.5ex}}
\renewcommand{\SpaceAfterSection}  {\vspace{0ex plus1ex minus 0.5ex}}
\renewcommand{\SpaceAfterSong}     {\vspace{0ex plus1ex minus 0.5ex}}
\renewcommand{\SpaceAfterVerse}    {\vspace{0ex plus1ex minus 0.5ex}}

% Tell LaTeX that \medskip is a good place to make a page break
\let\oldmedskip\medskip
\renewcommand{\medskip}{\oldmedskip\pagebreak[2]}

%%%
% Turn on/off index-file generation.  Uncomment the \makeindex line to
% turn index generation on;  comment it out to turn index generation
% off.
%%%
%\makeTitleIndex         %% Title and First Line Index.
%\makeTitleContents      %% Table of Contents.
%\makeKeyIndex           %% Index of song by key.
% \makeArtistIndex	%% Index of song by artist.
% \newcommand{\SBThechapter}[0]{}
% \newcommand{\SBChapter}[1]{
%     \startcontents
%     \chapter*{#1} 
%     % \input{unf-sangbog.toc}
%       \begin{minipage}{.8\textwidth}
%         \printcontents{}{1}{}
%       \end{minipage}%
%     \renewcommand{\SBThechapter}{#1}
%     \clearpage
% }

% \titleformat{\chapter}
% [display]
% {}
% {%\vspace*{\fill}
%  % \titlerule[1pt]%
%  % \vspace{1pt}%
%  % \titlerule
%  % \vspace{1pc}%
%  \chaptertitlename}
% {}
% {\Huge}



%%%%%%%%%%%%%%%%%%%%%%%%%%%%%%%%%%%%%%%%%%%%%%%%%%%%%%%%%%
%%%%%%%%%%%%%%%%%%%%%%%%%%%%%%%%%%%%%%%%%%%%%%%%%%%%%%%%%%
%%                                                      %%
%%           D O C U M E N T   B E G I N S              %%
%%                                                      %%
%%%%%%%%%%%%%%%%%%%%%%%%%%%%%%%%%%%%%%%%%%%%%%%%%%%%%%%%%%
%%%%%%%%%%%%%%%%%%%%%%%%%%%%%%%%%%%%%%%%%%%%%%%%%%%%%%%%%%
\begin{document}

%%%
% Uncomment "\maketitle" statement to make a title page.
%%%
%\maketitle
% \begin{titlepage}
%   \centering
%   \vspace{5cm}
% 	\includegraphics[width=1\textwidth]{unf_logo.jpeg}\par\vspace{1cm}
% 	{\scshape\LARGE Sangbog \par}
% 	\vspace{1cm}
% 	{\scshape\Large UNF Computer Science Camp 2019\par}
	
% 	\vfill

% % Bottom of the page
% 	{\large \today\par}
% \end{titlepage}
% \mainmatter
% \ifWordBk
%   \twocolumn
% \fi


%%% Kolofon
%\thispagestyle{empty}
%Sammensat til UNF Computer Science Camp 2019 - csc.unf.dk\\
%Redaktør: Andreas Mosbæk Jensen m.fl. efter tidligere sangbog af Steffen Strunge Mathiesen\\
%Indhold opsat i \LaTeX. 
%Digital version og kildekode: github.com/steffen555/UNF-sangbog\\
%Revision 1 med stave fejl korrektioner
%\par\vspace*{\fill}
%Hvis du har forslag til sange, rettelser, ris og ros, eller hvis du kender en ukendt forfatter, så skriv til sangbog@unf.dk.

%%%
% Turn on and define fancy page heading/footing definition.
%%%
% \pagestyle{fancy}

% \ifChordBk
%   % It's a words & chords songbook...
%   \addtolength{\headwidth}{\marginparsep}
%   \addtolength{\headwidth}{\marginparwidth}
%   \renewcommand{\headrulewidth}{0.4pt}
%   \renewcommand{\footrulewidth}{0.4pt}
%   \fancyhead[LE,RO]{\LHeadFont\emph{\leftmark\/}\SBContinueMark}
%   \fancyhead[CE,CO]{\CHeadFont\thepage}
%   \fancyhead[RE,LO]{\RHeadFont \chaptermark}
% \else\ifOverhead
%   % It's an overhead...
%   \renewcommand{\footrulewidth}{0pt}
%   \renewcommand{\headrulewidth}{0pt}
%   \fancyhead[LE,RO]{}
%   \fancyhead[CE,CO]{}
%   \fancyhead[RE,LO]{}
% \else\ifWordBk
%   % It's a words only songbook...
%   \addtolength{\headwidth}{\marginparsep}
%   \addtolength{\headwidth}{\marginparwidth}
%   \renewcommand{\headrulewidth}{0.4pt}
%   \renewcommand{\footrulewidth}{0.4pt}
%   \fancyhead[LE,RO]{\LHeadFont Naturvidenskab revy sange}
%   \fancyhead[CE,CO]{\CHeadFont\thepage}
%   \fancyhead[RE,LO]{\RHeadFont \SBThechapter}
% \fi\fi\fi

% \fancyfoot[LE,RO]{\LFootFont Computer Science Camp 2019}
% \ifSongEject
%   \fancyfoot[CE,CO]{\CFootFont Last Revised:  \RevDate}
% \else
%   \fancyfoot[CE,CO]{\CFootFont}
% \fi
% \fancyfoot[RE,LO]{\RFootFont Synges på eget ansvar}

%%%
% Table of contents
%%%

% \clearpage
% \twocolumn
% \font\myTinySF=cmss8    at  8pt
% \font\myHugeSF=cmssbx10 at 25pt
% \newcommand{\CpyRtInfoFont}{\tiny\myTinySF}
% \newcommand{\myTitleFont}{\Huge\myHugeSF}
% \newcommand{\mySubTitleFont}{\large\sf}
% \renewcommand{\indexspace}{\medskip}

% % {\parindent 8pt
% %   {\myTitleFont Indhold}}\par
% % \vskip 5pt
% \renewcommand{\SBThechapter}{Indhold}
% % {\parindent 20pt
% %   {\mySubTitleFont --- with first lines in italic ---}}
% % \vskip 20pt
% \let\olditem\item
% \let\oldsubitem\subitem
% \let\oldsubsubitem\subsubitem
% \renewcommand{\item}{\par\hangindent=40pt}
% \renewcommand{\subitem}{\par\hangindent=40pt \hspace*{20pt}}
% \renewcommand{\subsubitem}{\par\hangindent=40pt \hspace*{30pt}}

% %\input{unf-sangbog.tocx}

% \renewcommand{\item}{\olditem}
% \renewcommand{\subitem}{\oldsubitem}
% \renewcommand{\subsubitem}{\oldsubsubitem}

%%%
% Songbook begins.
%%%

\twocolumn
%It's just one page, don't print page numbers etc.
\pagestyle{empty}
%Songs included
\input{songs/matmatik.tex}
\input{songs/taal_daj.tex}
\input{songs/linieskriverdriver.tex}
\input{songs/steve_hawking.tex}
\input{songs/ode_til_kode.tex}
\input{songs/se_min_kode.tex}
\input{songs/vaabenfysik_kort.tex}
%Maybe include:
%\input{songs/kvanter_i_maaneskin.tex}
%\input{songs/mest_matematiske_dyr.tex}

% \input{songs/vi_kan_ikke_li.tex}
% \input{songs/selektionssangen.tex}
% \input{songs/alfabetsangen.tex}
% \input{songs/sciencecamps.tex}
% \input{songs/hvad_maa_man.tex}


% \input{songs/lambda_kalkylen.tex}
% \input{songs/puslespil.tex}
% \input{songs/null.tex}
% \input{songs/fasebal.tex}

% \input{songs/chifitter.tex}

% \input{songs/kun_fysik.tex}



% \input{songs/kanoniske.tex}
% \input{songs/jeg_er_en_matematiker_fra_hcoe.tex}


% \input{songs/rekursiv_skovsang.tex}
% \input{songs/laerkerede.tex}


% \clearpage
% \font\myTinySF=cmss8    at  8pt
% \font\myHugeSF=cmssbx10 at 25pt
% % \newcommand{\CpyRtInfoFont}{\tiny\myTinySF}
% % \newcommand{\myTitleFont}{\Huge\myHugeSF}
% % \newcommand{\mySubTitleFont}{\large\sf}
% \renewcommand{\indexspace}{\medskip}

% {\parindent 8pt
%   {\myTitleFont Index}}\par
% \vskip 5pt
% \renewcommand{\SBThechapter}{Index}
% % {\parindent 20pt
% %   {\mySubTitleFont --- with first lines in italic ---}}
% % \vskip 20pt
% \renewcommand{\item}{\par\hangindent=40pt}
% \renewcommand{\subitem}{\par\hangindent=40pt \hspace*{20pt}}
% \renewcommand{\subsubitem}{\par\hangindent=40pt \hspace*{30pt}}

%\input{unf-sangbog.tdx}

\end{document}
\bye
%
%%%
% Document ends.
%%%


\end{document}
\bye
%
%%%
% Document ends.
%%%


\end{document}
\bye
%
%%%
% Document ends.
%%%


% \renewcommand{\item}{\olditem}
% \renewcommand{\subitem}{\oldsubitem}
% \renewcommand{\subsubitem}{\oldsubsubitem}

%%%
% Songbook begins.
%%%

\twocolumn
%It's just one page, don't print page numbers etc.
\pagestyle{empty}
%Songs included
\input{songs/matmatik.tex}
\input{songs/taal_daj.tex}
\input{songs/linieskriverdriver.tex}
\input{songs/steve_hawking.tex}
\input{songs/ode_til_kode.tex}
\input{songs/se_min_kode.tex}
\input{songs/vaabenfysik_kort.tex}
%Maybe include:
%\input{songs/kvanter_i_maaneskin.tex}
%\input{songs/mest_matematiske_dyr.tex}

% \input{songs/vi_kan_ikke_li.tex}
% \input{songs/selektionssangen.tex}
% \input{songs/alfabetsangen.tex}
% \input{songs/sciencecamps.tex}
% \input{songs/hvad_maa_man.tex}


% \input{songs/lambda_kalkylen.tex}
% \input{songs/puslespil.tex}
% \input{songs/null.tex}
% \input{songs/fasebal.tex}

% \input{songs/chifitter.tex}

% \input{songs/kun_fysik.tex}



% \input{songs/kanoniske.tex}
% \input{songs/jeg_er_en_matematiker_fra_hcoe.tex}


% \input{songs/rekursiv_skovsang.tex}
% \input{songs/laerkerede.tex}


% \clearpage
% \font\myTinySF=cmss8    at  8pt
% \font\myHugeSF=cmssbx10 at 25pt
% % \newcommand{\CpyRtInfoFont}{\tiny\myTinySF}
% % \newcommand{\myTitleFont}{\Huge\myHugeSF}
% % \newcommand{\mySubTitleFont}{\large\sf}
% \renewcommand{\indexspace}{\medskip}

% {\parindent 8pt
%   {\myTitleFont Index}}\par
% \vskip 5pt
% \renewcommand{\SBThechapter}{Index}
% % {\parindent 20pt
% %   {\mySubTitleFont --- with first lines in italic ---}}
% % \vskip 20pt
% \renewcommand{\item}{\par\hangindent=40pt}
% \renewcommand{\subitem}{\par\hangindent=40pt \hspace*{20pt}}
% \renewcommand{\subsubitem}{\par\hangindent=40pt \hspace*{30pt}}

%%%%%%% rcsid = @(#)$Id: sample-sb.tex,v 1.23 2010-04-12 18:04:11 rathc Exp $
%%%%%%
%%
%%      ===============================
%%      Sample Songbook (sample-sb.tex)
%%      ===============================
%%
%%      Version 4.5, 30 April, 2010
%%
%%      Copyright 1992--2010 Christopher Rath <christopher@rath.ca>
%%
%%      This package is free software; you can redistribute it and/or
%%      modify it under the terms of version 2.1 of the GNU Lesser
%%	General Public License as published by the Free Software 
%%	Foundation.
%%
%%      This package is distributed in the hope that it will be
%%      useful, but WITHOUT ANY WARRANTY; without even the implied
%%      warranty of MERCHANTABILITY or FITNESS FOR A PARTICULAR
%%      PURPOSE.  See the GNU Lesser General Public License for more
%%      details.
%%
%%      This file contains a subset of the songbook we distribute
%%      at our church.  To the best of my knowledge, all of the lyrics
%%      contained herein are freely distributable.  This file has been
%%      provided as a sample of what can be produced by the chordbk,
%%      wordbk, and overhead LaTeX styles.
%%
%%      NEEDED:  The fancyhdr LaTeX style is required to properly
%%              format this file.  If you don't have that then comment
%%              out the commands in the preamble which deal with the
%%              fancyhdr style.
%%
%%%%%%
%%%%%%
%%
%%      1. Chord notation.  Within this songbook the following
%%         conventions have been adopted:
%%
%%              "Minor" is entered as "m";
%%                      e.g. Cm7 for C minor 7th.
%%              "Major" is entered as "M";
%%                      e.g. CM7 for C major 7th.
%%
%%%%%%
%%%%%%
%%      ============
%%      Bibliography
%%      ============
%%
%%      Exalt Him!: Exalt Him!  Compiled by Tom Fettke.  (c)1989
%%                      Word Music.
%%
%%      Hosanna! Music Books: Hosanna! Music Books #1--#6.
%%                      (c)1987--92 Integrity Music, Inc.
%%
%%      Worship Him II: Worship Him II.  Compiled by Jesse Peterson
%%                      and Bruce Ballinger.  (c)1989 Tempo Music
%%                      Publications.
%%
%%      Worship Songs Of The Vineyard: Worship Songs Of The Vineyard
%%                      --- Volume 2.  (c)1989 Vineyard Ministries
%%                      International.
%%
%%%%%%
%%%%%%

%%%%%%%%%%%%%%%%%%%%%%%%%%%%%%%%%%%%%%%%%%%%%%%%%%%%%%%%%%
%%%%%%%%%%%%%%%%%%%%%%%%%%%%%%%%%%%%%%%%%%%%%%%%%%%%%%%%%%
%%                                                      %%
%%           P R E A M B L E   B E G I N S              %%
%%                                                      %%
%%%%%%%%%%%%%%%%%%%%%%%%%%%%%%%%%%%%%%%%%%%%%%%%%%%%%%%%%%
%%%%%%%%%%%%%%%%%%%%%%%%%%%%%%%%%%%%%%%%%%%%%%%%%%%%%%%%%%

\documentclass[a5paper]{book}
\usepackage{latexsym,
            fancyhdr,
            titlesec,
            amsmath,
            amssymb,
            multicol,
            amsthm,
            stmaryrd,
            amsthm,
            color,
            needspace,
            stackengine,
            wasysym}
\usepackage[utf8]{inputenc}
\usepackage[T1]{fontenc}
% \usepackage[chordbk]{songbook}                  %% Words & Chords edition.
%%\usepackage[compactallsongs,chordbk]{songbook}    %% Words & Chords edition.
\usepackage[wordbk]{songbook}                 %% Words Only edition.
%%\usepackage[overhead]{songbook}               %% Overhead Transparency edition.
\usepackage{titletoc}
\usepackage{tket}  % Draws "TÅGEKAMMERET" correctly

%%%
% Revision Date and Release Date definitions.
%
%       \RelDate - The last time this songbook was released.  Set this
%                  date each time a new release/update of the songbook
%                  is generated.
%       \RevDate - The last time a particular song was revised in any
%                  way.  This command will be renewed inside every
%                  song.
%%%
\newcommand{\RelDate}{31~August,~2003}
\newcommand{\RevDate}{\today}

%%%
% C.C.L.I. license number definition; for copyright licensing info.
% One of these macros will be manually inserted into the {SBMel}
% parameter of the {song} environment.
%
%       \CCLInumber - The actual copyright license number.  Don't
%               insert this command in the {SBMel} parameter, use one
%               of the others.
%       \CCLIed - Indicates a song falls under our CCLI license.
%       \NotCCLIed - Indicates a song doesn't fall under our CCLI
%               license.  Public Domain songs fall into this category.
%       \PGranted - We have received specific permission from the
%               copyright holder to use this song.
%       \PPending - We are in the process of obtaining permission to
%               use this song.
%%%
\newcommand{\CCLInumber}{Your CCLI Number}
\newcommand{\CCLIed}{{\SBMelInfoFont (CCLI \CCLInumber)}}
\newcommand{\NotCCLIed}{\relax}
\newcommand{\PGranted}{\relax}
\newcommand{\PPending}{{\SBMelInfoFont (Permission Pending)}}

%%%
% Title page information.
%%%
%\title{UNF Computer Science Camp 2019 Sangbog}
%\author{}
%\date{Revideret:  \RevDate}

%%%
% Redefine fonts from SongBook style that I don't like.
%%%
\font\myTinySF=cmss8 at 8pt
\renewcommand{\SBMelInfoFont}{\tiny\myTinySF}

%%%
% Define fonts to use in the headers and footers of the songbook.
%%%
\newcommand{\LHeadFont}{\normalsize}            % = cmr12  at 12pt
\newcommand{\CHeadFont}{\normalsize\rm}         % = cmr12  at 12pt
\newcommand{\RHeadFont}{\normalsize}            % = cmr12  at 12pt
\newcommand{\LFootFont}{\scriptsize}            % = cmr8   at  8pt
\newcommand{\CFootFont}{\tiny\myTinySF}         % = cmss8  at  8pt
\newcommand{\RFootFont}{\scriptsize}            % = cmr8   at  8pt

\def\repeat{%
  \stackanchor{.}{.}%
  \rule[-\dp\strutbox]{.3pt}{\normalbaselineskip}%
  \kern0.5pt%
  \rule[-\dp\strutbox]{1pt}{\normalbaselineskip}%
  \kern1pt%
}
\def\frepeat{%
  \kern1pt%
  \rule[-\dp\strutbox]{1pt}{\normalbaselineskip}%
  \kern0.5pt%
  \rule[-\dp\strutbox]{.3pt}{\normalbaselineskip}%
  \stackanchor{.}{.}%
}
% \newcommand{\SBRepeat}[1]{#1\\#1}
\newcommand{\SBRepeat}[1]{\frepeat #1\repeat}
\setcounter{SBSongCnt}{-1}
\renewcommand{\SBWAndMTag}{Forfatter:}
\renewcommand{\SBUnknownTag}{Ukendt}
\renewcommand{\SBChorusTag}{Ref.}
\renewcommand{\SBOrgMel}{Originalmelodi}
\renewcommand{\SpaceAfterChorus}   {\vspace{0ex plus1ex minus 0.5ex}}
\renewcommand{\SpaceAfterOpGroup}  {\vspace{0ex plus1ex minus 0.5ex}}
\renewcommand{\SpaceAfterSBBracket}{\vspace{0ex plus1ex minus 0.5ex}}
\renewcommand{\SpaceAfterSection}  {\vspace{0ex plus1ex minus 0.5ex}}
\renewcommand{\SpaceAfterSong}     {\vspace{0ex plus1ex minus 0.5ex}}
\renewcommand{\SpaceAfterVerse}    {\vspace{0ex plus1ex minus 0.5ex}}

% Tell LaTeX that \medskip is a good place to make a page break
\let\oldmedskip\medskip
\renewcommand{\medskip}{\oldmedskip\pagebreak[2]}

%%%
% Turn on/off index-file generation.  Uncomment the \makeindex line to
% turn index generation on;  comment it out to turn index generation
% off.
%%%
%\makeTitleIndex         %% Title and First Line Index.
%\makeTitleContents      %% Table of Contents.
%\makeKeyIndex           %% Index of song by key.
% \makeArtistIndex	%% Index of song by artist.
% \newcommand{\SBThechapter}[0]{}
% \newcommand{\SBChapter}[1]{
%     \startcontents
%     \chapter*{#1} 
%     % %%%%%% rcsid = @(#)$Id: sample-sb.tex,v 1.23 2010-04-12 18:04:11 rathc Exp $
%%%%%%
%%
%%      ===============================
%%      Sample Songbook (sample-sb.tex)
%%      ===============================
%%
%%      Version 4.5, 30 April, 2010
%%
%%      Copyright 1992--2010 Christopher Rath <christopher@rath.ca>
%%
%%      This package is free software; you can redistribute it and/or
%%      modify it under the terms of version 2.1 of the GNU Lesser
%%	General Public License as published by the Free Software 
%%	Foundation.
%%
%%      This package is distributed in the hope that it will be
%%      useful, but WITHOUT ANY WARRANTY; without even the implied
%%      warranty of MERCHANTABILITY or FITNESS FOR A PARTICULAR
%%      PURPOSE.  See the GNU Lesser General Public License for more
%%      details.
%%
%%      This file contains a subset of the songbook we distribute
%%      at our church.  To the best of my knowledge, all of the lyrics
%%      contained herein are freely distributable.  This file has been
%%      provided as a sample of what can be produced by the chordbk,
%%      wordbk, and overhead LaTeX styles.
%%
%%      NEEDED:  The fancyhdr LaTeX style is required to properly
%%              format this file.  If you don't have that then comment
%%              out the commands in the preamble which deal with the
%%              fancyhdr style.
%%
%%%%%%
%%%%%%
%%
%%      1. Chord notation.  Within this songbook the following
%%         conventions have been adopted:
%%
%%              "Minor" is entered as "m";
%%                      e.g. Cm7 for C minor 7th.
%%              "Major" is entered as "M";
%%                      e.g. CM7 for C major 7th.
%%
%%%%%%
%%%%%%
%%      ============
%%      Bibliography
%%      ============
%%
%%      Exalt Him!: Exalt Him!  Compiled by Tom Fettke.  (c)1989
%%                      Word Music.
%%
%%      Hosanna! Music Books: Hosanna! Music Books #1--#6.
%%                      (c)1987--92 Integrity Music, Inc.
%%
%%      Worship Him II: Worship Him II.  Compiled by Jesse Peterson
%%                      and Bruce Ballinger.  (c)1989 Tempo Music
%%                      Publications.
%%
%%      Worship Songs Of The Vineyard: Worship Songs Of The Vineyard
%%                      --- Volume 2.  (c)1989 Vineyard Ministries
%%                      International.
%%
%%%%%%
%%%%%%

%%%%%%%%%%%%%%%%%%%%%%%%%%%%%%%%%%%%%%%%%%%%%%%%%%%%%%%%%%
%%%%%%%%%%%%%%%%%%%%%%%%%%%%%%%%%%%%%%%%%%%%%%%%%%%%%%%%%%
%%                                                      %%
%%           P R E A M B L E   B E G I N S              %%
%%                                                      %%
%%%%%%%%%%%%%%%%%%%%%%%%%%%%%%%%%%%%%%%%%%%%%%%%%%%%%%%%%%
%%%%%%%%%%%%%%%%%%%%%%%%%%%%%%%%%%%%%%%%%%%%%%%%%%%%%%%%%%

\documentclass[a5paper]{book}
\usepackage{latexsym,
            fancyhdr,
            titlesec,
            amsmath,
            amssymb,
            multicol,
            amsthm,
            stmaryrd,
            amsthm,
            color,
            needspace,
            stackengine,
            wasysym}
\usepackage[utf8]{inputenc}
\usepackage[T1]{fontenc}
% \usepackage[chordbk]{songbook}                  %% Words & Chords edition.
%%\usepackage[compactallsongs,chordbk]{songbook}    %% Words & Chords edition.
\usepackage[wordbk]{songbook}                 %% Words Only edition.
%%\usepackage[overhead]{songbook}               %% Overhead Transparency edition.
\usepackage{titletoc}
\usepackage{tket}  % Draws "TÅGEKAMMERET" correctly

%%%
% Revision Date and Release Date definitions.
%
%       \RelDate - The last time this songbook was released.  Set this
%                  date each time a new release/update of the songbook
%                  is generated.
%       \RevDate - The last time a particular song was revised in any
%                  way.  This command will be renewed inside every
%                  song.
%%%
\newcommand{\RelDate}{31~August,~2003}
\newcommand{\RevDate}{\today}

%%%
% C.C.L.I. license number definition; for copyright licensing info.
% One of these macros will be manually inserted into the {SBMel}
% parameter of the {song} environment.
%
%       \CCLInumber - The actual copyright license number.  Don't
%               insert this command in the {SBMel} parameter, use one
%               of the others.
%       \CCLIed - Indicates a song falls under our CCLI license.
%       \NotCCLIed - Indicates a song doesn't fall under our CCLI
%               license.  Public Domain songs fall into this category.
%       \PGranted - We have received specific permission from the
%               copyright holder to use this song.
%       \PPending - We are in the process of obtaining permission to
%               use this song.
%%%
\newcommand{\CCLInumber}{Your CCLI Number}
\newcommand{\CCLIed}{{\SBMelInfoFont (CCLI \CCLInumber)}}
\newcommand{\NotCCLIed}{\relax}
\newcommand{\PGranted}{\relax}
\newcommand{\PPending}{{\SBMelInfoFont (Permission Pending)}}

%%%
% Title page information.
%%%
%\title{UNF Computer Science Camp 2019 Sangbog}
%\author{}
%\date{Revideret:  \RevDate}

%%%
% Redefine fonts from SongBook style that I don't like.
%%%
\font\myTinySF=cmss8 at 8pt
\renewcommand{\SBMelInfoFont}{\tiny\myTinySF}

%%%
% Define fonts to use in the headers and footers of the songbook.
%%%
\newcommand{\LHeadFont}{\normalsize}            % = cmr12  at 12pt
\newcommand{\CHeadFont}{\normalsize\rm}         % = cmr12  at 12pt
\newcommand{\RHeadFont}{\normalsize}            % = cmr12  at 12pt
\newcommand{\LFootFont}{\scriptsize}            % = cmr8   at  8pt
\newcommand{\CFootFont}{\tiny\myTinySF}         % = cmss8  at  8pt
\newcommand{\RFootFont}{\scriptsize}            % = cmr8   at  8pt

\def\repeat{%
  \stackanchor{.}{.}%
  \rule[-\dp\strutbox]{.3pt}{\normalbaselineskip}%
  \kern0.5pt%
  \rule[-\dp\strutbox]{1pt}{\normalbaselineskip}%
  \kern1pt%
}
\def\frepeat{%
  \kern1pt%
  \rule[-\dp\strutbox]{1pt}{\normalbaselineskip}%
  \kern0.5pt%
  \rule[-\dp\strutbox]{.3pt}{\normalbaselineskip}%
  \stackanchor{.}{.}%
}
% \newcommand{\SBRepeat}[1]{#1\\#1}
\newcommand{\SBRepeat}[1]{\frepeat #1\repeat}
\setcounter{SBSongCnt}{-1}
\renewcommand{\SBWAndMTag}{Forfatter:}
\renewcommand{\SBUnknownTag}{Ukendt}
\renewcommand{\SBChorusTag}{Ref.}
\renewcommand{\SBOrgMel}{Originalmelodi}
\renewcommand{\SpaceAfterChorus}   {\vspace{0ex plus1ex minus 0.5ex}}
\renewcommand{\SpaceAfterOpGroup}  {\vspace{0ex plus1ex minus 0.5ex}}
\renewcommand{\SpaceAfterSBBracket}{\vspace{0ex plus1ex minus 0.5ex}}
\renewcommand{\SpaceAfterSection}  {\vspace{0ex plus1ex minus 0.5ex}}
\renewcommand{\SpaceAfterSong}     {\vspace{0ex plus1ex minus 0.5ex}}
\renewcommand{\SpaceAfterVerse}    {\vspace{0ex plus1ex minus 0.5ex}}

% Tell LaTeX that \medskip is a good place to make a page break
\let\oldmedskip\medskip
\renewcommand{\medskip}{\oldmedskip\pagebreak[2]}

%%%
% Turn on/off index-file generation.  Uncomment the \makeindex line to
% turn index generation on;  comment it out to turn index generation
% off.
%%%
%\makeTitleIndex         %% Title and First Line Index.
%\makeTitleContents      %% Table of Contents.
%\makeKeyIndex           %% Index of song by key.
% \makeArtistIndex	%% Index of song by artist.
% \newcommand{\SBThechapter}[0]{}
% \newcommand{\SBChapter}[1]{
%     \startcontents
%     \chapter*{#1} 
%     % %%%%%% rcsid = @(#)$Id: sample-sb.tex,v 1.23 2010-04-12 18:04:11 rathc Exp $
%%%%%%
%%
%%      ===============================
%%      Sample Songbook (sample-sb.tex)
%%      ===============================
%%
%%      Version 4.5, 30 April, 2010
%%
%%      Copyright 1992--2010 Christopher Rath <christopher@rath.ca>
%%
%%      This package is free software; you can redistribute it and/or
%%      modify it under the terms of version 2.1 of the GNU Lesser
%%	General Public License as published by the Free Software 
%%	Foundation.
%%
%%      This package is distributed in the hope that it will be
%%      useful, but WITHOUT ANY WARRANTY; without even the implied
%%      warranty of MERCHANTABILITY or FITNESS FOR A PARTICULAR
%%      PURPOSE.  See the GNU Lesser General Public License for more
%%      details.
%%
%%      This file contains a subset of the songbook we distribute
%%      at our church.  To the best of my knowledge, all of the lyrics
%%      contained herein are freely distributable.  This file has been
%%      provided as a sample of what can be produced by the chordbk,
%%      wordbk, and overhead LaTeX styles.
%%
%%      NEEDED:  The fancyhdr LaTeX style is required to properly
%%              format this file.  If you don't have that then comment
%%              out the commands in the preamble which deal with the
%%              fancyhdr style.
%%
%%%%%%
%%%%%%
%%
%%      1. Chord notation.  Within this songbook the following
%%         conventions have been adopted:
%%
%%              "Minor" is entered as "m";
%%                      e.g. Cm7 for C minor 7th.
%%              "Major" is entered as "M";
%%                      e.g. CM7 for C major 7th.
%%
%%%%%%
%%%%%%
%%      ============
%%      Bibliography
%%      ============
%%
%%      Exalt Him!: Exalt Him!  Compiled by Tom Fettke.  (c)1989
%%                      Word Music.
%%
%%      Hosanna! Music Books: Hosanna! Music Books #1--#6.
%%                      (c)1987--92 Integrity Music, Inc.
%%
%%      Worship Him II: Worship Him II.  Compiled by Jesse Peterson
%%                      and Bruce Ballinger.  (c)1989 Tempo Music
%%                      Publications.
%%
%%      Worship Songs Of The Vineyard: Worship Songs Of The Vineyard
%%                      --- Volume 2.  (c)1989 Vineyard Ministries
%%                      International.
%%
%%%%%%
%%%%%%

%%%%%%%%%%%%%%%%%%%%%%%%%%%%%%%%%%%%%%%%%%%%%%%%%%%%%%%%%%
%%%%%%%%%%%%%%%%%%%%%%%%%%%%%%%%%%%%%%%%%%%%%%%%%%%%%%%%%%
%%                                                      %%
%%           P R E A M B L E   B E G I N S              %%
%%                                                      %%
%%%%%%%%%%%%%%%%%%%%%%%%%%%%%%%%%%%%%%%%%%%%%%%%%%%%%%%%%%
%%%%%%%%%%%%%%%%%%%%%%%%%%%%%%%%%%%%%%%%%%%%%%%%%%%%%%%%%%

\documentclass[a5paper]{book}
\usepackage{latexsym,
            fancyhdr,
            titlesec,
            amsmath,
            amssymb,
            multicol,
            amsthm,
            stmaryrd,
            amsthm,
            color,
            needspace,
            stackengine,
            wasysym}
\usepackage[utf8]{inputenc}
\usepackage[T1]{fontenc}
% \usepackage[chordbk]{songbook}                  %% Words & Chords edition.
%%\usepackage[compactallsongs,chordbk]{songbook}    %% Words & Chords edition.
\usepackage[wordbk]{songbook}                 %% Words Only edition.
%%\usepackage[overhead]{songbook}               %% Overhead Transparency edition.
\usepackage{titletoc}
\usepackage{tket}  % Draws "TÅGEKAMMERET" correctly

%%%
% Revision Date and Release Date definitions.
%
%       \RelDate - The last time this songbook was released.  Set this
%                  date each time a new release/update of the songbook
%                  is generated.
%       \RevDate - The last time a particular song was revised in any
%                  way.  This command will be renewed inside every
%                  song.
%%%
\newcommand{\RelDate}{31~August,~2003}
\newcommand{\RevDate}{\today}

%%%
% C.C.L.I. license number definition; for copyright licensing info.
% One of these macros will be manually inserted into the {SBMel}
% parameter of the {song} environment.
%
%       \CCLInumber - The actual copyright license number.  Don't
%               insert this command in the {SBMel} parameter, use one
%               of the others.
%       \CCLIed - Indicates a song falls under our CCLI license.
%       \NotCCLIed - Indicates a song doesn't fall under our CCLI
%               license.  Public Domain songs fall into this category.
%       \PGranted - We have received specific permission from the
%               copyright holder to use this song.
%       \PPending - We are in the process of obtaining permission to
%               use this song.
%%%
\newcommand{\CCLInumber}{Your CCLI Number}
\newcommand{\CCLIed}{{\SBMelInfoFont (CCLI \CCLInumber)}}
\newcommand{\NotCCLIed}{\relax}
\newcommand{\PGranted}{\relax}
\newcommand{\PPending}{{\SBMelInfoFont (Permission Pending)}}

%%%
% Title page information.
%%%
%\title{UNF Computer Science Camp 2019 Sangbog}
%\author{}
%\date{Revideret:  \RevDate}

%%%
% Redefine fonts from SongBook style that I don't like.
%%%
\font\myTinySF=cmss8 at 8pt
\renewcommand{\SBMelInfoFont}{\tiny\myTinySF}

%%%
% Define fonts to use in the headers and footers of the songbook.
%%%
\newcommand{\LHeadFont}{\normalsize}            % = cmr12  at 12pt
\newcommand{\CHeadFont}{\normalsize\rm}         % = cmr12  at 12pt
\newcommand{\RHeadFont}{\normalsize}            % = cmr12  at 12pt
\newcommand{\LFootFont}{\scriptsize}            % = cmr8   at  8pt
\newcommand{\CFootFont}{\tiny\myTinySF}         % = cmss8  at  8pt
\newcommand{\RFootFont}{\scriptsize}            % = cmr8   at  8pt

\def\repeat{%
  \stackanchor{.}{.}%
  \rule[-\dp\strutbox]{.3pt}{\normalbaselineskip}%
  \kern0.5pt%
  \rule[-\dp\strutbox]{1pt}{\normalbaselineskip}%
  \kern1pt%
}
\def\frepeat{%
  \kern1pt%
  \rule[-\dp\strutbox]{1pt}{\normalbaselineskip}%
  \kern0.5pt%
  \rule[-\dp\strutbox]{.3pt}{\normalbaselineskip}%
  \stackanchor{.}{.}%
}
% \newcommand{\SBRepeat}[1]{#1\\#1}
\newcommand{\SBRepeat}[1]{\frepeat #1\repeat}
\setcounter{SBSongCnt}{-1}
\renewcommand{\SBWAndMTag}{Forfatter:}
\renewcommand{\SBUnknownTag}{Ukendt}
\renewcommand{\SBChorusTag}{Ref.}
\renewcommand{\SBOrgMel}{Originalmelodi}
\renewcommand{\SpaceAfterChorus}   {\vspace{0ex plus1ex minus 0.5ex}}
\renewcommand{\SpaceAfterOpGroup}  {\vspace{0ex plus1ex minus 0.5ex}}
\renewcommand{\SpaceAfterSBBracket}{\vspace{0ex plus1ex minus 0.5ex}}
\renewcommand{\SpaceAfterSection}  {\vspace{0ex plus1ex minus 0.5ex}}
\renewcommand{\SpaceAfterSong}     {\vspace{0ex plus1ex minus 0.5ex}}
\renewcommand{\SpaceAfterVerse}    {\vspace{0ex plus1ex minus 0.5ex}}

% Tell LaTeX that \medskip is a good place to make a page break
\let\oldmedskip\medskip
\renewcommand{\medskip}{\oldmedskip\pagebreak[2]}

%%%
% Turn on/off index-file generation.  Uncomment the \makeindex line to
% turn index generation on;  comment it out to turn index generation
% off.
%%%
%\makeTitleIndex         %% Title and First Line Index.
%\makeTitleContents      %% Table of Contents.
%\makeKeyIndex           %% Index of song by key.
% \makeArtistIndex	%% Index of song by artist.
% \newcommand{\SBThechapter}[0]{}
% \newcommand{\SBChapter}[1]{
%     \startcontents
%     \chapter*{#1} 
%     % \input{unf-sangbog.toc}
%       \begin{minipage}{.8\textwidth}
%         \printcontents{}{1}{}
%       \end{minipage}%
%     \renewcommand{\SBThechapter}{#1}
%     \clearpage
% }

% \titleformat{\chapter}
% [display]
% {}
% {%\vspace*{\fill}
%  % \titlerule[1pt]%
%  % \vspace{1pt}%
%  % \titlerule
%  % \vspace{1pc}%
%  \chaptertitlename}
% {}
% {\Huge}



%%%%%%%%%%%%%%%%%%%%%%%%%%%%%%%%%%%%%%%%%%%%%%%%%%%%%%%%%%
%%%%%%%%%%%%%%%%%%%%%%%%%%%%%%%%%%%%%%%%%%%%%%%%%%%%%%%%%%
%%                                                      %%
%%           D O C U M E N T   B E G I N S              %%
%%                                                      %%
%%%%%%%%%%%%%%%%%%%%%%%%%%%%%%%%%%%%%%%%%%%%%%%%%%%%%%%%%%
%%%%%%%%%%%%%%%%%%%%%%%%%%%%%%%%%%%%%%%%%%%%%%%%%%%%%%%%%%
\begin{document}

%%%
% Uncomment "\maketitle" statement to make a title page.
%%%
%\maketitle
% \begin{titlepage}
%   \centering
%   \vspace{5cm}
% 	\includegraphics[width=1\textwidth]{unf_logo.jpeg}\par\vspace{1cm}
% 	{\scshape\LARGE Sangbog \par}
% 	\vspace{1cm}
% 	{\scshape\Large UNF Computer Science Camp 2019\par}
	
% 	\vfill

% % Bottom of the page
% 	{\large \today\par}
% \end{titlepage}
% \mainmatter
% \ifWordBk
%   \twocolumn
% \fi


%%% Kolofon
%\thispagestyle{empty}
%Sammensat til UNF Computer Science Camp 2019 - csc.unf.dk\\
%Redaktør: Andreas Mosbæk Jensen m.fl. efter tidligere sangbog af Steffen Strunge Mathiesen\\
%Indhold opsat i \LaTeX. 
%Digital version og kildekode: github.com/steffen555/UNF-sangbog\\
%Revision 1 med stave fejl korrektioner
%\par\vspace*{\fill}
%Hvis du har forslag til sange, rettelser, ris og ros, eller hvis du kender en ukendt forfatter, så skriv til sangbog@unf.dk.

%%%
% Turn on and define fancy page heading/footing definition.
%%%
% \pagestyle{fancy}

% \ifChordBk
%   % It's a words & chords songbook...
%   \addtolength{\headwidth}{\marginparsep}
%   \addtolength{\headwidth}{\marginparwidth}
%   \renewcommand{\headrulewidth}{0.4pt}
%   \renewcommand{\footrulewidth}{0.4pt}
%   \fancyhead[LE,RO]{\LHeadFont\emph{\leftmark\/}\SBContinueMark}
%   \fancyhead[CE,CO]{\CHeadFont\thepage}
%   \fancyhead[RE,LO]{\RHeadFont \chaptermark}
% \else\ifOverhead
%   % It's an overhead...
%   \renewcommand{\footrulewidth}{0pt}
%   \renewcommand{\headrulewidth}{0pt}
%   \fancyhead[LE,RO]{}
%   \fancyhead[CE,CO]{}
%   \fancyhead[RE,LO]{}
% \else\ifWordBk
%   % It's a words only songbook...
%   \addtolength{\headwidth}{\marginparsep}
%   \addtolength{\headwidth}{\marginparwidth}
%   \renewcommand{\headrulewidth}{0.4pt}
%   \renewcommand{\footrulewidth}{0.4pt}
%   \fancyhead[LE,RO]{\LHeadFont Naturvidenskab revy sange}
%   \fancyhead[CE,CO]{\CHeadFont\thepage}
%   \fancyhead[RE,LO]{\RHeadFont \SBThechapter}
% \fi\fi\fi

% \fancyfoot[LE,RO]{\LFootFont Computer Science Camp 2019}
% \ifSongEject
%   \fancyfoot[CE,CO]{\CFootFont Last Revised:  \RevDate}
% \else
%   \fancyfoot[CE,CO]{\CFootFont}
% \fi
% \fancyfoot[RE,LO]{\RFootFont Synges på eget ansvar}

%%%
% Table of contents
%%%

% \clearpage
% \twocolumn
% \font\myTinySF=cmss8    at  8pt
% \font\myHugeSF=cmssbx10 at 25pt
% \newcommand{\CpyRtInfoFont}{\tiny\myTinySF}
% \newcommand{\myTitleFont}{\Huge\myHugeSF}
% \newcommand{\mySubTitleFont}{\large\sf}
% \renewcommand{\indexspace}{\medskip}

% % {\parindent 8pt
% %   {\myTitleFont Indhold}}\par
% % \vskip 5pt
% \renewcommand{\SBThechapter}{Indhold}
% % {\parindent 20pt
% %   {\mySubTitleFont --- with first lines in italic ---}}
% % \vskip 20pt
% \let\olditem\item
% \let\oldsubitem\subitem
% \let\oldsubsubitem\subsubitem
% \renewcommand{\item}{\par\hangindent=40pt}
% \renewcommand{\subitem}{\par\hangindent=40pt \hspace*{20pt}}
% \renewcommand{\subsubitem}{\par\hangindent=40pt \hspace*{30pt}}

% %\input{unf-sangbog.tocx}

% \renewcommand{\item}{\olditem}
% \renewcommand{\subitem}{\oldsubitem}
% \renewcommand{\subsubitem}{\oldsubsubitem}

%%%
% Songbook begins.
%%%

\twocolumn
%It's just one page, don't print page numbers etc.
\pagestyle{empty}
%Songs included
\input{songs/matmatik.tex}
\input{songs/taal_daj.tex}
\input{songs/linieskriverdriver.tex}
\input{songs/steve_hawking.tex}
\input{songs/ode_til_kode.tex}
\input{songs/se_min_kode.tex}
\input{songs/vaabenfysik_kort.tex}
%Maybe include:
%\input{songs/kvanter_i_maaneskin.tex}
%\input{songs/mest_matematiske_dyr.tex}

% \input{songs/vi_kan_ikke_li.tex}
% \input{songs/selektionssangen.tex}
% \input{songs/alfabetsangen.tex}
% \input{songs/sciencecamps.tex}
% \input{songs/hvad_maa_man.tex}


% \input{songs/lambda_kalkylen.tex}
% \input{songs/puslespil.tex}
% \input{songs/null.tex}
% \input{songs/fasebal.tex}

% \input{songs/chifitter.tex}

% \input{songs/kun_fysik.tex}



% \input{songs/kanoniske.tex}
% \input{songs/jeg_er_en_matematiker_fra_hcoe.tex}


% \input{songs/rekursiv_skovsang.tex}
% \input{songs/laerkerede.tex}


% \clearpage
% \font\myTinySF=cmss8    at  8pt
% \font\myHugeSF=cmssbx10 at 25pt
% % \newcommand{\CpyRtInfoFont}{\tiny\myTinySF}
% % \newcommand{\myTitleFont}{\Huge\myHugeSF}
% % \newcommand{\mySubTitleFont}{\large\sf}
% \renewcommand{\indexspace}{\medskip}

% {\parindent 8pt
%   {\myTitleFont Index}}\par
% \vskip 5pt
% \renewcommand{\SBThechapter}{Index}
% % {\parindent 20pt
% %   {\mySubTitleFont --- with first lines in italic ---}}
% % \vskip 20pt
% \renewcommand{\item}{\par\hangindent=40pt}
% \renewcommand{\subitem}{\par\hangindent=40pt \hspace*{20pt}}
% \renewcommand{\subsubitem}{\par\hangindent=40pt \hspace*{30pt}}

%\input{unf-sangbog.tdx}

\end{document}
\bye
%
%%%
% Document ends.
%%%

%       \begin{minipage}{.8\textwidth}
%         \printcontents{}{1}{}
%       \end{minipage}%
%     \renewcommand{\SBThechapter}{#1}
%     \clearpage
% }

% \titleformat{\chapter}
% [display]
% {}
% {%\vspace*{\fill}
%  % \titlerule[1pt]%
%  % \vspace{1pt}%
%  % \titlerule
%  % \vspace{1pc}%
%  \chaptertitlename}
% {}
% {\Huge}



%%%%%%%%%%%%%%%%%%%%%%%%%%%%%%%%%%%%%%%%%%%%%%%%%%%%%%%%%%
%%%%%%%%%%%%%%%%%%%%%%%%%%%%%%%%%%%%%%%%%%%%%%%%%%%%%%%%%%
%%                                                      %%
%%           D O C U M E N T   B E G I N S              %%
%%                                                      %%
%%%%%%%%%%%%%%%%%%%%%%%%%%%%%%%%%%%%%%%%%%%%%%%%%%%%%%%%%%
%%%%%%%%%%%%%%%%%%%%%%%%%%%%%%%%%%%%%%%%%%%%%%%%%%%%%%%%%%
\begin{document}

%%%
% Uncomment "\maketitle" statement to make a title page.
%%%
%\maketitle
% \begin{titlepage}
%   \centering
%   \vspace{5cm}
% 	\includegraphics[width=1\textwidth]{unf_logo.jpeg}\par\vspace{1cm}
% 	{\scshape\LARGE Sangbog \par}
% 	\vspace{1cm}
% 	{\scshape\Large UNF Computer Science Camp 2019\par}
	
% 	\vfill

% % Bottom of the page
% 	{\large \today\par}
% \end{titlepage}
% \mainmatter
% \ifWordBk
%   \twocolumn
% \fi


%%% Kolofon
%\thispagestyle{empty}
%Sammensat til UNF Computer Science Camp 2019 - csc.unf.dk\\
%Redaktør: Andreas Mosbæk Jensen m.fl. efter tidligere sangbog af Steffen Strunge Mathiesen\\
%Indhold opsat i \LaTeX. 
%Digital version og kildekode: github.com/steffen555/UNF-sangbog\\
%Revision 1 med stave fejl korrektioner
%\par\vspace*{\fill}
%Hvis du har forslag til sange, rettelser, ris og ros, eller hvis du kender en ukendt forfatter, så skriv til sangbog@unf.dk.

%%%
% Turn on and define fancy page heading/footing definition.
%%%
% \pagestyle{fancy}

% \ifChordBk
%   % It's a words & chords songbook...
%   \addtolength{\headwidth}{\marginparsep}
%   \addtolength{\headwidth}{\marginparwidth}
%   \renewcommand{\headrulewidth}{0.4pt}
%   \renewcommand{\footrulewidth}{0.4pt}
%   \fancyhead[LE,RO]{\LHeadFont\emph{\leftmark\/}\SBContinueMark}
%   \fancyhead[CE,CO]{\CHeadFont\thepage}
%   \fancyhead[RE,LO]{\RHeadFont \chaptermark}
% \else\ifOverhead
%   % It's an overhead...
%   \renewcommand{\footrulewidth}{0pt}
%   \renewcommand{\headrulewidth}{0pt}
%   \fancyhead[LE,RO]{}
%   \fancyhead[CE,CO]{}
%   \fancyhead[RE,LO]{}
% \else\ifWordBk
%   % It's a words only songbook...
%   \addtolength{\headwidth}{\marginparsep}
%   \addtolength{\headwidth}{\marginparwidth}
%   \renewcommand{\headrulewidth}{0.4pt}
%   \renewcommand{\footrulewidth}{0.4pt}
%   \fancyhead[LE,RO]{\LHeadFont Naturvidenskab revy sange}
%   \fancyhead[CE,CO]{\CHeadFont\thepage}
%   \fancyhead[RE,LO]{\RHeadFont \SBThechapter}
% \fi\fi\fi

% \fancyfoot[LE,RO]{\LFootFont Computer Science Camp 2019}
% \ifSongEject
%   \fancyfoot[CE,CO]{\CFootFont Last Revised:  \RevDate}
% \else
%   \fancyfoot[CE,CO]{\CFootFont}
% \fi
% \fancyfoot[RE,LO]{\RFootFont Synges på eget ansvar}

%%%
% Table of contents
%%%

% \clearpage
% \twocolumn
% \font\myTinySF=cmss8    at  8pt
% \font\myHugeSF=cmssbx10 at 25pt
% \newcommand{\CpyRtInfoFont}{\tiny\myTinySF}
% \newcommand{\myTitleFont}{\Huge\myHugeSF}
% \newcommand{\mySubTitleFont}{\large\sf}
% \renewcommand{\indexspace}{\medskip}

% % {\parindent 8pt
% %   {\myTitleFont Indhold}}\par
% % \vskip 5pt
% \renewcommand{\SBThechapter}{Indhold}
% % {\parindent 20pt
% %   {\mySubTitleFont --- with first lines in italic ---}}
% % \vskip 20pt
% \let\olditem\item
% \let\oldsubitem\subitem
% \let\oldsubsubitem\subsubitem
% \renewcommand{\item}{\par\hangindent=40pt}
% \renewcommand{\subitem}{\par\hangindent=40pt \hspace*{20pt}}
% \renewcommand{\subsubitem}{\par\hangindent=40pt \hspace*{30pt}}

% %%%%%%% rcsid = @(#)$Id: sample-sb.tex,v 1.23 2010-04-12 18:04:11 rathc Exp $
%%%%%%
%%
%%      ===============================
%%      Sample Songbook (sample-sb.tex)
%%      ===============================
%%
%%      Version 4.5, 30 April, 2010
%%
%%      Copyright 1992--2010 Christopher Rath <christopher@rath.ca>
%%
%%      This package is free software; you can redistribute it and/or
%%      modify it under the terms of version 2.1 of the GNU Lesser
%%	General Public License as published by the Free Software 
%%	Foundation.
%%
%%      This package is distributed in the hope that it will be
%%      useful, but WITHOUT ANY WARRANTY; without even the implied
%%      warranty of MERCHANTABILITY or FITNESS FOR A PARTICULAR
%%      PURPOSE.  See the GNU Lesser General Public License for more
%%      details.
%%
%%      This file contains a subset of the songbook we distribute
%%      at our church.  To the best of my knowledge, all of the lyrics
%%      contained herein are freely distributable.  This file has been
%%      provided as a sample of what can be produced by the chordbk,
%%      wordbk, and overhead LaTeX styles.
%%
%%      NEEDED:  The fancyhdr LaTeX style is required to properly
%%              format this file.  If you don't have that then comment
%%              out the commands in the preamble which deal with the
%%              fancyhdr style.
%%
%%%%%%
%%%%%%
%%
%%      1. Chord notation.  Within this songbook the following
%%         conventions have been adopted:
%%
%%              "Minor" is entered as "m";
%%                      e.g. Cm7 for C minor 7th.
%%              "Major" is entered as "M";
%%                      e.g. CM7 for C major 7th.
%%
%%%%%%
%%%%%%
%%      ============
%%      Bibliography
%%      ============
%%
%%      Exalt Him!: Exalt Him!  Compiled by Tom Fettke.  (c)1989
%%                      Word Music.
%%
%%      Hosanna! Music Books: Hosanna! Music Books #1--#6.
%%                      (c)1987--92 Integrity Music, Inc.
%%
%%      Worship Him II: Worship Him II.  Compiled by Jesse Peterson
%%                      and Bruce Ballinger.  (c)1989 Tempo Music
%%                      Publications.
%%
%%      Worship Songs Of The Vineyard: Worship Songs Of The Vineyard
%%                      --- Volume 2.  (c)1989 Vineyard Ministries
%%                      International.
%%
%%%%%%
%%%%%%

%%%%%%%%%%%%%%%%%%%%%%%%%%%%%%%%%%%%%%%%%%%%%%%%%%%%%%%%%%
%%%%%%%%%%%%%%%%%%%%%%%%%%%%%%%%%%%%%%%%%%%%%%%%%%%%%%%%%%
%%                                                      %%
%%           P R E A M B L E   B E G I N S              %%
%%                                                      %%
%%%%%%%%%%%%%%%%%%%%%%%%%%%%%%%%%%%%%%%%%%%%%%%%%%%%%%%%%%
%%%%%%%%%%%%%%%%%%%%%%%%%%%%%%%%%%%%%%%%%%%%%%%%%%%%%%%%%%

\documentclass[a5paper]{book}
\usepackage{latexsym,
            fancyhdr,
            titlesec,
            amsmath,
            amssymb,
            multicol,
            amsthm,
            stmaryrd,
            amsthm,
            color,
            needspace,
            stackengine,
            wasysym}
\usepackage[utf8]{inputenc}
\usepackage[T1]{fontenc}
% \usepackage[chordbk]{songbook}                  %% Words & Chords edition.
%%\usepackage[compactallsongs,chordbk]{songbook}    %% Words & Chords edition.
\usepackage[wordbk]{songbook}                 %% Words Only edition.
%%\usepackage[overhead]{songbook}               %% Overhead Transparency edition.
\usepackage{titletoc}
\usepackage{tket}  % Draws "TÅGEKAMMERET" correctly

%%%
% Revision Date and Release Date definitions.
%
%       \RelDate - The last time this songbook was released.  Set this
%                  date each time a new release/update of the songbook
%                  is generated.
%       \RevDate - The last time a particular song was revised in any
%                  way.  This command will be renewed inside every
%                  song.
%%%
\newcommand{\RelDate}{31~August,~2003}
\newcommand{\RevDate}{\today}

%%%
% C.C.L.I. license number definition; for copyright licensing info.
% One of these macros will be manually inserted into the {SBMel}
% parameter of the {song} environment.
%
%       \CCLInumber - The actual copyright license number.  Don't
%               insert this command in the {SBMel} parameter, use one
%               of the others.
%       \CCLIed - Indicates a song falls under our CCLI license.
%       \NotCCLIed - Indicates a song doesn't fall under our CCLI
%               license.  Public Domain songs fall into this category.
%       \PGranted - We have received specific permission from the
%               copyright holder to use this song.
%       \PPending - We are in the process of obtaining permission to
%               use this song.
%%%
\newcommand{\CCLInumber}{Your CCLI Number}
\newcommand{\CCLIed}{{\SBMelInfoFont (CCLI \CCLInumber)}}
\newcommand{\NotCCLIed}{\relax}
\newcommand{\PGranted}{\relax}
\newcommand{\PPending}{{\SBMelInfoFont (Permission Pending)}}

%%%
% Title page information.
%%%
%\title{UNF Computer Science Camp 2019 Sangbog}
%\author{}
%\date{Revideret:  \RevDate}

%%%
% Redefine fonts from SongBook style that I don't like.
%%%
\font\myTinySF=cmss8 at 8pt
\renewcommand{\SBMelInfoFont}{\tiny\myTinySF}

%%%
% Define fonts to use in the headers and footers of the songbook.
%%%
\newcommand{\LHeadFont}{\normalsize}            % = cmr12  at 12pt
\newcommand{\CHeadFont}{\normalsize\rm}         % = cmr12  at 12pt
\newcommand{\RHeadFont}{\normalsize}            % = cmr12  at 12pt
\newcommand{\LFootFont}{\scriptsize}            % = cmr8   at  8pt
\newcommand{\CFootFont}{\tiny\myTinySF}         % = cmss8  at  8pt
\newcommand{\RFootFont}{\scriptsize}            % = cmr8   at  8pt

\def\repeat{%
  \stackanchor{.}{.}%
  \rule[-\dp\strutbox]{.3pt}{\normalbaselineskip}%
  \kern0.5pt%
  \rule[-\dp\strutbox]{1pt}{\normalbaselineskip}%
  \kern1pt%
}
\def\frepeat{%
  \kern1pt%
  \rule[-\dp\strutbox]{1pt}{\normalbaselineskip}%
  \kern0.5pt%
  \rule[-\dp\strutbox]{.3pt}{\normalbaselineskip}%
  \stackanchor{.}{.}%
}
% \newcommand{\SBRepeat}[1]{#1\\#1}
\newcommand{\SBRepeat}[1]{\frepeat #1\repeat}
\setcounter{SBSongCnt}{-1}
\renewcommand{\SBWAndMTag}{Forfatter:}
\renewcommand{\SBUnknownTag}{Ukendt}
\renewcommand{\SBChorusTag}{Ref.}
\renewcommand{\SBOrgMel}{Originalmelodi}
\renewcommand{\SpaceAfterChorus}   {\vspace{0ex plus1ex minus 0.5ex}}
\renewcommand{\SpaceAfterOpGroup}  {\vspace{0ex plus1ex minus 0.5ex}}
\renewcommand{\SpaceAfterSBBracket}{\vspace{0ex plus1ex minus 0.5ex}}
\renewcommand{\SpaceAfterSection}  {\vspace{0ex plus1ex minus 0.5ex}}
\renewcommand{\SpaceAfterSong}     {\vspace{0ex plus1ex minus 0.5ex}}
\renewcommand{\SpaceAfterVerse}    {\vspace{0ex plus1ex minus 0.5ex}}

% Tell LaTeX that \medskip is a good place to make a page break
\let\oldmedskip\medskip
\renewcommand{\medskip}{\oldmedskip\pagebreak[2]}

%%%
% Turn on/off index-file generation.  Uncomment the \makeindex line to
% turn index generation on;  comment it out to turn index generation
% off.
%%%
%\makeTitleIndex         %% Title and First Line Index.
%\makeTitleContents      %% Table of Contents.
%\makeKeyIndex           %% Index of song by key.
% \makeArtistIndex	%% Index of song by artist.
% \newcommand{\SBThechapter}[0]{}
% \newcommand{\SBChapter}[1]{
%     \startcontents
%     \chapter*{#1} 
%     % \input{unf-sangbog.toc}
%       \begin{minipage}{.8\textwidth}
%         \printcontents{}{1}{}
%       \end{minipage}%
%     \renewcommand{\SBThechapter}{#1}
%     \clearpage
% }

% \titleformat{\chapter}
% [display]
% {}
% {%\vspace*{\fill}
%  % \titlerule[1pt]%
%  % \vspace{1pt}%
%  % \titlerule
%  % \vspace{1pc}%
%  \chaptertitlename}
% {}
% {\Huge}



%%%%%%%%%%%%%%%%%%%%%%%%%%%%%%%%%%%%%%%%%%%%%%%%%%%%%%%%%%
%%%%%%%%%%%%%%%%%%%%%%%%%%%%%%%%%%%%%%%%%%%%%%%%%%%%%%%%%%
%%                                                      %%
%%           D O C U M E N T   B E G I N S              %%
%%                                                      %%
%%%%%%%%%%%%%%%%%%%%%%%%%%%%%%%%%%%%%%%%%%%%%%%%%%%%%%%%%%
%%%%%%%%%%%%%%%%%%%%%%%%%%%%%%%%%%%%%%%%%%%%%%%%%%%%%%%%%%
\begin{document}

%%%
% Uncomment "\maketitle" statement to make a title page.
%%%
%\maketitle
% \begin{titlepage}
%   \centering
%   \vspace{5cm}
% 	\includegraphics[width=1\textwidth]{unf_logo.jpeg}\par\vspace{1cm}
% 	{\scshape\LARGE Sangbog \par}
% 	\vspace{1cm}
% 	{\scshape\Large UNF Computer Science Camp 2019\par}
	
% 	\vfill

% % Bottom of the page
% 	{\large \today\par}
% \end{titlepage}
% \mainmatter
% \ifWordBk
%   \twocolumn
% \fi


%%% Kolofon
%\thispagestyle{empty}
%Sammensat til UNF Computer Science Camp 2019 - csc.unf.dk\\
%Redaktør: Andreas Mosbæk Jensen m.fl. efter tidligere sangbog af Steffen Strunge Mathiesen\\
%Indhold opsat i \LaTeX. 
%Digital version og kildekode: github.com/steffen555/UNF-sangbog\\
%Revision 1 med stave fejl korrektioner
%\par\vspace*{\fill}
%Hvis du har forslag til sange, rettelser, ris og ros, eller hvis du kender en ukendt forfatter, så skriv til sangbog@unf.dk.

%%%
% Turn on and define fancy page heading/footing definition.
%%%
% \pagestyle{fancy}

% \ifChordBk
%   % It's a words & chords songbook...
%   \addtolength{\headwidth}{\marginparsep}
%   \addtolength{\headwidth}{\marginparwidth}
%   \renewcommand{\headrulewidth}{0.4pt}
%   \renewcommand{\footrulewidth}{0.4pt}
%   \fancyhead[LE,RO]{\LHeadFont\emph{\leftmark\/}\SBContinueMark}
%   \fancyhead[CE,CO]{\CHeadFont\thepage}
%   \fancyhead[RE,LO]{\RHeadFont \chaptermark}
% \else\ifOverhead
%   % It's an overhead...
%   \renewcommand{\footrulewidth}{0pt}
%   \renewcommand{\headrulewidth}{0pt}
%   \fancyhead[LE,RO]{}
%   \fancyhead[CE,CO]{}
%   \fancyhead[RE,LO]{}
% \else\ifWordBk
%   % It's a words only songbook...
%   \addtolength{\headwidth}{\marginparsep}
%   \addtolength{\headwidth}{\marginparwidth}
%   \renewcommand{\headrulewidth}{0.4pt}
%   \renewcommand{\footrulewidth}{0.4pt}
%   \fancyhead[LE,RO]{\LHeadFont Naturvidenskab revy sange}
%   \fancyhead[CE,CO]{\CHeadFont\thepage}
%   \fancyhead[RE,LO]{\RHeadFont \SBThechapter}
% \fi\fi\fi

% \fancyfoot[LE,RO]{\LFootFont Computer Science Camp 2019}
% \ifSongEject
%   \fancyfoot[CE,CO]{\CFootFont Last Revised:  \RevDate}
% \else
%   \fancyfoot[CE,CO]{\CFootFont}
% \fi
% \fancyfoot[RE,LO]{\RFootFont Synges på eget ansvar}

%%%
% Table of contents
%%%

% \clearpage
% \twocolumn
% \font\myTinySF=cmss8    at  8pt
% \font\myHugeSF=cmssbx10 at 25pt
% \newcommand{\CpyRtInfoFont}{\tiny\myTinySF}
% \newcommand{\myTitleFont}{\Huge\myHugeSF}
% \newcommand{\mySubTitleFont}{\large\sf}
% \renewcommand{\indexspace}{\medskip}

% % {\parindent 8pt
% %   {\myTitleFont Indhold}}\par
% % \vskip 5pt
% \renewcommand{\SBThechapter}{Indhold}
% % {\parindent 20pt
% %   {\mySubTitleFont --- with first lines in italic ---}}
% % \vskip 20pt
% \let\olditem\item
% \let\oldsubitem\subitem
% \let\oldsubsubitem\subsubitem
% \renewcommand{\item}{\par\hangindent=40pt}
% \renewcommand{\subitem}{\par\hangindent=40pt \hspace*{20pt}}
% \renewcommand{\subsubitem}{\par\hangindent=40pt \hspace*{30pt}}

% %\input{unf-sangbog.tocx}

% \renewcommand{\item}{\olditem}
% \renewcommand{\subitem}{\oldsubitem}
% \renewcommand{\subsubitem}{\oldsubsubitem}

%%%
% Songbook begins.
%%%

\twocolumn
%It's just one page, don't print page numbers etc.
\pagestyle{empty}
%Songs included
\input{songs/matmatik.tex}
\input{songs/taal_daj.tex}
\input{songs/linieskriverdriver.tex}
\input{songs/steve_hawking.tex}
\input{songs/ode_til_kode.tex}
\input{songs/se_min_kode.tex}
\input{songs/vaabenfysik_kort.tex}
%Maybe include:
%\input{songs/kvanter_i_maaneskin.tex}
%\input{songs/mest_matematiske_dyr.tex}

% \input{songs/vi_kan_ikke_li.tex}
% \input{songs/selektionssangen.tex}
% \input{songs/alfabetsangen.tex}
% \input{songs/sciencecamps.tex}
% \input{songs/hvad_maa_man.tex}


% \input{songs/lambda_kalkylen.tex}
% \input{songs/puslespil.tex}
% \input{songs/null.tex}
% \input{songs/fasebal.tex}

% \input{songs/chifitter.tex}

% \input{songs/kun_fysik.tex}



% \input{songs/kanoniske.tex}
% \input{songs/jeg_er_en_matematiker_fra_hcoe.tex}


% \input{songs/rekursiv_skovsang.tex}
% \input{songs/laerkerede.tex}


% \clearpage
% \font\myTinySF=cmss8    at  8pt
% \font\myHugeSF=cmssbx10 at 25pt
% % \newcommand{\CpyRtInfoFont}{\tiny\myTinySF}
% % \newcommand{\myTitleFont}{\Huge\myHugeSF}
% % \newcommand{\mySubTitleFont}{\large\sf}
% \renewcommand{\indexspace}{\medskip}

% {\parindent 8pt
%   {\myTitleFont Index}}\par
% \vskip 5pt
% \renewcommand{\SBThechapter}{Index}
% % {\parindent 20pt
% %   {\mySubTitleFont --- with first lines in italic ---}}
% % \vskip 20pt
% \renewcommand{\item}{\par\hangindent=40pt}
% \renewcommand{\subitem}{\par\hangindent=40pt \hspace*{20pt}}
% \renewcommand{\subsubitem}{\par\hangindent=40pt \hspace*{30pt}}

%\input{unf-sangbog.tdx}

\end{document}
\bye
%
%%%
% Document ends.
%%%


% \renewcommand{\item}{\olditem}
% \renewcommand{\subitem}{\oldsubitem}
% \renewcommand{\subsubitem}{\oldsubsubitem}

%%%
% Songbook begins.
%%%

\twocolumn
%It's just one page, don't print page numbers etc.
\pagestyle{empty}
%Songs included
\input{songs/matmatik.tex}
\input{songs/taal_daj.tex}
\input{songs/linieskriverdriver.tex}
\input{songs/steve_hawking.tex}
\input{songs/ode_til_kode.tex}
\input{songs/se_min_kode.tex}
\input{songs/vaabenfysik_kort.tex}
%Maybe include:
%\input{songs/kvanter_i_maaneskin.tex}
%\input{songs/mest_matematiske_dyr.tex}

% \input{songs/vi_kan_ikke_li.tex}
% \input{songs/selektionssangen.tex}
% \input{songs/alfabetsangen.tex}
% \input{songs/sciencecamps.tex}
% \input{songs/hvad_maa_man.tex}


% \input{songs/lambda_kalkylen.tex}
% \input{songs/puslespil.tex}
% \input{songs/null.tex}
% \input{songs/fasebal.tex}

% \input{songs/chifitter.tex}

% \input{songs/kun_fysik.tex}



% \input{songs/kanoniske.tex}
% \input{songs/jeg_er_en_matematiker_fra_hcoe.tex}


% \input{songs/rekursiv_skovsang.tex}
% \input{songs/laerkerede.tex}


% \clearpage
% \font\myTinySF=cmss8    at  8pt
% \font\myHugeSF=cmssbx10 at 25pt
% % \newcommand{\CpyRtInfoFont}{\tiny\myTinySF}
% % \newcommand{\myTitleFont}{\Huge\myHugeSF}
% % \newcommand{\mySubTitleFont}{\large\sf}
% \renewcommand{\indexspace}{\medskip}

% {\parindent 8pt
%   {\myTitleFont Index}}\par
% \vskip 5pt
% \renewcommand{\SBThechapter}{Index}
% % {\parindent 20pt
% %   {\mySubTitleFont --- with first lines in italic ---}}
% % \vskip 20pt
% \renewcommand{\item}{\par\hangindent=40pt}
% \renewcommand{\subitem}{\par\hangindent=40pt \hspace*{20pt}}
% \renewcommand{\subsubitem}{\par\hangindent=40pt \hspace*{30pt}}

%%%%%%% rcsid = @(#)$Id: sample-sb.tex,v 1.23 2010-04-12 18:04:11 rathc Exp $
%%%%%%
%%
%%      ===============================
%%      Sample Songbook (sample-sb.tex)
%%      ===============================
%%
%%      Version 4.5, 30 April, 2010
%%
%%      Copyright 1992--2010 Christopher Rath <christopher@rath.ca>
%%
%%      This package is free software; you can redistribute it and/or
%%      modify it under the terms of version 2.1 of the GNU Lesser
%%	General Public License as published by the Free Software 
%%	Foundation.
%%
%%      This package is distributed in the hope that it will be
%%      useful, but WITHOUT ANY WARRANTY; without even the implied
%%      warranty of MERCHANTABILITY or FITNESS FOR A PARTICULAR
%%      PURPOSE.  See the GNU Lesser General Public License for more
%%      details.
%%
%%      This file contains a subset of the songbook we distribute
%%      at our church.  To the best of my knowledge, all of the lyrics
%%      contained herein are freely distributable.  This file has been
%%      provided as a sample of what can be produced by the chordbk,
%%      wordbk, and overhead LaTeX styles.
%%
%%      NEEDED:  The fancyhdr LaTeX style is required to properly
%%              format this file.  If you don't have that then comment
%%              out the commands in the preamble which deal with the
%%              fancyhdr style.
%%
%%%%%%
%%%%%%
%%
%%      1. Chord notation.  Within this songbook the following
%%         conventions have been adopted:
%%
%%              "Minor" is entered as "m";
%%                      e.g. Cm7 for C minor 7th.
%%              "Major" is entered as "M";
%%                      e.g. CM7 for C major 7th.
%%
%%%%%%
%%%%%%
%%      ============
%%      Bibliography
%%      ============
%%
%%      Exalt Him!: Exalt Him!  Compiled by Tom Fettke.  (c)1989
%%                      Word Music.
%%
%%      Hosanna! Music Books: Hosanna! Music Books #1--#6.
%%                      (c)1987--92 Integrity Music, Inc.
%%
%%      Worship Him II: Worship Him II.  Compiled by Jesse Peterson
%%                      and Bruce Ballinger.  (c)1989 Tempo Music
%%                      Publications.
%%
%%      Worship Songs Of The Vineyard: Worship Songs Of The Vineyard
%%                      --- Volume 2.  (c)1989 Vineyard Ministries
%%                      International.
%%
%%%%%%
%%%%%%

%%%%%%%%%%%%%%%%%%%%%%%%%%%%%%%%%%%%%%%%%%%%%%%%%%%%%%%%%%
%%%%%%%%%%%%%%%%%%%%%%%%%%%%%%%%%%%%%%%%%%%%%%%%%%%%%%%%%%
%%                                                      %%
%%           P R E A M B L E   B E G I N S              %%
%%                                                      %%
%%%%%%%%%%%%%%%%%%%%%%%%%%%%%%%%%%%%%%%%%%%%%%%%%%%%%%%%%%
%%%%%%%%%%%%%%%%%%%%%%%%%%%%%%%%%%%%%%%%%%%%%%%%%%%%%%%%%%

\documentclass[a5paper]{book}
\usepackage{latexsym,
            fancyhdr,
            titlesec,
            amsmath,
            amssymb,
            multicol,
            amsthm,
            stmaryrd,
            amsthm,
            color,
            needspace,
            stackengine,
            wasysym}
\usepackage[utf8]{inputenc}
\usepackage[T1]{fontenc}
% \usepackage[chordbk]{songbook}                  %% Words & Chords edition.
%%\usepackage[compactallsongs,chordbk]{songbook}    %% Words & Chords edition.
\usepackage[wordbk]{songbook}                 %% Words Only edition.
%%\usepackage[overhead]{songbook}               %% Overhead Transparency edition.
\usepackage{titletoc}
\usepackage{tket}  % Draws "TÅGEKAMMERET" correctly

%%%
% Revision Date and Release Date definitions.
%
%       \RelDate - The last time this songbook was released.  Set this
%                  date each time a new release/update of the songbook
%                  is generated.
%       \RevDate - The last time a particular song was revised in any
%                  way.  This command will be renewed inside every
%                  song.
%%%
\newcommand{\RelDate}{31~August,~2003}
\newcommand{\RevDate}{\today}

%%%
% C.C.L.I. license number definition; for copyright licensing info.
% One of these macros will be manually inserted into the {SBMel}
% parameter of the {song} environment.
%
%       \CCLInumber - The actual copyright license number.  Don't
%               insert this command in the {SBMel} parameter, use one
%               of the others.
%       \CCLIed - Indicates a song falls under our CCLI license.
%       \NotCCLIed - Indicates a song doesn't fall under our CCLI
%               license.  Public Domain songs fall into this category.
%       \PGranted - We have received specific permission from the
%               copyright holder to use this song.
%       \PPending - We are in the process of obtaining permission to
%               use this song.
%%%
\newcommand{\CCLInumber}{Your CCLI Number}
\newcommand{\CCLIed}{{\SBMelInfoFont (CCLI \CCLInumber)}}
\newcommand{\NotCCLIed}{\relax}
\newcommand{\PGranted}{\relax}
\newcommand{\PPending}{{\SBMelInfoFont (Permission Pending)}}

%%%
% Title page information.
%%%
%\title{UNF Computer Science Camp 2019 Sangbog}
%\author{}
%\date{Revideret:  \RevDate}

%%%
% Redefine fonts from SongBook style that I don't like.
%%%
\font\myTinySF=cmss8 at 8pt
\renewcommand{\SBMelInfoFont}{\tiny\myTinySF}

%%%
% Define fonts to use in the headers and footers of the songbook.
%%%
\newcommand{\LHeadFont}{\normalsize}            % = cmr12  at 12pt
\newcommand{\CHeadFont}{\normalsize\rm}         % = cmr12  at 12pt
\newcommand{\RHeadFont}{\normalsize}            % = cmr12  at 12pt
\newcommand{\LFootFont}{\scriptsize}            % = cmr8   at  8pt
\newcommand{\CFootFont}{\tiny\myTinySF}         % = cmss8  at  8pt
\newcommand{\RFootFont}{\scriptsize}            % = cmr8   at  8pt

\def\repeat{%
  \stackanchor{.}{.}%
  \rule[-\dp\strutbox]{.3pt}{\normalbaselineskip}%
  \kern0.5pt%
  \rule[-\dp\strutbox]{1pt}{\normalbaselineskip}%
  \kern1pt%
}
\def\frepeat{%
  \kern1pt%
  \rule[-\dp\strutbox]{1pt}{\normalbaselineskip}%
  \kern0.5pt%
  \rule[-\dp\strutbox]{.3pt}{\normalbaselineskip}%
  \stackanchor{.}{.}%
}
% \newcommand{\SBRepeat}[1]{#1\\#1}
\newcommand{\SBRepeat}[1]{\frepeat #1\repeat}
\setcounter{SBSongCnt}{-1}
\renewcommand{\SBWAndMTag}{Forfatter:}
\renewcommand{\SBUnknownTag}{Ukendt}
\renewcommand{\SBChorusTag}{Ref.}
\renewcommand{\SBOrgMel}{Originalmelodi}
\renewcommand{\SpaceAfterChorus}   {\vspace{0ex plus1ex minus 0.5ex}}
\renewcommand{\SpaceAfterOpGroup}  {\vspace{0ex plus1ex minus 0.5ex}}
\renewcommand{\SpaceAfterSBBracket}{\vspace{0ex plus1ex minus 0.5ex}}
\renewcommand{\SpaceAfterSection}  {\vspace{0ex plus1ex minus 0.5ex}}
\renewcommand{\SpaceAfterSong}     {\vspace{0ex plus1ex minus 0.5ex}}
\renewcommand{\SpaceAfterVerse}    {\vspace{0ex plus1ex minus 0.5ex}}

% Tell LaTeX that \medskip is a good place to make a page break
\let\oldmedskip\medskip
\renewcommand{\medskip}{\oldmedskip\pagebreak[2]}

%%%
% Turn on/off index-file generation.  Uncomment the \makeindex line to
% turn index generation on;  comment it out to turn index generation
% off.
%%%
%\makeTitleIndex         %% Title and First Line Index.
%\makeTitleContents      %% Table of Contents.
%\makeKeyIndex           %% Index of song by key.
% \makeArtistIndex	%% Index of song by artist.
% \newcommand{\SBThechapter}[0]{}
% \newcommand{\SBChapter}[1]{
%     \startcontents
%     \chapter*{#1} 
%     % \input{unf-sangbog.toc}
%       \begin{minipage}{.8\textwidth}
%         \printcontents{}{1}{}
%       \end{minipage}%
%     \renewcommand{\SBThechapter}{#1}
%     \clearpage
% }

% \titleformat{\chapter}
% [display]
% {}
% {%\vspace*{\fill}
%  % \titlerule[1pt]%
%  % \vspace{1pt}%
%  % \titlerule
%  % \vspace{1pc}%
%  \chaptertitlename}
% {}
% {\Huge}



%%%%%%%%%%%%%%%%%%%%%%%%%%%%%%%%%%%%%%%%%%%%%%%%%%%%%%%%%%
%%%%%%%%%%%%%%%%%%%%%%%%%%%%%%%%%%%%%%%%%%%%%%%%%%%%%%%%%%
%%                                                      %%
%%           D O C U M E N T   B E G I N S              %%
%%                                                      %%
%%%%%%%%%%%%%%%%%%%%%%%%%%%%%%%%%%%%%%%%%%%%%%%%%%%%%%%%%%
%%%%%%%%%%%%%%%%%%%%%%%%%%%%%%%%%%%%%%%%%%%%%%%%%%%%%%%%%%
\begin{document}

%%%
% Uncomment "\maketitle" statement to make a title page.
%%%
%\maketitle
% \begin{titlepage}
%   \centering
%   \vspace{5cm}
% 	\includegraphics[width=1\textwidth]{unf_logo.jpeg}\par\vspace{1cm}
% 	{\scshape\LARGE Sangbog \par}
% 	\vspace{1cm}
% 	{\scshape\Large UNF Computer Science Camp 2019\par}
	
% 	\vfill

% % Bottom of the page
% 	{\large \today\par}
% \end{titlepage}
% \mainmatter
% \ifWordBk
%   \twocolumn
% \fi


%%% Kolofon
%\thispagestyle{empty}
%Sammensat til UNF Computer Science Camp 2019 - csc.unf.dk\\
%Redaktør: Andreas Mosbæk Jensen m.fl. efter tidligere sangbog af Steffen Strunge Mathiesen\\
%Indhold opsat i \LaTeX. 
%Digital version og kildekode: github.com/steffen555/UNF-sangbog\\
%Revision 1 med stave fejl korrektioner
%\par\vspace*{\fill}
%Hvis du har forslag til sange, rettelser, ris og ros, eller hvis du kender en ukendt forfatter, så skriv til sangbog@unf.dk.

%%%
% Turn on and define fancy page heading/footing definition.
%%%
% \pagestyle{fancy}

% \ifChordBk
%   % It's a words & chords songbook...
%   \addtolength{\headwidth}{\marginparsep}
%   \addtolength{\headwidth}{\marginparwidth}
%   \renewcommand{\headrulewidth}{0.4pt}
%   \renewcommand{\footrulewidth}{0.4pt}
%   \fancyhead[LE,RO]{\LHeadFont\emph{\leftmark\/}\SBContinueMark}
%   \fancyhead[CE,CO]{\CHeadFont\thepage}
%   \fancyhead[RE,LO]{\RHeadFont \chaptermark}
% \else\ifOverhead
%   % It's an overhead...
%   \renewcommand{\footrulewidth}{0pt}
%   \renewcommand{\headrulewidth}{0pt}
%   \fancyhead[LE,RO]{}
%   \fancyhead[CE,CO]{}
%   \fancyhead[RE,LO]{}
% \else\ifWordBk
%   % It's a words only songbook...
%   \addtolength{\headwidth}{\marginparsep}
%   \addtolength{\headwidth}{\marginparwidth}
%   \renewcommand{\headrulewidth}{0.4pt}
%   \renewcommand{\footrulewidth}{0.4pt}
%   \fancyhead[LE,RO]{\LHeadFont Naturvidenskab revy sange}
%   \fancyhead[CE,CO]{\CHeadFont\thepage}
%   \fancyhead[RE,LO]{\RHeadFont \SBThechapter}
% \fi\fi\fi

% \fancyfoot[LE,RO]{\LFootFont Computer Science Camp 2019}
% \ifSongEject
%   \fancyfoot[CE,CO]{\CFootFont Last Revised:  \RevDate}
% \else
%   \fancyfoot[CE,CO]{\CFootFont}
% \fi
% \fancyfoot[RE,LO]{\RFootFont Synges på eget ansvar}

%%%
% Table of contents
%%%

% \clearpage
% \twocolumn
% \font\myTinySF=cmss8    at  8pt
% \font\myHugeSF=cmssbx10 at 25pt
% \newcommand{\CpyRtInfoFont}{\tiny\myTinySF}
% \newcommand{\myTitleFont}{\Huge\myHugeSF}
% \newcommand{\mySubTitleFont}{\large\sf}
% \renewcommand{\indexspace}{\medskip}

% % {\parindent 8pt
% %   {\myTitleFont Indhold}}\par
% % \vskip 5pt
% \renewcommand{\SBThechapter}{Indhold}
% % {\parindent 20pt
% %   {\mySubTitleFont --- with first lines in italic ---}}
% % \vskip 20pt
% \let\olditem\item
% \let\oldsubitem\subitem
% \let\oldsubsubitem\subsubitem
% \renewcommand{\item}{\par\hangindent=40pt}
% \renewcommand{\subitem}{\par\hangindent=40pt \hspace*{20pt}}
% \renewcommand{\subsubitem}{\par\hangindent=40pt \hspace*{30pt}}

% %\input{unf-sangbog.tocx}

% \renewcommand{\item}{\olditem}
% \renewcommand{\subitem}{\oldsubitem}
% \renewcommand{\subsubitem}{\oldsubsubitem}

%%%
% Songbook begins.
%%%

\twocolumn
%It's just one page, don't print page numbers etc.
\pagestyle{empty}
%Songs included
\input{songs/matmatik.tex}
\input{songs/taal_daj.tex}
\input{songs/linieskriverdriver.tex}
\input{songs/steve_hawking.tex}
\input{songs/ode_til_kode.tex}
\input{songs/se_min_kode.tex}
\input{songs/vaabenfysik_kort.tex}
%Maybe include:
%\input{songs/kvanter_i_maaneskin.tex}
%\input{songs/mest_matematiske_dyr.tex}

% \input{songs/vi_kan_ikke_li.tex}
% \input{songs/selektionssangen.tex}
% \input{songs/alfabetsangen.tex}
% \input{songs/sciencecamps.tex}
% \input{songs/hvad_maa_man.tex}


% \input{songs/lambda_kalkylen.tex}
% \input{songs/puslespil.tex}
% \input{songs/null.tex}
% \input{songs/fasebal.tex}

% \input{songs/chifitter.tex}

% \input{songs/kun_fysik.tex}



% \input{songs/kanoniske.tex}
% \input{songs/jeg_er_en_matematiker_fra_hcoe.tex}


% \input{songs/rekursiv_skovsang.tex}
% \input{songs/laerkerede.tex}


% \clearpage
% \font\myTinySF=cmss8    at  8pt
% \font\myHugeSF=cmssbx10 at 25pt
% % \newcommand{\CpyRtInfoFont}{\tiny\myTinySF}
% % \newcommand{\myTitleFont}{\Huge\myHugeSF}
% % \newcommand{\mySubTitleFont}{\large\sf}
% \renewcommand{\indexspace}{\medskip}

% {\parindent 8pt
%   {\myTitleFont Index}}\par
% \vskip 5pt
% \renewcommand{\SBThechapter}{Index}
% % {\parindent 20pt
% %   {\mySubTitleFont --- with first lines in italic ---}}
% % \vskip 20pt
% \renewcommand{\item}{\par\hangindent=40pt}
% \renewcommand{\subitem}{\par\hangindent=40pt \hspace*{20pt}}
% \renewcommand{\subsubitem}{\par\hangindent=40pt \hspace*{30pt}}

%\input{unf-sangbog.tdx}

\end{document}
\bye
%
%%%
% Document ends.
%%%


\end{document}
\bye
%
%%%
% Document ends.
%%%

%       \begin{minipage}{.8\textwidth}
%         \printcontents{}{1}{}
%       \end{minipage}%
%     \renewcommand{\SBThechapter}{#1}
%     \clearpage
% }

% \titleformat{\chapter}
% [display]
% {}
% {%\vspace*{\fill}
%  % \titlerule[1pt]%
%  % \vspace{1pt}%
%  % \titlerule
%  % \vspace{1pc}%
%  \chaptertitlename}
% {}
% {\Huge}



%%%%%%%%%%%%%%%%%%%%%%%%%%%%%%%%%%%%%%%%%%%%%%%%%%%%%%%%%%
%%%%%%%%%%%%%%%%%%%%%%%%%%%%%%%%%%%%%%%%%%%%%%%%%%%%%%%%%%
%%                                                      %%
%%           D O C U M E N T   B E G I N S              %%
%%                                                      %%
%%%%%%%%%%%%%%%%%%%%%%%%%%%%%%%%%%%%%%%%%%%%%%%%%%%%%%%%%%
%%%%%%%%%%%%%%%%%%%%%%%%%%%%%%%%%%%%%%%%%%%%%%%%%%%%%%%%%%
\begin{document}

%%%
% Uncomment "\maketitle" statement to make a title page.
%%%
%\maketitle
% \begin{titlepage}
%   \centering
%   \vspace{5cm}
% 	\includegraphics[width=1\textwidth]{unf_logo.jpeg}\par\vspace{1cm}
% 	{\scshape\LARGE Sangbog \par}
% 	\vspace{1cm}
% 	{\scshape\Large UNF Computer Science Camp 2019\par}
	
% 	\vfill

% % Bottom of the page
% 	{\large \today\par}
% \end{titlepage}
% \mainmatter
% \ifWordBk
%   \twocolumn
% \fi


%%% Kolofon
%\thispagestyle{empty}
%Sammensat til UNF Computer Science Camp 2019 - csc.unf.dk\\
%Redaktør: Andreas Mosbæk Jensen m.fl. efter tidligere sangbog af Steffen Strunge Mathiesen\\
%Indhold opsat i \LaTeX. 
%Digital version og kildekode: github.com/steffen555/UNF-sangbog\\
%Revision 1 med stave fejl korrektioner
%\par\vspace*{\fill}
%Hvis du har forslag til sange, rettelser, ris og ros, eller hvis du kender en ukendt forfatter, så skriv til sangbog@unf.dk.

%%%
% Turn on and define fancy page heading/footing definition.
%%%
% \pagestyle{fancy}

% \ifChordBk
%   % It's a words & chords songbook...
%   \addtolength{\headwidth}{\marginparsep}
%   \addtolength{\headwidth}{\marginparwidth}
%   \renewcommand{\headrulewidth}{0.4pt}
%   \renewcommand{\footrulewidth}{0.4pt}
%   \fancyhead[LE,RO]{\LHeadFont\emph{\leftmark\/}\SBContinueMark}
%   \fancyhead[CE,CO]{\CHeadFont\thepage}
%   \fancyhead[RE,LO]{\RHeadFont \chaptermark}
% \else\ifOverhead
%   % It's an overhead...
%   \renewcommand{\footrulewidth}{0pt}
%   \renewcommand{\headrulewidth}{0pt}
%   \fancyhead[LE,RO]{}
%   \fancyhead[CE,CO]{}
%   \fancyhead[RE,LO]{}
% \else\ifWordBk
%   % It's a words only songbook...
%   \addtolength{\headwidth}{\marginparsep}
%   \addtolength{\headwidth}{\marginparwidth}
%   \renewcommand{\headrulewidth}{0.4pt}
%   \renewcommand{\footrulewidth}{0.4pt}
%   \fancyhead[LE,RO]{\LHeadFont Naturvidenskab revy sange}
%   \fancyhead[CE,CO]{\CHeadFont\thepage}
%   \fancyhead[RE,LO]{\RHeadFont \SBThechapter}
% \fi\fi\fi

% \fancyfoot[LE,RO]{\LFootFont Computer Science Camp 2019}
% \ifSongEject
%   \fancyfoot[CE,CO]{\CFootFont Last Revised:  \RevDate}
% \else
%   \fancyfoot[CE,CO]{\CFootFont}
% \fi
% \fancyfoot[RE,LO]{\RFootFont Synges på eget ansvar}

%%%
% Table of contents
%%%

% \clearpage
% \twocolumn
% \font\myTinySF=cmss8    at  8pt
% \font\myHugeSF=cmssbx10 at 25pt
% \newcommand{\CpyRtInfoFont}{\tiny\myTinySF}
% \newcommand{\myTitleFont}{\Huge\myHugeSF}
% \newcommand{\mySubTitleFont}{\large\sf}
% \renewcommand{\indexspace}{\medskip}

% % {\parindent 8pt
% %   {\myTitleFont Indhold}}\par
% % \vskip 5pt
% \renewcommand{\SBThechapter}{Indhold}
% % {\parindent 20pt
% %   {\mySubTitleFont --- with first lines in italic ---}}
% % \vskip 20pt
% \let\olditem\item
% \let\oldsubitem\subitem
% \let\oldsubsubitem\subsubitem
% \renewcommand{\item}{\par\hangindent=40pt}
% \renewcommand{\subitem}{\par\hangindent=40pt \hspace*{20pt}}
% \renewcommand{\subsubitem}{\par\hangindent=40pt \hspace*{30pt}}

% %%%%%%% rcsid = @(#)$Id: sample-sb.tex,v 1.23 2010-04-12 18:04:11 rathc Exp $
%%%%%%
%%
%%      ===============================
%%      Sample Songbook (sample-sb.tex)
%%      ===============================
%%
%%      Version 4.5, 30 April, 2010
%%
%%      Copyright 1992--2010 Christopher Rath <christopher@rath.ca>
%%
%%      This package is free software; you can redistribute it and/or
%%      modify it under the terms of version 2.1 of the GNU Lesser
%%	General Public License as published by the Free Software 
%%	Foundation.
%%
%%      This package is distributed in the hope that it will be
%%      useful, but WITHOUT ANY WARRANTY; without even the implied
%%      warranty of MERCHANTABILITY or FITNESS FOR A PARTICULAR
%%      PURPOSE.  See the GNU Lesser General Public License for more
%%      details.
%%
%%      This file contains a subset of the songbook we distribute
%%      at our church.  To the best of my knowledge, all of the lyrics
%%      contained herein are freely distributable.  This file has been
%%      provided as a sample of what can be produced by the chordbk,
%%      wordbk, and overhead LaTeX styles.
%%
%%      NEEDED:  The fancyhdr LaTeX style is required to properly
%%              format this file.  If you don't have that then comment
%%              out the commands in the preamble which deal with the
%%              fancyhdr style.
%%
%%%%%%
%%%%%%
%%
%%      1. Chord notation.  Within this songbook the following
%%         conventions have been adopted:
%%
%%              "Minor" is entered as "m";
%%                      e.g. Cm7 for C minor 7th.
%%              "Major" is entered as "M";
%%                      e.g. CM7 for C major 7th.
%%
%%%%%%
%%%%%%
%%      ============
%%      Bibliography
%%      ============
%%
%%      Exalt Him!: Exalt Him!  Compiled by Tom Fettke.  (c)1989
%%                      Word Music.
%%
%%      Hosanna! Music Books: Hosanna! Music Books #1--#6.
%%                      (c)1987--92 Integrity Music, Inc.
%%
%%      Worship Him II: Worship Him II.  Compiled by Jesse Peterson
%%                      and Bruce Ballinger.  (c)1989 Tempo Music
%%                      Publications.
%%
%%      Worship Songs Of The Vineyard: Worship Songs Of The Vineyard
%%                      --- Volume 2.  (c)1989 Vineyard Ministries
%%                      International.
%%
%%%%%%
%%%%%%

%%%%%%%%%%%%%%%%%%%%%%%%%%%%%%%%%%%%%%%%%%%%%%%%%%%%%%%%%%
%%%%%%%%%%%%%%%%%%%%%%%%%%%%%%%%%%%%%%%%%%%%%%%%%%%%%%%%%%
%%                                                      %%
%%           P R E A M B L E   B E G I N S              %%
%%                                                      %%
%%%%%%%%%%%%%%%%%%%%%%%%%%%%%%%%%%%%%%%%%%%%%%%%%%%%%%%%%%
%%%%%%%%%%%%%%%%%%%%%%%%%%%%%%%%%%%%%%%%%%%%%%%%%%%%%%%%%%

\documentclass[a5paper]{book}
\usepackage{latexsym,
            fancyhdr,
            titlesec,
            amsmath,
            amssymb,
            multicol,
            amsthm,
            stmaryrd,
            amsthm,
            color,
            needspace,
            stackengine,
            wasysym}
\usepackage[utf8]{inputenc}
\usepackage[T1]{fontenc}
% \usepackage[chordbk]{songbook}                  %% Words & Chords edition.
%%\usepackage[compactallsongs,chordbk]{songbook}    %% Words & Chords edition.
\usepackage[wordbk]{songbook}                 %% Words Only edition.
%%\usepackage[overhead]{songbook}               %% Overhead Transparency edition.
\usepackage{titletoc}
\usepackage{tket}  % Draws "TÅGEKAMMERET" correctly

%%%
% Revision Date and Release Date definitions.
%
%       \RelDate - The last time this songbook was released.  Set this
%                  date each time a new release/update of the songbook
%                  is generated.
%       \RevDate - The last time a particular song was revised in any
%                  way.  This command will be renewed inside every
%                  song.
%%%
\newcommand{\RelDate}{31~August,~2003}
\newcommand{\RevDate}{\today}

%%%
% C.C.L.I. license number definition; for copyright licensing info.
% One of these macros will be manually inserted into the {SBMel}
% parameter of the {song} environment.
%
%       \CCLInumber - The actual copyright license number.  Don't
%               insert this command in the {SBMel} parameter, use one
%               of the others.
%       \CCLIed - Indicates a song falls under our CCLI license.
%       \NotCCLIed - Indicates a song doesn't fall under our CCLI
%               license.  Public Domain songs fall into this category.
%       \PGranted - We have received specific permission from the
%               copyright holder to use this song.
%       \PPending - We are in the process of obtaining permission to
%               use this song.
%%%
\newcommand{\CCLInumber}{Your CCLI Number}
\newcommand{\CCLIed}{{\SBMelInfoFont (CCLI \CCLInumber)}}
\newcommand{\NotCCLIed}{\relax}
\newcommand{\PGranted}{\relax}
\newcommand{\PPending}{{\SBMelInfoFont (Permission Pending)}}

%%%
% Title page information.
%%%
%\title{UNF Computer Science Camp 2019 Sangbog}
%\author{}
%\date{Revideret:  \RevDate}

%%%
% Redefine fonts from SongBook style that I don't like.
%%%
\font\myTinySF=cmss8 at 8pt
\renewcommand{\SBMelInfoFont}{\tiny\myTinySF}

%%%
% Define fonts to use in the headers and footers of the songbook.
%%%
\newcommand{\LHeadFont}{\normalsize}            % = cmr12  at 12pt
\newcommand{\CHeadFont}{\normalsize\rm}         % = cmr12  at 12pt
\newcommand{\RHeadFont}{\normalsize}            % = cmr12  at 12pt
\newcommand{\LFootFont}{\scriptsize}            % = cmr8   at  8pt
\newcommand{\CFootFont}{\tiny\myTinySF}         % = cmss8  at  8pt
\newcommand{\RFootFont}{\scriptsize}            % = cmr8   at  8pt

\def\repeat{%
  \stackanchor{.}{.}%
  \rule[-\dp\strutbox]{.3pt}{\normalbaselineskip}%
  \kern0.5pt%
  \rule[-\dp\strutbox]{1pt}{\normalbaselineskip}%
  \kern1pt%
}
\def\frepeat{%
  \kern1pt%
  \rule[-\dp\strutbox]{1pt}{\normalbaselineskip}%
  \kern0.5pt%
  \rule[-\dp\strutbox]{.3pt}{\normalbaselineskip}%
  \stackanchor{.}{.}%
}
% \newcommand{\SBRepeat}[1]{#1\\#1}
\newcommand{\SBRepeat}[1]{\frepeat #1\repeat}
\setcounter{SBSongCnt}{-1}
\renewcommand{\SBWAndMTag}{Forfatter:}
\renewcommand{\SBUnknownTag}{Ukendt}
\renewcommand{\SBChorusTag}{Ref.}
\renewcommand{\SBOrgMel}{Originalmelodi}
\renewcommand{\SpaceAfterChorus}   {\vspace{0ex plus1ex minus 0.5ex}}
\renewcommand{\SpaceAfterOpGroup}  {\vspace{0ex plus1ex minus 0.5ex}}
\renewcommand{\SpaceAfterSBBracket}{\vspace{0ex plus1ex minus 0.5ex}}
\renewcommand{\SpaceAfterSection}  {\vspace{0ex plus1ex minus 0.5ex}}
\renewcommand{\SpaceAfterSong}     {\vspace{0ex plus1ex minus 0.5ex}}
\renewcommand{\SpaceAfterVerse}    {\vspace{0ex plus1ex minus 0.5ex}}

% Tell LaTeX that \medskip is a good place to make a page break
\let\oldmedskip\medskip
\renewcommand{\medskip}{\oldmedskip\pagebreak[2]}

%%%
% Turn on/off index-file generation.  Uncomment the \makeindex line to
% turn index generation on;  comment it out to turn index generation
% off.
%%%
%\makeTitleIndex         %% Title and First Line Index.
%\makeTitleContents      %% Table of Contents.
%\makeKeyIndex           %% Index of song by key.
% \makeArtistIndex	%% Index of song by artist.
% \newcommand{\SBThechapter}[0]{}
% \newcommand{\SBChapter}[1]{
%     \startcontents
%     \chapter*{#1} 
%     % %%%%%% rcsid = @(#)$Id: sample-sb.tex,v 1.23 2010-04-12 18:04:11 rathc Exp $
%%%%%%
%%
%%      ===============================
%%      Sample Songbook (sample-sb.tex)
%%      ===============================
%%
%%      Version 4.5, 30 April, 2010
%%
%%      Copyright 1992--2010 Christopher Rath <christopher@rath.ca>
%%
%%      This package is free software; you can redistribute it and/or
%%      modify it under the terms of version 2.1 of the GNU Lesser
%%	General Public License as published by the Free Software 
%%	Foundation.
%%
%%      This package is distributed in the hope that it will be
%%      useful, but WITHOUT ANY WARRANTY; without even the implied
%%      warranty of MERCHANTABILITY or FITNESS FOR A PARTICULAR
%%      PURPOSE.  See the GNU Lesser General Public License for more
%%      details.
%%
%%      This file contains a subset of the songbook we distribute
%%      at our church.  To the best of my knowledge, all of the lyrics
%%      contained herein are freely distributable.  This file has been
%%      provided as a sample of what can be produced by the chordbk,
%%      wordbk, and overhead LaTeX styles.
%%
%%      NEEDED:  The fancyhdr LaTeX style is required to properly
%%              format this file.  If you don't have that then comment
%%              out the commands in the preamble which deal with the
%%              fancyhdr style.
%%
%%%%%%
%%%%%%
%%
%%      1. Chord notation.  Within this songbook the following
%%         conventions have been adopted:
%%
%%              "Minor" is entered as "m";
%%                      e.g. Cm7 for C minor 7th.
%%              "Major" is entered as "M";
%%                      e.g. CM7 for C major 7th.
%%
%%%%%%
%%%%%%
%%      ============
%%      Bibliography
%%      ============
%%
%%      Exalt Him!: Exalt Him!  Compiled by Tom Fettke.  (c)1989
%%                      Word Music.
%%
%%      Hosanna! Music Books: Hosanna! Music Books #1--#6.
%%                      (c)1987--92 Integrity Music, Inc.
%%
%%      Worship Him II: Worship Him II.  Compiled by Jesse Peterson
%%                      and Bruce Ballinger.  (c)1989 Tempo Music
%%                      Publications.
%%
%%      Worship Songs Of The Vineyard: Worship Songs Of The Vineyard
%%                      --- Volume 2.  (c)1989 Vineyard Ministries
%%                      International.
%%
%%%%%%
%%%%%%

%%%%%%%%%%%%%%%%%%%%%%%%%%%%%%%%%%%%%%%%%%%%%%%%%%%%%%%%%%
%%%%%%%%%%%%%%%%%%%%%%%%%%%%%%%%%%%%%%%%%%%%%%%%%%%%%%%%%%
%%                                                      %%
%%           P R E A M B L E   B E G I N S              %%
%%                                                      %%
%%%%%%%%%%%%%%%%%%%%%%%%%%%%%%%%%%%%%%%%%%%%%%%%%%%%%%%%%%
%%%%%%%%%%%%%%%%%%%%%%%%%%%%%%%%%%%%%%%%%%%%%%%%%%%%%%%%%%

\documentclass[a5paper]{book}
\usepackage{latexsym,
            fancyhdr,
            titlesec,
            amsmath,
            amssymb,
            multicol,
            amsthm,
            stmaryrd,
            amsthm,
            color,
            needspace,
            stackengine,
            wasysym}
\usepackage[utf8]{inputenc}
\usepackage[T1]{fontenc}
% \usepackage[chordbk]{songbook}                  %% Words & Chords edition.
%%\usepackage[compactallsongs,chordbk]{songbook}    %% Words & Chords edition.
\usepackage[wordbk]{songbook}                 %% Words Only edition.
%%\usepackage[overhead]{songbook}               %% Overhead Transparency edition.
\usepackage{titletoc}
\usepackage{tket}  % Draws "TÅGEKAMMERET" correctly

%%%
% Revision Date and Release Date definitions.
%
%       \RelDate - The last time this songbook was released.  Set this
%                  date each time a new release/update of the songbook
%                  is generated.
%       \RevDate - The last time a particular song was revised in any
%                  way.  This command will be renewed inside every
%                  song.
%%%
\newcommand{\RelDate}{31~August,~2003}
\newcommand{\RevDate}{\today}

%%%
% C.C.L.I. license number definition; for copyright licensing info.
% One of these macros will be manually inserted into the {SBMel}
% parameter of the {song} environment.
%
%       \CCLInumber - The actual copyright license number.  Don't
%               insert this command in the {SBMel} parameter, use one
%               of the others.
%       \CCLIed - Indicates a song falls under our CCLI license.
%       \NotCCLIed - Indicates a song doesn't fall under our CCLI
%               license.  Public Domain songs fall into this category.
%       \PGranted - We have received specific permission from the
%               copyright holder to use this song.
%       \PPending - We are in the process of obtaining permission to
%               use this song.
%%%
\newcommand{\CCLInumber}{Your CCLI Number}
\newcommand{\CCLIed}{{\SBMelInfoFont (CCLI \CCLInumber)}}
\newcommand{\NotCCLIed}{\relax}
\newcommand{\PGranted}{\relax}
\newcommand{\PPending}{{\SBMelInfoFont (Permission Pending)}}

%%%
% Title page information.
%%%
%\title{UNF Computer Science Camp 2019 Sangbog}
%\author{}
%\date{Revideret:  \RevDate}

%%%
% Redefine fonts from SongBook style that I don't like.
%%%
\font\myTinySF=cmss8 at 8pt
\renewcommand{\SBMelInfoFont}{\tiny\myTinySF}

%%%
% Define fonts to use in the headers and footers of the songbook.
%%%
\newcommand{\LHeadFont}{\normalsize}            % = cmr12  at 12pt
\newcommand{\CHeadFont}{\normalsize\rm}         % = cmr12  at 12pt
\newcommand{\RHeadFont}{\normalsize}            % = cmr12  at 12pt
\newcommand{\LFootFont}{\scriptsize}            % = cmr8   at  8pt
\newcommand{\CFootFont}{\tiny\myTinySF}         % = cmss8  at  8pt
\newcommand{\RFootFont}{\scriptsize}            % = cmr8   at  8pt

\def\repeat{%
  \stackanchor{.}{.}%
  \rule[-\dp\strutbox]{.3pt}{\normalbaselineskip}%
  \kern0.5pt%
  \rule[-\dp\strutbox]{1pt}{\normalbaselineskip}%
  \kern1pt%
}
\def\frepeat{%
  \kern1pt%
  \rule[-\dp\strutbox]{1pt}{\normalbaselineskip}%
  \kern0.5pt%
  \rule[-\dp\strutbox]{.3pt}{\normalbaselineskip}%
  \stackanchor{.}{.}%
}
% \newcommand{\SBRepeat}[1]{#1\\#1}
\newcommand{\SBRepeat}[1]{\frepeat #1\repeat}
\setcounter{SBSongCnt}{-1}
\renewcommand{\SBWAndMTag}{Forfatter:}
\renewcommand{\SBUnknownTag}{Ukendt}
\renewcommand{\SBChorusTag}{Ref.}
\renewcommand{\SBOrgMel}{Originalmelodi}
\renewcommand{\SpaceAfterChorus}   {\vspace{0ex plus1ex minus 0.5ex}}
\renewcommand{\SpaceAfterOpGroup}  {\vspace{0ex plus1ex minus 0.5ex}}
\renewcommand{\SpaceAfterSBBracket}{\vspace{0ex plus1ex minus 0.5ex}}
\renewcommand{\SpaceAfterSection}  {\vspace{0ex plus1ex minus 0.5ex}}
\renewcommand{\SpaceAfterSong}     {\vspace{0ex plus1ex minus 0.5ex}}
\renewcommand{\SpaceAfterVerse}    {\vspace{0ex plus1ex minus 0.5ex}}

% Tell LaTeX that \medskip is a good place to make a page break
\let\oldmedskip\medskip
\renewcommand{\medskip}{\oldmedskip\pagebreak[2]}

%%%
% Turn on/off index-file generation.  Uncomment the \makeindex line to
% turn index generation on;  comment it out to turn index generation
% off.
%%%
%\makeTitleIndex         %% Title and First Line Index.
%\makeTitleContents      %% Table of Contents.
%\makeKeyIndex           %% Index of song by key.
% \makeArtistIndex	%% Index of song by artist.
% \newcommand{\SBThechapter}[0]{}
% \newcommand{\SBChapter}[1]{
%     \startcontents
%     \chapter*{#1} 
%     % \input{unf-sangbog.toc}
%       \begin{minipage}{.8\textwidth}
%         \printcontents{}{1}{}
%       \end{minipage}%
%     \renewcommand{\SBThechapter}{#1}
%     \clearpage
% }

% \titleformat{\chapter}
% [display]
% {}
% {%\vspace*{\fill}
%  % \titlerule[1pt]%
%  % \vspace{1pt}%
%  % \titlerule
%  % \vspace{1pc}%
%  \chaptertitlename}
% {}
% {\Huge}



%%%%%%%%%%%%%%%%%%%%%%%%%%%%%%%%%%%%%%%%%%%%%%%%%%%%%%%%%%
%%%%%%%%%%%%%%%%%%%%%%%%%%%%%%%%%%%%%%%%%%%%%%%%%%%%%%%%%%
%%                                                      %%
%%           D O C U M E N T   B E G I N S              %%
%%                                                      %%
%%%%%%%%%%%%%%%%%%%%%%%%%%%%%%%%%%%%%%%%%%%%%%%%%%%%%%%%%%
%%%%%%%%%%%%%%%%%%%%%%%%%%%%%%%%%%%%%%%%%%%%%%%%%%%%%%%%%%
\begin{document}

%%%
% Uncomment "\maketitle" statement to make a title page.
%%%
%\maketitle
% \begin{titlepage}
%   \centering
%   \vspace{5cm}
% 	\includegraphics[width=1\textwidth]{unf_logo.jpeg}\par\vspace{1cm}
% 	{\scshape\LARGE Sangbog \par}
% 	\vspace{1cm}
% 	{\scshape\Large UNF Computer Science Camp 2019\par}
	
% 	\vfill

% % Bottom of the page
% 	{\large \today\par}
% \end{titlepage}
% \mainmatter
% \ifWordBk
%   \twocolumn
% \fi


%%% Kolofon
%\thispagestyle{empty}
%Sammensat til UNF Computer Science Camp 2019 - csc.unf.dk\\
%Redaktør: Andreas Mosbæk Jensen m.fl. efter tidligere sangbog af Steffen Strunge Mathiesen\\
%Indhold opsat i \LaTeX. 
%Digital version og kildekode: github.com/steffen555/UNF-sangbog\\
%Revision 1 med stave fejl korrektioner
%\par\vspace*{\fill}
%Hvis du har forslag til sange, rettelser, ris og ros, eller hvis du kender en ukendt forfatter, så skriv til sangbog@unf.dk.

%%%
% Turn on and define fancy page heading/footing definition.
%%%
% \pagestyle{fancy}

% \ifChordBk
%   % It's a words & chords songbook...
%   \addtolength{\headwidth}{\marginparsep}
%   \addtolength{\headwidth}{\marginparwidth}
%   \renewcommand{\headrulewidth}{0.4pt}
%   \renewcommand{\footrulewidth}{0.4pt}
%   \fancyhead[LE,RO]{\LHeadFont\emph{\leftmark\/}\SBContinueMark}
%   \fancyhead[CE,CO]{\CHeadFont\thepage}
%   \fancyhead[RE,LO]{\RHeadFont \chaptermark}
% \else\ifOverhead
%   % It's an overhead...
%   \renewcommand{\footrulewidth}{0pt}
%   \renewcommand{\headrulewidth}{0pt}
%   \fancyhead[LE,RO]{}
%   \fancyhead[CE,CO]{}
%   \fancyhead[RE,LO]{}
% \else\ifWordBk
%   % It's a words only songbook...
%   \addtolength{\headwidth}{\marginparsep}
%   \addtolength{\headwidth}{\marginparwidth}
%   \renewcommand{\headrulewidth}{0.4pt}
%   \renewcommand{\footrulewidth}{0.4pt}
%   \fancyhead[LE,RO]{\LHeadFont Naturvidenskab revy sange}
%   \fancyhead[CE,CO]{\CHeadFont\thepage}
%   \fancyhead[RE,LO]{\RHeadFont \SBThechapter}
% \fi\fi\fi

% \fancyfoot[LE,RO]{\LFootFont Computer Science Camp 2019}
% \ifSongEject
%   \fancyfoot[CE,CO]{\CFootFont Last Revised:  \RevDate}
% \else
%   \fancyfoot[CE,CO]{\CFootFont}
% \fi
% \fancyfoot[RE,LO]{\RFootFont Synges på eget ansvar}

%%%
% Table of contents
%%%

% \clearpage
% \twocolumn
% \font\myTinySF=cmss8    at  8pt
% \font\myHugeSF=cmssbx10 at 25pt
% \newcommand{\CpyRtInfoFont}{\tiny\myTinySF}
% \newcommand{\myTitleFont}{\Huge\myHugeSF}
% \newcommand{\mySubTitleFont}{\large\sf}
% \renewcommand{\indexspace}{\medskip}

% % {\parindent 8pt
% %   {\myTitleFont Indhold}}\par
% % \vskip 5pt
% \renewcommand{\SBThechapter}{Indhold}
% % {\parindent 20pt
% %   {\mySubTitleFont --- with first lines in italic ---}}
% % \vskip 20pt
% \let\olditem\item
% \let\oldsubitem\subitem
% \let\oldsubsubitem\subsubitem
% \renewcommand{\item}{\par\hangindent=40pt}
% \renewcommand{\subitem}{\par\hangindent=40pt \hspace*{20pt}}
% \renewcommand{\subsubitem}{\par\hangindent=40pt \hspace*{30pt}}

% %\input{unf-sangbog.tocx}

% \renewcommand{\item}{\olditem}
% \renewcommand{\subitem}{\oldsubitem}
% \renewcommand{\subsubitem}{\oldsubsubitem}

%%%
% Songbook begins.
%%%

\twocolumn
%It's just one page, don't print page numbers etc.
\pagestyle{empty}
%Songs included
\input{songs/matmatik.tex}
\input{songs/taal_daj.tex}
\input{songs/linieskriverdriver.tex}
\input{songs/steve_hawking.tex}
\input{songs/ode_til_kode.tex}
\input{songs/se_min_kode.tex}
\input{songs/vaabenfysik_kort.tex}
%Maybe include:
%\input{songs/kvanter_i_maaneskin.tex}
%\input{songs/mest_matematiske_dyr.tex}

% \input{songs/vi_kan_ikke_li.tex}
% \input{songs/selektionssangen.tex}
% \input{songs/alfabetsangen.tex}
% \input{songs/sciencecamps.tex}
% \input{songs/hvad_maa_man.tex}


% \input{songs/lambda_kalkylen.tex}
% \input{songs/puslespil.tex}
% \input{songs/null.tex}
% \input{songs/fasebal.tex}

% \input{songs/chifitter.tex}

% \input{songs/kun_fysik.tex}



% \input{songs/kanoniske.tex}
% \input{songs/jeg_er_en_matematiker_fra_hcoe.tex}


% \input{songs/rekursiv_skovsang.tex}
% \input{songs/laerkerede.tex}


% \clearpage
% \font\myTinySF=cmss8    at  8pt
% \font\myHugeSF=cmssbx10 at 25pt
% % \newcommand{\CpyRtInfoFont}{\tiny\myTinySF}
% % \newcommand{\myTitleFont}{\Huge\myHugeSF}
% % \newcommand{\mySubTitleFont}{\large\sf}
% \renewcommand{\indexspace}{\medskip}

% {\parindent 8pt
%   {\myTitleFont Index}}\par
% \vskip 5pt
% \renewcommand{\SBThechapter}{Index}
% % {\parindent 20pt
% %   {\mySubTitleFont --- with first lines in italic ---}}
% % \vskip 20pt
% \renewcommand{\item}{\par\hangindent=40pt}
% \renewcommand{\subitem}{\par\hangindent=40pt \hspace*{20pt}}
% \renewcommand{\subsubitem}{\par\hangindent=40pt \hspace*{30pt}}

%\input{unf-sangbog.tdx}

\end{document}
\bye
%
%%%
% Document ends.
%%%

%       \begin{minipage}{.8\textwidth}
%         \printcontents{}{1}{}
%       \end{minipage}%
%     \renewcommand{\SBThechapter}{#1}
%     \clearpage
% }

% \titleformat{\chapter}
% [display]
% {}
% {%\vspace*{\fill}
%  % \titlerule[1pt]%
%  % \vspace{1pt}%
%  % \titlerule
%  % \vspace{1pc}%
%  \chaptertitlename}
% {}
% {\Huge}



%%%%%%%%%%%%%%%%%%%%%%%%%%%%%%%%%%%%%%%%%%%%%%%%%%%%%%%%%%
%%%%%%%%%%%%%%%%%%%%%%%%%%%%%%%%%%%%%%%%%%%%%%%%%%%%%%%%%%
%%                                                      %%
%%           D O C U M E N T   B E G I N S              %%
%%                                                      %%
%%%%%%%%%%%%%%%%%%%%%%%%%%%%%%%%%%%%%%%%%%%%%%%%%%%%%%%%%%
%%%%%%%%%%%%%%%%%%%%%%%%%%%%%%%%%%%%%%%%%%%%%%%%%%%%%%%%%%
\begin{document}

%%%
% Uncomment "\maketitle" statement to make a title page.
%%%
%\maketitle
% \begin{titlepage}
%   \centering
%   \vspace{5cm}
% 	\includegraphics[width=1\textwidth]{unf_logo.jpeg}\par\vspace{1cm}
% 	{\scshape\LARGE Sangbog \par}
% 	\vspace{1cm}
% 	{\scshape\Large UNF Computer Science Camp 2019\par}
	
% 	\vfill

% % Bottom of the page
% 	{\large \today\par}
% \end{titlepage}
% \mainmatter
% \ifWordBk
%   \twocolumn
% \fi


%%% Kolofon
%\thispagestyle{empty}
%Sammensat til UNF Computer Science Camp 2019 - csc.unf.dk\\
%Redaktør: Andreas Mosbæk Jensen m.fl. efter tidligere sangbog af Steffen Strunge Mathiesen\\
%Indhold opsat i \LaTeX. 
%Digital version og kildekode: github.com/steffen555/UNF-sangbog\\
%Revision 1 med stave fejl korrektioner
%\par\vspace*{\fill}
%Hvis du har forslag til sange, rettelser, ris og ros, eller hvis du kender en ukendt forfatter, så skriv til sangbog@unf.dk.

%%%
% Turn on and define fancy page heading/footing definition.
%%%
% \pagestyle{fancy}

% \ifChordBk
%   % It's a words & chords songbook...
%   \addtolength{\headwidth}{\marginparsep}
%   \addtolength{\headwidth}{\marginparwidth}
%   \renewcommand{\headrulewidth}{0.4pt}
%   \renewcommand{\footrulewidth}{0.4pt}
%   \fancyhead[LE,RO]{\LHeadFont\emph{\leftmark\/}\SBContinueMark}
%   \fancyhead[CE,CO]{\CHeadFont\thepage}
%   \fancyhead[RE,LO]{\RHeadFont \chaptermark}
% \else\ifOverhead
%   % It's an overhead...
%   \renewcommand{\footrulewidth}{0pt}
%   \renewcommand{\headrulewidth}{0pt}
%   \fancyhead[LE,RO]{}
%   \fancyhead[CE,CO]{}
%   \fancyhead[RE,LO]{}
% \else\ifWordBk
%   % It's a words only songbook...
%   \addtolength{\headwidth}{\marginparsep}
%   \addtolength{\headwidth}{\marginparwidth}
%   \renewcommand{\headrulewidth}{0.4pt}
%   \renewcommand{\footrulewidth}{0.4pt}
%   \fancyhead[LE,RO]{\LHeadFont Naturvidenskab revy sange}
%   \fancyhead[CE,CO]{\CHeadFont\thepage}
%   \fancyhead[RE,LO]{\RHeadFont \SBThechapter}
% \fi\fi\fi

% \fancyfoot[LE,RO]{\LFootFont Computer Science Camp 2019}
% \ifSongEject
%   \fancyfoot[CE,CO]{\CFootFont Last Revised:  \RevDate}
% \else
%   \fancyfoot[CE,CO]{\CFootFont}
% \fi
% \fancyfoot[RE,LO]{\RFootFont Synges på eget ansvar}

%%%
% Table of contents
%%%

% \clearpage
% \twocolumn
% \font\myTinySF=cmss8    at  8pt
% \font\myHugeSF=cmssbx10 at 25pt
% \newcommand{\CpyRtInfoFont}{\tiny\myTinySF}
% \newcommand{\myTitleFont}{\Huge\myHugeSF}
% \newcommand{\mySubTitleFont}{\large\sf}
% \renewcommand{\indexspace}{\medskip}

% % {\parindent 8pt
% %   {\myTitleFont Indhold}}\par
% % \vskip 5pt
% \renewcommand{\SBThechapter}{Indhold}
% % {\parindent 20pt
% %   {\mySubTitleFont --- with first lines in italic ---}}
% % \vskip 20pt
% \let\olditem\item
% \let\oldsubitem\subitem
% \let\oldsubsubitem\subsubitem
% \renewcommand{\item}{\par\hangindent=40pt}
% \renewcommand{\subitem}{\par\hangindent=40pt \hspace*{20pt}}
% \renewcommand{\subsubitem}{\par\hangindent=40pt \hspace*{30pt}}

% %%%%%%% rcsid = @(#)$Id: sample-sb.tex,v 1.23 2010-04-12 18:04:11 rathc Exp $
%%%%%%
%%
%%      ===============================
%%      Sample Songbook (sample-sb.tex)
%%      ===============================
%%
%%      Version 4.5, 30 April, 2010
%%
%%      Copyright 1992--2010 Christopher Rath <christopher@rath.ca>
%%
%%      This package is free software; you can redistribute it and/or
%%      modify it under the terms of version 2.1 of the GNU Lesser
%%	General Public License as published by the Free Software 
%%	Foundation.
%%
%%      This package is distributed in the hope that it will be
%%      useful, but WITHOUT ANY WARRANTY; without even the implied
%%      warranty of MERCHANTABILITY or FITNESS FOR A PARTICULAR
%%      PURPOSE.  See the GNU Lesser General Public License for more
%%      details.
%%
%%      This file contains a subset of the songbook we distribute
%%      at our church.  To the best of my knowledge, all of the lyrics
%%      contained herein are freely distributable.  This file has been
%%      provided as a sample of what can be produced by the chordbk,
%%      wordbk, and overhead LaTeX styles.
%%
%%      NEEDED:  The fancyhdr LaTeX style is required to properly
%%              format this file.  If you don't have that then comment
%%              out the commands in the preamble which deal with the
%%              fancyhdr style.
%%
%%%%%%
%%%%%%
%%
%%      1. Chord notation.  Within this songbook the following
%%         conventions have been adopted:
%%
%%              "Minor" is entered as "m";
%%                      e.g. Cm7 for C minor 7th.
%%              "Major" is entered as "M";
%%                      e.g. CM7 for C major 7th.
%%
%%%%%%
%%%%%%
%%      ============
%%      Bibliography
%%      ============
%%
%%      Exalt Him!: Exalt Him!  Compiled by Tom Fettke.  (c)1989
%%                      Word Music.
%%
%%      Hosanna! Music Books: Hosanna! Music Books #1--#6.
%%                      (c)1987--92 Integrity Music, Inc.
%%
%%      Worship Him II: Worship Him II.  Compiled by Jesse Peterson
%%                      and Bruce Ballinger.  (c)1989 Tempo Music
%%                      Publications.
%%
%%      Worship Songs Of The Vineyard: Worship Songs Of The Vineyard
%%                      --- Volume 2.  (c)1989 Vineyard Ministries
%%                      International.
%%
%%%%%%
%%%%%%

%%%%%%%%%%%%%%%%%%%%%%%%%%%%%%%%%%%%%%%%%%%%%%%%%%%%%%%%%%
%%%%%%%%%%%%%%%%%%%%%%%%%%%%%%%%%%%%%%%%%%%%%%%%%%%%%%%%%%
%%                                                      %%
%%           P R E A M B L E   B E G I N S              %%
%%                                                      %%
%%%%%%%%%%%%%%%%%%%%%%%%%%%%%%%%%%%%%%%%%%%%%%%%%%%%%%%%%%
%%%%%%%%%%%%%%%%%%%%%%%%%%%%%%%%%%%%%%%%%%%%%%%%%%%%%%%%%%

\documentclass[a5paper]{book}
\usepackage{latexsym,
            fancyhdr,
            titlesec,
            amsmath,
            amssymb,
            multicol,
            amsthm,
            stmaryrd,
            amsthm,
            color,
            needspace,
            stackengine,
            wasysym}
\usepackage[utf8]{inputenc}
\usepackage[T1]{fontenc}
% \usepackage[chordbk]{songbook}                  %% Words & Chords edition.
%%\usepackage[compactallsongs,chordbk]{songbook}    %% Words & Chords edition.
\usepackage[wordbk]{songbook}                 %% Words Only edition.
%%\usepackage[overhead]{songbook}               %% Overhead Transparency edition.
\usepackage{titletoc}
\usepackage{tket}  % Draws "TÅGEKAMMERET" correctly

%%%
% Revision Date and Release Date definitions.
%
%       \RelDate - The last time this songbook was released.  Set this
%                  date each time a new release/update of the songbook
%                  is generated.
%       \RevDate - The last time a particular song was revised in any
%                  way.  This command will be renewed inside every
%                  song.
%%%
\newcommand{\RelDate}{31~August,~2003}
\newcommand{\RevDate}{\today}

%%%
% C.C.L.I. license number definition; for copyright licensing info.
% One of these macros will be manually inserted into the {SBMel}
% parameter of the {song} environment.
%
%       \CCLInumber - The actual copyright license number.  Don't
%               insert this command in the {SBMel} parameter, use one
%               of the others.
%       \CCLIed - Indicates a song falls under our CCLI license.
%       \NotCCLIed - Indicates a song doesn't fall under our CCLI
%               license.  Public Domain songs fall into this category.
%       \PGranted - We have received specific permission from the
%               copyright holder to use this song.
%       \PPending - We are in the process of obtaining permission to
%               use this song.
%%%
\newcommand{\CCLInumber}{Your CCLI Number}
\newcommand{\CCLIed}{{\SBMelInfoFont (CCLI \CCLInumber)}}
\newcommand{\NotCCLIed}{\relax}
\newcommand{\PGranted}{\relax}
\newcommand{\PPending}{{\SBMelInfoFont (Permission Pending)}}

%%%
% Title page information.
%%%
%\title{UNF Computer Science Camp 2019 Sangbog}
%\author{}
%\date{Revideret:  \RevDate}

%%%
% Redefine fonts from SongBook style that I don't like.
%%%
\font\myTinySF=cmss8 at 8pt
\renewcommand{\SBMelInfoFont}{\tiny\myTinySF}

%%%
% Define fonts to use in the headers and footers of the songbook.
%%%
\newcommand{\LHeadFont}{\normalsize}            % = cmr12  at 12pt
\newcommand{\CHeadFont}{\normalsize\rm}         % = cmr12  at 12pt
\newcommand{\RHeadFont}{\normalsize}            % = cmr12  at 12pt
\newcommand{\LFootFont}{\scriptsize}            % = cmr8   at  8pt
\newcommand{\CFootFont}{\tiny\myTinySF}         % = cmss8  at  8pt
\newcommand{\RFootFont}{\scriptsize}            % = cmr8   at  8pt

\def\repeat{%
  \stackanchor{.}{.}%
  \rule[-\dp\strutbox]{.3pt}{\normalbaselineskip}%
  \kern0.5pt%
  \rule[-\dp\strutbox]{1pt}{\normalbaselineskip}%
  \kern1pt%
}
\def\frepeat{%
  \kern1pt%
  \rule[-\dp\strutbox]{1pt}{\normalbaselineskip}%
  \kern0.5pt%
  \rule[-\dp\strutbox]{.3pt}{\normalbaselineskip}%
  \stackanchor{.}{.}%
}
% \newcommand{\SBRepeat}[1]{#1\\#1}
\newcommand{\SBRepeat}[1]{\frepeat #1\repeat}
\setcounter{SBSongCnt}{-1}
\renewcommand{\SBWAndMTag}{Forfatter:}
\renewcommand{\SBUnknownTag}{Ukendt}
\renewcommand{\SBChorusTag}{Ref.}
\renewcommand{\SBOrgMel}{Originalmelodi}
\renewcommand{\SpaceAfterChorus}   {\vspace{0ex plus1ex minus 0.5ex}}
\renewcommand{\SpaceAfterOpGroup}  {\vspace{0ex plus1ex minus 0.5ex}}
\renewcommand{\SpaceAfterSBBracket}{\vspace{0ex plus1ex minus 0.5ex}}
\renewcommand{\SpaceAfterSection}  {\vspace{0ex plus1ex minus 0.5ex}}
\renewcommand{\SpaceAfterSong}     {\vspace{0ex plus1ex minus 0.5ex}}
\renewcommand{\SpaceAfterVerse}    {\vspace{0ex plus1ex minus 0.5ex}}

% Tell LaTeX that \medskip is a good place to make a page break
\let\oldmedskip\medskip
\renewcommand{\medskip}{\oldmedskip\pagebreak[2]}

%%%
% Turn on/off index-file generation.  Uncomment the \makeindex line to
% turn index generation on;  comment it out to turn index generation
% off.
%%%
%\makeTitleIndex         %% Title and First Line Index.
%\makeTitleContents      %% Table of Contents.
%\makeKeyIndex           %% Index of song by key.
% \makeArtistIndex	%% Index of song by artist.
% \newcommand{\SBThechapter}[0]{}
% \newcommand{\SBChapter}[1]{
%     \startcontents
%     \chapter*{#1} 
%     % \input{unf-sangbog.toc}
%       \begin{minipage}{.8\textwidth}
%         \printcontents{}{1}{}
%       \end{minipage}%
%     \renewcommand{\SBThechapter}{#1}
%     \clearpage
% }

% \titleformat{\chapter}
% [display]
% {}
% {%\vspace*{\fill}
%  % \titlerule[1pt]%
%  % \vspace{1pt}%
%  % \titlerule
%  % \vspace{1pc}%
%  \chaptertitlename}
% {}
% {\Huge}



%%%%%%%%%%%%%%%%%%%%%%%%%%%%%%%%%%%%%%%%%%%%%%%%%%%%%%%%%%
%%%%%%%%%%%%%%%%%%%%%%%%%%%%%%%%%%%%%%%%%%%%%%%%%%%%%%%%%%
%%                                                      %%
%%           D O C U M E N T   B E G I N S              %%
%%                                                      %%
%%%%%%%%%%%%%%%%%%%%%%%%%%%%%%%%%%%%%%%%%%%%%%%%%%%%%%%%%%
%%%%%%%%%%%%%%%%%%%%%%%%%%%%%%%%%%%%%%%%%%%%%%%%%%%%%%%%%%
\begin{document}

%%%
% Uncomment "\maketitle" statement to make a title page.
%%%
%\maketitle
% \begin{titlepage}
%   \centering
%   \vspace{5cm}
% 	\includegraphics[width=1\textwidth]{unf_logo.jpeg}\par\vspace{1cm}
% 	{\scshape\LARGE Sangbog \par}
% 	\vspace{1cm}
% 	{\scshape\Large UNF Computer Science Camp 2019\par}
	
% 	\vfill

% % Bottom of the page
% 	{\large \today\par}
% \end{titlepage}
% \mainmatter
% \ifWordBk
%   \twocolumn
% \fi


%%% Kolofon
%\thispagestyle{empty}
%Sammensat til UNF Computer Science Camp 2019 - csc.unf.dk\\
%Redaktør: Andreas Mosbæk Jensen m.fl. efter tidligere sangbog af Steffen Strunge Mathiesen\\
%Indhold opsat i \LaTeX. 
%Digital version og kildekode: github.com/steffen555/UNF-sangbog\\
%Revision 1 med stave fejl korrektioner
%\par\vspace*{\fill}
%Hvis du har forslag til sange, rettelser, ris og ros, eller hvis du kender en ukendt forfatter, så skriv til sangbog@unf.dk.

%%%
% Turn on and define fancy page heading/footing definition.
%%%
% \pagestyle{fancy}

% \ifChordBk
%   % It's a words & chords songbook...
%   \addtolength{\headwidth}{\marginparsep}
%   \addtolength{\headwidth}{\marginparwidth}
%   \renewcommand{\headrulewidth}{0.4pt}
%   \renewcommand{\footrulewidth}{0.4pt}
%   \fancyhead[LE,RO]{\LHeadFont\emph{\leftmark\/}\SBContinueMark}
%   \fancyhead[CE,CO]{\CHeadFont\thepage}
%   \fancyhead[RE,LO]{\RHeadFont \chaptermark}
% \else\ifOverhead
%   % It's an overhead...
%   \renewcommand{\footrulewidth}{0pt}
%   \renewcommand{\headrulewidth}{0pt}
%   \fancyhead[LE,RO]{}
%   \fancyhead[CE,CO]{}
%   \fancyhead[RE,LO]{}
% \else\ifWordBk
%   % It's a words only songbook...
%   \addtolength{\headwidth}{\marginparsep}
%   \addtolength{\headwidth}{\marginparwidth}
%   \renewcommand{\headrulewidth}{0.4pt}
%   \renewcommand{\footrulewidth}{0.4pt}
%   \fancyhead[LE,RO]{\LHeadFont Naturvidenskab revy sange}
%   \fancyhead[CE,CO]{\CHeadFont\thepage}
%   \fancyhead[RE,LO]{\RHeadFont \SBThechapter}
% \fi\fi\fi

% \fancyfoot[LE,RO]{\LFootFont Computer Science Camp 2019}
% \ifSongEject
%   \fancyfoot[CE,CO]{\CFootFont Last Revised:  \RevDate}
% \else
%   \fancyfoot[CE,CO]{\CFootFont}
% \fi
% \fancyfoot[RE,LO]{\RFootFont Synges på eget ansvar}

%%%
% Table of contents
%%%

% \clearpage
% \twocolumn
% \font\myTinySF=cmss8    at  8pt
% \font\myHugeSF=cmssbx10 at 25pt
% \newcommand{\CpyRtInfoFont}{\tiny\myTinySF}
% \newcommand{\myTitleFont}{\Huge\myHugeSF}
% \newcommand{\mySubTitleFont}{\large\sf}
% \renewcommand{\indexspace}{\medskip}

% % {\parindent 8pt
% %   {\myTitleFont Indhold}}\par
% % \vskip 5pt
% \renewcommand{\SBThechapter}{Indhold}
% % {\parindent 20pt
% %   {\mySubTitleFont --- with first lines in italic ---}}
% % \vskip 20pt
% \let\olditem\item
% \let\oldsubitem\subitem
% \let\oldsubsubitem\subsubitem
% \renewcommand{\item}{\par\hangindent=40pt}
% \renewcommand{\subitem}{\par\hangindent=40pt \hspace*{20pt}}
% \renewcommand{\subsubitem}{\par\hangindent=40pt \hspace*{30pt}}

% %\input{unf-sangbog.tocx}

% \renewcommand{\item}{\olditem}
% \renewcommand{\subitem}{\oldsubitem}
% \renewcommand{\subsubitem}{\oldsubsubitem}

%%%
% Songbook begins.
%%%

\twocolumn
%It's just one page, don't print page numbers etc.
\pagestyle{empty}
%Songs included
\input{songs/matmatik.tex}
\input{songs/taal_daj.tex}
\input{songs/linieskriverdriver.tex}
\input{songs/steve_hawking.tex}
\input{songs/ode_til_kode.tex}
\input{songs/se_min_kode.tex}
\input{songs/vaabenfysik_kort.tex}
%Maybe include:
%\input{songs/kvanter_i_maaneskin.tex}
%\input{songs/mest_matematiske_dyr.tex}

% \input{songs/vi_kan_ikke_li.tex}
% \input{songs/selektionssangen.tex}
% \input{songs/alfabetsangen.tex}
% \input{songs/sciencecamps.tex}
% \input{songs/hvad_maa_man.tex}


% \input{songs/lambda_kalkylen.tex}
% \input{songs/puslespil.tex}
% \input{songs/null.tex}
% \input{songs/fasebal.tex}

% \input{songs/chifitter.tex}

% \input{songs/kun_fysik.tex}



% \input{songs/kanoniske.tex}
% \input{songs/jeg_er_en_matematiker_fra_hcoe.tex}


% \input{songs/rekursiv_skovsang.tex}
% \input{songs/laerkerede.tex}


% \clearpage
% \font\myTinySF=cmss8    at  8pt
% \font\myHugeSF=cmssbx10 at 25pt
% % \newcommand{\CpyRtInfoFont}{\tiny\myTinySF}
% % \newcommand{\myTitleFont}{\Huge\myHugeSF}
% % \newcommand{\mySubTitleFont}{\large\sf}
% \renewcommand{\indexspace}{\medskip}

% {\parindent 8pt
%   {\myTitleFont Index}}\par
% \vskip 5pt
% \renewcommand{\SBThechapter}{Index}
% % {\parindent 20pt
% %   {\mySubTitleFont --- with first lines in italic ---}}
% % \vskip 20pt
% \renewcommand{\item}{\par\hangindent=40pt}
% \renewcommand{\subitem}{\par\hangindent=40pt \hspace*{20pt}}
% \renewcommand{\subsubitem}{\par\hangindent=40pt \hspace*{30pt}}

%\input{unf-sangbog.tdx}

\end{document}
\bye
%
%%%
% Document ends.
%%%


% \renewcommand{\item}{\olditem}
% \renewcommand{\subitem}{\oldsubitem}
% \renewcommand{\subsubitem}{\oldsubsubitem}

%%%
% Songbook begins.
%%%

\twocolumn
%It's just one page, don't print page numbers etc.
\pagestyle{empty}
%Songs included
\input{songs/matmatik.tex}
\input{songs/taal_daj.tex}
\input{songs/linieskriverdriver.tex}
\input{songs/steve_hawking.tex}
\input{songs/ode_til_kode.tex}
\input{songs/se_min_kode.tex}
\input{songs/vaabenfysik_kort.tex}
%Maybe include:
%\input{songs/kvanter_i_maaneskin.tex}
%\input{songs/mest_matematiske_dyr.tex}

% \input{songs/vi_kan_ikke_li.tex}
% \input{songs/selektionssangen.tex}
% \input{songs/alfabetsangen.tex}
% \input{songs/sciencecamps.tex}
% \input{songs/hvad_maa_man.tex}


% \input{songs/lambda_kalkylen.tex}
% \input{songs/puslespil.tex}
% \input{songs/null.tex}
% \input{songs/fasebal.tex}

% \input{songs/chifitter.tex}

% \input{songs/kun_fysik.tex}



% \input{songs/kanoniske.tex}
% \input{songs/jeg_er_en_matematiker_fra_hcoe.tex}


% \input{songs/rekursiv_skovsang.tex}
% \input{songs/laerkerede.tex}


% \clearpage
% \font\myTinySF=cmss8    at  8pt
% \font\myHugeSF=cmssbx10 at 25pt
% % \newcommand{\CpyRtInfoFont}{\tiny\myTinySF}
% % \newcommand{\myTitleFont}{\Huge\myHugeSF}
% % \newcommand{\mySubTitleFont}{\large\sf}
% \renewcommand{\indexspace}{\medskip}

% {\parindent 8pt
%   {\myTitleFont Index}}\par
% \vskip 5pt
% \renewcommand{\SBThechapter}{Index}
% % {\parindent 20pt
% %   {\mySubTitleFont --- with first lines in italic ---}}
% % \vskip 20pt
% \renewcommand{\item}{\par\hangindent=40pt}
% \renewcommand{\subitem}{\par\hangindent=40pt \hspace*{20pt}}
% \renewcommand{\subsubitem}{\par\hangindent=40pt \hspace*{30pt}}

%%%%%%% rcsid = @(#)$Id: sample-sb.tex,v 1.23 2010-04-12 18:04:11 rathc Exp $
%%%%%%
%%
%%      ===============================
%%      Sample Songbook (sample-sb.tex)
%%      ===============================
%%
%%      Version 4.5, 30 April, 2010
%%
%%      Copyright 1992--2010 Christopher Rath <christopher@rath.ca>
%%
%%      This package is free software; you can redistribute it and/or
%%      modify it under the terms of version 2.1 of the GNU Lesser
%%	General Public License as published by the Free Software 
%%	Foundation.
%%
%%      This package is distributed in the hope that it will be
%%      useful, but WITHOUT ANY WARRANTY; without even the implied
%%      warranty of MERCHANTABILITY or FITNESS FOR A PARTICULAR
%%      PURPOSE.  See the GNU Lesser General Public License for more
%%      details.
%%
%%      This file contains a subset of the songbook we distribute
%%      at our church.  To the best of my knowledge, all of the lyrics
%%      contained herein are freely distributable.  This file has been
%%      provided as a sample of what can be produced by the chordbk,
%%      wordbk, and overhead LaTeX styles.
%%
%%      NEEDED:  The fancyhdr LaTeX style is required to properly
%%              format this file.  If you don't have that then comment
%%              out the commands in the preamble which deal with the
%%              fancyhdr style.
%%
%%%%%%
%%%%%%
%%
%%      1. Chord notation.  Within this songbook the following
%%         conventions have been adopted:
%%
%%              "Minor" is entered as "m";
%%                      e.g. Cm7 for C minor 7th.
%%              "Major" is entered as "M";
%%                      e.g. CM7 for C major 7th.
%%
%%%%%%
%%%%%%
%%      ============
%%      Bibliography
%%      ============
%%
%%      Exalt Him!: Exalt Him!  Compiled by Tom Fettke.  (c)1989
%%                      Word Music.
%%
%%      Hosanna! Music Books: Hosanna! Music Books #1--#6.
%%                      (c)1987--92 Integrity Music, Inc.
%%
%%      Worship Him II: Worship Him II.  Compiled by Jesse Peterson
%%                      and Bruce Ballinger.  (c)1989 Tempo Music
%%                      Publications.
%%
%%      Worship Songs Of The Vineyard: Worship Songs Of The Vineyard
%%                      --- Volume 2.  (c)1989 Vineyard Ministries
%%                      International.
%%
%%%%%%
%%%%%%

%%%%%%%%%%%%%%%%%%%%%%%%%%%%%%%%%%%%%%%%%%%%%%%%%%%%%%%%%%
%%%%%%%%%%%%%%%%%%%%%%%%%%%%%%%%%%%%%%%%%%%%%%%%%%%%%%%%%%
%%                                                      %%
%%           P R E A M B L E   B E G I N S              %%
%%                                                      %%
%%%%%%%%%%%%%%%%%%%%%%%%%%%%%%%%%%%%%%%%%%%%%%%%%%%%%%%%%%
%%%%%%%%%%%%%%%%%%%%%%%%%%%%%%%%%%%%%%%%%%%%%%%%%%%%%%%%%%

\documentclass[a5paper]{book}
\usepackage{latexsym,
            fancyhdr,
            titlesec,
            amsmath,
            amssymb,
            multicol,
            amsthm,
            stmaryrd,
            amsthm,
            color,
            needspace,
            stackengine,
            wasysym}
\usepackage[utf8]{inputenc}
\usepackage[T1]{fontenc}
% \usepackage[chordbk]{songbook}                  %% Words & Chords edition.
%%\usepackage[compactallsongs,chordbk]{songbook}    %% Words & Chords edition.
\usepackage[wordbk]{songbook}                 %% Words Only edition.
%%\usepackage[overhead]{songbook}               %% Overhead Transparency edition.
\usepackage{titletoc}
\usepackage{tket}  % Draws "TÅGEKAMMERET" correctly

%%%
% Revision Date and Release Date definitions.
%
%       \RelDate - The last time this songbook was released.  Set this
%                  date each time a new release/update of the songbook
%                  is generated.
%       \RevDate - The last time a particular song was revised in any
%                  way.  This command will be renewed inside every
%                  song.
%%%
\newcommand{\RelDate}{31~August,~2003}
\newcommand{\RevDate}{\today}

%%%
% C.C.L.I. license number definition; for copyright licensing info.
% One of these macros will be manually inserted into the {SBMel}
% parameter of the {song} environment.
%
%       \CCLInumber - The actual copyright license number.  Don't
%               insert this command in the {SBMel} parameter, use one
%               of the others.
%       \CCLIed - Indicates a song falls under our CCLI license.
%       \NotCCLIed - Indicates a song doesn't fall under our CCLI
%               license.  Public Domain songs fall into this category.
%       \PGranted - We have received specific permission from the
%               copyright holder to use this song.
%       \PPending - We are in the process of obtaining permission to
%               use this song.
%%%
\newcommand{\CCLInumber}{Your CCLI Number}
\newcommand{\CCLIed}{{\SBMelInfoFont (CCLI \CCLInumber)}}
\newcommand{\NotCCLIed}{\relax}
\newcommand{\PGranted}{\relax}
\newcommand{\PPending}{{\SBMelInfoFont (Permission Pending)}}

%%%
% Title page information.
%%%
%\title{UNF Computer Science Camp 2019 Sangbog}
%\author{}
%\date{Revideret:  \RevDate}

%%%
% Redefine fonts from SongBook style that I don't like.
%%%
\font\myTinySF=cmss8 at 8pt
\renewcommand{\SBMelInfoFont}{\tiny\myTinySF}

%%%
% Define fonts to use in the headers and footers of the songbook.
%%%
\newcommand{\LHeadFont}{\normalsize}            % = cmr12  at 12pt
\newcommand{\CHeadFont}{\normalsize\rm}         % = cmr12  at 12pt
\newcommand{\RHeadFont}{\normalsize}            % = cmr12  at 12pt
\newcommand{\LFootFont}{\scriptsize}            % = cmr8   at  8pt
\newcommand{\CFootFont}{\tiny\myTinySF}         % = cmss8  at  8pt
\newcommand{\RFootFont}{\scriptsize}            % = cmr8   at  8pt

\def\repeat{%
  \stackanchor{.}{.}%
  \rule[-\dp\strutbox]{.3pt}{\normalbaselineskip}%
  \kern0.5pt%
  \rule[-\dp\strutbox]{1pt}{\normalbaselineskip}%
  \kern1pt%
}
\def\frepeat{%
  \kern1pt%
  \rule[-\dp\strutbox]{1pt}{\normalbaselineskip}%
  \kern0.5pt%
  \rule[-\dp\strutbox]{.3pt}{\normalbaselineskip}%
  \stackanchor{.}{.}%
}
% \newcommand{\SBRepeat}[1]{#1\\#1}
\newcommand{\SBRepeat}[1]{\frepeat #1\repeat}
\setcounter{SBSongCnt}{-1}
\renewcommand{\SBWAndMTag}{Forfatter:}
\renewcommand{\SBUnknownTag}{Ukendt}
\renewcommand{\SBChorusTag}{Ref.}
\renewcommand{\SBOrgMel}{Originalmelodi}
\renewcommand{\SpaceAfterChorus}   {\vspace{0ex plus1ex minus 0.5ex}}
\renewcommand{\SpaceAfterOpGroup}  {\vspace{0ex plus1ex minus 0.5ex}}
\renewcommand{\SpaceAfterSBBracket}{\vspace{0ex plus1ex minus 0.5ex}}
\renewcommand{\SpaceAfterSection}  {\vspace{0ex plus1ex minus 0.5ex}}
\renewcommand{\SpaceAfterSong}     {\vspace{0ex plus1ex minus 0.5ex}}
\renewcommand{\SpaceAfterVerse}    {\vspace{0ex plus1ex minus 0.5ex}}

% Tell LaTeX that \medskip is a good place to make a page break
\let\oldmedskip\medskip
\renewcommand{\medskip}{\oldmedskip\pagebreak[2]}

%%%
% Turn on/off index-file generation.  Uncomment the \makeindex line to
% turn index generation on;  comment it out to turn index generation
% off.
%%%
%\makeTitleIndex         %% Title and First Line Index.
%\makeTitleContents      %% Table of Contents.
%\makeKeyIndex           %% Index of song by key.
% \makeArtistIndex	%% Index of song by artist.
% \newcommand{\SBThechapter}[0]{}
% \newcommand{\SBChapter}[1]{
%     \startcontents
%     \chapter*{#1} 
%     % \input{unf-sangbog.toc}
%       \begin{minipage}{.8\textwidth}
%         \printcontents{}{1}{}
%       \end{minipage}%
%     \renewcommand{\SBThechapter}{#1}
%     \clearpage
% }

% \titleformat{\chapter}
% [display]
% {}
% {%\vspace*{\fill}
%  % \titlerule[1pt]%
%  % \vspace{1pt}%
%  % \titlerule
%  % \vspace{1pc}%
%  \chaptertitlename}
% {}
% {\Huge}



%%%%%%%%%%%%%%%%%%%%%%%%%%%%%%%%%%%%%%%%%%%%%%%%%%%%%%%%%%
%%%%%%%%%%%%%%%%%%%%%%%%%%%%%%%%%%%%%%%%%%%%%%%%%%%%%%%%%%
%%                                                      %%
%%           D O C U M E N T   B E G I N S              %%
%%                                                      %%
%%%%%%%%%%%%%%%%%%%%%%%%%%%%%%%%%%%%%%%%%%%%%%%%%%%%%%%%%%
%%%%%%%%%%%%%%%%%%%%%%%%%%%%%%%%%%%%%%%%%%%%%%%%%%%%%%%%%%
\begin{document}

%%%
% Uncomment "\maketitle" statement to make a title page.
%%%
%\maketitle
% \begin{titlepage}
%   \centering
%   \vspace{5cm}
% 	\includegraphics[width=1\textwidth]{unf_logo.jpeg}\par\vspace{1cm}
% 	{\scshape\LARGE Sangbog \par}
% 	\vspace{1cm}
% 	{\scshape\Large UNF Computer Science Camp 2019\par}
	
% 	\vfill

% % Bottom of the page
% 	{\large \today\par}
% \end{titlepage}
% \mainmatter
% \ifWordBk
%   \twocolumn
% \fi


%%% Kolofon
%\thispagestyle{empty}
%Sammensat til UNF Computer Science Camp 2019 - csc.unf.dk\\
%Redaktør: Andreas Mosbæk Jensen m.fl. efter tidligere sangbog af Steffen Strunge Mathiesen\\
%Indhold opsat i \LaTeX. 
%Digital version og kildekode: github.com/steffen555/UNF-sangbog\\
%Revision 1 med stave fejl korrektioner
%\par\vspace*{\fill}
%Hvis du har forslag til sange, rettelser, ris og ros, eller hvis du kender en ukendt forfatter, så skriv til sangbog@unf.dk.

%%%
% Turn on and define fancy page heading/footing definition.
%%%
% \pagestyle{fancy}

% \ifChordBk
%   % It's a words & chords songbook...
%   \addtolength{\headwidth}{\marginparsep}
%   \addtolength{\headwidth}{\marginparwidth}
%   \renewcommand{\headrulewidth}{0.4pt}
%   \renewcommand{\footrulewidth}{0.4pt}
%   \fancyhead[LE,RO]{\LHeadFont\emph{\leftmark\/}\SBContinueMark}
%   \fancyhead[CE,CO]{\CHeadFont\thepage}
%   \fancyhead[RE,LO]{\RHeadFont \chaptermark}
% \else\ifOverhead
%   % It's an overhead...
%   \renewcommand{\footrulewidth}{0pt}
%   \renewcommand{\headrulewidth}{0pt}
%   \fancyhead[LE,RO]{}
%   \fancyhead[CE,CO]{}
%   \fancyhead[RE,LO]{}
% \else\ifWordBk
%   % It's a words only songbook...
%   \addtolength{\headwidth}{\marginparsep}
%   \addtolength{\headwidth}{\marginparwidth}
%   \renewcommand{\headrulewidth}{0.4pt}
%   \renewcommand{\footrulewidth}{0.4pt}
%   \fancyhead[LE,RO]{\LHeadFont Naturvidenskab revy sange}
%   \fancyhead[CE,CO]{\CHeadFont\thepage}
%   \fancyhead[RE,LO]{\RHeadFont \SBThechapter}
% \fi\fi\fi

% \fancyfoot[LE,RO]{\LFootFont Computer Science Camp 2019}
% \ifSongEject
%   \fancyfoot[CE,CO]{\CFootFont Last Revised:  \RevDate}
% \else
%   \fancyfoot[CE,CO]{\CFootFont}
% \fi
% \fancyfoot[RE,LO]{\RFootFont Synges på eget ansvar}

%%%
% Table of contents
%%%

% \clearpage
% \twocolumn
% \font\myTinySF=cmss8    at  8pt
% \font\myHugeSF=cmssbx10 at 25pt
% \newcommand{\CpyRtInfoFont}{\tiny\myTinySF}
% \newcommand{\myTitleFont}{\Huge\myHugeSF}
% \newcommand{\mySubTitleFont}{\large\sf}
% \renewcommand{\indexspace}{\medskip}

% % {\parindent 8pt
% %   {\myTitleFont Indhold}}\par
% % \vskip 5pt
% \renewcommand{\SBThechapter}{Indhold}
% % {\parindent 20pt
% %   {\mySubTitleFont --- with first lines in italic ---}}
% % \vskip 20pt
% \let\olditem\item
% \let\oldsubitem\subitem
% \let\oldsubsubitem\subsubitem
% \renewcommand{\item}{\par\hangindent=40pt}
% \renewcommand{\subitem}{\par\hangindent=40pt \hspace*{20pt}}
% \renewcommand{\subsubitem}{\par\hangindent=40pt \hspace*{30pt}}

% %\input{unf-sangbog.tocx}

% \renewcommand{\item}{\olditem}
% \renewcommand{\subitem}{\oldsubitem}
% \renewcommand{\subsubitem}{\oldsubsubitem}

%%%
% Songbook begins.
%%%

\twocolumn
%It's just one page, don't print page numbers etc.
\pagestyle{empty}
%Songs included
\input{songs/matmatik.tex}
\input{songs/taal_daj.tex}
\input{songs/linieskriverdriver.tex}
\input{songs/steve_hawking.tex}
\input{songs/ode_til_kode.tex}
\input{songs/se_min_kode.tex}
\input{songs/vaabenfysik_kort.tex}
%Maybe include:
%\input{songs/kvanter_i_maaneskin.tex}
%\input{songs/mest_matematiske_dyr.tex}

% \input{songs/vi_kan_ikke_li.tex}
% \input{songs/selektionssangen.tex}
% \input{songs/alfabetsangen.tex}
% \input{songs/sciencecamps.tex}
% \input{songs/hvad_maa_man.tex}


% \input{songs/lambda_kalkylen.tex}
% \input{songs/puslespil.tex}
% \input{songs/null.tex}
% \input{songs/fasebal.tex}

% \input{songs/chifitter.tex}

% \input{songs/kun_fysik.tex}



% \input{songs/kanoniske.tex}
% \input{songs/jeg_er_en_matematiker_fra_hcoe.tex}


% \input{songs/rekursiv_skovsang.tex}
% \input{songs/laerkerede.tex}


% \clearpage
% \font\myTinySF=cmss8    at  8pt
% \font\myHugeSF=cmssbx10 at 25pt
% % \newcommand{\CpyRtInfoFont}{\tiny\myTinySF}
% % \newcommand{\myTitleFont}{\Huge\myHugeSF}
% % \newcommand{\mySubTitleFont}{\large\sf}
% \renewcommand{\indexspace}{\medskip}

% {\parindent 8pt
%   {\myTitleFont Index}}\par
% \vskip 5pt
% \renewcommand{\SBThechapter}{Index}
% % {\parindent 20pt
% %   {\mySubTitleFont --- with first lines in italic ---}}
% % \vskip 20pt
% \renewcommand{\item}{\par\hangindent=40pt}
% \renewcommand{\subitem}{\par\hangindent=40pt \hspace*{20pt}}
% \renewcommand{\subsubitem}{\par\hangindent=40pt \hspace*{30pt}}

%\input{unf-sangbog.tdx}

\end{document}
\bye
%
%%%
% Document ends.
%%%


\end{document}
\bye
%
%%%
% Document ends.
%%%


% \renewcommand{\item}{\olditem}
% \renewcommand{\subitem}{\oldsubitem}
% \renewcommand{\subsubitem}{\oldsubsubitem}

%%%
% Songbook begins.
%%%

\twocolumn
%It's just one page, don't print page numbers etc.
\pagestyle{empty}
%Songs included
\input{songs/matmatik.tex}
\input{songs/taal_daj.tex}
\input{songs/linieskriverdriver.tex}
\input{songs/steve_hawking.tex}
\input{songs/ode_til_kode.tex}
\input{songs/se_min_kode.tex}
\input{songs/vaabenfysik_kort.tex}
%Maybe include:
%\input{songs/kvanter_i_maaneskin.tex}
%\input{songs/mest_matematiske_dyr.tex}

% \input{songs/vi_kan_ikke_li.tex}
% \input{songs/selektionssangen.tex}
% \input{songs/alfabetsangen.tex}
% \input{songs/sciencecamps.tex}
% \input{songs/hvad_maa_man.tex}


% \input{songs/lambda_kalkylen.tex}
% \input{songs/puslespil.tex}
% \input{songs/null.tex}
% \input{songs/fasebal.tex}

% \input{songs/chifitter.tex}

% \input{songs/kun_fysik.tex}



% \input{songs/kanoniske.tex}
% \input{songs/jeg_er_en_matematiker_fra_hcoe.tex}


% \input{songs/rekursiv_skovsang.tex}
% \input{songs/laerkerede.tex}


% \clearpage
% \font\myTinySF=cmss8    at  8pt
% \font\myHugeSF=cmssbx10 at 25pt
% % \newcommand{\CpyRtInfoFont}{\tiny\myTinySF}
% % \newcommand{\myTitleFont}{\Huge\myHugeSF}
% % \newcommand{\mySubTitleFont}{\large\sf}
% \renewcommand{\indexspace}{\medskip}

% {\parindent 8pt
%   {\myTitleFont Index}}\par
% \vskip 5pt
% \renewcommand{\SBThechapter}{Index}
% % {\parindent 20pt
% %   {\mySubTitleFont --- with first lines in italic ---}}
% % \vskip 20pt
% \renewcommand{\item}{\par\hangindent=40pt}
% \renewcommand{\subitem}{\par\hangindent=40pt \hspace*{20pt}}
% \renewcommand{\subsubitem}{\par\hangindent=40pt \hspace*{30pt}}

%%%%%%% rcsid = @(#)$Id: sample-sb.tex,v 1.23 2010-04-12 18:04:11 rathc Exp $
%%%%%%
%%
%%      ===============================
%%      Sample Songbook (sample-sb.tex)
%%      ===============================
%%
%%      Version 4.5, 30 April, 2010
%%
%%      Copyright 1992--2010 Christopher Rath <christopher@rath.ca>
%%
%%      This package is free software; you can redistribute it and/or
%%      modify it under the terms of version 2.1 of the GNU Lesser
%%	General Public License as published by the Free Software 
%%	Foundation.
%%
%%      This package is distributed in the hope that it will be
%%      useful, but WITHOUT ANY WARRANTY; without even the implied
%%      warranty of MERCHANTABILITY or FITNESS FOR A PARTICULAR
%%      PURPOSE.  See the GNU Lesser General Public License for more
%%      details.
%%
%%      This file contains a subset of the songbook we distribute
%%      at our church.  To the best of my knowledge, all of the lyrics
%%      contained herein are freely distributable.  This file has been
%%      provided as a sample of what can be produced by the chordbk,
%%      wordbk, and overhead LaTeX styles.
%%
%%      NEEDED:  The fancyhdr LaTeX style is required to properly
%%              format this file.  If you don't have that then comment
%%              out the commands in the preamble which deal with the
%%              fancyhdr style.
%%
%%%%%%
%%%%%%
%%
%%      1. Chord notation.  Within this songbook the following
%%         conventions have been adopted:
%%
%%              "Minor" is entered as "m";
%%                      e.g. Cm7 for C minor 7th.
%%              "Major" is entered as "M";
%%                      e.g. CM7 for C major 7th.
%%
%%%%%%
%%%%%%
%%      ============
%%      Bibliography
%%      ============
%%
%%      Exalt Him!: Exalt Him!  Compiled by Tom Fettke.  (c)1989
%%                      Word Music.
%%
%%      Hosanna! Music Books: Hosanna! Music Books #1--#6.
%%                      (c)1987--92 Integrity Music, Inc.
%%
%%      Worship Him II: Worship Him II.  Compiled by Jesse Peterson
%%                      and Bruce Ballinger.  (c)1989 Tempo Music
%%                      Publications.
%%
%%      Worship Songs Of The Vineyard: Worship Songs Of The Vineyard
%%                      --- Volume 2.  (c)1989 Vineyard Ministries
%%                      International.
%%
%%%%%%
%%%%%%

%%%%%%%%%%%%%%%%%%%%%%%%%%%%%%%%%%%%%%%%%%%%%%%%%%%%%%%%%%
%%%%%%%%%%%%%%%%%%%%%%%%%%%%%%%%%%%%%%%%%%%%%%%%%%%%%%%%%%
%%                                                      %%
%%           P R E A M B L E   B E G I N S              %%
%%                                                      %%
%%%%%%%%%%%%%%%%%%%%%%%%%%%%%%%%%%%%%%%%%%%%%%%%%%%%%%%%%%
%%%%%%%%%%%%%%%%%%%%%%%%%%%%%%%%%%%%%%%%%%%%%%%%%%%%%%%%%%

\documentclass[a5paper]{book}
\usepackage{latexsym,
            fancyhdr,
            titlesec,
            amsmath,
            amssymb,
            multicol,
            amsthm,
            stmaryrd,
            amsthm,
            color,
            needspace,
            stackengine,
            wasysym}
\usepackage[utf8]{inputenc}
\usepackage[T1]{fontenc}
% \usepackage[chordbk]{songbook}                  %% Words & Chords edition.
%%\usepackage[compactallsongs,chordbk]{songbook}    %% Words & Chords edition.
\usepackage[wordbk]{songbook}                 %% Words Only edition.
%%\usepackage[overhead]{songbook}               %% Overhead Transparency edition.
\usepackage{titletoc}
\usepackage{tket}  % Draws "TÅGEKAMMERET" correctly

%%%
% Revision Date and Release Date definitions.
%
%       \RelDate - The last time this songbook was released.  Set this
%                  date each time a new release/update of the songbook
%                  is generated.
%       \RevDate - The last time a particular song was revised in any
%                  way.  This command will be renewed inside every
%                  song.
%%%
\newcommand{\RelDate}{31~August,~2003}
\newcommand{\RevDate}{\today}

%%%
% C.C.L.I. license number definition; for copyright licensing info.
% One of these macros will be manually inserted into the {SBMel}
% parameter of the {song} environment.
%
%       \CCLInumber - The actual copyright license number.  Don't
%               insert this command in the {SBMel} parameter, use one
%               of the others.
%       \CCLIed - Indicates a song falls under our CCLI license.
%       \NotCCLIed - Indicates a song doesn't fall under our CCLI
%               license.  Public Domain songs fall into this category.
%       \PGranted - We have received specific permission from the
%               copyright holder to use this song.
%       \PPending - We are in the process of obtaining permission to
%               use this song.
%%%
\newcommand{\CCLInumber}{Your CCLI Number}
\newcommand{\CCLIed}{{\SBMelInfoFont (CCLI \CCLInumber)}}
\newcommand{\NotCCLIed}{\relax}
\newcommand{\PGranted}{\relax}
\newcommand{\PPending}{{\SBMelInfoFont (Permission Pending)}}

%%%
% Title page information.
%%%
%\title{UNF Computer Science Camp 2019 Sangbog}
%\author{}
%\date{Revideret:  \RevDate}

%%%
% Redefine fonts from SongBook style that I don't like.
%%%
\font\myTinySF=cmss8 at 8pt
\renewcommand{\SBMelInfoFont}{\tiny\myTinySF}

%%%
% Define fonts to use in the headers and footers of the songbook.
%%%
\newcommand{\LHeadFont}{\normalsize}            % = cmr12  at 12pt
\newcommand{\CHeadFont}{\normalsize\rm}         % = cmr12  at 12pt
\newcommand{\RHeadFont}{\normalsize}            % = cmr12  at 12pt
\newcommand{\LFootFont}{\scriptsize}            % = cmr8   at  8pt
\newcommand{\CFootFont}{\tiny\myTinySF}         % = cmss8  at  8pt
\newcommand{\RFootFont}{\scriptsize}            % = cmr8   at  8pt

\def\repeat{%
  \stackanchor{.}{.}%
  \rule[-\dp\strutbox]{.3pt}{\normalbaselineskip}%
  \kern0.5pt%
  \rule[-\dp\strutbox]{1pt}{\normalbaselineskip}%
  \kern1pt%
}
\def\frepeat{%
  \kern1pt%
  \rule[-\dp\strutbox]{1pt}{\normalbaselineskip}%
  \kern0.5pt%
  \rule[-\dp\strutbox]{.3pt}{\normalbaselineskip}%
  \stackanchor{.}{.}%
}
% \newcommand{\SBRepeat}[1]{#1\\#1}
\newcommand{\SBRepeat}[1]{\frepeat #1\repeat}
\setcounter{SBSongCnt}{-1}
\renewcommand{\SBWAndMTag}{Forfatter:}
\renewcommand{\SBUnknownTag}{Ukendt}
\renewcommand{\SBChorusTag}{Ref.}
\renewcommand{\SBOrgMel}{Originalmelodi}
\renewcommand{\SpaceAfterChorus}   {\vspace{0ex plus1ex minus 0.5ex}}
\renewcommand{\SpaceAfterOpGroup}  {\vspace{0ex plus1ex minus 0.5ex}}
\renewcommand{\SpaceAfterSBBracket}{\vspace{0ex plus1ex minus 0.5ex}}
\renewcommand{\SpaceAfterSection}  {\vspace{0ex plus1ex minus 0.5ex}}
\renewcommand{\SpaceAfterSong}     {\vspace{0ex plus1ex minus 0.5ex}}
\renewcommand{\SpaceAfterVerse}    {\vspace{0ex plus1ex minus 0.5ex}}

% Tell LaTeX that \medskip is a good place to make a page break
\let\oldmedskip\medskip
\renewcommand{\medskip}{\oldmedskip\pagebreak[2]}

%%%
% Turn on/off index-file generation.  Uncomment the \makeindex line to
% turn index generation on;  comment it out to turn index generation
% off.
%%%
%\makeTitleIndex         %% Title and First Line Index.
%\makeTitleContents      %% Table of Contents.
%\makeKeyIndex           %% Index of song by key.
% \makeArtistIndex	%% Index of song by artist.
% \newcommand{\SBThechapter}[0]{}
% \newcommand{\SBChapter}[1]{
%     \startcontents
%     \chapter*{#1} 
%     % %%%%%% rcsid = @(#)$Id: sample-sb.tex,v 1.23 2010-04-12 18:04:11 rathc Exp $
%%%%%%
%%
%%      ===============================
%%      Sample Songbook (sample-sb.tex)
%%      ===============================
%%
%%      Version 4.5, 30 April, 2010
%%
%%      Copyright 1992--2010 Christopher Rath <christopher@rath.ca>
%%
%%      This package is free software; you can redistribute it and/or
%%      modify it under the terms of version 2.1 of the GNU Lesser
%%	General Public License as published by the Free Software 
%%	Foundation.
%%
%%      This package is distributed in the hope that it will be
%%      useful, but WITHOUT ANY WARRANTY; without even the implied
%%      warranty of MERCHANTABILITY or FITNESS FOR A PARTICULAR
%%      PURPOSE.  See the GNU Lesser General Public License for more
%%      details.
%%
%%      This file contains a subset of the songbook we distribute
%%      at our church.  To the best of my knowledge, all of the lyrics
%%      contained herein are freely distributable.  This file has been
%%      provided as a sample of what can be produced by the chordbk,
%%      wordbk, and overhead LaTeX styles.
%%
%%      NEEDED:  The fancyhdr LaTeX style is required to properly
%%              format this file.  If you don't have that then comment
%%              out the commands in the preamble which deal with the
%%              fancyhdr style.
%%
%%%%%%
%%%%%%
%%
%%      1. Chord notation.  Within this songbook the following
%%         conventions have been adopted:
%%
%%              "Minor" is entered as "m";
%%                      e.g. Cm7 for C minor 7th.
%%              "Major" is entered as "M";
%%                      e.g. CM7 for C major 7th.
%%
%%%%%%
%%%%%%
%%      ============
%%      Bibliography
%%      ============
%%
%%      Exalt Him!: Exalt Him!  Compiled by Tom Fettke.  (c)1989
%%                      Word Music.
%%
%%      Hosanna! Music Books: Hosanna! Music Books #1--#6.
%%                      (c)1987--92 Integrity Music, Inc.
%%
%%      Worship Him II: Worship Him II.  Compiled by Jesse Peterson
%%                      and Bruce Ballinger.  (c)1989 Tempo Music
%%                      Publications.
%%
%%      Worship Songs Of The Vineyard: Worship Songs Of The Vineyard
%%                      --- Volume 2.  (c)1989 Vineyard Ministries
%%                      International.
%%
%%%%%%
%%%%%%

%%%%%%%%%%%%%%%%%%%%%%%%%%%%%%%%%%%%%%%%%%%%%%%%%%%%%%%%%%
%%%%%%%%%%%%%%%%%%%%%%%%%%%%%%%%%%%%%%%%%%%%%%%%%%%%%%%%%%
%%                                                      %%
%%           P R E A M B L E   B E G I N S              %%
%%                                                      %%
%%%%%%%%%%%%%%%%%%%%%%%%%%%%%%%%%%%%%%%%%%%%%%%%%%%%%%%%%%
%%%%%%%%%%%%%%%%%%%%%%%%%%%%%%%%%%%%%%%%%%%%%%%%%%%%%%%%%%

\documentclass[a5paper]{book}
\usepackage{latexsym,
            fancyhdr,
            titlesec,
            amsmath,
            amssymb,
            multicol,
            amsthm,
            stmaryrd,
            amsthm,
            color,
            needspace,
            stackengine,
            wasysym}
\usepackage[utf8]{inputenc}
\usepackage[T1]{fontenc}
% \usepackage[chordbk]{songbook}                  %% Words & Chords edition.
%%\usepackage[compactallsongs,chordbk]{songbook}    %% Words & Chords edition.
\usepackage[wordbk]{songbook}                 %% Words Only edition.
%%\usepackage[overhead]{songbook}               %% Overhead Transparency edition.
\usepackage{titletoc}
\usepackage{tket}  % Draws "TÅGEKAMMERET" correctly

%%%
% Revision Date and Release Date definitions.
%
%       \RelDate - The last time this songbook was released.  Set this
%                  date each time a new release/update of the songbook
%                  is generated.
%       \RevDate - The last time a particular song was revised in any
%                  way.  This command will be renewed inside every
%                  song.
%%%
\newcommand{\RelDate}{31~August,~2003}
\newcommand{\RevDate}{\today}

%%%
% C.C.L.I. license number definition; for copyright licensing info.
% One of these macros will be manually inserted into the {SBMel}
% parameter of the {song} environment.
%
%       \CCLInumber - The actual copyright license number.  Don't
%               insert this command in the {SBMel} parameter, use one
%               of the others.
%       \CCLIed - Indicates a song falls under our CCLI license.
%       \NotCCLIed - Indicates a song doesn't fall under our CCLI
%               license.  Public Domain songs fall into this category.
%       \PGranted - We have received specific permission from the
%               copyright holder to use this song.
%       \PPending - We are in the process of obtaining permission to
%               use this song.
%%%
\newcommand{\CCLInumber}{Your CCLI Number}
\newcommand{\CCLIed}{{\SBMelInfoFont (CCLI \CCLInumber)}}
\newcommand{\NotCCLIed}{\relax}
\newcommand{\PGranted}{\relax}
\newcommand{\PPending}{{\SBMelInfoFont (Permission Pending)}}

%%%
% Title page information.
%%%
%\title{UNF Computer Science Camp 2019 Sangbog}
%\author{}
%\date{Revideret:  \RevDate}

%%%
% Redefine fonts from SongBook style that I don't like.
%%%
\font\myTinySF=cmss8 at 8pt
\renewcommand{\SBMelInfoFont}{\tiny\myTinySF}

%%%
% Define fonts to use in the headers and footers of the songbook.
%%%
\newcommand{\LHeadFont}{\normalsize}            % = cmr12  at 12pt
\newcommand{\CHeadFont}{\normalsize\rm}         % = cmr12  at 12pt
\newcommand{\RHeadFont}{\normalsize}            % = cmr12  at 12pt
\newcommand{\LFootFont}{\scriptsize}            % = cmr8   at  8pt
\newcommand{\CFootFont}{\tiny\myTinySF}         % = cmss8  at  8pt
\newcommand{\RFootFont}{\scriptsize}            % = cmr8   at  8pt

\def\repeat{%
  \stackanchor{.}{.}%
  \rule[-\dp\strutbox]{.3pt}{\normalbaselineskip}%
  \kern0.5pt%
  \rule[-\dp\strutbox]{1pt}{\normalbaselineskip}%
  \kern1pt%
}
\def\frepeat{%
  \kern1pt%
  \rule[-\dp\strutbox]{1pt}{\normalbaselineskip}%
  \kern0.5pt%
  \rule[-\dp\strutbox]{.3pt}{\normalbaselineskip}%
  \stackanchor{.}{.}%
}
% \newcommand{\SBRepeat}[1]{#1\\#1}
\newcommand{\SBRepeat}[1]{\frepeat #1\repeat}
\setcounter{SBSongCnt}{-1}
\renewcommand{\SBWAndMTag}{Forfatter:}
\renewcommand{\SBUnknownTag}{Ukendt}
\renewcommand{\SBChorusTag}{Ref.}
\renewcommand{\SBOrgMel}{Originalmelodi}
\renewcommand{\SpaceAfterChorus}   {\vspace{0ex plus1ex minus 0.5ex}}
\renewcommand{\SpaceAfterOpGroup}  {\vspace{0ex plus1ex minus 0.5ex}}
\renewcommand{\SpaceAfterSBBracket}{\vspace{0ex plus1ex minus 0.5ex}}
\renewcommand{\SpaceAfterSection}  {\vspace{0ex plus1ex minus 0.5ex}}
\renewcommand{\SpaceAfterSong}     {\vspace{0ex plus1ex minus 0.5ex}}
\renewcommand{\SpaceAfterVerse}    {\vspace{0ex plus1ex minus 0.5ex}}

% Tell LaTeX that \medskip is a good place to make a page break
\let\oldmedskip\medskip
\renewcommand{\medskip}{\oldmedskip\pagebreak[2]}

%%%
% Turn on/off index-file generation.  Uncomment the \makeindex line to
% turn index generation on;  comment it out to turn index generation
% off.
%%%
%\makeTitleIndex         %% Title and First Line Index.
%\makeTitleContents      %% Table of Contents.
%\makeKeyIndex           %% Index of song by key.
% \makeArtistIndex	%% Index of song by artist.
% \newcommand{\SBThechapter}[0]{}
% \newcommand{\SBChapter}[1]{
%     \startcontents
%     \chapter*{#1} 
%     % \input{unf-sangbog.toc}
%       \begin{minipage}{.8\textwidth}
%         \printcontents{}{1}{}
%       \end{minipage}%
%     \renewcommand{\SBThechapter}{#1}
%     \clearpage
% }

% \titleformat{\chapter}
% [display]
% {}
% {%\vspace*{\fill}
%  % \titlerule[1pt]%
%  % \vspace{1pt}%
%  % \titlerule
%  % \vspace{1pc}%
%  \chaptertitlename}
% {}
% {\Huge}



%%%%%%%%%%%%%%%%%%%%%%%%%%%%%%%%%%%%%%%%%%%%%%%%%%%%%%%%%%
%%%%%%%%%%%%%%%%%%%%%%%%%%%%%%%%%%%%%%%%%%%%%%%%%%%%%%%%%%
%%                                                      %%
%%           D O C U M E N T   B E G I N S              %%
%%                                                      %%
%%%%%%%%%%%%%%%%%%%%%%%%%%%%%%%%%%%%%%%%%%%%%%%%%%%%%%%%%%
%%%%%%%%%%%%%%%%%%%%%%%%%%%%%%%%%%%%%%%%%%%%%%%%%%%%%%%%%%
\begin{document}

%%%
% Uncomment "\maketitle" statement to make a title page.
%%%
%\maketitle
% \begin{titlepage}
%   \centering
%   \vspace{5cm}
% 	\includegraphics[width=1\textwidth]{unf_logo.jpeg}\par\vspace{1cm}
% 	{\scshape\LARGE Sangbog \par}
% 	\vspace{1cm}
% 	{\scshape\Large UNF Computer Science Camp 2019\par}
	
% 	\vfill

% % Bottom of the page
% 	{\large \today\par}
% \end{titlepage}
% \mainmatter
% \ifWordBk
%   \twocolumn
% \fi


%%% Kolofon
%\thispagestyle{empty}
%Sammensat til UNF Computer Science Camp 2019 - csc.unf.dk\\
%Redaktør: Andreas Mosbæk Jensen m.fl. efter tidligere sangbog af Steffen Strunge Mathiesen\\
%Indhold opsat i \LaTeX. 
%Digital version og kildekode: github.com/steffen555/UNF-sangbog\\
%Revision 1 med stave fejl korrektioner
%\par\vspace*{\fill}
%Hvis du har forslag til sange, rettelser, ris og ros, eller hvis du kender en ukendt forfatter, så skriv til sangbog@unf.dk.

%%%
% Turn on and define fancy page heading/footing definition.
%%%
% \pagestyle{fancy}

% \ifChordBk
%   % It's a words & chords songbook...
%   \addtolength{\headwidth}{\marginparsep}
%   \addtolength{\headwidth}{\marginparwidth}
%   \renewcommand{\headrulewidth}{0.4pt}
%   \renewcommand{\footrulewidth}{0.4pt}
%   \fancyhead[LE,RO]{\LHeadFont\emph{\leftmark\/}\SBContinueMark}
%   \fancyhead[CE,CO]{\CHeadFont\thepage}
%   \fancyhead[RE,LO]{\RHeadFont \chaptermark}
% \else\ifOverhead
%   % It's an overhead...
%   \renewcommand{\footrulewidth}{0pt}
%   \renewcommand{\headrulewidth}{0pt}
%   \fancyhead[LE,RO]{}
%   \fancyhead[CE,CO]{}
%   \fancyhead[RE,LO]{}
% \else\ifWordBk
%   % It's a words only songbook...
%   \addtolength{\headwidth}{\marginparsep}
%   \addtolength{\headwidth}{\marginparwidth}
%   \renewcommand{\headrulewidth}{0.4pt}
%   \renewcommand{\footrulewidth}{0.4pt}
%   \fancyhead[LE,RO]{\LHeadFont Naturvidenskab revy sange}
%   \fancyhead[CE,CO]{\CHeadFont\thepage}
%   \fancyhead[RE,LO]{\RHeadFont \SBThechapter}
% \fi\fi\fi

% \fancyfoot[LE,RO]{\LFootFont Computer Science Camp 2019}
% \ifSongEject
%   \fancyfoot[CE,CO]{\CFootFont Last Revised:  \RevDate}
% \else
%   \fancyfoot[CE,CO]{\CFootFont}
% \fi
% \fancyfoot[RE,LO]{\RFootFont Synges på eget ansvar}

%%%
% Table of contents
%%%

% \clearpage
% \twocolumn
% \font\myTinySF=cmss8    at  8pt
% \font\myHugeSF=cmssbx10 at 25pt
% \newcommand{\CpyRtInfoFont}{\tiny\myTinySF}
% \newcommand{\myTitleFont}{\Huge\myHugeSF}
% \newcommand{\mySubTitleFont}{\large\sf}
% \renewcommand{\indexspace}{\medskip}

% % {\parindent 8pt
% %   {\myTitleFont Indhold}}\par
% % \vskip 5pt
% \renewcommand{\SBThechapter}{Indhold}
% % {\parindent 20pt
% %   {\mySubTitleFont --- with first lines in italic ---}}
% % \vskip 20pt
% \let\olditem\item
% \let\oldsubitem\subitem
% \let\oldsubsubitem\subsubitem
% \renewcommand{\item}{\par\hangindent=40pt}
% \renewcommand{\subitem}{\par\hangindent=40pt \hspace*{20pt}}
% \renewcommand{\subsubitem}{\par\hangindent=40pt \hspace*{30pt}}

% %\input{unf-sangbog.tocx}

% \renewcommand{\item}{\olditem}
% \renewcommand{\subitem}{\oldsubitem}
% \renewcommand{\subsubitem}{\oldsubsubitem}

%%%
% Songbook begins.
%%%

\twocolumn
%It's just one page, don't print page numbers etc.
\pagestyle{empty}
%Songs included
\input{songs/matmatik.tex}
\input{songs/taal_daj.tex}
\input{songs/linieskriverdriver.tex}
\input{songs/steve_hawking.tex}
\input{songs/ode_til_kode.tex}
\input{songs/se_min_kode.tex}
\input{songs/vaabenfysik_kort.tex}
%Maybe include:
%\input{songs/kvanter_i_maaneskin.tex}
%\input{songs/mest_matematiske_dyr.tex}

% \input{songs/vi_kan_ikke_li.tex}
% \input{songs/selektionssangen.tex}
% \input{songs/alfabetsangen.tex}
% \input{songs/sciencecamps.tex}
% \input{songs/hvad_maa_man.tex}


% \input{songs/lambda_kalkylen.tex}
% \input{songs/puslespil.tex}
% \input{songs/null.tex}
% \input{songs/fasebal.tex}

% \input{songs/chifitter.tex}

% \input{songs/kun_fysik.tex}



% \input{songs/kanoniske.tex}
% \input{songs/jeg_er_en_matematiker_fra_hcoe.tex}


% \input{songs/rekursiv_skovsang.tex}
% \input{songs/laerkerede.tex}


% \clearpage
% \font\myTinySF=cmss8    at  8pt
% \font\myHugeSF=cmssbx10 at 25pt
% % \newcommand{\CpyRtInfoFont}{\tiny\myTinySF}
% % \newcommand{\myTitleFont}{\Huge\myHugeSF}
% % \newcommand{\mySubTitleFont}{\large\sf}
% \renewcommand{\indexspace}{\medskip}

% {\parindent 8pt
%   {\myTitleFont Index}}\par
% \vskip 5pt
% \renewcommand{\SBThechapter}{Index}
% % {\parindent 20pt
% %   {\mySubTitleFont --- with first lines in italic ---}}
% % \vskip 20pt
% \renewcommand{\item}{\par\hangindent=40pt}
% \renewcommand{\subitem}{\par\hangindent=40pt \hspace*{20pt}}
% \renewcommand{\subsubitem}{\par\hangindent=40pt \hspace*{30pt}}

%\input{unf-sangbog.tdx}

\end{document}
\bye
%
%%%
% Document ends.
%%%

%       \begin{minipage}{.8\textwidth}
%         \printcontents{}{1}{}
%       \end{minipage}%
%     \renewcommand{\SBThechapter}{#1}
%     \clearpage
% }

% \titleformat{\chapter}
% [display]
% {}
% {%\vspace*{\fill}
%  % \titlerule[1pt]%
%  % \vspace{1pt}%
%  % \titlerule
%  % \vspace{1pc}%
%  \chaptertitlename}
% {}
% {\Huge}



%%%%%%%%%%%%%%%%%%%%%%%%%%%%%%%%%%%%%%%%%%%%%%%%%%%%%%%%%%
%%%%%%%%%%%%%%%%%%%%%%%%%%%%%%%%%%%%%%%%%%%%%%%%%%%%%%%%%%
%%                                                      %%
%%           D O C U M E N T   B E G I N S              %%
%%                                                      %%
%%%%%%%%%%%%%%%%%%%%%%%%%%%%%%%%%%%%%%%%%%%%%%%%%%%%%%%%%%
%%%%%%%%%%%%%%%%%%%%%%%%%%%%%%%%%%%%%%%%%%%%%%%%%%%%%%%%%%
\begin{document}

%%%
% Uncomment "\maketitle" statement to make a title page.
%%%
%\maketitle
% \begin{titlepage}
%   \centering
%   \vspace{5cm}
% 	\includegraphics[width=1\textwidth]{unf_logo.jpeg}\par\vspace{1cm}
% 	{\scshape\LARGE Sangbog \par}
% 	\vspace{1cm}
% 	{\scshape\Large UNF Computer Science Camp 2019\par}
	
% 	\vfill

% % Bottom of the page
% 	{\large \today\par}
% \end{titlepage}
% \mainmatter
% \ifWordBk
%   \twocolumn
% \fi


%%% Kolofon
%\thispagestyle{empty}
%Sammensat til UNF Computer Science Camp 2019 - csc.unf.dk\\
%Redaktør: Andreas Mosbæk Jensen m.fl. efter tidligere sangbog af Steffen Strunge Mathiesen\\
%Indhold opsat i \LaTeX. 
%Digital version og kildekode: github.com/steffen555/UNF-sangbog\\
%Revision 1 med stave fejl korrektioner
%\par\vspace*{\fill}
%Hvis du har forslag til sange, rettelser, ris og ros, eller hvis du kender en ukendt forfatter, så skriv til sangbog@unf.dk.

%%%
% Turn on and define fancy page heading/footing definition.
%%%
% \pagestyle{fancy}

% \ifChordBk
%   % It's a words & chords songbook...
%   \addtolength{\headwidth}{\marginparsep}
%   \addtolength{\headwidth}{\marginparwidth}
%   \renewcommand{\headrulewidth}{0.4pt}
%   \renewcommand{\footrulewidth}{0.4pt}
%   \fancyhead[LE,RO]{\LHeadFont\emph{\leftmark\/}\SBContinueMark}
%   \fancyhead[CE,CO]{\CHeadFont\thepage}
%   \fancyhead[RE,LO]{\RHeadFont \chaptermark}
% \else\ifOverhead
%   % It's an overhead...
%   \renewcommand{\footrulewidth}{0pt}
%   \renewcommand{\headrulewidth}{0pt}
%   \fancyhead[LE,RO]{}
%   \fancyhead[CE,CO]{}
%   \fancyhead[RE,LO]{}
% \else\ifWordBk
%   % It's a words only songbook...
%   \addtolength{\headwidth}{\marginparsep}
%   \addtolength{\headwidth}{\marginparwidth}
%   \renewcommand{\headrulewidth}{0.4pt}
%   \renewcommand{\footrulewidth}{0.4pt}
%   \fancyhead[LE,RO]{\LHeadFont Naturvidenskab revy sange}
%   \fancyhead[CE,CO]{\CHeadFont\thepage}
%   \fancyhead[RE,LO]{\RHeadFont \SBThechapter}
% \fi\fi\fi

% \fancyfoot[LE,RO]{\LFootFont Computer Science Camp 2019}
% \ifSongEject
%   \fancyfoot[CE,CO]{\CFootFont Last Revised:  \RevDate}
% \else
%   \fancyfoot[CE,CO]{\CFootFont}
% \fi
% \fancyfoot[RE,LO]{\RFootFont Synges på eget ansvar}

%%%
% Table of contents
%%%

% \clearpage
% \twocolumn
% \font\myTinySF=cmss8    at  8pt
% \font\myHugeSF=cmssbx10 at 25pt
% \newcommand{\CpyRtInfoFont}{\tiny\myTinySF}
% \newcommand{\myTitleFont}{\Huge\myHugeSF}
% \newcommand{\mySubTitleFont}{\large\sf}
% \renewcommand{\indexspace}{\medskip}

% % {\parindent 8pt
% %   {\myTitleFont Indhold}}\par
% % \vskip 5pt
% \renewcommand{\SBThechapter}{Indhold}
% % {\parindent 20pt
% %   {\mySubTitleFont --- with first lines in italic ---}}
% % \vskip 20pt
% \let\olditem\item
% \let\oldsubitem\subitem
% \let\oldsubsubitem\subsubitem
% \renewcommand{\item}{\par\hangindent=40pt}
% \renewcommand{\subitem}{\par\hangindent=40pt \hspace*{20pt}}
% \renewcommand{\subsubitem}{\par\hangindent=40pt \hspace*{30pt}}

% %%%%%%% rcsid = @(#)$Id: sample-sb.tex,v 1.23 2010-04-12 18:04:11 rathc Exp $
%%%%%%
%%
%%      ===============================
%%      Sample Songbook (sample-sb.tex)
%%      ===============================
%%
%%      Version 4.5, 30 April, 2010
%%
%%      Copyright 1992--2010 Christopher Rath <christopher@rath.ca>
%%
%%      This package is free software; you can redistribute it and/or
%%      modify it under the terms of version 2.1 of the GNU Lesser
%%	General Public License as published by the Free Software 
%%	Foundation.
%%
%%      This package is distributed in the hope that it will be
%%      useful, but WITHOUT ANY WARRANTY; without even the implied
%%      warranty of MERCHANTABILITY or FITNESS FOR A PARTICULAR
%%      PURPOSE.  See the GNU Lesser General Public License for more
%%      details.
%%
%%      This file contains a subset of the songbook we distribute
%%      at our church.  To the best of my knowledge, all of the lyrics
%%      contained herein are freely distributable.  This file has been
%%      provided as a sample of what can be produced by the chordbk,
%%      wordbk, and overhead LaTeX styles.
%%
%%      NEEDED:  The fancyhdr LaTeX style is required to properly
%%              format this file.  If you don't have that then comment
%%              out the commands in the preamble which deal with the
%%              fancyhdr style.
%%
%%%%%%
%%%%%%
%%
%%      1. Chord notation.  Within this songbook the following
%%         conventions have been adopted:
%%
%%              "Minor" is entered as "m";
%%                      e.g. Cm7 for C minor 7th.
%%              "Major" is entered as "M";
%%                      e.g. CM7 for C major 7th.
%%
%%%%%%
%%%%%%
%%      ============
%%      Bibliography
%%      ============
%%
%%      Exalt Him!: Exalt Him!  Compiled by Tom Fettke.  (c)1989
%%                      Word Music.
%%
%%      Hosanna! Music Books: Hosanna! Music Books #1--#6.
%%                      (c)1987--92 Integrity Music, Inc.
%%
%%      Worship Him II: Worship Him II.  Compiled by Jesse Peterson
%%                      and Bruce Ballinger.  (c)1989 Tempo Music
%%                      Publications.
%%
%%      Worship Songs Of The Vineyard: Worship Songs Of The Vineyard
%%                      --- Volume 2.  (c)1989 Vineyard Ministries
%%                      International.
%%
%%%%%%
%%%%%%

%%%%%%%%%%%%%%%%%%%%%%%%%%%%%%%%%%%%%%%%%%%%%%%%%%%%%%%%%%
%%%%%%%%%%%%%%%%%%%%%%%%%%%%%%%%%%%%%%%%%%%%%%%%%%%%%%%%%%
%%                                                      %%
%%           P R E A M B L E   B E G I N S              %%
%%                                                      %%
%%%%%%%%%%%%%%%%%%%%%%%%%%%%%%%%%%%%%%%%%%%%%%%%%%%%%%%%%%
%%%%%%%%%%%%%%%%%%%%%%%%%%%%%%%%%%%%%%%%%%%%%%%%%%%%%%%%%%

\documentclass[a5paper]{book}
\usepackage{latexsym,
            fancyhdr,
            titlesec,
            amsmath,
            amssymb,
            multicol,
            amsthm,
            stmaryrd,
            amsthm,
            color,
            needspace,
            stackengine,
            wasysym}
\usepackage[utf8]{inputenc}
\usepackage[T1]{fontenc}
% \usepackage[chordbk]{songbook}                  %% Words & Chords edition.
%%\usepackage[compactallsongs,chordbk]{songbook}    %% Words & Chords edition.
\usepackage[wordbk]{songbook}                 %% Words Only edition.
%%\usepackage[overhead]{songbook}               %% Overhead Transparency edition.
\usepackage{titletoc}
\usepackage{tket}  % Draws "TÅGEKAMMERET" correctly

%%%
% Revision Date and Release Date definitions.
%
%       \RelDate - The last time this songbook was released.  Set this
%                  date each time a new release/update of the songbook
%                  is generated.
%       \RevDate - The last time a particular song was revised in any
%                  way.  This command will be renewed inside every
%                  song.
%%%
\newcommand{\RelDate}{31~August,~2003}
\newcommand{\RevDate}{\today}

%%%
% C.C.L.I. license number definition; for copyright licensing info.
% One of these macros will be manually inserted into the {SBMel}
% parameter of the {song} environment.
%
%       \CCLInumber - The actual copyright license number.  Don't
%               insert this command in the {SBMel} parameter, use one
%               of the others.
%       \CCLIed - Indicates a song falls under our CCLI license.
%       \NotCCLIed - Indicates a song doesn't fall under our CCLI
%               license.  Public Domain songs fall into this category.
%       \PGranted - We have received specific permission from the
%               copyright holder to use this song.
%       \PPending - We are in the process of obtaining permission to
%               use this song.
%%%
\newcommand{\CCLInumber}{Your CCLI Number}
\newcommand{\CCLIed}{{\SBMelInfoFont (CCLI \CCLInumber)}}
\newcommand{\NotCCLIed}{\relax}
\newcommand{\PGranted}{\relax}
\newcommand{\PPending}{{\SBMelInfoFont (Permission Pending)}}

%%%
% Title page information.
%%%
%\title{UNF Computer Science Camp 2019 Sangbog}
%\author{}
%\date{Revideret:  \RevDate}

%%%
% Redefine fonts from SongBook style that I don't like.
%%%
\font\myTinySF=cmss8 at 8pt
\renewcommand{\SBMelInfoFont}{\tiny\myTinySF}

%%%
% Define fonts to use in the headers and footers of the songbook.
%%%
\newcommand{\LHeadFont}{\normalsize}            % = cmr12  at 12pt
\newcommand{\CHeadFont}{\normalsize\rm}         % = cmr12  at 12pt
\newcommand{\RHeadFont}{\normalsize}            % = cmr12  at 12pt
\newcommand{\LFootFont}{\scriptsize}            % = cmr8   at  8pt
\newcommand{\CFootFont}{\tiny\myTinySF}         % = cmss8  at  8pt
\newcommand{\RFootFont}{\scriptsize}            % = cmr8   at  8pt

\def\repeat{%
  \stackanchor{.}{.}%
  \rule[-\dp\strutbox]{.3pt}{\normalbaselineskip}%
  \kern0.5pt%
  \rule[-\dp\strutbox]{1pt}{\normalbaselineskip}%
  \kern1pt%
}
\def\frepeat{%
  \kern1pt%
  \rule[-\dp\strutbox]{1pt}{\normalbaselineskip}%
  \kern0.5pt%
  \rule[-\dp\strutbox]{.3pt}{\normalbaselineskip}%
  \stackanchor{.}{.}%
}
% \newcommand{\SBRepeat}[1]{#1\\#1}
\newcommand{\SBRepeat}[1]{\frepeat #1\repeat}
\setcounter{SBSongCnt}{-1}
\renewcommand{\SBWAndMTag}{Forfatter:}
\renewcommand{\SBUnknownTag}{Ukendt}
\renewcommand{\SBChorusTag}{Ref.}
\renewcommand{\SBOrgMel}{Originalmelodi}
\renewcommand{\SpaceAfterChorus}   {\vspace{0ex plus1ex minus 0.5ex}}
\renewcommand{\SpaceAfterOpGroup}  {\vspace{0ex plus1ex minus 0.5ex}}
\renewcommand{\SpaceAfterSBBracket}{\vspace{0ex plus1ex minus 0.5ex}}
\renewcommand{\SpaceAfterSection}  {\vspace{0ex plus1ex minus 0.5ex}}
\renewcommand{\SpaceAfterSong}     {\vspace{0ex plus1ex minus 0.5ex}}
\renewcommand{\SpaceAfterVerse}    {\vspace{0ex plus1ex minus 0.5ex}}

% Tell LaTeX that \medskip is a good place to make a page break
\let\oldmedskip\medskip
\renewcommand{\medskip}{\oldmedskip\pagebreak[2]}

%%%
% Turn on/off index-file generation.  Uncomment the \makeindex line to
% turn index generation on;  comment it out to turn index generation
% off.
%%%
%\makeTitleIndex         %% Title and First Line Index.
%\makeTitleContents      %% Table of Contents.
%\makeKeyIndex           %% Index of song by key.
% \makeArtistIndex	%% Index of song by artist.
% \newcommand{\SBThechapter}[0]{}
% \newcommand{\SBChapter}[1]{
%     \startcontents
%     \chapter*{#1} 
%     % \input{unf-sangbog.toc}
%       \begin{minipage}{.8\textwidth}
%         \printcontents{}{1}{}
%       \end{minipage}%
%     \renewcommand{\SBThechapter}{#1}
%     \clearpage
% }

% \titleformat{\chapter}
% [display]
% {}
% {%\vspace*{\fill}
%  % \titlerule[1pt]%
%  % \vspace{1pt}%
%  % \titlerule
%  % \vspace{1pc}%
%  \chaptertitlename}
% {}
% {\Huge}



%%%%%%%%%%%%%%%%%%%%%%%%%%%%%%%%%%%%%%%%%%%%%%%%%%%%%%%%%%
%%%%%%%%%%%%%%%%%%%%%%%%%%%%%%%%%%%%%%%%%%%%%%%%%%%%%%%%%%
%%                                                      %%
%%           D O C U M E N T   B E G I N S              %%
%%                                                      %%
%%%%%%%%%%%%%%%%%%%%%%%%%%%%%%%%%%%%%%%%%%%%%%%%%%%%%%%%%%
%%%%%%%%%%%%%%%%%%%%%%%%%%%%%%%%%%%%%%%%%%%%%%%%%%%%%%%%%%
\begin{document}

%%%
% Uncomment "\maketitle" statement to make a title page.
%%%
%\maketitle
% \begin{titlepage}
%   \centering
%   \vspace{5cm}
% 	\includegraphics[width=1\textwidth]{unf_logo.jpeg}\par\vspace{1cm}
% 	{\scshape\LARGE Sangbog \par}
% 	\vspace{1cm}
% 	{\scshape\Large UNF Computer Science Camp 2019\par}
	
% 	\vfill

% % Bottom of the page
% 	{\large \today\par}
% \end{titlepage}
% \mainmatter
% \ifWordBk
%   \twocolumn
% \fi


%%% Kolofon
%\thispagestyle{empty}
%Sammensat til UNF Computer Science Camp 2019 - csc.unf.dk\\
%Redaktør: Andreas Mosbæk Jensen m.fl. efter tidligere sangbog af Steffen Strunge Mathiesen\\
%Indhold opsat i \LaTeX. 
%Digital version og kildekode: github.com/steffen555/UNF-sangbog\\
%Revision 1 med stave fejl korrektioner
%\par\vspace*{\fill}
%Hvis du har forslag til sange, rettelser, ris og ros, eller hvis du kender en ukendt forfatter, så skriv til sangbog@unf.dk.

%%%
% Turn on and define fancy page heading/footing definition.
%%%
% \pagestyle{fancy}

% \ifChordBk
%   % It's a words & chords songbook...
%   \addtolength{\headwidth}{\marginparsep}
%   \addtolength{\headwidth}{\marginparwidth}
%   \renewcommand{\headrulewidth}{0.4pt}
%   \renewcommand{\footrulewidth}{0.4pt}
%   \fancyhead[LE,RO]{\LHeadFont\emph{\leftmark\/}\SBContinueMark}
%   \fancyhead[CE,CO]{\CHeadFont\thepage}
%   \fancyhead[RE,LO]{\RHeadFont \chaptermark}
% \else\ifOverhead
%   % It's an overhead...
%   \renewcommand{\footrulewidth}{0pt}
%   \renewcommand{\headrulewidth}{0pt}
%   \fancyhead[LE,RO]{}
%   \fancyhead[CE,CO]{}
%   \fancyhead[RE,LO]{}
% \else\ifWordBk
%   % It's a words only songbook...
%   \addtolength{\headwidth}{\marginparsep}
%   \addtolength{\headwidth}{\marginparwidth}
%   \renewcommand{\headrulewidth}{0.4pt}
%   \renewcommand{\footrulewidth}{0.4pt}
%   \fancyhead[LE,RO]{\LHeadFont Naturvidenskab revy sange}
%   \fancyhead[CE,CO]{\CHeadFont\thepage}
%   \fancyhead[RE,LO]{\RHeadFont \SBThechapter}
% \fi\fi\fi

% \fancyfoot[LE,RO]{\LFootFont Computer Science Camp 2019}
% \ifSongEject
%   \fancyfoot[CE,CO]{\CFootFont Last Revised:  \RevDate}
% \else
%   \fancyfoot[CE,CO]{\CFootFont}
% \fi
% \fancyfoot[RE,LO]{\RFootFont Synges på eget ansvar}

%%%
% Table of contents
%%%

% \clearpage
% \twocolumn
% \font\myTinySF=cmss8    at  8pt
% \font\myHugeSF=cmssbx10 at 25pt
% \newcommand{\CpyRtInfoFont}{\tiny\myTinySF}
% \newcommand{\myTitleFont}{\Huge\myHugeSF}
% \newcommand{\mySubTitleFont}{\large\sf}
% \renewcommand{\indexspace}{\medskip}

% % {\parindent 8pt
% %   {\myTitleFont Indhold}}\par
% % \vskip 5pt
% \renewcommand{\SBThechapter}{Indhold}
% % {\parindent 20pt
% %   {\mySubTitleFont --- with first lines in italic ---}}
% % \vskip 20pt
% \let\olditem\item
% \let\oldsubitem\subitem
% \let\oldsubsubitem\subsubitem
% \renewcommand{\item}{\par\hangindent=40pt}
% \renewcommand{\subitem}{\par\hangindent=40pt \hspace*{20pt}}
% \renewcommand{\subsubitem}{\par\hangindent=40pt \hspace*{30pt}}

% %\input{unf-sangbog.tocx}

% \renewcommand{\item}{\olditem}
% \renewcommand{\subitem}{\oldsubitem}
% \renewcommand{\subsubitem}{\oldsubsubitem}

%%%
% Songbook begins.
%%%

\twocolumn
%It's just one page, don't print page numbers etc.
\pagestyle{empty}
%Songs included
\input{songs/matmatik.tex}
\input{songs/taal_daj.tex}
\input{songs/linieskriverdriver.tex}
\input{songs/steve_hawking.tex}
\input{songs/ode_til_kode.tex}
\input{songs/se_min_kode.tex}
\input{songs/vaabenfysik_kort.tex}
%Maybe include:
%\input{songs/kvanter_i_maaneskin.tex}
%\input{songs/mest_matematiske_dyr.tex}

% \input{songs/vi_kan_ikke_li.tex}
% \input{songs/selektionssangen.tex}
% \input{songs/alfabetsangen.tex}
% \input{songs/sciencecamps.tex}
% \input{songs/hvad_maa_man.tex}


% \input{songs/lambda_kalkylen.tex}
% \input{songs/puslespil.tex}
% \input{songs/null.tex}
% \input{songs/fasebal.tex}

% \input{songs/chifitter.tex}

% \input{songs/kun_fysik.tex}



% \input{songs/kanoniske.tex}
% \input{songs/jeg_er_en_matematiker_fra_hcoe.tex}


% \input{songs/rekursiv_skovsang.tex}
% \input{songs/laerkerede.tex}


% \clearpage
% \font\myTinySF=cmss8    at  8pt
% \font\myHugeSF=cmssbx10 at 25pt
% % \newcommand{\CpyRtInfoFont}{\tiny\myTinySF}
% % \newcommand{\myTitleFont}{\Huge\myHugeSF}
% % \newcommand{\mySubTitleFont}{\large\sf}
% \renewcommand{\indexspace}{\medskip}

% {\parindent 8pt
%   {\myTitleFont Index}}\par
% \vskip 5pt
% \renewcommand{\SBThechapter}{Index}
% % {\parindent 20pt
% %   {\mySubTitleFont --- with first lines in italic ---}}
% % \vskip 20pt
% \renewcommand{\item}{\par\hangindent=40pt}
% \renewcommand{\subitem}{\par\hangindent=40pt \hspace*{20pt}}
% \renewcommand{\subsubitem}{\par\hangindent=40pt \hspace*{30pt}}

%\input{unf-sangbog.tdx}

\end{document}
\bye
%
%%%
% Document ends.
%%%


% \renewcommand{\item}{\olditem}
% \renewcommand{\subitem}{\oldsubitem}
% \renewcommand{\subsubitem}{\oldsubsubitem}

%%%
% Songbook begins.
%%%

\twocolumn
%It's just one page, don't print page numbers etc.
\pagestyle{empty}
%Songs included
\input{songs/matmatik.tex}
\input{songs/taal_daj.tex}
\input{songs/linieskriverdriver.tex}
\input{songs/steve_hawking.tex}
\input{songs/ode_til_kode.tex}
\input{songs/se_min_kode.tex}
\input{songs/vaabenfysik_kort.tex}
%Maybe include:
%\input{songs/kvanter_i_maaneskin.tex}
%\input{songs/mest_matematiske_dyr.tex}

% \input{songs/vi_kan_ikke_li.tex}
% \input{songs/selektionssangen.tex}
% \input{songs/alfabetsangen.tex}
% \input{songs/sciencecamps.tex}
% \input{songs/hvad_maa_man.tex}


% \input{songs/lambda_kalkylen.tex}
% \input{songs/puslespil.tex}
% \input{songs/null.tex}
% \input{songs/fasebal.tex}

% \input{songs/chifitter.tex}

% \input{songs/kun_fysik.tex}



% \input{songs/kanoniske.tex}
% \input{songs/jeg_er_en_matematiker_fra_hcoe.tex}


% \input{songs/rekursiv_skovsang.tex}
% \input{songs/laerkerede.tex}


% \clearpage
% \font\myTinySF=cmss8    at  8pt
% \font\myHugeSF=cmssbx10 at 25pt
% % \newcommand{\CpyRtInfoFont}{\tiny\myTinySF}
% % \newcommand{\myTitleFont}{\Huge\myHugeSF}
% % \newcommand{\mySubTitleFont}{\large\sf}
% \renewcommand{\indexspace}{\medskip}

% {\parindent 8pt
%   {\myTitleFont Index}}\par
% \vskip 5pt
% \renewcommand{\SBThechapter}{Index}
% % {\parindent 20pt
% %   {\mySubTitleFont --- with first lines in italic ---}}
% % \vskip 20pt
% \renewcommand{\item}{\par\hangindent=40pt}
% \renewcommand{\subitem}{\par\hangindent=40pt \hspace*{20pt}}
% \renewcommand{\subsubitem}{\par\hangindent=40pt \hspace*{30pt}}

%%%%%%% rcsid = @(#)$Id: sample-sb.tex,v 1.23 2010-04-12 18:04:11 rathc Exp $
%%%%%%
%%
%%      ===============================
%%      Sample Songbook (sample-sb.tex)
%%      ===============================
%%
%%      Version 4.5, 30 April, 2010
%%
%%      Copyright 1992--2010 Christopher Rath <christopher@rath.ca>
%%
%%      This package is free software; you can redistribute it and/or
%%      modify it under the terms of version 2.1 of the GNU Lesser
%%	General Public License as published by the Free Software 
%%	Foundation.
%%
%%      This package is distributed in the hope that it will be
%%      useful, but WITHOUT ANY WARRANTY; without even the implied
%%      warranty of MERCHANTABILITY or FITNESS FOR A PARTICULAR
%%      PURPOSE.  See the GNU Lesser General Public License for more
%%      details.
%%
%%      This file contains a subset of the songbook we distribute
%%      at our church.  To the best of my knowledge, all of the lyrics
%%      contained herein are freely distributable.  This file has been
%%      provided as a sample of what can be produced by the chordbk,
%%      wordbk, and overhead LaTeX styles.
%%
%%      NEEDED:  The fancyhdr LaTeX style is required to properly
%%              format this file.  If you don't have that then comment
%%              out the commands in the preamble which deal with the
%%              fancyhdr style.
%%
%%%%%%
%%%%%%
%%
%%      1. Chord notation.  Within this songbook the following
%%         conventions have been adopted:
%%
%%              "Minor" is entered as "m";
%%                      e.g. Cm7 for C minor 7th.
%%              "Major" is entered as "M";
%%                      e.g. CM7 for C major 7th.
%%
%%%%%%
%%%%%%
%%      ============
%%      Bibliography
%%      ============
%%
%%      Exalt Him!: Exalt Him!  Compiled by Tom Fettke.  (c)1989
%%                      Word Music.
%%
%%      Hosanna! Music Books: Hosanna! Music Books #1--#6.
%%                      (c)1987--92 Integrity Music, Inc.
%%
%%      Worship Him II: Worship Him II.  Compiled by Jesse Peterson
%%                      and Bruce Ballinger.  (c)1989 Tempo Music
%%                      Publications.
%%
%%      Worship Songs Of The Vineyard: Worship Songs Of The Vineyard
%%                      --- Volume 2.  (c)1989 Vineyard Ministries
%%                      International.
%%
%%%%%%
%%%%%%

%%%%%%%%%%%%%%%%%%%%%%%%%%%%%%%%%%%%%%%%%%%%%%%%%%%%%%%%%%
%%%%%%%%%%%%%%%%%%%%%%%%%%%%%%%%%%%%%%%%%%%%%%%%%%%%%%%%%%
%%                                                      %%
%%           P R E A M B L E   B E G I N S              %%
%%                                                      %%
%%%%%%%%%%%%%%%%%%%%%%%%%%%%%%%%%%%%%%%%%%%%%%%%%%%%%%%%%%
%%%%%%%%%%%%%%%%%%%%%%%%%%%%%%%%%%%%%%%%%%%%%%%%%%%%%%%%%%

\documentclass[a5paper]{book}
\usepackage{latexsym,
            fancyhdr,
            titlesec,
            amsmath,
            amssymb,
            multicol,
            amsthm,
            stmaryrd,
            amsthm,
            color,
            needspace,
            stackengine,
            wasysym}
\usepackage[utf8]{inputenc}
\usepackage[T1]{fontenc}
% \usepackage[chordbk]{songbook}                  %% Words & Chords edition.
%%\usepackage[compactallsongs,chordbk]{songbook}    %% Words & Chords edition.
\usepackage[wordbk]{songbook}                 %% Words Only edition.
%%\usepackage[overhead]{songbook}               %% Overhead Transparency edition.
\usepackage{titletoc}
\usepackage{tket}  % Draws "TÅGEKAMMERET" correctly

%%%
% Revision Date and Release Date definitions.
%
%       \RelDate - The last time this songbook was released.  Set this
%                  date each time a new release/update of the songbook
%                  is generated.
%       \RevDate - The last time a particular song was revised in any
%                  way.  This command will be renewed inside every
%                  song.
%%%
\newcommand{\RelDate}{31~August,~2003}
\newcommand{\RevDate}{\today}

%%%
% C.C.L.I. license number definition; for copyright licensing info.
% One of these macros will be manually inserted into the {SBMel}
% parameter of the {song} environment.
%
%       \CCLInumber - The actual copyright license number.  Don't
%               insert this command in the {SBMel} parameter, use one
%               of the others.
%       \CCLIed - Indicates a song falls under our CCLI license.
%       \NotCCLIed - Indicates a song doesn't fall under our CCLI
%               license.  Public Domain songs fall into this category.
%       \PGranted - We have received specific permission from the
%               copyright holder to use this song.
%       \PPending - We are in the process of obtaining permission to
%               use this song.
%%%
\newcommand{\CCLInumber}{Your CCLI Number}
\newcommand{\CCLIed}{{\SBMelInfoFont (CCLI \CCLInumber)}}
\newcommand{\NotCCLIed}{\relax}
\newcommand{\PGranted}{\relax}
\newcommand{\PPending}{{\SBMelInfoFont (Permission Pending)}}

%%%
% Title page information.
%%%
%\title{UNF Computer Science Camp 2019 Sangbog}
%\author{}
%\date{Revideret:  \RevDate}

%%%
% Redefine fonts from SongBook style that I don't like.
%%%
\font\myTinySF=cmss8 at 8pt
\renewcommand{\SBMelInfoFont}{\tiny\myTinySF}

%%%
% Define fonts to use in the headers and footers of the songbook.
%%%
\newcommand{\LHeadFont}{\normalsize}            % = cmr12  at 12pt
\newcommand{\CHeadFont}{\normalsize\rm}         % = cmr12  at 12pt
\newcommand{\RHeadFont}{\normalsize}            % = cmr12  at 12pt
\newcommand{\LFootFont}{\scriptsize}            % = cmr8   at  8pt
\newcommand{\CFootFont}{\tiny\myTinySF}         % = cmss8  at  8pt
\newcommand{\RFootFont}{\scriptsize}            % = cmr8   at  8pt

\def\repeat{%
  \stackanchor{.}{.}%
  \rule[-\dp\strutbox]{.3pt}{\normalbaselineskip}%
  \kern0.5pt%
  \rule[-\dp\strutbox]{1pt}{\normalbaselineskip}%
  \kern1pt%
}
\def\frepeat{%
  \kern1pt%
  \rule[-\dp\strutbox]{1pt}{\normalbaselineskip}%
  \kern0.5pt%
  \rule[-\dp\strutbox]{.3pt}{\normalbaselineskip}%
  \stackanchor{.}{.}%
}
% \newcommand{\SBRepeat}[1]{#1\\#1}
\newcommand{\SBRepeat}[1]{\frepeat #1\repeat}
\setcounter{SBSongCnt}{-1}
\renewcommand{\SBWAndMTag}{Forfatter:}
\renewcommand{\SBUnknownTag}{Ukendt}
\renewcommand{\SBChorusTag}{Ref.}
\renewcommand{\SBOrgMel}{Originalmelodi}
\renewcommand{\SpaceAfterChorus}   {\vspace{0ex plus1ex minus 0.5ex}}
\renewcommand{\SpaceAfterOpGroup}  {\vspace{0ex plus1ex minus 0.5ex}}
\renewcommand{\SpaceAfterSBBracket}{\vspace{0ex plus1ex minus 0.5ex}}
\renewcommand{\SpaceAfterSection}  {\vspace{0ex plus1ex minus 0.5ex}}
\renewcommand{\SpaceAfterSong}     {\vspace{0ex plus1ex minus 0.5ex}}
\renewcommand{\SpaceAfterVerse}    {\vspace{0ex plus1ex minus 0.5ex}}

% Tell LaTeX that \medskip is a good place to make a page break
\let\oldmedskip\medskip
\renewcommand{\medskip}{\oldmedskip\pagebreak[2]}

%%%
% Turn on/off index-file generation.  Uncomment the \makeindex line to
% turn index generation on;  comment it out to turn index generation
% off.
%%%
%\makeTitleIndex         %% Title and First Line Index.
%\makeTitleContents      %% Table of Contents.
%\makeKeyIndex           %% Index of song by key.
% \makeArtistIndex	%% Index of song by artist.
% \newcommand{\SBThechapter}[0]{}
% \newcommand{\SBChapter}[1]{
%     \startcontents
%     \chapter*{#1} 
%     % \input{unf-sangbog.toc}
%       \begin{minipage}{.8\textwidth}
%         \printcontents{}{1}{}
%       \end{minipage}%
%     \renewcommand{\SBThechapter}{#1}
%     \clearpage
% }

% \titleformat{\chapter}
% [display]
% {}
% {%\vspace*{\fill}
%  % \titlerule[1pt]%
%  % \vspace{1pt}%
%  % \titlerule
%  % \vspace{1pc}%
%  \chaptertitlename}
% {}
% {\Huge}



%%%%%%%%%%%%%%%%%%%%%%%%%%%%%%%%%%%%%%%%%%%%%%%%%%%%%%%%%%
%%%%%%%%%%%%%%%%%%%%%%%%%%%%%%%%%%%%%%%%%%%%%%%%%%%%%%%%%%
%%                                                      %%
%%           D O C U M E N T   B E G I N S              %%
%%                                                      %%
%%%%%%%%%%%%%%%%%%%%%%%%%%%%%%%%%%%%%%%%%%%%%%%%%%%%%%%%%%
%%%%%%%%%%%%%%%%%%%%%%%%%%%%%%%%%%%%%%%%%%%%%%%%%%%%%%%%%%
\begin{document}

%%%
% Uncomment "\maketitle" statement to make a title page.
%%%
%\maketitle
% \begin{titlepage}
%   \centering
%   \vspace{5cm}
% 	\includegraphics[width=1\textwidth]{unf_logo.jpeg}\par\vspace{1cm}
% 	{\scshape\LARGE Sangbog \par}
% 	\vspace{1cm}
% 	{\scshape\Large UNF Computer Science Camp 2019\par}
	
% 	\vfill

% % Bottom of the page
% 	{\large \today\par}
% \end{titlepage}
% \mainmatter
% \ifWordBk
%   \twocolumn
% \fi


%%% Kolofon
%\thispagestyle{empty}
%Sammensat til UNF Computer Science Camp 2019 - csc.unf.dk\\
%Redaktør: Andreas Mosbæk Jensen m.fl. efter tidligere sangbog af Steffen Strunge Mathiesen\\
%Indhold opsat i \LaTeX. 
%Digital version og kildekode: github.com/steffen555/UNF-sangbog\\
%Revision 1 med stave fejl korrektioner
%\par\vspace*{\fill}
%Hvis du har forslag til sange, rettelser, ris og ros, eller hvis du kender en ukendt forfatter, så skriv til sangbog@unf.dk.

%%%
% Turn on and define fancy page heading/footing definition.
%%%
% \pagestyle{fancy}

% \ifChordBk
%   % It's a words & chords songbook...
%   \addtolength{\headwidth}{\marginparsep}
%   \addtolength{\headwidth}{\marginparwidth}
%   \renewcommand{\headrulewidth}{0.4pt}
%   \renewcommand{\footrulewidth}{0.4pt}
%   \fancyhead[LE,RO]{\LHeadFont\emph{\leftmark\/}\SBContinueMark}
%   \fancyhead[CE,CO]{\CHeadFont\thepage}
%   \fancyhead[RE,LO]{\RHeadFont \chaptermark}
% \else\ifOverhead
%   % It's an overhead...
%   \renewcommand{\footrulewidth}{0pt}
%   \renewcommand{\headrulewidth}{0pt}
%   \fancyhead[LE,RO]{}
%   \fancyhead[CE,CO]{}
%   \fancyhead[RE,LO]{}
% \else\ifWordBk
%   % It's a words only songbook...
%   \addtolength{\headwidth}{\marginparsep}
%   \addtolength{\headwidth}{\marginparwidth}
%   \renewcommand{\headrulewidth}{0.4pt}
%   \renewcommand{\footrulewidth}{0.4pt}
%   \fancyhead[LE,RO]{\LHeadFont Naturvidenskab revy sange}
%   \fancyhead[CE,CO]{\CHeadFont\thepage}
%   \fancyhead[RE,LO]{\RHeadFont \SBThechapter}
% \fi\fi\fi

% \fancyfoot[LE,RO]{\LFootFont Computer Science Camp 2019}
% \ifSongEject
%   \fancyfoot[CE,CO]{\CFootFont Last Revised:  \RevDate}
% \else
%   \fancyfoot[CE,CO]{\CFootFont}
% \fi
% \fancyfoot[RE,LO]{\RFootFont Synges på eget ansvar}

%%%
% Table of contents
%%%

% \clearpage
% \twocolumn
% \font\myTinySF=cmss8    at  8pt
% \font\myHugeSF=cmssbx10 at 25pt
% \newcommand{\CpyRtInfoFont}{\tiny\myTinySF}
% \newcommand{\myTitleFont}{\Huge\myHugeSF}
% \newcommand{\mySubTitleFont}{\large\sf}
% \renewcommand{\indexspace}{\medskip}

% % {\parindent 8pt
% %   {\myTitleFont Indhold}}\par
% % \vskip 5pt
% \renewcommand{\SBThechapter}{Indhold}
% % {\parindent 20pt
% %   {\mySubTitleFont --- with first lines in italic ---}}
% % \vskip 20pt
% \let\olditem\item
% \let\oldsubitem\subitem
% \let\oldsubsubitem\subsubitem
% \renewcommand{\item}{\par\hangindent=40pt}
% \renewcommand{\subitem}{\par\hangindent=40pt \hspace*{20pt}}
% \renewcommand{\subsubitem}{\par\hangindent=40pt \hspace*{30pt}}

% %\input{unf-sangbog.tocx}

% \renewcommand{\item}{\olditem}
% \renewcommand{\subitem}{\oldsubitem}
% \renewcommand{\subsubitem}{\oldsubsubitem}

%%%
% Songbook begins.
%%%

\twocolumn
%It's just one page, don't print page numbers etc.
\pagestyle{empty}
%Songs included
\input{songs/matmatik.tex}
\input{songs/taal_daj.tex}
\input{songs/linieskriverdriver.tex}
\input{songs/steve_hawking.tex}
\input{songs/ode_til_kode.tex}
\input{songs/se_min_kode.tex}
\input{songs/vaabenfysik_kort.tex}
%Maybe include:
%\input{songs/kvanter_i_maaneskin.tex}
%\input{songs/mest_matematiske_dyr.tex}

% \input{songs/vi_kan_ikke_li.tex}
% \input{songs/selektionssangen.tex}
% \input{songs/alfabetsangen.tex}
% \input{songs/sciencecamps.tex}
% \input{songs/hvad_maa_man.tex}


% \input{songs/lambda_kalkylen.tex}
% \input{songs/puslespil.tex}
% \input{songs/null.tex}
% \input{songs/fasebal.tex}

% \input{songs/chifitter.tex}

% \input{songs/kun_fysik.tex}



% \input{songs/kanoniske.tex}
% \input{songs/jeg_er_en_matematiker_fra_hcoe.tex}


% \input{songs/rekursiv_skovsang.tex}
% \input{songs/laerkerede.tex}


% \clearpage
% \font\myTinySF=cmss8    at  8pt
% \font\myHugeSF=cmssbx10 at 25pt
% % \newcommand{\CpyRtInfoFont}{\tiny\myTinySF}
% % \newcommand{\myTitleFont}{\Huge\myHugeSF}
% % \newcommand{\mySubTitleFont}{\large\sf}
% \renewcommand{\indexspace}{\medskip}

% {\parindent 8pt
%   {\myTitleFont Index}}\par
% \vskip 5pt
% \renewcommand{\SBThechapter}{Index}
% % {\parindent 20pt
% %   {\mySubTitleFont --- with first lines in italic ---}}
% % \vskip 20pt
% \renewcommand{\item}{\par\hangindent=40pt}
% \renewcommand{\subitem}{\par\hangindent=40pt \hspace*{20pt}}
% \renewcommand{\subsubitem}{\par\hangindent=40pt \hspace*{30pt}}

%\input{unf-sangbog.tdx}

\end{document}
\bye
%
%%%
% Document ends.
%%%


\end{document}
\bye
%
%%%
% Document ends.
%%%


\end{document}
\bye
%
%%%
% Document ends.
%%%


\end{document}
\bye
%
%%%
% Document ends.
%%%
